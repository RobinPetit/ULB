\documentclass{article}

\usepackage{palatino,eulervm}
\usepackage[T1]{fontenc}
\usepackage[utf8]{inputenc}
\usepackage{fullpage}
\usepackage[parfill]{parskip}
\usepackage[bottom]{footmisc}
% math packages
\usepackage{amsmath, amsthm, amssymb}
\usepackage{mathtools}
\usepackage{mathdots}
\usepackage{interval}
\usepackage{commath}
\usepackage{hyperref}
\usepackage{stmaryrd}

\makeatletter
\def\thm@space@setup{%
	\thm@preskip=.4cm%
	\thm@postskip=\thm@preskip%
}
\makeatother

\title{Algèbre linéaire et géométrie}
\author{R. Petit}
\date{année académique 2015 - 2016}

\DeclareMathOperator{\tq}{\text{ t.q. }}
\DeclareMathOperator{\Id}{Id}
\DeclareMathOperator{\GCD}{GCD}
\DeclareMathOperator{\Ker}{Ker}
\DeclareMathOperator{\Imf}{Im}
\DeclareMathOperator{\rang}{rang}
\DeclareMathOperator{\sign}{sign}
\DeclareMathOperator{\Hom}{Hom}
\DeclareMathOperator{\adj}{adj}

\newcommand{\N}{\mathbb N}
\newcommand{\Z}{\mathbb Z}
\newcommand{\Q}{\mathbb Q}
\newcommand{\R}{\mathbb R}
\newcommand{\C}{\mathbb C}
\newcommand{\K}{\mathbb K}
\newcommand{\scpr}[2]{\left\langle #1, #2\right\rangle}
\newcommand{\eng}[1]{\left\langle#1\right\rangle}
\renewcommand{\interval}[2]{\left[#1, #2\right]}
\newcommand{\M}[3]{M_{#1 \times #2}(#3)}
\newcommand{\Perm}{\mathfrak{S}}
\newcommand{\Aor}{A^{or}}
\newcommand{\Vor}{V^{or}}
\newcommand{\restr}[2]{ %
	\left.\kern-\nulldelimiterspace %
	#1 %
	\vphantom{\big|} %
	\right|_{#2} %
}

% amsthm
\newtheorem{thm}{Théorème}[section]
\newtheorem{prp}[thm]{Proposition}
\newtheorem{cor}[thm]{Corollaire}
\newtheorem{lem}[thm]{Lemme}
\renewcommand{\proofname}{\it{Démonstration}}
\theoremstyle{definition}
\newtheorem{alg}[thm]{Algorithme}
\newtheorem{déf}[thm]{Définition}
\theoremstyle{remark}
\newtheorem*{rmq}{Remarque}

\begin{document}

\pagenumbering{Roman}
\maketitle
\tableofcontents
\newpage
\pagenumbering{arabic}

\section{Rappels}
	\subsection{Ensembles et fonctions}
		\begin{déf} Soient $X, Y$ deux ensembles. Une fonction $f : X \to Y$ est une \textit{correspondance} qui associe chaque élément de $X$ à un élément unique de $Y$
		que l'on appelle $f(x) \in Y$. $X$ est appelé le domaine de $f$ et $Y$ est le codomaine. \end{déf}

		\begin{déf}~
		\begin{itemize}
			\item L'image de $A \subseteq X$ par $f : X \to Y$ est $f(A) \coloneqq \{f(x) \tq x \in A\}$. L'image de $f$ est $f(X) \subseteq Y$.
			\item La préimage de $B \subseteq Y$ par $f$ est $f^{-1}(B) \coloneqq \{x \in X \tq f(x) \in B\}$. La préimage de $y \in Y$ est donnée par
			      $f^{-1}(y) \coloneqq f^{-1}(\{y\})$.
		\end{itemize}
		\end{déf}

		\begin{déf} Soit $f : X \to Y$.
		\begin{itemize}
			\item Une fonction $f : X \to Y$ est injective si $\forall x, x' \in X : f(x) = f(x') \Rightarrow x = x'$.
			\item Une fonction $f : X \to Y$ est surjective si $\forall y \in Y : \exists x \in X \tq f(x) = y$ (ou encore si $f(X) = Y$).
			\item Une fonction $f : X \to Y$ est bijective si elle est injective et surjective.
		\end{itemize}
		\end{déf}

		\begin{déf} Soit $f : X \to Y$ une fonction bijective. On définit la fonction inverse $f^{-1}$ de $f$ par $f^{-1} : Y \to X : y \mapsto x \tq f(x) = y$. \end{déf}

		\begin{déf} Soient $X, Y, Z$ trois ensembles. Soient $f : X \to Y, g : Y \to Z$ deux fonctions. On définit la composée $g \circ f : X \to Z : x \mapsto g(f(x))$.
		\end{déf}

		\begin{lem} La composée $f \circ f^{-1} = \Id_Y$ et la composée $f^{-1} \circ f = \Id_X$. \end{lem}

		\begin{déf} Soient $X, Y$ deux ensembles. On définit le produit cartésien~: \[X \times Y \coloneqq \{(x, y) \tq x \in X,\;y\in Y\}.\] \end{déf}

		\begin{déf} Soit $\{X_i\}_{i \in I}$ une collection d'ensembles. On définit~: \[\prod_{i \in I} X_i \coloneqq \{(x_i)_{i \in I} \tq x_i \in X_i\}.\] \end{déf}

		\begin{déf} Soit $f : X \to Y$ une fonction. On définit le graphe de $f$ par~: \[\Gamma_f \coloneqq \{(x, f(x)) \tq x \in X\} \subseteq X \times Y.\] \end{déf}

	\subsection{Trigonométrie}
		\begin{déf} Les fonctions $\cos, \sin : \R \to \R$ représentent respectivement l'abscisse et l'ordonnée d'un point dont les coordonnées polaires sont
		$(1, \theta)$. \end{déf}

		\begin{lem} Soit $\theta \in \R$. Il existe un unique $k \in \Z$ tel que $\theta + 2k\pi \in [0, 2\pi[.$\end{lem}

		\begin{thm} $\forall \theta \in \R : \sin(\theta)^2 + \cos(\theta)^2 = 1$. \end{thm}

		\begin{proof} Par Pythagore. \end{proof}

		\begin{lem} Les fonctions $\cos, \sin$ peuvent être réduites à un domaine afin d'être bijectives (si ce domaine est continu, il doit être de longueur $\pi$).
		\end{lem}

		\begin{déf}~
		\begin{itemize}
			\item On définit les fonctions inverses $\arccos : \interval {-1}1 \to \interval 0\pi$ et $\arcsin : \interval {-1}1 \to \interval {\frac \pi2}{\frac 32\pi}$.
			\item On définit également la fonction $\tan : ]-\frac \pi2, \frac \pi2[ \to \R : \theta \mapsto \frac {\sin\theta}{\cos\theta}$ et son inverse
			      $\arctan : \R \to ]-\frac \pi2, \frac \pi2[$.
		\end{itemize}
		\end{déf}

	\subsection{Nombres complexes}
		\begin{déf} On définit l'ensemble des complexes $\C \coloneqq \{a + ib \tq (a, b) \in \R^2\}$. Où $i$ est l'unité imaginaire telle que $i^2 = -1$. \end{déf}

		\begin{déf} Soit $z = a + ib \in \C$ un complexe. On définit $\Re z \coloneqq a$ et $\Im z \coloneqq b$ respectivement la partie réelle et la partie
		imaginaire de $z$. \end{déf}

		\begin{déf} Soit $z \in \C$ un complexe. $z$ peut s'écrire soit sous sa forme cartésienne $a + ib$ soit sous sa forme polaire $\rho \exp(i\theta)$. $\rho$
		est le module de $z$ et $\theta$ est son argument. Il existe une relation entre $(a, b)$ et $(\rho, \theta)$ : $(a, b) = (\rho\cos\theta, \rho\sin\theta)$.
		\end{déf}

		\begin{rmq} On peut visualiser l'ensemble des réels par $\R \equiv \{z \in \C \tq \Im z = 0\} \subseteq \C$. \end{rmq}

		\begin{déf} On définit la somme et le produit de deux complexes $z = a+ib, w = c + id \in \C$ comme suit~:
		\begin{align*}
			w+z &\coloneqq (a+ib) + (c+id) = (a+c) + (b+d)i &\in \C \\
			w z &\coloneqq (a+ib)(c+id) = (ac - bd) + (ad+bc)i &\in \C
		\end{align*}
		On définit également le module d'un complexe $z = a+ib = \rho\exp(i\theta) \in \C$ par $\abs z = \sqrt {a^2+b^2} = \rho \in \R^+$. \end{déf}

		\begin{prp}[Propriétés des opérations complexes] Soient $v, w, z \in \C$. Alors~:
		\begin{enumerate}
			\item $z+w = w+z$~;
			\item $z+(w+v)=(z+w)+v$~;
			\item $z+0 = z$~;
			\item $z + (-1)z = z - z = 0$~;
			\item $zw = wz$~;
			\item $z(wv) = (zw)v$~;
			\item $z(w+v) = zw + zv$~;
			\item $1z = z$~;
			\item $\abs{z} \geq 0$ avec $\abs z = 0 \iff z = 0$~;
			\item $\abs{zw}  = \abs z\abs w$~;
			\item $\abs{z+w} \leq \abs z + \abs w$.
		\end{enumerate}
		\end{prp}

		\begin{proof} Les points 1 à 8 se démontrent par les mêmes propriétés sur les nombres réels. Le point 9 découle du fait qu'une racine carrée est toujours positive
		et $\sqrt {a^2+b^2} = 0 \iff a^2+b^2 = 0$. Le point 10 se montre par les propriétés de la racine carrée. Le point 11 vient de l'inégalité de Cauchy-Schwartz.
		\end{proof}

		\begin{rmq} La somme et le produit peuvent également être exprimés en coordonnées polaires. De plus, le produit est plus instinctif en coordonnées polaires~:
		$\forall z = \rho_z\exp(i\theta_z), w = \rho_w\exp(i\theta_w) \in \C : zw = \rho_z\rho_w\exp(i(\theta_z+\theta_w))$. \end{rmq}

		\begin{déf} Soit $z = a + ib = \rho\exp(i\theta) \in \C$ un complexe. On définit son conjugué $\overline z = a - ib = \rho\exp(-i\theta) \in \C$. \end{déf}

		\begin{prp}[Propriétés du conjugué] Soient $z, w, \overline z, \overline w \in \C$ deux complexes et leur conjugué.
		\begin{enumerate}
			\item $\overline {z+w} = \overline z + \overline w$~;
			\item $\overline {zw} = \overline z\cdot\overline w$~;
			\item $z\overline z = \abs z^2$.
		\end{enumerate}
		\end{prp}

		\begin{proof} Trivial par la définition. \end{proof}

		\begin{déf} Par la propriété 3 ci-dessus, on définit pour tout $z \in \C \setminus \{0\}$ un inverse multiplicatif~:
		\[z^{-1} \coloneqq \frac {\overline z}{\abs z^2}.\] \end{déf}

		\begin{déf}[L'identité d'Euler] On définit l'exponentielle complexe par $\exp(i\theta) \coloneqq \cos\theta + i\sin\theta$. \end{déf}

		\begin{prp} Soient $z, w \in \C$ deux complexes. Alors $zw = 0 \Rightarrow z=0 \lor w=0$. \end{prp}

		\begin{proof} Soient $z, w \in \C$. On sait que $zw = \rho_z\rho_w\exp(i(\theta_z+\theta_w)) = 0$. Or $\exp$ est toujours strictement positive.
		Dès lors, $\rho_z=0$ ou $\rho_w=0$. \end{proof}

		\begin{thm}\label{thmFondamental} Tout polynôme à coefficients complexes de degré $n$ a $n$ racines complexes. \end{thm}

		\begin{prp} Soit $z = \rho \exp(i\theta) \in \C$. Le polynôme $x^n - z$ admet $n$ racines complexes distinctes données par~:
		\[w_k = \sqrt[n]\rho\exp\left(i\frac {\theta+2k\pi}n\right).\]\end{prp}

		\begin{proof} On sait par le théorème~\ref{thmFondamental} que le polynôme admet $n$ racines. Montrons que $w_k$ est une racine de $x^n-z$. En étendant la
		définition du produit vu ci-dessus, on sait que~: \[\prod_{j=1}^nz_j = \prod_{j=1}^n(\rho_j)\exp\left(i\sum_{j=1}^n\theta_j\right).\]

		On peut étendre cette formule pour la puissance~: \[z^n = \rho^n\exp(ni\theta).\]

		Dès lors, en prenant une racine $w_k$, on a~:
		\[w_k^n = \left(\sqrt[n]\rho\exp\left(i\frac {\theta+2k\pi}n\right)\right)^n = \sqrt[n]\rho^n\exp\left(in\frac {\theta+2k\pi}n\right) = \rho\exp(i(\theta+2k\pi)).\]

		On a donc effectivement $w_k^n-z = 0$. \end{proof}

\newpage
\section{Les polynômes}
	\subsection{L'algorithme d'Euclide sur des entiers}
		\begin{déf} Soient $(a, b) \in \Z \times \Z_0$. Il existe un unique $d \in \Z_0$ tel que $d$ est le plus grand nombre qui divise $a$ et $b$ simultanément.
		On note ce nombre $GCD(a, b)$.\end{déf}

		\begin{rmq} Pour tout couple $(a, b)$, notons que $0 \leq GCD(a, b) < \max\{\abs a, \abs b\}$. \end{rmq}

		\begin{déf} Soient $(a, b) \in \Z \times \Z_0$. Il existe un unique $q \in \Z$ tel que $r \coloneqq a - qb$ satisfait $0 \leq r < \abs b$. $q$ et $r$ sont
		respectivement le reste et le quotient de la division euclidienne de $a$ par $b$. \end{déf}

		\begin{alg}[Algorithme d'Euclide]\label{algEuclide} L'algorithme d'Euclide est un moyen très efficace pour déterminer le GCD entre deux nombres entiers. En posant
		$r_{-1} \coloneqq a$ et $r_0 \coloneqq b$. Ensuite pour $i > 0$, on trouve les uniques $q_i, r_i$ tels que~: \[r_{i-2} = q_ir_{i-1} + r_i,\] où
		$0 \leq r_i < \abs{r_{i-1}}$. On s'arrête lorsque $r_i = 0$. \end{alg}

		\begin{thm} Soient $a, b$ deux entiers avec $b$ non-nul. Prenons $(r_i)_i, (q_i)_i$ donnés par l'algorithme d'Euclide. Soit $k \geq 0$ le plus petit entier tel que
		$r_{k+1} = 0$. Alors $\GCD(a, b) \coloneqq \abs {r_k}$\end{thm}

		\begin{proof} Montrons d'abord que tout diviseur de $a$ et de $b$ divise $r_k$ et ensuite montrons que $r_k$ divise $a$ et $b$.

		On sait pour tout $i$ que $r_{i-2} = q_ir_{i-1} + r_i$. Donc si $d$ divise $r_{i-1}$ et $r_{i-2}$, il doit diviser $r_i$. Dès lors les diviseurs (en particulier
		le GCD) de $a$ et $b$ doivent diviser $r_1$ et donc $r_2$, etc. jusque $r_k$.

		Montrons que $r_k$ divise $a$ et $b$. On sait que $r_{k+1} = 0$. Donc $r_{k-1} = q_{k+1}r_k + r_{k+1} = q_{k+1}r_k$. Donc $r_k$ divise $r_{k-1}$. De plus,
		$r_{k-2} = q_kr_{k-1} + r_k$. $r_k$ divise donc $r_{k-2}$, etc. jusque $r_{-1}$.

		On a donc montré que tous les diviseurs de $a$ et de $b$ divisent $r_k$ et que $r_k$ divise $a$ et $b$. \end{proof}

		\begin{cor} Soient $a, b$ deux entiers. Il existe deux uniques entiers $y, z$ tels que $ya + zb = \GCD(a, b)$. \end{cor}

		\begin{proof} Par l'algorithme d'Euclide, on sait qu'il existe des uniques $(r_i)_i, (q_i)_i$. Soit $k \geq 0$ le plus petit entier tel que $r_{k+1} = 0$. On sait
		que $r_k = r_{k-2} - q_kr_{k-1}$. Or $r_{k-1}$ peut s'exprimer en fonction de $r_{k-2}$ et $r_{k-3}$. Donc $r_k = \alpha_2r_{k-2}+\beta_3r_{k-3}$.
		On répète ce procédé $k$ fois. On a donc $\GCD(a, b) = r_k = \alpha_kr_0 + \beta_{k+1}r_{-1} = \alpha_kb + \beta_{k+1}a$ avec $\alpha, \beta \in \Z$. \end{proof}

	\subsection{L'algorithme d'Euclide sur des polynômes}
		\begin{déf} On définit l'anneau des polynômes à coefficients réels par $\R[x]$. \end{déf}

		\begin{déf} Soit $f \in \R[x]$. $\displaystyle f(x) = \sum_{i=1}^na_ix^i$ où $a_i \in \R$ pour tout $i$ et $a_n \neq 0$. $n$ est le degré du polynôme.
		On note~: \[\deg f(x) \coloneqq n.\] \end{déf}

		\begin{rmq} Par convention, le degré du polynôme $x \mapsto 0$ est $-\infty$. \end{rmq}

		\begin{déf} Soit $f \in \R[x]$ un polynôme de degré $n$. $f$ est monique (ou unitaire) si le coefficient du monôme de degré $n$ vaut 1 (si $a_n = 1$). \end{déf}

		\begin{déf} Soient $f, g \in \R[x]$. On définit $\GCD(f(x), g(x))$ par l'unique polynôme monique de degré maximal qui divise $f(x)$ et $g(x)$. \end{déf}

		\begin{déf} Soit $f \in \R[x]$ un polynôme de degré $n$. Pour tout $0 \leq k \leq n$, on définit $[x^k]f(x)$ par le coefficient du monôme de degré $k$ de $f$.
		\end{déf}

		\begin{thm} Soient $f, g \in \R[x]$ tels que $g(x) \neq 0$. Il existe deux uniques polynômes $q, r \in \R[x]$ tels que $f(x) = q(x)g(x) + r(x)$ satisfaisant
		$\deg r(x) < \deg g(x)$. $q$ est appelé le quotient et $r$ le reste de la division de $f$ par $g$. \end{thm}

		\begin{alg}[Algorithme d'Euclide sur des polynômes] Soient $f, g \in \R[x]$ tels que $g(x) \neq 0$. Comme pour l'algorithme sur des entiers, posons
		$r_{-1}(x) \coloneqq f(x)$ et $r_0(x) = g(x)$. Dès lors, pour tout $i > 0$, il existe des uniques $q_i, r_i$ tels que~:\[r_{i-2}(x) = q_i(x)r_{i-1}(x) + r_i(x)\]
		où $\deg r_i(x) < \deg r_{i-1}(x)$. On s'arrête lorsque $r_i(x) = 0$. \end{alg}

		\begin{thm} Soient $f, g \in \R[x]$. Prenons $(r_i)_i, (q_i)_i \subset \R[x]$ donnés par l'algorithme de division de $f$ par $g$. Soit le plus petit $k \in \N$
		tel que $r_{k+1}(x) = 0$. $r_k(x)$ est le GCD non-normalisé de $f(x)$ et $g(x)$. Et on définit~:
		\[\GCD(f(x), g(x)) \coloneqq \frac {r_k(x)}{[x^{\deg r_k(x)}]r_k(x)}.\] \end{thm}

		\begin{proof} Comme pour l'algorithme sur les nombres entiers, si un polynôme divise $f(x)$ et $g(x)$, alors il doit diviser $r_i(x)$ pour tout $i$. De plus,
		comme par hypothèse $r_{k+1}(x) = 0$, $r_{k-1}(x) = q_{k+1}(x)r_k(x)$. Donc $r_k(x)$ divise $r_{k-1}(x)$. Et donc $r_k(x)$ divise également $r_{k-2}(x)$, etc.
		jusque $f(x)$ et $g(x)$. On a donc bien que tout diviseur simultané de $f$ et $g$ divise $r_k$ et $r_k$ divise à la fois $f$ et $g$. Dès plus, en divisant
		$r_k$ par le coefficient de son monôme dominant, on en fait un polynôme monique. \end{proof}

		\begin{cor} Soient $f, g \in \R[x]$. Il existe deux uniques polynômes $y(x), z(x)$ tels que $y(x)f(x)+z(x)g(x) = \GCD(f(x), g(x))$. \end{cor}

		\begin{proof} Preuve similaire à l'algorithme sur des entiers. \end{proof}

	\subsection{Racines}
		\begin{déf} Soit $P \in \C[x]$ un polynôme complexe. Une racine de $P$ est un complexe $z \in \C$ est une complexe tel que $P(z) = 0$. \end{déf}

		\begin{rmq} Le théorème fondamental (théorème~\ref{thmFondamental}) dit qu'un polynôme complexe $f(x) \in \C[x]$ a exactement $\deg f(x)$ racines complexes en
		comptant la multiplicité. On peut donc réécrire le polynôme $f$ comme~: \[f(x) = c\prod_{i=1}^{\deg f(x)}(x-z_i)\] où $\{z_i \tq 1 \leq i \leq \deg(f(x))\}$ est
		l'ensemble des racines (pas obligatoirement distinctes) de $f$.\end{rmq}

		\begin{déf} Soit $f \in \C[x]$ un polynôme complexe définie par~: \[f(x) = \sum_{i=1}^na_ix^i.\] On définit son polynôme conjugué par~:
		\[\overline f(x) = \sum_{i=1}^n\overline {a_i}x^i.\] \end{déf}

		\begin{rmq} Si $f \in \R[x] \subset \C[x]$, alors son conjugué $\overline f(x) = f(x)$ car $\forall x \in \R : \overline x = x$. \end{rmq}

		\begin{lem} Soit $f \in \C[x]$. Alors $\overline f(\overline z) = \overline {f(z)}$. \end{lem}

		\begin{proof} \[\overline f(\overline z) = \sum_{i=1}^n\overline {a_i}\cdot\overline z = \sum_{i=1}^n\overline {a_iz} = \overline {\sum_{i=1}^na_iz} =
		\overline {f(z)}.\] \end{proof}

		\begin{prp} Soit $f \in \R[x]$ un polynôme à coefficients réels. Le théorème fondamental (théorème~\ref{thmFondamental}) dit que $f$ a exactement $\deg(f(x))$
		racines complexes. Soit $z$ une telle racine. Alors le conjugué de $z$ est également une racine de $f$. \end{prp}

		\begin{proof} Soient $f \in \R[x]$ et $z$ une racine de $f$. Alors $0 = f(z) = \overline {f(z)} = \overline f(\overline z) = f(\overline z)$. \end{proof}

		\begin{prp} Soit $f \in \R[x]$ un polynôme complexe. Soit $z_0 \in \C$ une racine complexe de $f$ telle que $\Im z \neq 0$. Alors le polynôme suivant~:
		\[(x-z_0)(z-\overline {z_0})\] est un diviseur de $f(x)$. \end{prp}

		\begin{proof} Par hypothèse, une telle racine $z_0$ existe. Dès lors le polynôme $(x-z_0)(x-\overline {z_0})$ est un polynôme réel. En effet~:
		$(x-z_0)(x-\overline {z_0}) = x^2 -z_0x - \overline {z_0}x + z_0\overline z_0 = x^2 - 2\Re z_0 - \abs{z_0}^2 \in \R$. Dès lors, on peut trouver les polynômes
		quotient et reste tels que~: \[f(x) = q(x)(x-z_0)(x-\overline {z_0}) + r(x).\]

		Ensuite, montrons que $\deg r(x) = 0$. En effet, $0 \leq \deg r(x) < 2$ car 2 est le degré du polynôme diviseur. Supposons par l'absurde que $\deg r(x) = 1$.
		Alors $r(x) = ax+b$ pour $a, b \in \R$. L'évaluation en $z_0$ donne donc $0 = f(z_0) = r(z_0) = az_0+b$. Donc $\Im r(x) = \Im(az_0+b) = a\Im z_0 \neq 0$. Il y
		a donc contradiction.

		Le reste étant constant (de degré nul), le polynôme $(x-z_0)(x-\overline{z_0})$ doit diviser $f(x)$. \end{proof}

		\begin{cor} Tout polynôme $f(x) \in \R[x]$ se factorise en un produit de polynômes de degré 2 à racines purement complexes conjuguées et de polynômes de
		degré 1 à racines réelles. \end{cor}

		\begin{rmq} Si $f(x) \in \R[x]$ est de degré impair, alors il doit avoir une racine réelle. \end{rmq}

	\subsection{Fonctions symétriques}
		\begin{déf} Soit $f \in \C[x]$ un polynôme monique de degré $n$. Soient $z_1, \dotsc, z_n$ ses $n$ racines, multiplicité comptée. On définit, pour tout
		$1 \leq k \leq n$, les fonctions symétriques élémentaires de $f$ par~:
		\[e_k(z_1, \dotsc, z_n) \coloneqq \sum_{1 \leq i_1 < \ldots < i_k \leq n}\prod_{j=1}^kz_{i_j}.\]
		De plus, on définit $e_0(z_1, \dotsc, z_n) \coloneqq 1$. \end{déf}

		\begin{lem} Soit un polynôme complexe monique $f \in \C[x]$ de degré $n$ et ses fonctions symétriques élémentaires $e_k : \C^n \to \C$. Alors pour tout
		$1 \leq k \leq n$, on a: \[e_k(z_1, \dotsc, z_n) = e_k(z_1, \dotsc, z_{n-1}) + z_ne_{k-1}(z_1, \dotsc, z_{n-1}).\]\end{lem}

		\begin{proof} Cette formule donne deux termes~:
		\[\sum_{1 \leq i_1 < \ldots < i_k \leq n-1}\prod_{j=1}^kz_{i_j} + z_n\sum_{1 \leq i_1 < \ldots < i_{k-1} \leq n-1}\prod_{j=1}^{k-1}z_{i_j}.\]
		En posant $i_k = n$ dans le terme de droite, on obtient~:
		\[e_k(z_1, \dotsc, z_n) = \sum_{1 \leq i_1 < \ldots < i_k \leq n-1}\prod_{j=1}^kz_{i_j} + \sum_{1 \leq i_1 < \ldots < i_{k-1} \leq n-1}\prod_{j=1}^kz_{i_j}.\]
		On a, dans le terme de gauche les $\binom {n-1}k$ termes ne contenant pas $z_n$ et dans le terme de droite les $\binom {n-1}{k-1}$ termes contenant $z_n$.
		On a donc bien les $\binom nk$ termes de $e_k$. \end{proof}

		\begin{prp} Soit un polynôme monique $f \in \C[x]$ de degré $n$, soient $(z_i)_{1 \leq i \leq n}$ ses racines et soient $e_k : \C^n \to \C$ ses fonctions
		symétriques élémentaires. Alors, si $f$ est défini par~: \[f(x) = \sum_{i=1}^na_ix^i,\] avec $a_i \in \C$ et $a_n \neq 0$, alors, pour tout $0 \leq i \leq n$~:
		\[e_{n-i}(z_1, \dotsc, z_n) = (-1)^{n-i}a_i.\] \end{prp}

		\begin{proof} Prouvons le par récurrence sur $n$. Montrons d'abord pour $n=1$ :\[f(x) = x-z_1 = a_1x + a_0.\] De plus~:
		\begin{align*}
			&(-1)^{1-0}a_0 = e_{1-0}(z_1) \iff a_0 = -e_1(z_1) = z_1. \\
			&(-1)^{1-1}a_1 = e_{1-1}(z_1) \iff a_1 = 1.
		\end{align*}
		Le cas initial est donc bon. Montrons maintenant le pas de récurrence. Supposons cela vrai pour $n-1$ et montrons-le pour $n$~:
		\begin{align*}
			f(x) &= \prod_{i=1}^n(x-z_i) = x\prod_{i=1}^{n-1}(x-z_i) - z_n\prod_{i=1}^{n-1}(x-z_i) = x\sum_{i=0}^{n-1}b_ix^i - z_n\sum_{i=0}^{n-1}b_ix^i \\
			     &\stackrel{hyp. rec.}= x\sum_{i=0}^{n-1}(-1)^{n-1-i}e_{n-1-i}(z_1, \dotsc, z_{n-1})x^i - z_n\sum_{i=0}^{n-1}(-1)^{n-1-i}e_{n-i-1}(z_1, \dotsc, z_{n-1})x^i \\
			     &= \sum_{i=1}^n(-1)^{n-i}e_{n-i}(z_1, \dotsc, z_{n-1})x^i + z_n\sum_{i=0}^{n-1}(-1)^{n-i}e_{n-i-1}(z_1, \dotsc, z_{n-1})x^i \\
			     &= (-1)^ne_{n-1}(z_1, \dotsc, z_{n-1})z_n + \sum_{i=1}^{n-1}(-1)^{n-i}(e_{n-i}(z_1, \dotsc, z_{n-1}) + z_ne_{n-i-1}(z_1, \dotsc, z_{n-1}))x^i + x^n \\
			     &= \sum_{i=0}^n(-1)^{n-i}e_{n-i}(z_1, \dotsc, z_n)x^i.
		\end{align*}

		On a donc bien prouvé l'égalité pour chaque coefficient, donc pour tout $i$ et en faisant la preuve par récurrence sur $n$. \end{proof}

\newpage
\section{Géométrie analytique dans $\R^3$~: rappels}
	\subsection{Vecteurs}
		\begin{déf} Soit $\R^3$, l'espace à trois dimensions. Pour tout $v \in \R^3$, on définit (naïvement) le vecteur $v$ comme une flèche partant de l'origine
		$\mathcal O = (0, 0, 0)$ au point $v$. On définit donc les coordonnées de $v$ comme étant $v_1, v_2, v_3 \in \R$ tels que $v = (v_1, v_2, v_3)$. \end{déf}

		\begin{déf} Soient $v, w \in \R^3$, deux vecteurs et $\lambda \in \R$ un scalaire . On définit la somme interne et le produit externe par~:
		\begin{align*}
			v+w &\coloneqq (v_1+w_1, v_2+w_2, v_3+w_3), \\
			\lambda v &\coloneqq (\lambda v_1, \lambda v_2, \lambda v_3).
		\end{align*}
		\end{déf}

		\begin{rmq} Pour tout $\lambda \in \R$ scalaire, le vecteur $\lambda v$ se trouve sur la droite engendrée par le vecteur $v$ (droite passant par $\mathcal O$
		et $v$). On parle alors de droite vectorielle $v\R$. \end{rmq}

		\begin{prp} Ces définitions d'opérations ont les propriétés suivantes. Pour tout $u, v, w \in \R^3, \lambda, \mu \in \R$~:
		\begin{enumerate}
			\item $v+w = w+v$~;
			\item $(u+v)+w = u+(v+w)$~;
			\item $\lambda(\mu v) = (\lambda\mu)v$~;
			\item $0v = 0$~;
			\item $0 + v = v$~;
			\item $1v = v$~;
			\item $v - v \coloneqq v + (-v) = v + (-1)v = 0 = (0, 0, 0)$~;
			\item $\lambda (v+w) = \lambda v + \lambda w$~;
			\item $(\lambda + \mu) v = \lambda v + \mu v$.
		\end{enumerate}
		\end{prp}

		\begin{proof} Trivial par ces mêmes opérations sur $\R$. \end{proof}

	\subsection{Droites et plans}
		\begin{déf} Soit $v \in \R^3$ un vecteur. Alors $v\R$ est la droite engendrée par le vecteur $v$. Son équation vectorielle est $D : \lambda v$.
		Soit $w \in \R^3$ un autre vecteur. L'équation vectorielle de la droite parallèle à $D$ et passant par $w$ est $D_2 : w + \lambda v$. $v$ est appelé le
		vecteur directeur de ces droites. \end{déf}

		\begin{déf} Soient $v, w \in \R^3$ deux vecteurs non colinéaires. L'équation vectorielle du plan regroupant tous les points de la droite $D : \lambda v$
		et ses translations par les vecteurs $\mu w$ est $\pi : \lambda v + \mu w$. Ce plan passe par l'origine car il contient tous les points de $\lambda v$.
		Soit $u$ un troisième vecteur. L'équation vectorielle du plan parallèle à $\Pi$ et passant par $u$ est $\Pi_u : u + \lambda v + \mu w$. \end{déf}

		\begin{déf} Les deux définitions précédentes concernent les équations dites vectorielles. Une droite ou un plan peuvent également être exprimés en coordonnées
		paramétriques ou cartésiennes. Soit $P = (x, y, z) \in \R^3$. Soient $D : \lambda v + w$, $\Pi : \lambda v + \mu w + u$.
		\begin{enumerate}
			\item \[P \in D \iff \exists \lambda \in \R \tq \left\{\begin{aligned}x &= \lambda v_1 + w_1 \\y &= \lambda v_2 + w_2 \\z &= \lambda v_3 + w_3\end{aligned}\right.~;\]
			\item \[P \in \Pi \iff \exists (\lambda, \mu) \in \R^2 \tq
			\left\{\begin{aligned}
				x &= \lambda v_1 + \mu v_2 + u_1 \\y &= \lambda v_2 + \mu v_2 + u_2 \\z &= \lambda v_3 + \mu w_3 + u_3
			\end{aligned}\right..\]
		\end{enumerate}

		Ces équations sont les équation paramétriques. Pour trouver les équations cartésiennes, il faut résoudre le système (en $\lambda, \mu$ pour le plan et en $\lambda$
		pour la droite). On a donc~: \[\Pi : ax + by + cz = d.\] Et pour la droite $D$, on a:
		\[D : \left\{\begin{aligned}a_1x + b_1y + c_1z &= d_1 \\a_2x + b_2y + c_2z &= d_2\end{aligned}\right..\] \end{déf}

		\begin{rmq} L'équation cartésienne d'une droite est un couple de coordonnées cartésiennes de plan. Cela peut se comprendre comme suit~: l'ensemble des points de
		la droite sont les points satisfaisant simultanément la première équation et la seconde, ou encore tous les points étant dans l'intersection entre les deux plans.
		En coordonnées cartésiennes, une droite est définie par deux plans et un point par trois plans (non-parallèles). \end{rmq}

	\subsection{Produit scalaire et orthogonalité}
		\begin{déf} Soient $v, w \in \R^3$ deux vecteurs. On définit le produit scalaire entre $v$ et $w$ par \[\scpr vw \coloneqq \sum_{i=1}^3v_iw_i.\] \end{déf}

		\begin{prp} Soient $v, w \in \R^3$. Le produit scalaire $\scpr vw$ satisfait les propriétés suivantes pour tout $\lambda \in \R, u, v ,w \in \R^3$~:
		\begin{enumerate}
			\item $\scpr vw = \scpr wv$~;
			\item $\scpr {\lambda v+u}w = \lambda \scpr vw + \scpr uw$~;
			\item $\scpr vv \geq 0$ avec égalité $\iff v = 0$.
		\end{enumerate}
		\end{prp}

		\begin{proof} Le point 1 est trivial, le second se démontre par~:
		\[\scpr {\lambda v+u}w \equiv \sum_{i=1}^3(\lambda v_i + u_i)w_i = \lambda \sum_{i=1}^3v_iw_i + \sum_{i=1}^3u_iw_i = \lambda \scpr vw + \scpr uw,\]
		et le dernier point découle du fait que $\scpr vv$ est une somme de carrés ne pouvant donc être nulle que quand tous les termes sont nuls. \end{proof}

		\begin{déf} Soient deux vecteurs $v, w \in \R$. On dit que $v$ et $w$ sont orthogonaux si $\scpr vw = 0$. \end{déf}

		\begin{rmq} Un plan peut être défini uniquement par un vecteur normal et un point appartenant au plan (au lieu de deux vecteurs directeurs et un point). \end{rmq}

		\begin{prp} Soit un plan $\pi : ax + by + cz = d$. Alors le vecteur $a = (a, b, c)$ est normal à $\pi$. \end{prp}

		\begin{proof} Soit $\pi$ un plan défini par le point $v$ et le vecteur normal $a$. Alors pour tout vecteur $w = (x, y, z) \in \pi$, le vecteur $v-w$ est un vecteur
		directeur du plan. Dès lors, $\scpr {v-w}a = 0$, ou encore $\scpr va = \scpr wa$. Posons $d \coloneqq \scpr va$. On a donc $d = \scpr va = \scpr wa = ax+by+cz$.
		Dès lors, si un plan est de coordonnée cartésienne $\pi : ax + by + cz = d$, alors le vecteur $a$ est normal au plan $\pi$. \end{proof}

		\begin{déf} Soient deux droites $D_1, D_2$ de vecteur directeur respectif $v_1, v_2$.
		\begin{enumerate}
			\item Si $D_1 \cap D_2 = \emptyset$ et $\exists k \in \R \tq v_1 = kv_2$, alors $D_1$ et $D_2$ sont parallèles.
			\item Si $\abs{D_1 \cap D_2} = 1$ et $\scpr {v_1}{v_2} = 0$, alors $D_1$ et $D_2$ sont orthogonales.
		\end{enumerate}
		\end{déf}

		\begin{déf} Soient deux plans $\pi_1, \pi_2$ de vecteur orthogonal respectif $a_1, a_2$.
		\begin{enumerate}
			\item Si $\pi_1 \cap \pi_2 = \emptyset$, alors $\pi_1$ et $\pi_2$ sont parallèles (et donc $\exists k \in \R \tq a_1 = ka_2$).
			\item Si $\scpr {a_1}{a_2} = 0$ , alors $\pi_1$ et $\pi_2$ sont orthogonaux.
		\end{enumerate}
		\end{déf}

		\begin{déf} Soient $D$ une droite de vecteur directeur $v$ et $\pi$ un plan de vecteur normal $a$.
		\begin{enumerate}
			\item Si $\scpr av = 0$, alors le vecteur normal du plan et le vecteur directeur de la droite et le vecteur normal du plan sont orthogonaux, donc la droite
			      et le plan sont parallèles.
			\item Si $\exists k \in \R \tq a = kv$, alors le vecteur directeurs de la droite et le vecteur normal du plan sont colinéaires, donc la droite est orthogonale
			      au plan.
		\end{enumerate}
		\end{déf}

	\subsection{Distances dans $\R^3$}
		\begin{déf} Soit $v \in \R^3$ un vecteur. On définit sa norme (sa longueur) par~: \[\norm v \coloneqq \sqrt {\scpr vv}.\] \end{déf}

		\begin{rmq} Comme $\norm v$ représente la longueur de $v$, c'est également la distance entre $v$ et l'origine $\mathcal O$. Donc la distance entre deux points
		$p$ et $q$est donnée par $\norm {p-q}$. \end{rmq}

		\begin{prp}[Propriétés de la norme] Soit $v \in \R^3$ un vecteur. Alors~:
		\begin{enumerate}
			\item $\norm v \geq 0$ et $\norm v = 0 \iff v = 0$~;
			\item $\norm {\lambda v} = \lambda \norm v$.
		\end{enumerate}
		\end{prp}

		\begin{proof} La proposition 1 est directe par la définition de la norme et la proposition 2 se montre par~:
		\[\norm {\lambda v} = \sqrt {\scpr {\lambda v}{\lambda v}} = \sqrt {\lambda^2\scpr vv} = \lambda \sqrt {\scpr vv} = \lambda \norm v.\] \end{proof}

		\begin{thm}[Inégalité de Cauchy-Schwartz] Soient deux vecteurs $v, w \in \R^3$. Alors~:
		\[\abs{\scpr vw} \leq \norm v\norm w,\]
		avec égalité si et seulement si $v$ et $w$ sont colinéaires. \end{thm}

		\begin{proof} Si un des deux vecteurs vaut $0$, alors la preuve est évidente. S'ils sont tous les deux différents de $0$, alors pour tout $\lambda \in \R$, on a:
		\[0 \leq \norm{v-\lambda w}^2 = \scpr {v-\lambda w}{v-\lambda w} = \scpr vv -2\lambda \scpr vw + \lambda^2\scpr ww = \norm v^2 -2\lambda \scpr vw + \lambda^2\norm w^2.\]
		Cette équation de degré 2 en $\lambda$ a pour discriminant $(2\scpr vw)^2 - 4\norm w^2\norm v^2$. Et comme cette équation doit être positive, le discriminant
		doit être négatif (pas de racine) ou nul (une unique racine $\lambda_0$ telle que $\norm {v-\lambda_0w} = 0$). Il faut donc
		$4\scpr vw^2 - 4\norm v^2\norm w^2 \leq 0$, ou encore $\abs{\scpr vw} \leq \norm v\norm w$. \end{proof}

		\begin{prp} Soient $v, w \in \R^3$. Alors $\norm {v+w} \leq \norm v+\norm w$. \end{prp}

		\begin{proof} \[\norm {v+w}^2 = \scpr {v+w}{v+w} = \scpr vv + 2\scpr vw + \scpr ww = \norm v^2 + 2\scpr vw + \norm w^2 \leq \norm v^2 + 2\norm v\norm w + \norm w^2
		= (\norm v + \norm w)^2.\] \end{proof}

	\subsection{Angles dans $\R^3$}
		\begin{déf} Soient $v, w \in \R^3$. Alors On définit l'angle entre $v$ et $w$ par~:
		\[\alpha \coloneqq \arccos\left(\frac {\scpr vw}{\norm v\norm w}\right).\] \end{déf}

		\begin{rmq} Cette définition est bonne car deux vecteurs orthogonaux ont un produit scalaire nul, dès lors $\alpha = \arccos(0) = \frac \pi2$. De même, deux
		vecteurs colinéaires ont un produit scalaire valant $\norm v\norm w$ (ou $-\norm v\norm w$ selon l'orientation des vecteurs). Dès lors,
		$\alpha = \arccos(\pm 1) = 0$ ou $\pi$. \end{rmq}

		\begin{rmq} On remarque que cette définition implique que $\scpr vw = \norm v\norm w \cos \alpha$. \end{rmq}

		\begin{lem} Soient deux vecteurs $v, w \in \R^3$. Alors~:\[\beta \coloneqq \arccos\left(\frac {\abs {\scpr vw}}{\norm v\norm w}\right) \in \interval 0{\frac \pi2}\]
		\end{lem}

		\begin{proof} Par la définition de la fonction $\arccos$, pour tout $x \in \interval 0{\frac \pi2}$, on a $\arccos(x) \geq 0$ et c'est une bijection. \end{proof}

		\begin{prp}[Parallélisme et orthogonalité entre droites et plans]~
		\begin{enumerate}
			\item Soient deux droites $D_1, D_2$ de vecteurs directeurs $v_1, v_2 \in \R^3$. Si $\abs{D_1 \cap D_2} = 1$, alors l'angle entre ces deux droites est
			      donné par~: \[\alpha \coloneqq \arccos\left(\frac {\abs {\scpr {v_1}{v_2}}}{\norm {v_1}\norm {v_2}}\right) \in \left]0, \frac \pi2\right].\]
			\item Soient deux plans $\pi_1, \pi_2$ de vecteurs normaux $a_1, a_2 \in \R^3$. Si $\pi_1 \cap \pi_2$ est une droite, alors l'angle entre $\pi_1$ et $\pi_2$
			      est donné par~: \[\alpha \coloneqq \arccos\left(\frac {\abs {\scpr {a_1}{a_2}}}{\norm {a_1}\norm {a_2}}\right) \in \left]0, \frac \pi2\right].\]
			\item Soient une droite $D$ de vecteur directeur $v \in \R^3$ et un plan $\pi$ de vecteur normal $a \in \R^3$. Si $\abs{D \cap \pi} = 1$, alors l'angle entre
			      $\pi$ et $D$ est donné par~: \[\alpha \coloneqq \frac \pi2 - \arccos\left(\frac {\abs {\scpr av}}{\norm a\norm v}\right) \in \left]0, \frac \pi2\right].\]
		\end{enumerate}
		\end{prp}

	\subsection{Produit vectoriel}
		\begin{déf} Soient $v, w \in \R^3$ deux vecteurs. On définit le produit vectoriel entre $v$ et $w$ par~:
		\[v \times w \coloneqq (v_2w_3 - v_3w_2, v_3w_1 - v_1w_3, v_1w_2 - v_2w_1).\] \end{déf}

		\begin{prp}[Propriétés du produit vectoriel] Soient $u, v, w \in \R^3$ trois vecteurs et $\lambda \in \R$ un scalaire.
		\begin{enumerate}
			\item $(u+v) \times w = (u \times w) + (v \times w)$~;
			\item $(\lambda v) \times w = \lambda (v \times w)$~;
			\item $v \times w = - (w \times v)$~;
			\item $v \times (\lambda v) = 0$.
		\end{enumerate}
		\end{prp}

		\begin{proof}~
		\begin{enumerate}
			\item Pour la proposition 1, notons que~:
			      \begin{align*}(u+v) \times w &= ((u_2+v_2)w_3 - w_2(u_3+v_3), (u_3+v_3)w_1 - w_3(u_1+v_1), (u_1+v_1)w_2 - w_1(u_2+v_2)) \\
			      	&= (u_2w_3-w_2u_3 + v_2w_3-w_2v_3, u_3w_1-w_3u_1 + v_3w_1-w_3v_1, u_1w_2-w_1u_2 + v_1w_2-w_1v_2) \\
			      	&= (u_2w_3-w_2u_3, u_3w_1-w_3u_1, u_1w_2-w_1u_2) + (v_2w_3-w_2v_3, v_3w_1-w_3v_1, v_1w_2-w_1v_2) \\
			      	&= (u \times w) + (v \times w).
			      \end{align*}
			\item Pour la proposition 2, il suffit de mettre $\lambda$ en évidence~:
			      \[(\lambda v) \times w = (\lambda(v_2w_3-w_2v_3), \lambda(v_3w_1-w_3v_1), \lambda(v_1w_2-w_1v_2)) = \lambda(v \times w).\]
			\item Pour la proposition 3, il suffit de regarder les indices pour s'en convaincre.
			\item Pour la proposition 4, on a:
			      \[v \times (\lambda v) = (v_2\lambda v_3-v_3\lambda v_2, v_3\lambda v_1-v_1\lambda v_3, v_1\lambda v_2-v_2\lambda v_3) = (0, 0, 0) = 0.\]
		\end{enumerate}
		\end{proof}

		\begin{prp} Soient $v, w \in \R^3$ deux vecteurs. Alors leur produit vectoriel est orthogonal à $v$ et à $w$. \end{prp}

		\begin{proof} Trivial par la définition (expansion de la formule du produit scalaire et arriver à 0). \end{proof}

		\begin{lem}\label{prodsVecSc} Soient $v, w \in \R^3$ deux vecteurs. Alors~:\[\norm v^2\norm w^2 - (\scpr vw)^2 = \norm{v \times w}^2.\] \end{lem}

		\begin{proof} Trivial par expansion de la formule. \end{proof}

		\begin{cor}[du lemme précédent] Soient $v, w \in \R^3$. Alors $\norm {v \times w} = \norm v^2\norm w^2\sin(\alpha)^2$. \end{cor}

		\begin{proof} Par le lemme ~\ref{prodsVecSc}, on a:
		\[\norm{v \times w}^2 = (\norm v\norm w)^2 - (\scpr vw)^2 = (\norm v\norm w)^2\left(1 - \left(\frac {\scpr vw}{\norm v\norm w}\right)^2\right)
		= (\norm v\norm w)^2(1-\cos(\alpha)^2) = \norm v^2\norm w^2 \sin(\alpha)^2.\] \end{proof}

		\begin{prp} Soient $v, w \in \R^3$ deux vecteurs. La projection $p \in \R^3$ de $v$ sur $w$ est donnée par~:\[p \coloneqq \frac {\scpr vw}{\scpr ww}w.\] \end{prp}

		\begin{proof} On veut trouver $\lambda$ tel que $p = \lambda w$. Géométriquement, il faut que $v-p$ et $w$ soient orthogonaux~:
		\[0 = \scpr {v-p}w = \scpr {v-\lambda w}w = \scpr vw - \lambda \scpr ww.\] On a donc $\lambda = \frac {\scpr vw}{\scpr ww}$. \end{proof}

\newpage
\section{Systèmes d'équations linéaires et algèbre matricielle}
	\subsection{Définitions}
		\begin{déf} Soient $s, t \in \N_0$. Un système linéaires de $s$ équations à $t$ inconnues $x_j$ et à coefficients réels $a_{ij}, b_i \in \R$ est~:
		\[(S) : \left\{\begin{aligned}
			a_{11}x_1 + a_{12}x_2 + &\ldots + a_{1t}x_t = b_1 \\
			&\ldots \\
			a_{s1}x_1 + a_{s2}x_2 + &\ldots + a_{st}x_t = b_t
		\end{aligned}\right.\]

		Que l'on peut également noter~:
		\[(S) : \sum_{j=1}^ta_{ij}x_j = b_i\tag*{$i \in [s]$}.\]
		\end{déf}

		\begin{déf} Soit $(S)$ un système d'équations linéaire. Si $b_i = 0$, alors la $i$eme équation est dite homogène. Si toutes les équations sont homogènes ($b = 0$),
		alors le système $(S)$ est dit homogène. \end{déf}

		\begin{déf} L'ensemble $W$ des solutions de $(S)$ est l'ensemble des t-uples $(x_1, \dotsc, x_t) \in \R^t$ qui satisfont toutes les équations de $(S)$. \end{déf}

		\begin{déf} Deux systèmes linéaires sont dits équivalents s'ils ont le même ensemble de solutions \end{déf}

	\subsection{La méthode de Gauss}
		\begin{prp}\label{opérationsFondamentalesGauss}
		Soit $(S)$ un système linéaires d'équations. Il existe trois opérations fondamentales applicables sur $(S)$ qui donnent un système équivalent~:
		\begin{enumerate}
			\item échanger deux équations du système~;
			\item remplacer une équation par une combinaison linéaire d'équations du système~;
			\item multiplier une équation par une constante $\lambda \neq 0$.
		\end{enumerate}
		\end{prp}

		\begin{rmq} Les trois opérations fondamentales ci-dessus sont toutes inversibles et leur inverse restent fondamentales. \end{rmq}

		\begin{déf} la matrice associée à un système linéaire d'équations $(S)$ est la matrice de dimensions $s \times (t+1)$ contenant tous les coefficients du système~:
		\[\begin{bmatrix}a_{11} & \ldots & a_{1t} & -b_1 \\
		                 a_{21} & \ldots & a_{2t} & -b_2 \\
										 \ldots & \ldots & \ldots & \ldots \\
										 a_{s1} & \ldots & a_{st} & -b_t
			\end{bmatrix}.\]
		\end{déf}

		\begin{déf} Soit $\mathcal M$ la matrice associée à un système linéaire d'équations $(S)$. Si~:
		\begin{enumerate}
			\item dans chaque ligne de $\mathcal M$, si le premier coefficient non-nul (s'il existe) est 1 (on l'appelle le pivot de la ligne)~;
			\item dans la colonne de chaque pivot, tous les coefficients n'étant pas le pivot valent 0~;
			\item Le pivot d'une ligne est à gauche des pivots des lignes suivantes (s'ils existent),
		\end{enumerate}
		alors on dit que $\mathcal M$ est sous sa forme échelonnée réduite.
		\end{déf}

		\begin{alg}[Méthode de Gauss]\label{algoGauss} L'algorithme de Gauss permet de transformer une matrice en sa forme échelonnée réduite en utilisant les opérations
		fondamentales de la proposition~\ref{opérationsFondamentalesGauss}. Le principe de cet algorithme est pour chaque ligne $\ell$, s'il existe un coefficient non-nul
		dans la ligne, notons $r$ l'indice de ce coefficient. Ensuite, il faut diviser la ligne $\ell$ par ce même coefficient ($\mathcal M_{\ell r}$) puis, pour toutes
		ligne $i$ différentes de $\ell$, soustraire à la ligne $i$ la ligne $\ell$ multipliée par le coefficient $\mathcal M_{i r}$.
\begin{verbatim}
Soit M une matrice de dimensions l fois r.
Pour tout 0 < i < l+1
    Si M(i) != (0, 0, ..., 0)
        p = indice du premier coefficient non nul de M(i)
        M(i) /= M(i)(p)
        Pour tout 0 < j < l+1, j != i
            M(j) -= M(i)*M(j)(p)
\end{verbatim}
		\end{alg}

		\begin{déf} L'ensemble des matrices réelles de dimensions $s \times t$ est donné par $\M st\R$. On note $a \coloneqq [a_{ij}] \in M_{s \times t}(\R)$
		une matrice quelconque de dimension $s \times t$. $a_{ij}$ représente le coefficient de la $i$eme ligne et de la $j$eme colonne. \end{déf}

		\begin{déf} Soit $a \in \M nn\R$ une matrice. $a$ est une matrice triangulaire si pour tout $i > j : a_{i\,j} = 0$. \end{déf}

		\begin{thm}[Unicité de la forme échelonée réduite] Soient trois matrices $a, b, c \in \M st\R$. Si les matrices $b$ et $c$ sont toutes deux obtenues par une suite
		d'opérations fondamentales sur $a$ et si $b$ et $c$ sont toutes deux sous leur forme échelonnée réduite, alors $b = c$. \end{thm}

		\begin{proof} Les matrices $b$ et $c$ sont obtenues par combinaisons linéaires des lignes de la matrice $a$. Dès lors, les lignes de $b$ sont également des
		combinaisons linéaires des lignes de $c$. Dès lors, les indices $1 \leq i_1 < i_2 < \ldots < i_k \leq t$ des pivots de la matrice $b$ doivent être les mêmes que
		les indices des pivots de la matrice $c$. En effet~: soient $1 \leq j_1 < j_2 < \ldots <  j_k \leq t$ les indices des pivots de la matrice $c$.

		Prouvons-le par récurrence. 

		\underline{Cas de base~:} supposons $i_1 < j_1$. Alors, il faudrait que la $i_1$eme colonne de $c$ soit vide. Or $b$ est une combinaison linéaire de $c$ et la
		première ligne de $b$ contient à la $i_1$eme colonne le coefficient 1 ce qui est une contradiction avec le fait que $b$ et $c$ soient combinaisons linéaires
		l'un de l'autre. En supposant $i_1 > j_1$, on obtient la même contradiction. Il faut donc $i_1 = j_1$.

		\underline{Pas de récurrence~:} soit $\gamma \in \{1, \dotsc, t\}$. Supposons $i_\gamma < j_\gamma$. Par le même argument, on montre que $i_\gamma = j_\gamma$.
		\end{proof}

	\subsection{Calcul matriciel}
		\begin{rmq} Soit $x \in \R^t$ un vecteur. $x \in \M t1\R$ donc $x$ est une matrice colonne. \end{rmq}

		\begin{déf} Soient $v_1, \dotsc, v_k \in \R^t$ $k$ vecteurs et $\lambda_1, \dotsc, \lambda_k$ $k$ scalaires. Le vecteur \[w = \sum_{i=1}^k\lambda_iv_i\]
		est une combinaison linéaire des vecteurs $v_i$. \end{déf}

		\begin{déf} Soient $a \in \M st\R$ une matrice et $b \in \R^t$ un vecteur (une matrice colonne). On définit le produit entre $a$ et $b$ par~:
		\[ab \coloneqq \left[\sum_{k=1}^sa_{ik}b_k\right] \in \R^s \simeq \M s1\R.\] \end{déf}

		\begin{déf} À l'aide de cette notation, on peut maintenant noter le système d'équations linéaire $(S) : ax = b$ où $a \in \M st\R$, $x \in \R^t$ et $b \in \R^s$.
		\end{déf}

		\begin{déf} Soit $a = [a_{ij}] \in \M st\R$ une matrice. On définit la transformation linéaire qui lui est associée par~:
		\[A : \R^t \to \R^s : v \mapsto av.\] \end{déf}

		\begin{lem} Soit $(S) : ax + b$ un système linéaire d'équations et soit $A : \R^t \to \R^s$ la transformation linéaire associée à $a$. Alors l'ensemble $W$ des
		solutions du système correspond à la préimage du vecteur $b \in \R^s$. \end{lem}

		\begin{proof} On peut constater~: \[A^{-1}(b) = \{x \in \R^t \tq A(x) = b\} = \{x \in \R^t \tq ax = b\} = W.\] \end{proof}

		\begin{prp}[Propriétés de la transformation linéaire associée] Soient $a = [a_{ij}] \in \M st\R$ une matrice et $A : \R^t \to \R^s$ sa transformation
		linéaire associée (TLA). Soient $v = [v_i], w = [w_i] \in \R^t, \lambda \in \R$. On définit les propriétés de linéarité de $A$ par~:
		\begin{enumerate}
			\item $A(v+w) = A(v) + A(w)$~;
			\item $A(\lambda v) = \lambda A(v)$.
		\end{enumerate}
		\end{prp}

		\begin{proof} En repartant de la définition de la TLA, on voit que~:
		\begin{align*}
			A(v+w) &= a(v+w) = \left[\sum_{k=1}^ta_{ik}(v_k+w_k)\right] = \left[\sum_{k=1}^ta_{ik}v_k+\sum_{k=1}^ta_{ik}w_k\right] = A(v) + A(w), \\
			A(\lambda v) &= a(\lambda v) = \left[\sum_{k=1}^ta_{ik}\lambda v_k\right] = \lambda \left[\sum_{k=1}^ta_{ik}v_k\right] = \lambda A(v).
		\end{align*}
		\end{proof}

		\begin{rmq} On remarque que $A(0) = A(v + (-1)v) = A(v) - A(v) = 0 \in \R^s$. \end{rmq}

	\subsection{Structure des solutions d'un système linéaire d'équations}
		\begin{déf} Soit $(S) : ax = b$ un système linéaire d'équations. On définit le système homogène associé par $(S') : ax = 0$. \end{déf}

		\begin{déf} Soit $(S) : ax = b$ et $(S')$ son système homogène associé. Alors l'ensemble des solutions de $(S')$ est appelé $W_0$. \end{déf}

		\begin{prp}\label{combiliW0} Soient $(S)$ et $(S')$ un système linéaire et son système homogène associé. Soient $x, y \in W_0$ deux solutions de $(S')$. Alors
		toute combinaison linéaire de ces deux solutions est également une solution de $(S')$. \end{prp}

		\begin{proof} Effectivement, si $x$ et $y$ sont deux solutions de $(S')$, on a $A(\lambda x + \mu y) = \lambda A(x) + \mu A(y) = \lambda 0 + \mu 0 = 0$. \end{proof}

		\begin{lem} L'ensemble des solutions de $(S')$, le système homogène associé à $(S) : ax = b$ est~:
		\[W_0 = \left\{\sum_{i=1}^k\lambda_i v_i \tq \lambda_i \in \R\right\},\]
		où les vecteurs $v_i$ sont les vecteurs donnés par l'algorithme de Gauss. \end{lem}

		\begin{proof} On sait que les vecteurs $v_i$ sont les solutions de $(S')$ et donc sont dans $W_0$. Or, par la proposition~\ref{combiliW0}, on sait que toutes les
		combinaisons linéaires de ces vecteurs sont toujours des solutions. \end{proof}

		\begin{rmq} Géométriquement, selon le degré de liberté $k$ du système, l'ensemble des solutions est un espace de dimension $k$ passant par l'origine. En effet, si
		$k=0$, alors l'ensemble des solutions $W_0$ est l'ensemble contenant uniquement l'origine, si $k=1$, l'ensemble est la droite contenant tous les multiples
		$\lambda v$ de $v$, solution donnée par Gauss, si $k=3$, $W_0$ est un plan ayant pour vecteurs directeurs $v_1, v_2$ donnés par Gauss et passant par l'origine.
		Etc. \end{rmq}

		\begin{lem}\label{W-W=W_0} Soit $(S)$ un système linéaire d'équations et $(S')$ son système homogène associé. Si $x, y \in \R^t$ sont des solutions de $(S)$,
		alors $x-y$ est une solution de $(S')$. \end{lem}

		\begin{proof} Soit $a$ la matrice associée à $(S)$ et $A$ la TLA à $a$. Si $x, y \in W$, alors $A(x) = A(y) = b$. Dès lors, $A(x-y) = A(x) - A(y) = b-b = 0$.
		\end{proof}

		\begin{cor} Soit $\widetilde x \in \R^t$ une solution du système linéaire $(S)$. Alors toute solution $y \in W$ de $(S)$ est sous la forme suivante
		$\widetilde x + y$ où $y \in W_0$. Ou encore, l'ensemble des solutions de $(S)$ est l'ensemble~:
		\[W = \widetilde x + W_0 \coloneqq \{\widetilde x + y \tq y \in W_0\}.\] \end{cor}

		\begin{proof} Montrons d'abord que $\widetilde x + W_0 \subseteq W$ et ensuite montrons que $W \subseteq \widetilde x + W_0$. Soit
		$\widetilde x + y \in \widetilde x + W_0$. On sait que $A(\widetilde x + y) = A(\widetilde x) + A(y) = b + 0 = b$. Dès lors, $\widetilde x + y$ est une
		solution de $(S)$ et donc $\widetilde x + y \in W$.

		Soit $x \in W$. On sait que $A(x) = A(\widetilde x) = b$. Par le lemme~\ref{W-W=W_0}, on sait que le vecteur $x-\widetilde \in \R^t$ est une solution du système
		homogène associé, donc $x-\widetilde \in W_0$. Dès lors, on peut réécrire $x = \widetilde x + (x-\widetilde x)$. On peut donc dire $x \in \widetilde x + W_0$, ou
		encore $W \subseteq \widetilde x + W_0$. Donc $W = \widetilde x + W_0$. \end{proof}

		\begin{rmq} L'ensemble des solutions de $(S)$ est donc l'ensemble \[W \coloneqq \left\{\sum_{i=1}^k\lambda_iv_i \tq \lambda_i \in \R\right\} + \widetilde x\] où
		les vecteurs $\widetilde x, v_i$ sont donnés par l'algorithme de Gauss (le vecteur $\widetilde x$ est le vecteur correspondant à la $(t+1)$eme colonne de
		la matrice sous sa forme échelonnée réduite). Géométriquement, cela correspond à une translation de l'espace des solutions de $(S')$ par le vecteur
		$\widetilde x$. C'est donc un point, une droite, un plan, etc. comme pour $(S')$ mais ne devant pas contenir l'origine. \end{rmq}

\newpage
\section{espaces vectoriels}
	\subsection{Définitions}
		\begin{déf} Soit $\mathbb K$ un corps. Un espace vectoriel $V$ sur $\mathbb K$\footnote{Il sera question ici presque exclusivement d'espaces vectoriels sur $\R$.}
		est un sous-ensemble de vecteurs muni d'une addition interne et d'un produit externe définis par~:
		\begin{enumerate}
			\item $+ : V \times V \to V : (v, w) \mapsto v+w$~;
			\item $\cdot : \mathbb K \times V \to V : (\lambda, v) \mapsto \lambda v$.
		\end{enumerate}
		telles que pour tout $u, v, w \in V, \lambda, \mu \in \mathbb K$~:
		\begin{enumerate}
			\item $u + (v + w) = (u + v) + w$~;
			\item $\exists 0 \in V \tq v + 0 = v$~;
			\item $\exists (-v) \in V \tq v + (-v) = 0$~;
			\item $v + w = w + v$~;
			\item $\lambda(\mu v) = (\lambda\mu)v$~;
			\item $\lambda (v+w) = \lambda v + \lambda w$~;
			\item $(\lambda + \mu)v = \lambda v + \mu v$~;
			\item $1v = v$.
		\end{enumerate}
		\end{déf}

		\begin{déf} Soit $W \subseteq V$ un sous-ensemble de $V$. $W$ est un sous-espace vectoriel de $V$ si $W$ est également un espace vectoriel muni des mêmes
		opérations de somme et de produit. \end{déf}

		\begin{prp}\label{conditionsSousEspace} Soit $V$ un espace vectoriel réel. $W \subseteq V$ est un sous-espace vectoriel de $V$ si et seulement si pour tout
		$v, w \in W, \lambda \in \R$ on a~:
		\begin{enumerate}
			\item $0 \in W$~;
			\item $v+w \in W$~;
			\item $\lambda v \in W$.
		\end{enumerate}
		\end{prp}

		\begin{proof} Montrons d'abord que ces trois hypothèses font de $W$ un sous-espace vectoriel. Montrons ensuite que si $W$ est un sous-espace, alors ces trois
		propositions sont vérifiées.

		Soit $W \subseteq V$ tel que $\forall v, w \in W, \lambda \in \R : (v+w \in W) \land (\lambda v \in W) \land (0 \in W)$. Les propriétés 1, 4, 5, 6, 7, 8
		viennent de la définition des opérations et des hypothèses 2 et 3. La propriété 2 vient de l'hypothèse 0 et la propriété 3 vient de l'hypothèse 3 pour
		$\lambda = -1$.

		Soit $W \subseteq V$ un sous-espace de $V$. Les hypothèses sont vérifiées de manière triviale. \end{proof}

		\begin{déf} Soient $V, W$ deux espaces vectoriels. La fonction $f : V \to W$ est une application linéaire si pour tout $v, w \in V, \lambda, \mu \in \R$, on a:
		$f(\lambda x + \mu y) = \lambda f(x) + \mu f(y)$. \end{déf}

		\begin{prp} Soit $V$ un espace vectoriel réel. Soient $v, v_1, \dotsc, v_k \in V, \lambda, \lambda_1, \lambda_k \in \R$. Alors~:
		\begin{enumerate}
			\item $\lambda 0 = 0$~;
			\item $0v = 0$~;
			\item $\lambda v = 0 \Rightarrow (\lambda = 0) \lor (v = 0)$~;
			\item $(-\lambda)v = \lambda(-v) = -\lambda v$~;
			\item $\displaystyle v\sum_{i=1}^k\lambda_i = \sum_{i=1}^k(\lambda_i v)$~;
			\item $\lambda \displaystyle \sum_{i=1}^kv_i = \sum_{i=1}^k(\lambda v_i)$
		\end{enumerate}
		\end{prp}

		\begin{proof}~
			\begin{enumerate}
				\item $0 + \lambda0 = \lambda0 = \lambda(0+0) = \lambda0 + \lambda0$. Dès lors, par l'existence de l'inverse additif, on a $0 = \lambda 0$~;
				\item $0 + 1v = 1v = v = v(1+0) = 1v + 0v$. Dès lors, par l'existence d'un inverse additif, on a $0 = 0v$~;
				\item Supposons $\lambda v = 0$ et $\lambda \neq 0$. Alors $0 = \lambda^{-1}0 = \lambda^{-1}(\lambda v) = 1v = v$~;
				\item $-\lambda v = (-1 \cdot \lambda)v = (-\lambda)v = (\lambda \cdot -1)v = \lambda(-v)$~;
				\item par les propriétés 6 et 7 d'un espace vectoriels, les propriétés 5 et 6 de la proposition sont vérifiées.
			\end{enumerate}
		\end{proof}

		\begin{déf} Soient $V, W$ deux espaces vectoriels. Soit $A : V \to W$ une application linéaire. On définit~:
		\begin{enumerate}
			\item $\Ker(A) \coloneqq \{v \in V \tq A(v) = 0 \in W\} \subseteq V = A^{-1}(0)$~;
			\item $\Imf(A) \coloneqq \{A(v) \tq v \in V\} \subseteq W = A(V)$.
		\end{enumerate}
		\end{déf}

	\subsection{Sous-espaces}
		\begin{thm}\label{sousEspacesApplicationLinéaire} Soient $V, W$ deux espaces vectoriels et $A : V \to W$ une application linéaire. Alors~:
		\begin{enumerate}
			\item si $Y \subseteq W$ est un sous-espace vectoriel de $W$, alors $A^{-1}(Y)$ est un sous-espace vectoriel de $V$~;
			\item si $Z \subseteq V$ est un sous-espace vectoriel de $V$, alors $A(Z)$ est un sous-espace vectoriel de $W$.
		\end{enumerate}
		\end{thm}

		\begin{proof} Pour montrer qu'un sous-ensemble est un sous-espace vectoriel, on peut utiliser la proposition~\ref{conditionsSousEspace}. Dès lors, montrons que
		$A(Z)$ est un sous-espace de $W$ . Soient $y, z \in Z$. Notons $v = A(y), w = A(z)$. Alors, comme $Z$ est un sous-espace, on sait que $y + \lambda z \in Z$. Donc
		$A(Z) \ni A(y + \lambda z) = A(y) + \lambda A(z) = v + \lambda w$.

		Montrons maintenant que $A^{-1}(Y)$ est un sous-espace de $V$. Soient $x, y \in Y$. Comme $Y$ est un sous-espace, $x + \lambda y \in Y$. Prenons $v, w$ tels que
		$A(v) = x$ et $A(w) = y$. Donc $A^{-1}(Y) \ni A^{-1}(x + \lambda y) = A^{-1}(A(v) + \lambda A(w)) = A^{-1}(A(v + \lambda w)) = v+\lambda w$. \end{proof}

		\begin{rmq} La preuve ci-dessus utilise le fait que $0_V$ est envoyé sur $0_W$ par une application linéaire, ce qui garantit la présence de $0_W$ dans $A(Z)$
		et la présence de $0_V$ dans $A^{-1}(Y)$. \end{rmq}

		\begin{cor} Soient $V, W$ deux espaces vectoriels et $A : V \to W$ une application linéaire. Alors $\Ker(A)$ est un sous-espace de $W$ et $\Imf(A)$ est un
		sous-espace de $V$. \end{cor}

		\begin{proof} $V$ est un sous-espace de lui-même et $\{0\}$ est un sous-espace de $W$. Donc $\Ker(A) = A^{-1}(\{0\})$ est un sous-espace de $V$ et
		$\Imf(A) = A(V)$ est un sous-espace de $W$ par le théorème~\ref{sousEspacesApplicationLinéaire}.\end{proof}

		\begin{déf} Soient $V$ un e.v. et $W_1, W_2$ deux sous-espaces de $V$. On définit la somme de ces sous-espaces par~:
		\[W = W_1 + W_2 \coloneqq \{w_1 + w_2 \tq w_1 \in W_1, w_2 \in W_2\} \subseteq V.\] \end{déf}

		\begin{déf} Soit $V$ un e.v. et $W_1, W_2$ deux sous-espaces. La somme $W_1 + W_2$ est dite directe si $W_1 \cap W_2 = \emptyset$. \end{déf}

		\begin{prp} Soit $V$ un e.v. réel et $W_1, W_2$ deux sous-espaces. La somme $W_1 + W_2$ est un sous-espace de $V$. \end{prp}

		\begin{proof} Les deux sous-espaces contiennent $0_V$ donc $0_V \in W_1+W_2$. De plus, soient $x_1, x_2 \in W_1, \lambda \in \R, w_1 w_2 \in W_2$. On sait que
		$v_1 = x_1 + w_1, v_2 = x_2 + w_2 \in W_1+W_2$. De plus, $v_1 + \lambda v_2 = (x_1 + \lambda x_2) \in W_1 + (w_1 + \lambda w_2) \in W_2$. On a donc bien
		$v_1 + \lambda v_2 \in W_1 + W_2$. \end{proof}

		\begin{déf} Soit $V$ un e.v. Soit $\{W_i\}_{i \in I}$ une famille de sous-espaces. On définit l'intersection de ces sous-espaces par~:
		\[\bigcap_{i \in I}W_i \coloneqq \left\{v \in V \tq \forall i \in I : v \in W_i\right\}.\]\end{déf}

		\begin{prp} Soit $V$ un e.v. et $\{W_i\}_{i \in I}$ une famille de sous-espaces. Alors l'espace intersection $\cap_{i \in I}W_i$ est un sous-espace de $V$.
		\end{prp}

		\begin{proof} Soient $V$ un e.v. et $\{W_i\}_{i \in I}$ une famille de sous-espaces. Soient $w_1, w_2 \in \cap_{i \in I}W_i$ deux vecteurs dans l'intersection.
		On sait dès lors que $w_1, w_2 \in W_i$ pour tout $i \in I$. S'ils le sont, alors $w_1 + \lambda w_2 \in W_i$ pour tout $i \in I$ également. Dès lors,
		$w_1 + \lambda w_2 \in \cap_{i \in I}W_i$. \end{proof}

	\subsection{Parties libres et génératrices}
		\begin{déf} Soient $V$ un espace vectoriel et $X \subseteq V$ un sous-ensemble de $V$. On définit le sous-espace engendré par $X$ par~:
		\[\eng X \coloneqq \left\{\sum_{i=1}^k\lambda_iv_i \tq k \in \N, \lambda_i \in \R, v_i \in X\right\}.\] Si $X = \emptyset$, on pose $\eng X \coloneqq \{0\}$.
		\end{déf}

		\begin{prp} Soient $V$ un e.v. et $X \subseteq V$ un sous-ensemble de $V$. Alors $\eng X$ est un sous-espace de $V$. \end{prp}

		\begin{proof} Si $X$ est l'ensemble vide, alors $\eng X = \{0\}$ et est un sous-espace de $V$. Si $\abs X = n > 0$, alors prenons deux vecteurs $v_1, v_2$ dans
		$\eng X$. On sait que $v_1$ et $v_2$ sont des combinaisons linéaires des vecteurs $x_i$ de $X$. On sait~:
		\begin{align*}
			v_1 &= \sum_{i=1}^n\lambda_ix_i, \\
			v_2 &= \sum_{i=1}^n\mu_ix_i.
		\end{align*}
		Dès lors, on trouve que~: \[v_1 + \beta v_2 = \sum_{i=1}^n\lambda_ix_i + \beta\sum_{i=1}^n\mu_ix_i = \sum_{i=1}^n(\lambda_i + \beta\mu_i)x_i \in \eng X.\]
		Effectivement, $v_1 + \beta v_2$ est une combinaison linéaire des vecteurs de $X$, il fait donc partie de $\eng X$. \end{proof}

		\begin{lem}\label{X<<X>} Soit $X \subseteq V$ un sous-ensemble d'un espace vectoriel. Alors $X \subseteq \eng X$ \end{lem}

		\begin{proof} En prenant les $n$ vecteurs $x_i$ de $X$, on sait que pour tout $x_i \in X$, on a~:\[x_i = \sum_{j=1}^n\delta_{ij}x_j \in \eng X.\] \end{proof}

		\begin{prp} Soit un espace vectoriel $V$, un sous-ensemble $X \subseteq V$ et $W$ l'intersection des sous-espaces de $V$ contenant $X$. Alors~: \[W = \eng X.\]
		\end{prp}

		\begin{proof} Si $X = \emptyset$, alors tous les sous-espaces contiennent $X$. Donc l'intersection de tous ces sous-espaces est $\{0\}$. Si $X$ est non-vide,
		montrons que $\eng X \subseteq W$ et puis montrons que $W \subseteq \eng X$. Soit $v = \sum_{i=1}^k\lambda_iv_i \in \eng X$. Puisque les $v_i \in X$, $v$ est
		dans tout sous-espace contenant $X$ et est donc dans l'intersection de ceux-ci. On a donc $\eng X \subseteq W$.

		Par le lemme~\ref{X<<X>}, on sait que $\eng X$ est un sous-espace de $V$ contenant $X$. Dès lors, $W \subseteq \eng X$. On a alors montré l'égalité. \end{proof}

		\begin{déf} Soient $V$ un e.v. et $X \subseteq V$. On appelle $X$ une partie génératrice si $\eng X = V$. \end{déf}

		\begin{prp}\label{engXDansEngY} Soient $V$ un e.v. et $X \subseteq Y \subseteq V$ deux sous-ensembles de $V$. Alors $\eng X \subseteq \eng Y$. \end{prp}

		\begin{proof} Supposons $\abs X = n$ et $\abs Y = k$ avec $k \geq n$. Soit $v \in \eng X$. On sait que \[v = \sum_{i=1}^n\alpha_ix_i,\] où $x_i \in X$.
		Soient $y_i$ pour $i \in \{1, \dotsc, k\}$ les vecteurs de $Y$. On sait par hypothèse que pour $i \in \{1, \dotsc, n\}$, on a $x_i = y_i$. Donc un vecteur de
		$\eng Y$ s'écrit sous la forme \[w = \sum_{i=1}^k\lambda_iy_i = \sum_{i=1}^n\lambda_ix_i + \sum_{i=1}^{k-n}\mu_iy_{i+n}.\] En prenant $\mu_i = 0$ pour tout
		$i \in \{1, \dotsc, k-n\}$, on sait écrire \[v = \sum_{i=1}^n\alpha_ix_i + \sum_{i=1}^{k-n}\mu_iy_{i+n} \in \eng Y.\] On a donc bien $\eng X \subseteq \eng Y$.
		\end{proof}

		\begin{prp} Soit $V$ un e.v. réel et $X \subseteq V$. Alors $\eng {\eng X} = \eng X$. \end{prp}

		\begin{proof} On sait que
		\begin{align*}
			\eng {\eng X} = \left\{\sum_{i=1}^k\lambda_iv_i \tq k \in \N, \lambda_i \in \R, v_i \in \eng X\right\}
			&= \left\{\sum_{i=1}^k\left(\lambda_i\sum_{j=1}^\ell\mu_jw_j\right) \tq k, \ell \in \N, w_j \in X, \lambda_i, \mu_j \in \R\right\} \\
			&= \left\{\sum_{j=1}^\ell\left(\mu_j\sum_{i=1}^k\lambda_i\right)w_j \tq k, \ell \in \N, w_j \in X, \lambda_i, \mu_j \in \R\right\}.
		\end{align*}
		En posant~: \[\alpha_j \coloneqq \mu_j\sum_{i=1}^k\lambda_i,\] on peut réécrire~:
		\[\eng {\eng X} = \left\{\sum_{j=1}^\ell\alpha_jw_j \tq \ell \in \N, w_j \in X, \alpha_j \in \R\right\} = \eng X.\] \end{proof}

		\begin{lem} Soit $V$ un e.v. et $X \subseteq Y \subseteq V$. Si $X$ est une partie génératrice de $V$, alors $Y$ est également une partie génératrice de $V$.
		\end{lem}

		\begin{proof} Par la proposition~\ref{engXDansEngY}, on sait que $\eng X \subseteq \eng Y$. Or $\eng X = V$. Donc $V \subseteq \eng Y$. Donc $Y$ est une partie
		génératrice de $V$. \end{proof}

		\begin{prp} Soit $V$ un e.v. et $X$ une partie génératrice de $V$. S'il existe $x \in X \setminus \{x\}$ tel que $x \in \eng {X \setminus \{x\}}$, alors
		$X \setminus \{x\}$ est une partie génératrice de $V$. \end{prp}

		\begin{proof} Supposons qu'il existe un tel $x$. On peut donc réécrire \[x = \sum_{i=1}^k\lambda_iv_i\] où $v_i \in X \setminus \{x\}$. Un vecteur quelconque
		$w \in V$ peut s'écrire comme une combinaison linéaire de vecteurs de $X$ (car $X$ est génératrice). On a donc~: \[v = \sum_{i=1}^k\mu_iv_i + \mu x\] où
		$v_i \in X \setminus \{x\}$. Or, comme $x$ est également une combinaison linéaire des vecteurs $v_i$, on peut réécrire~:
		\[v = \sum_{i=1}^k\mu_iv_i + \sum_{i=1}^k\lambda_iv_i = \sum_{i=1}^k(\lambda_i+\mu_i)v_i.\]
		Et comme $v_i \in X \setminus \{x\}$, l'ensemble $X \setminus \{x\}$ est une partie génératrice de $V$. \end{proof}

		\begin{déf} Soit $V$ un e.v. et $X \subseteq V$ un sous-ensemble de $V$. $X$ est appelé partie libre si pour tout $v \in \eng X$, on a
		$v = 0 \Rightarrow \lambda_i = 0 \forall i$. \end{déf}

		\begin{prp}\label{conditionsPartieLibre} Soit $X$ un sous-ensemble de l'e.v. $V$. $X$ est une partie libre de $V$ si et seulement si pour tout
		$x \in X : x \not \in \eng {X \setminus \{x\}}$. \end{prp}

		\begin{proof} Montrons que si $X$ est une partie libre, alors $\forall x \in X : x \not \in \eng {X \setminus \{x\}}$. Soit $X$ une partie libre de $V$.
		Supposons par l'absurde qu'il existe $x \in X$ tel que $x \not \in \eng {X \setminus \{x\}}$. Dès lors, il existe $v_1, \dotsc, v_k \in X \setminus \{x\}$ et
		$\lambda_1, \dotsc, \lambda_k \in \R$ tels que $x = \lambda_1v_1 + \ldots + \lambda_kv_k$. La combinaison linéaire $x - \lambda_1v_1 - \ldots - \lambda_kv_k = 0$
		est donc une combinaison nulle non-triviale, ce qui contredit le fait que $X$ est libre.

		Montrons maintenant que si $\forall x \in X : x \not \in \eng {X \setminus \{x\}}$, alors $X$ est libre. Supposons par l'absurde que $X$ n'est pas libre. Alors
		il existe une combinaison linéaire $\lambda_1v_1 + \ldots + \lambda_kv_k = 0$ avec au moins un coefficient non-nul. Appelons-le $\lambda_i$. On a donc~:
		\[v_i = -\lambda_i^{-1}\left(\sum_{j=1, j \neq i}^k\lambda_jv_j\right) \in X \setminus \{v_i\},\] ce qui contredit l'hypothèse. \end{proof}

		\begin{prp} Soit $V$ un e.v. et $X \subseteq Y \subseteq V$. Si $Y$ est une partie libre, alors $X$ est également une partie libre. \end{prp}

		\begin{proof} Soit une combinaison linéaire des vecteurs de $Y$~: \[y = \sum_{i=1}^k\lambda_iv_i\] où $v_i \in Y$. Comme $Y$ est libre, la seule combinaison
		linéaire nulle possible est $\lambda_1 = \lambda_2 = \ldots = 0$. Soit une combinaison linéaire des vecteurs de $X$~: \[x = \sum_{i=1}^n\mu_iv_i\] où
		$n \leq k$ et $v_i \in X$. Si $x = 0$, on a déjà $\mu_t = 0$ pour $t \in \{n+1, \dotsc, k\}$. Par hypothèse, il faut donc que les autres $\lambda_i$ valent $0$.
		\end{proof}

		\begin{prp}\label{ajoutPartieLibre} Soit $V$ un e.v. et $X$ une partie libre de $V$. S'il existe $v \in V \setminus \eng X$, alors $X \cup \{v\}$ est
		également une partie libre. \end{prp}

		\begin{proof} Supposons que $X \cup \{v\}$ ne soit pas libre. Alors il existe $v_1, \dotsc, v_k \in X, \lambda, \lambda_1, \dotsc, \lambda_k \in \R$ tels que~:
		\[\lambda v + \sum_{i=1}^k\lambda_iv_i = 0.\] Si $\lambda = 0$, il y a contradiction car on a supposé $X \cup \{v\}$ non libre. Si $\lambda \neq 0$, alors $v$
		est une combinaison linéaire des vecteurs $v_i$ de $X$. Or par hypothèse, $v \in V \setminus \eng X$. Il y a donc à nouveau contradiction. Un tel $\lambda$
		n'existe pas, et donc $X \cup \{v\}$ est libre. \end{proof}

	\subsection{Bases et dimension}
		\begin{thm}\label{equivBases} Soient $V$ un e.v. et $X \subseteq V$. Alors les trois propositions suivantes sont équivalentes~:
		\begin{enumerate}
			\item $X$ est une partie génératrice minimale (il n'existe pas de $x \in X$ tel que $x \in \eng {X \setminus \{x\}}$)~;
			\item $X$ est une partie libre maximale (il n'existe pas de $v \in V$ tel que $X \cup \{v\}$ soit libre)~;
			\item $X$ est une partie libre et génératrice.
		\end{enumerate}
		\end{thm}

		\begin{proof} Montrons que $(1)$ implique $(3)$~: par la proposition~\ref{conditionsPartieLibre}, s'il n'existe pas de $x \in X$ tel que
		$x \in \eng {X \setminus \{x\}}$ , alors $X$ est libre.

		Montrons que $(3)$ implique $(2)$~: $X$ est libre et génératrice. Dès lors, si $v \in V$, alors $v \in \eng X$. Donc $X \cup \{v\}$ ne peut être libre.

		Montrons que $(2)$ implique $(1)$~: $X$ est libre maximale. Donc il n'existe pas de $v \in V$ tel que $v \not \in \eng X$. Donc $\eng X = V$ ou encore $X$ est
		génératrice. Mais $X$ est génératrice minimale car s'il existait $x \in X$ tel que $x \in \eng {X \setminus \{x\}}$, alors cela voudrait dire que $X$ n'est
		pas libre maximale. \end{proof}

		\begin{déf} Soit un espace vectoriel $V$ et $X \subseteq V$. Si $X$ est une une partie libre et génératrice, on appelle $X$ une base de $V$. \end{déf}

		\begin{lem}[lemme de Zorn] Soit $(S, \leq)$ un ensemble partiellement ordonné. Si tout sous-ensemble totalement ordonné $X \subseteq S$ a une borne supérieure
		dans $S$ (un $M \in S$ tel que $\forall x \in X : x \leq M$), alors il existe un élément maximal $m \in S$. \end{lem}

		\begin{thm}\label{existenceBaseEspaceGénéré} Soient $V$ un espace vectoriel et $X \subseteq Y \subseteq V$ tels que $X$ est libre. Alors il existe une base
		$B$ de $\eng Y$ telle que $X \subseteq B \subseteq \eng Y$. \end{thm}

		\begin{proof} Soit $\mathcal S \coloneqq \{Z \subseteq V \tq X \subseteq Z \subseteq Y, Z \text{ partie libre de } V\}$. Associons une relation d'ordre partiel
		à $\mathcal S$ par l'inclusion. On sait $\mathcal S$ non vide car $X \in \mathcal S$.

		Soient $\{Z_i\}_{i \in I}$ une famille totalement ordonnée de $\mathcal S$ et $Z \coloneqq \cup_{i \in I}Z_i$. On sait que $Z \in \mathcal S$ car
		$X \subseteq Z \subseteq Y$ (en effet~: comme $Z$ est défini par l'union, $X \subseteq Z$ et de plus, il n'existe aucun $z \in Z$ tel qu'il n'existe pas de
		$i$ pour $z \in Z_i$ dès lors $Z \subseteq Y$) et $Z$ est libre car défini par l'inclusion de parties libres \textbf{incluses}. De plus $Z$ est une borne
		supérieure de $\{Z_i\}_{i \in I}$. Dès lors, par le lemme de Zorn, on sait qu'il existe un élément maximal $B \in \mathcal S$.

		On remarque que $B$ est une base de $\eng Y$ car sinon il existerait $v \in V \setminus \eng Y$ tel que $B \cup \{v\}$ serait dans $\mathcal S$, or $B$
		est l'élément maximal. Donc $\eng B = \eng Y$, ou encore $B$ est une base de $\eng Y$ telle que $X \subseteq B \subseteq Y$. \end{proof}

		\begin{cor} Tout espace vectoriel $V$ admet une base. \end{cor}

		\begin{proof} En posant $X = \emptyset$ et $Y = V$, par le théorème~\ref{existenceBaseEspaceGénéré}, il existe une base de $\eng V = V$. \end{proof}

		\begin{lem}\label{identitéSousEspaceInclus} Soi $V$ un e.v. et $W$ un sous-espace de $V$. Soit $v \in V \setminus W$. Si $Z \subseteq V$ est un sous-espace de
		$V$ tel que~: \[W \subseteq Z \subseteq \eng {V \cup \{v\}},\] alors, $W = Z$ ou $Z = \eng {W \cup \{v\}}$. \end{lem}

		\begin{proof} Soit $B$ une base de $W$. Comme $v \not \in W$, alors $B \cup \{v\}$ est une base de $\eng {W \cup \{v\}}$ car $B \cup \{v\}$ est une partie libre
		de $\eng {W \cup \{v\}}$ et est génératrice. 

		Si $Z = W$, alors la démonstration est faite. Supposons donc $Z \neq W$. Prenons $z \in Z \setminus W$. Puisque $z$ n'est pas dans $W$, on sait que
		$B \cup \{z\}$ est une partie libre de $\eng {W \cup \{v\}}$. Alors il faut $v \in \eng {B \cup \{z\}}$ sinon $B \cup \{z, v\}$ serait une partie libre de
		$\eng {W \cup \{v\}}$ ce qui contredirait le fait que $B \cup \{v\}$ en soit une base. On a donc~:
		\[\eng {W \cup \{v\}} = \eng {B \cup \{v\}} \subseteq \eng {B \cup \{z\}} \subseteq Z.\]
		Or par hypothèse, on avait $Z \subseteq \eng {W \cup \{v\}}$. Donc on sait $Z = \eng {W \cup \{v\}}$. \end{proof}

		\subsubsection{Cardinalité}
		\begin{déf} Soient deux ensembles $X$ et $Y$. On dit qu'ils sont de même cardinalité s'il existe une bijection $f : X \to Y$. Et on note $\abs X = \abs Y$.
		S'il existe une fonction injective $f : X \to Y$, on écrit $\abs X \leq \abs Y$. \end{déf}

		\begin{thm}[Théorème de Cantor-Bernstein]\label{Cantor-Bernstein} Soient deux ensembles $X$ et $Y$. Si $\abs X \leq \abs Y$ et $\abs Y \leq \abs X$,
		alors $\abs X = \abs Y$. \end{thm}

		\begin{proof} Soient $f : X \to Y, g : Y \to X$ deux fonctions injectives. Si $g$ est bijective, la preuve est finie. Posons $A_0 \coloneqq X \setminus g(Y)$.
		On sait $A_0$ non-vide car $g$ n'est pas bijective. On pose ensuite~: \[\forall i \in \N \setminus \{0\} : A_i \coloneqq (g \circ f)(A_{i-1}).\]
		On définit $A \coloneqq \cup_{i\in\N}A_i$ et $A' \coloneqq X \setminus A$. L'ensemble $A$ est non-vide.

		On sait~: \[(g \circ f)(A) = \cup_{i \in I}(g \circ f)(A_i) = \cup_{i\in\N}A_{i+1} = X \setminus X_0\subset A.\]
		Définissons ensuite~: \[\varphi : X \to Y : x \mapsto \left\{\begin{aligned}&f(x) &\text{ si } x \in A, \\&g^{-1}(x) &\text{ si $x \in A'$}\end{aligned}\right..\]
		On sait que $g^{-1}(x)$ existe car $x \in A'$ et $A' \subseteq g(Y)$.

		Montrons que $\varphi$ est bijective~:
		\begin{enumerate}
			\item montrons que $\varphi(A) \cap \varphi(A') = \emptyset$. Supposons qu'il existe $x \in \varphi(A) \cap \varphi(A')$. Puisque $x \in \varphi(A)$,
			      on sait que $x \in f(A) \subseteq Y$, donc il existe $y \in A$ tel que $f(y) = x$ donc $g(x) = g(f(y)) = (g \circ f)(y)$. On on a montré que l'image
				  de $(g \circ f)$ est incluse dans $A$. On a donc $g(x) \in A$.

			      Mais puisque $x \in \varphi(A')$, on sait qu'il existe $y' \in A'$ tel que $g(x) = y'$. Donc $g(x) \in A'$. Ce qui est une contradiction avec le fait
				  que $g(x) \in A$ car $A$ et $A'$ sont disjoints.

			\item Montrons que $\varphi$ est injective. Soient $x, x' \in X$ tels que $\varphi(x) = \varphi(x')$. Les combinaisons possibles sont~:
			\begin{enumerate}
				\item $x \in A$ et $x \in A'$ ou inversement ce qui est impossible car $\varphi(x) \in \varphi(A)$ et $\varphi(x') \in \varphi(A')$ qui sont disjoints~;
				\item $x, x' \in A$. Dès lors, $f(x) = \varphi(x) = \varphi(x') = f(x')$, et par injectivité de $f$, il faut $x=x'$~;
				\item $x, x' \in A'$. On sait que $g$ est une bijection entre $Y$ et $B \coloneqq g(Y)$. Donc $g^{-1} : g(Y) \to Y$ est également bijective. Dès lors,
				si $g^{-1}(x) = g^{-1}(x')$, alors$x = x'$.
			\end{enumerate}
			On a montré que dans tous les cas, si $\varphi(x) = \varphi(x')$, il faut $x=x'$. $\varphi$ est donc injective.

			\item Montrons que $\varphi$ est surjective. Soit $y \in Y$. On sait que $g(y) \in A \iff g(y) \not \in g(A')$.
			\begin{enumerate}
				\item Si $g(y) \in A$, alors on sait que $g(y) \not \in A_0$ car $A_0 \equiv X \setminus g(Y)$. Donc $g(y) \in X \setminus A_0 = (g \circ f)(A) \subset X$.
				Donc il existe $x \in X$ tel que $g(y) = (g \circ f)(x)$ ou encore tel que $y = f(x) = \varphi(x)$ (par l'injectivité de $g$).
				\item Si $g(y) \in A'$, alors il existe $x' \in A'$ tel que $g(y) = x'$ ou encore $y = g^{-1}(x') = \varphi(x')$.
			\end{enumerate}
			On a montré que pour tout $y \in Y$, il existe un $x \in X$ tel que $\varphi(x) = y$. $\varphi$ est donc surjective.
		\end{enumerate}
		\end{proof}

		\subsubsection{Dimension}
		\begin{rmq} Les deux théorèmes suivants ne sont pas démontrés et sont considérés comme admis. \end{rmq}

		\begin{thm}\label{cardPartiesFinies} Soit $X$ un ensemble infini. Soit $\mathcal F_X \coloneqq \{Y \subseteq X \tq Y \text {est une partie finie}\}$.
		Alors $\abs {\mathcal F_X} = \abs X$. \end{thm}

		\begin{thm}\label{unionEnsFinisIndicésInfini} Soient $I$ un ensemble d'indices infini et $\{X_i\}_{i \in I}$ une famille d'ensembles finis indicés par $I$.
		Alors~: \[\abs {\bigcup_{i \in I}X_i} \leq \abs I.\] \end{thm}

		\begin{thm} Soient $V$ un espace vectoriel et $E, E'$ deux bases quelconques de $V$. Alors $\abs E = \abs {E'}$. \end{thm}

		\begin{proof} Supposons d'abord $\abs E = n \in \N$ donc $E$ fini. Posons $k \coloneqq \abs {E \cap E'}$. Montrons par récurrence sur $n-k$ que
		$\abs E = \abs {E'}$.

		\underline{cas de base~:} si $n-k = 0$, alors $\abs {E \cap E'} = \abs {E}$ et il faut donc $E \subseteq E'$. Or $E$ et $E'$ sont tous deux des familles
		génératrices minimales donc il faut $E = E'$.

		\underline {pas de récurrence~:} si $n-k = i > 0$, on sait que $E \cap E' \subseteq E$ et donc si $E = \{e_1, \dotsc, e_n\}$, on a
		$E \cap E' = \{e_1, \dotsc, e_k\}$. De plus, on sait que $E$ est une partie génératrice minimale. Donc $E \setminus \{e_{k+1}\}$ n'est plus génératrice.
		Dès lors, on sait qu'il existe $y \in E' \setminus \eng {E \setminus \{e_{k+1}\}}$. Soit $E'' \coloneqq \{y\} \cup E \setminus \{e_{k+1}\}$. On sait donc dire~:
		\[\eng {E \setminus \{e_{k+1}\}} \subset \eng {E''} \subseteq \eng E = V.\] Par le lemme~\ref{identitéSousEspaceInclus}, on sait $\eng {E''} = V$. De plus, $E''$
		est libre. Donc $E''$ est une base de $V$. On a donc $\abs {E''} = \abs E = n$. De plus, puisque $E' \cap E'' = \{e_1, \dotsc, e_{k+1}\}$, peut dire~:
		$\abs {E''} - \abs {E' \cap E''} = n - (k+1) = n-k-1$. Alors Par hypothèse de récurrence, $\abs {E'} = \abs {E''}$.

		Supposons ensuite que $E$ est de cardinal infini (et donc $E'$ également). Posons $\mathcal F_E \coloneqq \{X \subseteq E \tq X \text{ partie finie}\}$ et
		$\mathcal F_{E'} \coloneqq \{Y \subseteq E' \tq Y \text{ partie finie}\}$. Par le théorème~\ref{cardPartiesFinies} (admis), on sait que
		$\abs {\mathcal F_E} = \abs E$ et $\abs {\mathcal F_{E'}} = \abs {E'}$.

		Soit $Y \in \mathcal F_E$. Alors, par définition de $Y$, on sait que $E' \cap \eng Y$ est fini. Dès lors, on sait qu'il existe une base $B$ de $\eng Y$
		telle que $E' \cap \eng Y \subseteq B \subseteq \eng Y$. Or $Y$ est également une base de $\eng Y$. Donc $\abs Y = \abs B$ et donc $Y$ est fini.

		Posons $f : \mathcal F_E \to \mathcal F_{E'} : Y \mapsto E' \cap \eng Y$. Soit $b \in E'$. On sait qu'il existe $Y \in \mathcal F_E$ tel que $b \in \eng Y$.
		Donc $b \in f(Y) = E' \cap \eng Y$. On peut déduire que $E'$ est égal à l'union de tous les éléments de $\mathcal F_E$ :
		\[E' = \bigcup_{Y \in f(\mathcal F_E)}Y.\] Dès lors, par le théorème~\ref{unionEnsFinisIndicésInfini}, on sait que $\abs {E'} \leq \abs {f(\mathcal F_E)}$.
		De plus, on sait que $f(\mathcal F_E) \subseteq \mathcal F_{E'}$. Donc $\abs {E'} \leq \abs {f(\mathcal F_E)} \leq \abs {\mathcal F_{E'}} = \abs {E'}$.
		Il faut donc nécessairement $\abs {f(\mathcal F_{E})} = \abs {\mathcal F_{E'}} = \abs {E'}$.

		Et comme, de plus, on a $\abs {f(\mathcal F_E)} \leq \abs {\mathcal F_E}$. On sait donc dire
		$\abs {E'} = \abs {f(\mathcal F_E)} \leq \abs {\mathcal F_E} = \abs E$, d'où $\abs {E'} \leq \abs E$.

		Par symétrie, on trouve la même chose pour $E$ : à savoir $\abs E \leq \abs {E'}$. Donc par le théorème de Cantor-Bernstein (théorème~\ref{Cantor-Bernstein}), on
		sait que $\abs E = \abs {E'}$. \end{proof}

		\begin{déf} Soit $V$ un e.v. On définit la dimension réelle de $V$ par le cardinal de ses bases~: $\dim_\R V = \abs E$ avec $E$, base de $V$. \end{déf}

		\subsubsection{Dépendance base et système de coordonnées}
		\begin{déf} Soit $V$ un espace vectoriel réel. Tout vecteur $v \in V$ peut s'écrire comme une combinaison linéaire de $\lambda_ie_i$ où $e_i$ sont les vecteurs
		de la base $E$ de $V$. On appelle le n-uple de $\lambda_i$ suivant $(\lambda_1, \dotsc, \lambda_n) \in \R^n$ les coordonnées de $v$ dans la base $E$. \end{déf}

		\begin{prp} Soit $V$ un espace vectoriel et $E$ une base de $V$. Alors chaque vecteur $v \in V$ s'écrit de manière unique comme une combinaison linéaire des
		vecteurs de la base $E$. \end{prp}

		\begin{proof} Par hypothèse, $E$ est génératrice. Soit $v \in V$. Alors~: \[v = \sum_{i=1}^n\lambda_ie_i = \sum_{i=1}^n\mu_ie_i.\] Alors, on peut écrire
		$v-v = 0$, d'où~: \[0 = v - v = \sum_{i=1}^n\lambda_ie_i - \sum_{i=1}^n\mu_ie_i = \sum_{i=1}^n(\lambda_i - \mu_i)e_i.\] Comme $E$ est libre, la seule combinaison
		nulle possible est la combinaison linéaire nulle triviale avec $\lambda_i-\mu_i=0$ pour tout $i$, ou encore $\lambda_i = \mu_i$. \end{proof}

		\begin{prp}\label{inclusionBases} Soient $V$ un e.v. et $W$ un sous-espace de $V$. Soient $E_V$ et $E_W$ des bases respectivement de $V$ et $W$. Alors on peut~:
		\begin{enumerate}
			\item compléter $E_W$ pour en faire une base de $V$~;
			\item réduire $E_V$ pour en faire une base de $W$.
		\end{enumerate}
		\end{prp}

		\begin{proof} Montrons d'abord comment étendre $E_W$. Si $V=W$, $E_W$ est une base de $V$. Supposons $W \neq V$. Alors il existe $v_1 \in V \setminus W =
		V \setminus \eng {E_W}$. De plus, $E_W \cup \{v_1\}$ est toujours libre. Si $\eng {E_W \cup \{v_1\}} = V$, alors c'est bon, sinon, il doit exister
		$v_2 \in V \setminus \eng {E_W \cup \{v_1\}}$. Mais $E_W \cup \{v_1, v_2\}$ est toujours libre. Si $\eng {E_W \cup \{v_1, v_2\}}$, il existe $v_3$ etc.
		Une fois $k = \dim_\R V - \dim_\R W$ vecteurs ajoutés, on obtient $\eng {E_W \cup \{v_1, \dotsc, v_k\}} = V$ où $E_W \cup \{v_1, \dotsc, v_k\}$ est libre.
		Donc on a complété $E_W$ pour en faire une base de $V$.

		Montrons maintenant comment réduire $E_V$ pour en faire une base de $W$. Si $V=W$, alors $E_V$ est une base de $W$. Supposons $W \neq V$. Alors il existe
		$v_1 \in \eng {E_V} \setminus W$ tel que $v_1$ est une combinaison linéaire des des vecteurs de $E_V \setminus \{v_1\}$. $E_V \setminus \{v_1\}$ est toujours
		génératrice. Si $E_V \setminus \{v_1\}$ n'est pas libre, c'est qu'il existe un vecteur $v_2 \in \eng {E_V \setminus \{v_1\}} \setminus W$ qui est une combinaison
		linéaire des vecteurs de $E_V \setminus \{v_1\}$. $E_V \setminus \{v_1, v_2\}$ est toujours génératrice. Et si elle n'est pas libre, alors il existe $v_3$ etc.
		Une fois $k=\dim_\R V-\dim_\R W$ vecteurs retirés, on obtient $E_V \setminus \{v_1, \dotsc, v_k\}$ est libre dans $W$ et est restée génératrice. Donc on a
		réduit $E_V$ pour en faire une base de $W$. \end{proof}

		\begin{cor} Soit $W$ un sous-espace de l'e.v. $V$. Alors $\dim_\R W \leq \dim_\R V$. \end{cor}

		\begin{proof} Par la proposition~\ref{inclusionBases}, on sait qu'il faut ajouter (ou retirer) un nombre positif de vecteurs pour passer d'une base du
		sous-espace (ou de l'espace) à une base de l'espace (ou du sous-espace). \end{proof}

		\begin{prp} Soit $V$ un e.v. réel de dimension finie $d$. Si $X$ est une partie libre de $V$, alors $\abs X \leq d$ et si $\abs X = d$, alors $X$ est une base.
		Si $X$ est une partie génératrice, alors $\abs X \geq d$ et si $\abs X = d$, alors $X$ est une base de $V$. \end{prp}

		\begin{proof} Soit $E$ une base de $V$. Supposons $X = \{x_1, \dotsc, x_k\}$ une partie libre. Si $k > d$, alors $x_i$ peut être écrit comme une combinaison
		linéaire des vecteurs de $E$ pour $1 \leq i \leq d$. Alors $x_{d+1}$ sait s'écrire comme une combinaison linéaire des $x_i$ pour $1 \leq i \leq d$, ce qui
		contredit le fait que $X$ est libre. Il faut donc $\abs X \leq d$ Si $\abs X = d$, $X$ est une partie libre maximale, donc une base par le
		théorème~\ref{equivBases}.

		Supposons $X = \{x_1, \dotsc, x_k\}$ une partie génératrice. Si $k < d$, alors Il existe $v \in V \setminus \eng X$ ce qui contredit le fait que $X$ est
		génératrice. Si $\abs X = d$, alors $X$ est une partie génératrice minimale, donc une base par le théorème~\ref{equivBases}. \end{proof}

\newpage
\section{Relations entre espaces vectoriels}
	\subsection{Isomorphismes}
		\begin{déf} Soient $V$, $W$ deux e.v. Soit $A : V \to W$ une application linéaire bijective. Alors $A$ est un isomorphisme. \end{déf}

		\begin{déf} Deux espaces vectoriels $V$ et $W$ sont dits isomorphes s'il existe un isomorphisme $I : V \to W$. \end{déf}

		\begin{lem}\label{invIsom} Soit $A : V \to W$ un isomorphisme entre deux espaces vectoriels. Alors l'inverse $A^{-1}$ de $A$ est également un isomorphisme.
		\end{lem}

		\begin{proof} Si $A$ est bijective, alors $A^{-1}$ est également bijective. Soient $\lambda \in \R, v_1, v_2 \in V$ et $w_1, w_2 \in W$ tels que $A(v_1) = w_1$ et
		$A(v_2) = w_2$. Alors, par la linéarité de $A$ : \[A^{-1}(w_1 + \lambda w_2) = A^{-1}(A(v_1) + \lambda A(v_2)) = A^{-1}(A(v_1 + \lambda v_2)) = v_1 + \lambda v_2
		= A^{-1}(w_1) + \lambda A^{-1}(w_2).\] On a bien $A^{-1} : W \to V$ linéaire et bijective. \end{proof}

		\begin{prp}\label{dimDIsomorpheRd} Soit $V$ un espace vectoriel réel de dimension $d$. Alors $V$ est isomorphe à $\R^d$. \end{prp}

		\begin{proof} Soient $E$ la base canonique de $\R^d$ et $F$ une base de $V$. Soit $\phi : V \to \R^d$ telle que si $v = \sum_{i=1}^d\lambda_if_i$, alors on
		définit~: \[f(v) \coloneqq (\lambda_i)_{i \in \{1, \dotsc, d\}}.\] La fonction $f$ est linéaire car si
		$v = \sum_{i=1}^n\lambda_if_i$ et $w = \sum_{i=1}^n\mu_if_i$, alors~:
		\[f(v + \alpha w) = f\left(\sum_{i=1}^n(\lambda_i + \alpha\mu_i)f_i\right) = (\lambda_i + \alpha\mu_i)_{i \in \{1, \dotsc, d\}}.\]
		De plus, $f$ est bijective car elle est injective~:
		\[(\lambda_1, \dotsc, \lambda_d) = (\mu_1, \dotsc, \mu_d) \Rightarrow \sum_{i=1}^d\lambda_if_i = \sum_{i=1}^d\mu_if_i,\]
		et elle est surjective~:
		\[\forall (\lambda_i)_{i \in \{1, \dotsc, d\}} : \exists v \in V \tq f(v) = (\lambda_i)_{i \in \{1, \dotsc, d\}}.\]
		Il suffit de prendre $v = \sum_{i=1}^n\lambda_if_i$. \end{proof}

		\begin{prp}\label{dimDIso} Tous les espaces vectoriels réels de dimension $d$ sont isomorphes entre eux. \end{prp}

		\begin{proof} Soient $V$ et $W$ deux espaces vectoriels de dimension $d$. Par la proposition~\ref{dimDIsomorpheRd}, on sait qu'il existe $f : V \to \R^d$
		et $g : W \to \R^d$ deux isomorphismes. Puisque $g^{-1}$ est également un isomorphisme (par le lemme~\ref{invIsom}), et puisque la composée de bijection reste
		une bijection ainsi que la composée de transformations linéaires reste une transformation linéaire, la fonction $(g^{-1} \circ f) : V \to W$ est un isomorphisme
		entre $V$ et $W$. \end{proof}

		\begin{prp} Soient $V, W$ deux espaces vectoriels réels de dimension finie. Alors~: \[\dim_\R(V+W) + \dim_\R(V \cap W) = \dim_\R V + \dim_\R W.\] \end{prp}

		\begin{proof} Soient $B_{V \cap W} = \{e_1, \dotsc, e_r\}$  une base de $V \cap W$, $B_V = \{e_1, \dotsc, e_r, x_1, \dotsc, x_n\}$ une base de $V$,
		$B_W = \{e_1, \dotsc, e_r, y_1, \dotsc, y_t\}$ une base de $W$. Soit $B \coloneqq B_V \cup B_W = \{e_1, \dotsc, e_r, x_1, \dotsc, x_n, y_1, \dotsc, y_t\}$.
		$B$ est une partie génératrice de $V+W$. Montrons que $B$ est également une partie libre de $V+W$. Supposons~:
		\[\sum_{i=1}^r\alpha_ie_i + \sum_{i=1}^n\beta_ix_i + \sum_{i=1}^t\gamma_iy_i = 0.\] Posons~:
		\[v \coloneqq \sum_{i=1}^r\alpha_ie_i + \sum_{i=1}^n\beta_ix_i \in V = - \sum_{i=1}^t\gamma_iy_i \in W.\]
		Si $v \in V$ et $v \in W$, alors $v \in V \cap W$. Donc il existe $(\lambda_1, \dotsc, \lambda_r) \in \R^r$ tels que~:
		\[v = \sum_{i=1}^r\lambda_ie_i.\]
		On a donc~:
		\[v \coloneqq \sum_{i=1}^r\alpha_ie_i + \sum_{i=1}^n\beta_ix_i = -\sum_{i=1}^t\gamma_iy_i = \sum_{i=1}^r\lambda_ie_i.\]
		Donc~: \[\sum_{i=1}^r\lambda_ie_i + \sum_{i=1}^t\gamma_iy_i = 0,\] ce qui implique $\lambda_i = \gamma_j = 0$ pour tout $i, j$ puisque $B_W$ est libre. Dès lors,
		$v = 0$, ou encore~: \[v = \sum_{i=1}^r\alpha_ie_i + \sum_{i=1}^n\beta_ix_i = 0.\] Mais puisque $B_V$ est libre, il faut $\alpha_i = \beta_j = 0$ pour tout
		$i, j$. La seule combinaison linéaire nulle possible est donc la combinaison linéaire nulle triviale. Alors $B$ est libre.

		Dès lors, on sait que $\dim_\R(V+W) + \dim_\R(V \cap W) = r+n+t + r = (r+n) + (r+t) = \dim_\R V + \dim_\R W$. \end{proof}

		\begin{prp}\label{dim=dim+dim} Soient $V, W$ deux espaces vectoriels réels. Soit $A : V \to W$ une application linéaire.
		Alors $\dim_\R V = \dim_\R \Ker(A) + \dim_\R\Imf(A)$. \end{prp}

		\begin{proof} Soient $B_{\Ker}\{e_1, \dotsc, e_m\}$ une base du $\Ker(A)$ et $B_{\Imf} = \{f_1, \dotsc, f_n\}$ une base de $\Imf(A)$. Soient
		$v_1, \dotsc, v_n \in V$ tels que $A(v_i) = f_i$ pour tout $1 \leq i \leq n$. Montrons que $B \coloneqq B_{\Ker} \cup \{v_, \dotsc, v_n\}$ est une base de $V$.
		Montrons d'abord que $B$ est génératrice.

		Soit $v \in V$. Alors on a: \[A(v) = \sum_{i=1}^n\lambda_if_i = A\left(\sum_{i=1}^n\lambda_iv_i\right).\] Dès lors, on sait construire $0 \in V$ par~:
		\[0 = A(v) - A\left(\sum_{i=1}^n\lambda_iv_i\right) = A\left(v-\sum_{i=1}^n\lambda_iv_i\right).\] Il en découle directement que le vecteur
		$v - \sum_{i=1}\mu_iv_i \in \Ker(A)$. Or, tout vecteur de $\Ker(A)$ peut s'exprimer comme~: \[\sum_{i=1}^m\mu_ie_i.\] On a donc~:
		\[v-\sum_{i=1}^n\lambda_iv_i = \sum_{i=1}^m\mu_ie_i \iff v = \sum_{i=1}^n\lambda_iv_i + \sum_{i=1}^m\mu_ie_i \in \eng B.\]

		Maintenant, montrons que $B$ est libre. Soient $\alpha_1, \dotsc, \alpha_m, \beta_1, \dotsc, \beta_n$ tels que~:
		\[\sum_{i=1}^m\alpha_ie_i + \sum_{i=1}^n\beta_iv_i = 0.\] Alors, on peut écrire~:
		\[0 = A(0) = A\left(\sum_{i=1}^m\alpha_ie_i + \sum_{i=1}^n\beta_if_i\right) = A\left(\sum_{i=1}^m\alpha_ie_i\right) + \sum_{i=1}^n\beta_iA(v_i)
		= A\left(\sum_{i=1}^m\alpha_ie_i\right) + \sum_{i=1}^n\beta_if_i = 0 + \sum_{i=1}^n\beta_if_i.\] Or $B_{\Imf}$ est libre, donc il faut $\beta_i = 0$ pour tout
		$i$. Il reste donc~: \[\sum_{i=1}^m\alpha_ie_i = 0,\] et comme $B_{\Ker}$ est libre, il faut $\alpha_i = 0$ pour tout $i$. On a donc $B$ libre. Puisque $B$ est
		libre et génératrice de $V$, alors $B$ est une base de $V$. Donc $\dim_\R V = \abs B = n+m = \dim_\R\Imf(A) + \dim_\R\ker(A)$. \end{proof}

		\begin{prp}\label{condIsom} Soient $V, W$ deux espaces vectoriels et $A : V \to W$ une transformation linéaire. Alors~:
		\begin{enumerate}
			\item $A$ est injective si et seulement si $\dim_\R \Ker(A) = 0$~;
			\item $A$ est surjective si et seulement si $\dim_\R \Imf(A) = \dim_\R W$~;
			\item $A$ est un isomorphisme si et seulement si $\dim_\R\Ker(A) = 0$ et $\dim_\R\Imf(A) = \dim_\R W$~;
			\item $A$ est un isomorphisme si et seulement si $\dim_\R\Ker(A) = 0$ et $\dim_\R V = \dim_\R W$~;
			\item $A$ est un isomorphisme si et seulement si $\dim_\R \Imf(A) = \dim_\R W$ et $\dim_\R V = \dim_\R W$.
		\end{enumerate}
		\end{prp}

		\begin{proof}~
		\begin{enumerate}
			\item Montrons que si $\dim_\R\Ker(A) = 0$, alors $A$ est injective. Si $\dim_\R\Ker(A) = 0$, alors $\Ker(A) = \{0_V\}$. Donc soient $v_1, v_2 \in V$.
			      Supposons que $f(v_1) = f(v_2)$. Dès lors, $0 = f(v_1)-f(v_2) = f(v_1-v_2) = f(0_V)$. Comme $0_V$ est le seul élément du noyau, il faut $v_1-v_2 = 0$,
				  ou encore $v_1 = v_2$.

			      Montrons maintenant que si $A$ est injective, alors $\dim_\R\Ker(A) = 0$. Soient $v_1, v_2 \in V$. On sait que $f(v_1)=f(v_2) \Rightarrow v_1=v_2$.
				  On sait également que $f(0_V) = 0_W$. Donc pour tout si $v \in V$, $f(v) = f(0) = 0$, alors $v=0$. Autrement dit, seul $0_V$ est envoyé sur $0_W$.
				  On a bien $\dim_\R\Ker(A)=0$.

			\item Montons que si $\dim_\R\Imf(A) = \dim_\R W$, alors $A$ est surjective. Soit $E \{e_1, \dotsc, e_m\}$ une base de $W$. Soit $F = \{f_1, \dotsc, f_n\}$
			      une base de $\Imf(A)$. Soient $v_1, \dotsc, v_m$ tels que $A(v_i) = f_i$. Par la proposition~\ref{dimDIso}, on sait que $\Imf(A)$ et $W$ sont
				  isomorphes. Donc soit $w \in W$, on peut écrire $w$ comme~:
				  \[w = \sum_{i=1}^m\lambda_ie_i = \sum_{i=1}^m\mu_if_i = \sum_{i=1}^m\mu_iA(v_i) = A\left(\sum_{i=1}^m\mu_iv_i\right).\]
				  Il existe donc un vecteur $v \in V$ tel que $A(v) = w$.

			      Montrons maintenant que si $A$ est surjective, alors $\dim_\R \Imf(A)=\dim_\R W$. Soit $w \in W$. On sait qu'il existe $v \in V$ tel que $A(v) = w$.
				  Prenons $E = \{e_1, \dotsc, e_m\}$ une base de $W$. Prenons $v_1, \dotsc, v_m \in V$ tels que $A(v_i) = e_i$. On a $\{A(v_1), \dotsc, A(v_m)\}$ est une
				  base de $\Imf(A)$ car pour tout vecteur $w \in W$, il existe $v \in V$ tel que~:
				  \[w = \sum_{i=1}^m\lambda_ie_i = \sum_{i=1}^m\lambda_iA(v_i) = A\left(\sum_{i=1}^m\lambda_iv_i\right) = A(v).\]

			\item Par les propositions 1 et 2.

			\item Montrons que si $A$ est un isomorphisme et $\dim_\R\Ker(A) = 0$ (par l'injectivité de $A$), alors $\dim_\R V = \dim_\R W$.
			      Par la proposition~\ref{dim=dim+dim}, on sait $\dim_\R V = \dim_\R \Ker(A) + \dim_\R\Imf(A)$. Donc si $\dim_\R \Ker(A) = 0$, on a
				  $\dim_\R V = \dim_\R W$.

			      Montrons maintenant que si $\dim_\R V = \dim_\R W$ et $\dim_\R \Ker(A) = 0$, alors $A$ est un isomorphisme. Soient $E_V = \{e_1, \dotsc, e_n\}$ une base
				  de $V$ et $E_W = \{f_1, \dotsc, f_n\}$ une base de $W$. En définissant $A$ par $A(e_i) = f_i$, on a bien un isomorphisme tel que $\dim_\R\Ker(A) = 0$
				  car seul $0_V$ est envoyé sur $0_W$.

			\item Montrons d'abord que si $A$ est un isomorphisme, alors $\dim_\R \Imf(A) = \dim_\R W$ et $\dim_\R V = \dim_\R W$. À nouveau, en partant de la
			      proposition~\ref{dim=dim+dim}, on sait que $\dim_\R V = \dim_\R \Imf(A) + \dim_\R\Ker(A)$. Or on sait $A$ surjectif. Donc on sait que
				  $\dim_\R W = \dim_\R\Imf(A)$. Et puisque $A$ est injectif, on a $\dim_\R\Ker(A) = 0$ donc $\dim_\R V = \dim_\R\Imf(A) = \dim_\R W$.

			      Si $A$ n'est pas un isomorphisme, alors $\dim_\R\Imf(A) \neq \dim_\R W$.
		\end{enumerate}
		\end{proof}

	\subsection{Somme directe}
		\begin{déf} Soient $V$ et $W$ deux espaces vectoriels. On définit la somme directe de $V$ et $W$ par $V \oplus W \coloneqq V \times W$ muni de l'addition interne et
		de la multiplication externe triviales membre à membre. \end{déf}

		\begin{prp} la somme directe de $n$ espaces vectoriels est un espace vectoriel. \end{prp}

		\begin{proof} Les 8 axiomes se prouvent grâce à leur homologue des espaces vectoriels pères. \end{proof}

		\begin{prp} Soit $V$ un espace vectoriel réel et soient $W_1, W_2$ deux sous-espaces de $V$. Si $W_1 \cap W_2 = \{0\}$, alors $W_1+W_2$ est isomorphe à
		$W_1 \oplus W_2$. \end{prp}

		\begin{proof} Soit $A : W_1 \oplus W_2 \to W_1 + W_2 : (v,w) \mapsto v+w$. $A$ est linéaire car pour tout $v_1, w_1 \in W_1, w_2, v_2 \in W_2, \lambda \in \R$,
		on a~: $A((w_1, w_2) + \lambda(v_1, v_2)) = A(w_1 + \lambda v_1, w_2 + \lambda v_2) = (w_1+\lambda v_1) + (w_2 + \lambda v_2) = (w_1 + w_2) + \lambda(v_1+v_2)
		= A(w_1, w_2) + \lambda A(v_1, v_2)$.

		De plus, $\Ker(A) = \{(w_1, w_2) \in W_1 \oplus W_2 \tq w_1 + w_2 = 0\}$. Or, par hypothèse, $W_1 \cap W_2 = \{0\}$. Donc la seule possibilité pour avoir
		$w_1 + w_2 = 0$ est d'avoir $w_1=w_2=0$. Donc $\dim_\R\Ker(A) = 0$. Soit $x \in W_1+W_2$. On sait qu'il existe $w_1 \in W_1, w_2 \in W_2$ tels que
		$x = w_1+w_2 = A(w_1, w_2)$. Donc $A$ est surjective. Dès lors, par la proposition~\ref{condIsom}, $A$ est un isomorphisme. \end{proof}

		\begin{prp} Soit $V$ un e.v. et $E = \{e_1, \dotsc, e_d\}$ une base de $V$. Soit $W$ un autre e.v. Soit $F =\{f_1, \dotsc, f_d\} \subset W$. Alors il existe
		une unique transformation linéaire $A : V \to W$ telle que $A(e_i) = f_i$ pour tout $1 \leq i \leq d$. \end{prp}

		\begin{proof} Soit $g : E \to F : e_i \mapsto f_i$. On définit $A_g : V \to W : v = \sum_{i=1}^d\lambda_ie_i \mapsto w = \sum_{i=1}^d\lambda_ig(e_i)$.
		Puisque $g$ est bijective, $A_g$ est également bijective. De plus, $A_g$ est linéaire~:
		\[A_g\left(\sum_{i=1}^d\lambda_ie_i + \alpha\sum_{i=1}^d\mu_ie_i\right) = \sum_{i=1}^d(\lambda_i + \alpha\mu_i)g(e_i)
		= A_g\left(\sum_{i=1}^d\lambda_ie_i\right) + \alpha A_g\left(\sum_{i=1}^d\mu_ie_i\right).\] \end{proof}

		\begin{déf} Soient $V, W$ deux e.v. et $A : V \to W$ une application linéaire. Alors on appelle $\dim_\R\Imf(A)$ le rang de $A$ et on le note $\rang A$.
		\end{déf}

	\subsection{Groupe linéaire}
		\begin{déf} Soit $V$ un espace vectoriel réel. Le groupe linéaire (général) est l'ensemble des isomorphismes $\{A : V \to V \tq A \text{ est un automorphisme}\}$.
		On le note $GL(V)$. Si $V = \R^d$, on note $GL_d(\R)$ ou $GL(d, \R)$. \end{déf}

		\begin{rmq} La composée de deux automorphismes est toujours un automorphisme car la composée de bijections reste bijective et si $f : V \to V$ et $g : V \to V$,
		alors $(f \circ g) : V \to V$ également. \end{rmq}

		\begin{prp} Soient deux espaces vectoriels $V$ et $W$. Soit $E = \{e_1, \dotsc, e_d\}$. L'application linéaire $A : V \to W$ est un isomorphisme si et
		seulement si $A(E) = \{A(e_1), A(e_2), \dotsc, A(e_d)\}$ est une base de $W$. \end{prp}

		\begin{proof} Montrons que si $A$ est un isomorphisme, alors $A(E)$ est une base de $W$. Les vecteurs de $A(E)$ sont des combinaisons linéaires des vecteurs
		de $E$. Donc $A(E)$ est libre. De plus, soit $v \in W$, il existe $w \in W$ tel que $w = A(v) = \sum_{i=1}^d\lambda_iA(e_i)$. Donc $A(E)$ est une partie
		génératrice de $W$.

		Montrons maintenant que si $A(E)$ est une base de $W$,a lors $A$ est un isomorphisme. Puisque $\dim_\R V = \dim_\R W$ et $\Imf(A) = W$, par la
		proposition~\ref{condIsom}, $A$ est un isomorphisme. \end{proof}

\newpage
\section{Algèbre matricielle}
	\subsection{Matrice associée à une transformation linéaire}
		\begin{déf} Soient deux espaces vectoriels $V$ et $W$. On définit $\Hom(V, W) \coloneqq \{A : V \to W \tq A \text { transformation linéaire}\}$. \end{déf}

		\begin{prp} Soient $V, W$ deux e.v. En munissant $\Hom(V, W)$ de l'addition interne \[(A+B)(v) \coloneqq A(v) + B(v),\] et de la multiplication externe
		\[(\lambda A)(v) \coloneqq \lambda(A(v)),\] $\Hom(V, W)$ est une espace vectoriel. \end{prp}

		\begin{proof} Les 8 propriétés sont montrées par les propriétés identiques de $V$ et $W$. \end{proof}

		\begin{déf} Soit $W = \Hom(V, V)$ un espace vectoriel. Les éléments de $V$ sont appelés opérateurs linéaires (ou endomorphismes) de $V$. \end{déf}

		\begin{déf} Soient $V, W$ deux espaces vectoriels de dimension finie, $A \in \Hom(V, W)$. Si $E = \{e_1, \dotsc, e_t\}, F = \{f_1, \dotsc, f_s\}$ sont une
		base de $V$ et $W$ respectivement, alors on définit $m_{F, E}(A) \coloneqq [A_{ij}] \in \M st\R$ telle que~: \[A(e_j) = \sum_{i=1}^sA_{ij}f_i.\] \end{déf}

	\subsection{Opérations sur les matrices}
		\begin{prp}\label{linéaritéMatrices} Soient $V, W$ deux espaces vectoriels avec une base respective $E, F$, soient $A, B \in \Hom(V, W)$,
		et $m_{F, E}(A), m_{F, E}(B)$ les matrices associées à $A$ et $B$. Alors~:
		\begin{align*}
			m_{F, E}(\lambda A) &= \lambda m_{F, E}(A) \\
			m_{F, E}(A + B) &= m_{F, E}(A) + m_{F, E}(B).
		\end{align*}
		\end{prp}

		\begin{proof} On définit $m_{F, E}(A + \lambda B) = [(A+ \lambda B)_{ij}]$ comme étant la matrice associée la transformation linéaire $A+\lambda B$. Donc elle est
		telle que~: \[(A+\lambda B)(e_j) = \sum_{i=1}^s(A + \lambda B)_{ij}f_i.\] Cependant, on sait que $(A+\lambda B)(v) = A(v) + \lambda B(v)$. Donc on sait~:
		\[\sum_{i=1}^s(A + \lambda B)_{ij}f_i = (A+\lambda B)(e_j) = A(e_j) + \lambda B(e_j) = \sum_{i=1}^sA_{ij}f_i + \lambda\sum_{i=1}^sB_{ij}f_i
		= \sum_{i=1}^s(A_{ij} + \lambda B_{ij})f_i.\] On a donc pour tout $i, j : (A+\lambda B)_{ij} = A_{ij} + \lambda B_{ij}$. \end{proof}

		\begin{rmq} Une fois les bases $E$ et $F$ fixées, $m_{F, E}$ est une fonction telle que
		$m_{F, E} : \Hom(V, W) \to \M {\dim_\R W}{\dim_\R V}\R : A \mapsto m_{F, E}(A)$.
		Et la proposition~\ref{linéaritéMatrices} montre que cette fonction est linéaire. \end{rmq}

		\begin{déf} Soient trois espaces vectoriels $U, V, W$ de base respective $D = \{d_1, \dotsc, d_s\}, E = \{e_1, \dotsc, e_t\}, F = \{f_1, \dotsc, f_u\}$.
		Soient $A \in \Hom(U, V)$ et $B \in \Hom(V, W)$. La composée $(B \circ A) \in \Hom(U, W)$. Soient les matrices $m_{E, D}(A) = [A_{ij}] \in \M ts\R$ et
		$m_{F, E}(B) = [B_{ij}] \in \M ut\R$. On définit le produit de ces deux matrices par $m_{F, E}(B \circ A) = [(BA)_{ij}] \in \M us\R$. Il faut que~:
		\[(B \circ A)(d_j) = \sum_{i=1}^u(BA)_{ij}f_i. \]

		Or, par la définition de la composée, on sait~:
		\[(B \circ A)(d_j) = B(A(d_j)) = B\left(\sum_{i=1}^tA_{ij}e_i\right) = \sum_{i=1}^tA_{ij}B(e_i)
		                   = \sum_{i=1}^tA_{ij}\sum_{k=1}^uB_{ki}f_k = \sum_{k=1}^u\left(\sum_{i=1}^tB_{ki}A_{ij}\right)f_k.\]

		On peut donc conclure pour tout $i, j$~: \[(BA)_{ij} = \sum_{k=1}^tB_{ik}A_{kj}.\] \end{déf}

		\begin{rmq} Selon cette définition, pour $A = [A_{ij}] \in \M st\R, B = [B_{ij}] \in \M tu\R$, on a
		$(AB) = [(AB)_{ij}] = \left[\sum_{k=1}^tA_{ik}B_{kj}\right] \in \M su\R$.

		Donc pour utiliser les notations des matrices d'applications, on peut écrire $m_{F, D}(B \circ A) = m_{F, E}(B)m_{E, D}(A)$. \end{rmq}

		\begin{prp} Si $V$ et $W$ sont deux espaces vectoriels. Soit $A : V \to W$ une transformation linéaire. Alors si $x \in V : A(x) = m_{F, E}(A)x$. \end{prp}

		\begin{proof} Soient $x \in V$ et $E = \{e_1, \dotsc, e_t\}$ une base de $V$. Soit $F = \{f_1, \dotsc, f_u\}$ une base de $W$. Dès lors
		$x = \sum_{i=1}^tx_ie_i$. On peut donc écrire~:
		\[A(x) = \sum_{i=1}^tx_iA(e_i) = \sum_{i=1}^tx_i\sum_{k=1}^uA_{ki}f_k = \sum_{k=1}^u\left(\sum_{i=1}^tA_{ki}x_i\right)f_k = m_{F, E}(A)x.\]
		En posant $x \in \M t1\R$ le vecteur colonne contenant les coordonnées de $x$ dans la base $E$. \end{proof}

		\begin{prp} Soient $V, W$ deux espaces vectoriels de dimension respective $t$ et $s$ avec une base $E, F$. La transformation
		$m_{F, E} : \Hom(V, W) \to \M st\R : A \mapsto m_{F, E}(A)$ est un isomorphisme. \end{prp}

		\begin{proof} La proposition~\ref{linéaritéMatrices} montre la linéarité de cette transformation. Montrons maintenant qu'elle est bijective.

		$m_{F, E}$ est injective car les matrices déterminent totalement les applications. Dès lors, deux matrices identiques donnent deux applications identiques
		également. Et $m_{F, E}$ est surjective car pour toute matrice $[A_{ij}] \in \M st\R$, on définit $A \in \Hom(V, W)$ telle que
		$A(x) \coloneqq \sum_{i=1}^s\sum_{j=1}^tA_{ij}x_j$. \end{proof}

		\begin{déf} Soit $V$ un espace vectoriel de dimension finie $n$ et une base $E$ de $V$. Soit $A \in \Hom(V, V)$. Alors il existe $A^{-1} \in \Hom(V, V)$ l'inverse
		de $A$. On définit la matrice inverse de $m_{E, E}(A)$ par $m_{E, E}(A)^{-1} \coloneqq m_{E, E}(A^{-1})$. \end{déf}

	\subsection{Méthode de Gauss pour inverser une matrice}
		\begin{déf} On définit la matrice $\Lambda_{k\ell} = [\lambda_{ij}]$ telle que $\lambda_{ij} = 1$ si $(i, j) = (k, \ell)$ et $\lambda_{ij} = 0$ sinon. \end{déf}

		\begin{rmq} En partant de l'algorithme de Gauss (algorithme~\ref{algoGauss}), et en se concentrant sur les définitions des opérations matricielles, on remarque
		que les opérations fondamentales de l'algorithme correspondent à des opérations algébriques sur les matrices.

		Changer la $i$eme ligne de la matrice $A = [A_{ij}]$ par $L_i + \mu L_j$ revient à faire~: \[(I + \mu \Lambda_{ij})A.\] De plus, multiplier à gauche par
		$(I + \mu\Lambda_{ii})$ correspond à multiplier la $i$eme ligne par $\mu+1$. Donc multiplier une ligne par un facteur $\alpha$ revient à faire~:
		\[(I + (\alpha-1)\Lambda_{ii})A.\] Finalement, remplacer la $i$eme ligne par la $j$eme revient à faire~:
		\[(I - (\Lambda_{ii} + \Lambda_{jj}) + \Lambda_{ij} + \Lambda_{ji})A.\] \end{rmq}

		\begin{lem} Soit une matrice $A = [A_{ij}] \in \M nn\R$. En appliquant l'algorithme de Gauss, on obtient une suite de matrices $b_k, \dotsc, b_1$ telles que~:
		\[I = b_k\ldots b_1A.\] Alors $b_k \ldots b_1$ est l'inverse à gauche $A^{-1}$ de $A$. \end{lem}

		\begin{proof} $A^{-1} \coloneqq b_k\ldots b_1$ est effectivement l'inverse à gauche car $A^{-1}A = b_k\ldots b_1A$ qui, par hypothèse, vaut $I$. \end{proof}

		\begin{prp} Soit $a \in \M nn\R$ une matrice inversible à gauche. Si $a^{-1}$ est l'inverse à gauche de $a$, alors $a^{-1}$ est également l'inverse à droite
		(et donc l'inverse) de $a$. \end{prp}

		\begin{proof} Soit $E$ une base de $V$. Soient $a \in \M nn\R$ une matrice et $a^{-1}$ son inverse à gauche. $a$ représente une application linéaire $\gamma$.
		$\gamma$ est injective puisque son inverse sur $\gamma(\R^n)$ existe. Donc $\dim_\R\Ker(\gamma) = 0$. De plus, $\gamma : \R^n \to \R^n$. Son domaine et son
		codomaine sont donc de même dimension. Alors par la proposition~\ref{condIsom}, $\gamma$ est un isomorphisme. Pareillement pour $a^{-1}$ qui est la matrice d'une
		application linéaire $\gamma^{-1}$ qui est bijective pour les mêmes raisons. Dès lors on peut dire
		$(\gamma \circ \gamma^{-1}) = Id_{\R^n} = (\gamma^{-1} \circ \gamma)$. On peut donc écrire~:
		\[aa^{-1} = m_{E, E}(\gamma)m_{E, E}(\gamma^{-1}) = m_{E, E}(\gamma \circ \gamma^{-1}) = m_{E, E}(Id_{\R^n}) = I_n.\]
		On vient donc de montrer que si $a^{-1}$ est l'inverse à gauche de $a$, alors $a^{-1}$ est l'inverse à droite de $a$. \end{proof}

		\begin{rmq} L'existence d'un inverse à gauche (respectivement droite) n'implique pas toujours l'existence d'un inverse à droite (respectivement à gauche).
		Le cas des matrices ou celui des réels est un exemple mais pas une généralité~: soit $f : X \to Y$ une fonction injective mais pas surjective. On peut trouver
		$g : f(X) \to X$ telle que $(g \circ f)$ soit l'identité sur $X$ mais il n'existe pas d'inverse à droite pour $f$ car elle n'est pas surjective. \end{rmq}

	\subsection{Changement de base}
		\begin{déf} Soit $V$ un e.v. et $E = \{e_1, \dotsc, e_n\}, F = \{f_1, \dotsc, f_n\}$ deux bases de $V$ et soit $A \in \Hom(V, V)$ un automorphisme de $V$.
		On définit la matrice de changement de base de $E$ vers $F$ exprimée avec les coordonnées de $E$ par $b \coloneqq [B_{ij}] \in \M nn\R$, matrice associée à
		l'application linéaire $B \in \Hom(V, V)$ où $B(e_j) = \sum_{i=1}^nB_{ij}f_i$. \end{déf}

		\begin{lem} Soient $V$ un e.v. avec deux bases $E$ et $F$. Soit $b$ la matrice de changement de base de $E$ vers $F$ exprimée dans les coordonnées de $E$.
		Alors si $\lambda_1, \dotsc, \lambda_n$ représente les coordonnées de $x \in V$ dans la base $E$ et $\mu_1, \dotsc, \mu_n$ représente ses coordonnées dans la
		base $F$, on a~: \[\begin{bmatrix}\mu_1 \\\vdots \\ \mu_n\end{bmatrix} = b\begin{bmatrix}\lambda_1 \\ \vdots \\ \lambda_n\end{bmatrix}.\]\end{lem}

		\begin{proof} On sait~: \[\sum_{j=1}^n\mu_jf_j = B(x) = \sum_{i=1}^n\lambda_iB(e_i) = \sum_{i=1}^n\lambda_i\sum_{j=1}^nB_{ji}f_j
		= \sum_{j=1}^n\left(\sum_{i=1}^nB_{ji}\lambda_i\right)f_j.\] Il faut donc avoir, pour tout $j$ : \[\mu_j = \sum_{i=1}^nB_{ji}\lambda_i,\] ou encore~:
		\[\begin{bmatrix}\mu_1 \\ \vdots \\ \mu_n\end{bmatrix} = b\begin{bmatrix}\lambda_1 \\ \vdots \\ \lambda_n\end{bmatrix}.\] \end{proof}

		\begin{prp} Soit $V$ un e.v., $E, F$ deux bases de $V$ et $b$ la matrice de changement de base de $E$ vers $F$ de $V$. Soit $A \in \Hom(V, V)$ un automorphisme
		de $V$. Soient $m_{E, E}(A) = [A_{ij}]$ est la matrice e l'application $A$ dans les coordonnées de la base $E$ et $m_{F, F}(A) = [A'_{ij}]$ la matrice de $A$
		dans la base $F$. Alors~: \[b \cdot m_{F, F}(A) = m_{E, E}(A) \cdot b.\] \end{prp}

		\begin{proof} Par définition des matrices $[A_{ij}]$ et $[A'_{ij}]$, on sait que~:
		\[A(e_j) = \sum_{k=1}^nA_{kj}e_k \qquad\text{ et }\qquad A(f_j) = \sum_{k=1}^nA'_{kj}f_k.\]

		Donc, en partant de $A(f_j)$, on obtient les égalités suivantes~:
		\begin{align*}
			A(f_j) &= \sum_{k=1}^nA'_{kj}f_k = \sum_{k=1}^nA'_{kj}B(e_k) = \sum_{k=1}^nA'_{kj}\sum_{i=1}^nB_{ik}e_i = \sum_{i=1}^n\left(\sum_{k=1}^nB_{ik}A'_{kj}\right)e_i, \\
			A(f_j) &= A(B(e_j)) = A\left(\sum_{k=1}^nB_{kj}e_k\right) = \sum_{k=1}^nB_{kj}A(e_k) = \sum_{k=1}^nB_{kj}\sum_{i=1}^nA_{ik}e_i
			= \sum_{i=1}^n\left(\sum_{k=1}^nA_{ik}B_{kj}\right)e_i.
		\end{align*}

		On sait donc que pour tout $i, j$, l'égalité suivante est vérifiée~: \[\sum_{k=1}^nB_{ik}A'_{kj} = \sum_{k=1}^nA_{ik}B_{kj},\] ce qui revient à faire l'égalité
		suivante~: \[b \cdot m_{F, F}(A) = [B_{ij}][A'_{ij}] = [A_{ij}][B_{ij}] = m_{E, E}(A) \cdot b.\] \end{proof}

	\subsection{Dualité}
		\begin{déf}[Définition naïve de la dualité] Soient deux objets $X$ et $Y$ et une application $A$. Si $A(X) = Y$ et $A(Y) = X$, on dit que $X$ et $Y$ sont duals.
		\end{déf}

		\begin{rmq} La géométrie projective comporte des exemples de dualité. On sait que deux points distincts désignent une droite et que deux droites distinctes
		(non-parallèles) désignent un point. \end{rmq}

		\begin{déf} Soit $V$ un espace vectoriel réel. On définit $V^*$ l'espace dual de $V$ par $V^* \coloneqq \Hom(V, \R)$. On appelle les éléments de $V^*$
		des formes linéaires de $V$. \end{déf}

		\begin{prp}\label{dualIsoMRn} Soit $V$ un espace vectoriel de dimension finie $n$. Alors l'espace $V^*$ est isomorphe à $\R^n$. \end{prp}

		\begin{proof} Soit $E = \{e_1, \dotsc, e_n\}$ une base de $V$ et soit $F = \{1\}$ la base canonique de $\R$. Soit $f \in V^*$ une forme linéaire sur $V$.
		La forme $f$ est caractérisée par la matrice $m_{F, E}(f) = [f(e_i)_i]$. La transformation linéaire $m_{F, E} : V^* \to \M 1n\R \sim \R^n$ est donc un
		isomorphisme. \end{proof}

		\begin{cor}\label{VisoMV*} Un espace vectoriel de dimension finie est isomorphe avec son dual. \end{cor}

		\begin{proof} Soient $V$ et $V^*$ un espace vectoriel et son dual. Si $V$ est de dimension finie $n$, alors par la proposition~\ref{dualIsoMRn}, on sait que
		$V^*$ est isomorphe à $\R^n$. Par la proposition~\ref{dimDIsomorpheRd}, on sait que $V$ est isomorphe à $\R^n$. Donc par composition des isomorphismes, il
		existe un isomorphisme entre $V$ et $V^*$. \end{proof}

		\begin{déf} Soient un espace vectoriel $V$ et une base $E = \{e_1, \dotsc, e_n\}$ de $V$ ainsi que $V^*$ le dual de $V$. On définit $E^*$ la base duale de $E$ par
		$E^* = \{e_1^*, \dotsc, e_n^*\}$ où pour tout $k, j$, on définit~: \[e_k^*(e_j) \coloneqq \delta_{kj}.\] \end{déf}

		\begin{prp} Soient $V, V^*$ et $E$ un espace vectoriel et son dual avec une base de $E$. Alors la base duale $E^*$ de $E$ est une base de $V^*$. \end{prp}

		\begin{proof} $E^*$ est libre car~: \[0 = \left(\sum_{i=1}^n\lambda_ie_i^*\right)(e_j) = \sum_{i=1}^n\lambda_ie_i^*(e_j) = \sum_{i=1}^n\lambda_i\delta_{ij}
		= \lambda_j.\] Montrons alors que $E^*$ est génératrice. Soit $f \in V^*$. En effet, $f = \sum_{i=1}^nf(e_i)e_i^*$~:
		\[\left(\sum_{i=1}^nf(e_i)e_i^*\right)(e_j) = \sum_{i=1}^nf(e_i)e_i^*(e_j) = \sum_{i=1}^nf(e_i)\delta_{ij} = f(e_j).\] \end{proof}

		\begin{rmq} Par rapport au corollaire~\ref{VisoMV*}, on a un isomorphisme $\phi : V \to V^*$ tel que $\phi(e_i) \coloneqq e_i^*$. \end{rmq}

		\begin{lem} Soient $V$ un espace vectoriel de dimension finie $n$ et $V^*$ son dual. Alors pour $f \in V^*$, si $f$ n'est pas la fonction constante nulle,
		alors $\dim_\R \Ker(f) = n-1$ \end{lem}

		\begin{proof} Si $f$ n'est pas la fonction constante nulle, on sait que $\dim_\R \Imf(f) \geq 1$. De plus, $\Imf(f) \subseteq \R$. Donc
		$\dim_\R\Imf(f) \leq \dim_\R \R = 1$. Donc par la proposition~\ref{dim=dim+dim}, on sait que $\dim_\R\Ker(f) = \dim_\R V - \dim_\R\Imf(f) = n-1$. \end{proof}

		\begin{rmq} Le noyau de $f$ est donc un hyperplan passant par l'origine et si $\lambda \in \R$, $f^{-1}(\lambda)$ est un hyperplan parallèle à $\Ker(f)$. \end{rmq}

		\begin{rmq} On sait que $f^{-1}(\lambda) = \{x \in V \tq f(x) = \lambda\}$. Soit $v \in V \setminus \Ker(f)$. Tout vecteur $w \in V$ peut s'écrire comme~:
		$v = u + \mu v$ avec $u \in \Ker(f)$ et $\mu \in \R$. Dès lors $f(w) = f(u) + \mu f(v) = \mu f(v)$. On sait donc déterminer $f$ par son noyau et un vecteur dans
		$V \setminus \Ker(f)$. 

		De même, avec un espace $V$, un hyperplan $W$ et $v \in V \setminus W$, il existe une unique forme linéaire $f \in V^*$ telle que $f(W) = \{0\}$ et $f(v) = \xi$
		pour $\xi$ fixé. \end{rmq}

		\begin{déf} Soit $V$ un espace vectoriel. On définit l'espace bidual de $V$ par $V^{**} \coloneqq (V^*)^* = \Hom(V^*, \R)$. \end{déf}

		\begin{prp} Si $V$ est un espace vectoriel de dimension finie, $V^{**}$ est isomorphe à $V$. \end{prp}

		\begin{proof} On définit une fonction $\alpha : V \to V^{**}$ telle que pour $v \in V$, $\alpha(v) \in \Hom(V^*, \R)$ est définie par $\alpha(v)(f) = f(v)$.
		Il est évident que $\alpha$ est linéaire~: \[\alpha(v+\lambda w)(f) = f(v+\lambda w) = f(v)+\lambda f(w) = \alpha(v)(f) + \lambda\alpha(w)(f).\]
		De plus, si $E = \{e_1, \dotsc, e_n\}$ est une base de $V$, on définit la base biduale de $E$ par $E^{**} = \{\alpha(e_1), \dotsc, \alpha(e_n)\}$.
		On sait que $E^{**}$ est une base de $V^{**}$ car $E^{**}$ est la base duale de $E^*$. En effet~: \[\alpha(e_i)(e_j^*) = e_j^*(e_i) = \delta_{ij}.\] \end{proof}

	\subsection{Matrices transposées}
		\begin{déf} Soit $A : V \to W$ une application linéaire. On définit la transposée de $A$ par $A^T : W^* \to V^* : f \mapsto (f \circ A)$. \end{déf}

		\begin{prp} L'application transposée est linéaire. \end{prp}

		\begin{proof} Montrons la linéarité~:
		\[A^T(f+\lambda g)(v) = ((f+\lambda g) \circ A)(v) = (f \circ A)(v) + \lambda (g \circ A)(v) = A^T(f)(v) + \lambda A^T(g)(v).\]
		\end{proof}

		\begin{déf} Soient $V, W$ deux e.v. avec des bases $E, F$. On définit la matrice $A^T \coloneqq m_{E^*, F^*}(A^T)  = [A^T_{ij}]$ comme la matrice transposée
		de $A$. \end{déf}

		\begin{prp} Soit $A \in \Hom(V, W)$ une application linéaire et $A = [A_{ij}]$ sa matrice associée. Alors la matrice transposée $A^T$ est définie telle que
		pour tout $i, j$, on a $A^T_{ij} = A_{ji}$. \end{prp}

		\begin{proof} Si $A^T = m_{E^*, F^*}(A^T)$ est la matrice de l'application transposée, alors~: \[A^T(f_j^*)(e_i) = (f_j^* \circ A)(e_i) = f_j^*(A(e_i))
		= f_j\left(\sum_{k=1}^nA_{ki}f_k\right) = \sum_{k=1}^nA_{ki}f_j^*(f_k) = \sum_{k=1}^nA_{ki}\delta_{jk} = A_{ji}.\]

		Mais par définition de la matrice, on a également~: \[A^T(f_j^*)(e_i) = \left(\sum_{k=1}^nA^T_{kj}e_k^*\right)(e_i) = \sum_{k=1}^nA^T_{kj}e_k^*(e_i)
		= \sum_{k=1}^nA^T_{kj}\delta_{ki} = A^T_{ij}.\]

		Il faut donc avoir $A^T_{ij} = A_{ji}$. \end{proof}

		\begin{rmq} On peut donc noter $m_{E*, F*}(A^T) = m_{F, E}(A)^T$. \end{rmq}

		\begin{déf} Soit $a \in \M nn\R$ une matrice. Si $a=a^T$, on dit que $a$ est symétrique. Si $a=-a^T$, on dit que $a$ est antisymétrique. \end{déf}

		\begin{déf} Soit $a = [a_{ij}] \in \M st\R$ une matrice. Soit $A : V \to W : v \mapsto av$ la transformation linéaire associée à $a$. On définit le rang de
		la matrice $a$ comme le rang de l'application $A$. Donc $\rang(a) = \rang(A) = \dim_\R\Imf(A)$. \end{déf}

		\begin{prp} Soit $A : V \to W$ une transformation linéaire. Alors $\rang(A) = \rang(A^T)$. \end{prp}

		\begin{proof} Soit $E = \{e_1, \dotsc, e_n\}$ une base de $V$. Soit $A(E) = \{A(e_1), \dotsc, A(e_r)\}$ une base de l'image de $A$. Soit
		$F = \{A(e_1), \dotsc, A(e_r), f_{r+1}, \dotsc, f_s\}$ une base de $W$. Dès lors, la matrice $m_{F, E}$ correspond à la matrice identité sur
		$[A_{ij}]_{1 \leq i, j \leq r}$. De plus, la matrice est nulle sur $[A_{ij}]_{r+1 \leq i \leq s, 1 \leq j \leq r}$. Donc forcément, le range de la transposée
		doit être au moins égal au rang de $A$. On a donc $\rang(A) \leq \rang(A^T)$. Par un raisonnement similaire, on trouve
		$\rang(A^T) \leq \rang\left(\left(A^T\right)^T\right)$. Or on sait que $\left(A^T\right)^T = A$. Donc on a $\rang(A) \leq \rang(A^T) \leq \rang(A)$. Il faut
		alors $\rang(A) = \rang(A^T)$. \end{proof}

\newpage
\section{Permutations et Déterminants}
	\subsection{Définitions}
		\begin{déf} Soit $n \in \mathbb N^*$. On note l'ensemble des permutations de $\{1, 2, \dotsc, n\}$~:
		\[\mathfrak{S}_n \coloneqq \left\{f : \{1, 2, \dotsc, n\} \to \{1, 2, \dotsc, n\} \tq f \text{ est bijective}\right\}.\]

		On appelle cet ensemble le \textbf{groupe symétrique de degré $n$}. \end{déf}

		\begin{déf} Soient $n$ un naturel non nul et $\mathfrak{S}_n$, le groupe symétrique de degré $n$. Soit $\sigma \in \mathfrak{S}_n$ une permutation de
		$\{1, \dotsc, n\}$. Il existe quatre notations pour $\sigma$~:

		\begin{enumerate}
			\item le tableau double~: \[\sigma = \begin{pmatrix}1 & 2 & \ldots & n \\ \sigma(1) & \sigma(2) & \ldots & \sigma(n) \end{pmatrix}\text{~;}\]
			\item la chaine~: \[\sigma = \sigma(1)\sigma(2)\ldots\sigma(n)\text{~;}\]
			\item la tresse~: deux lignes comportant les points $\{1, 2, \dotsc, n\}$ reliés par des arêtes $(i, \sigma(i))$~;
			\item le produit de cycles~: \[\sigma = \left(\sigma(k_{1, 1})\sigma(k_{1, 2})\ldots\sigma(k_{1, r})\right)\left(\sigma(k_{2, 1})\sigma(k_{2, 2})\ldots\sigma(k_{2, s})\right)\ldots\left(\sigma(k_{t, 1})\sigma(k_{t, 2})\ldots\sigma(k_{t, n})\right).\]
		\end{enumerate}\end{déf}

		\begin{rmq} Quand le degré du groupe symétrique $\mathfrak{S}_n$ n'est pas ambigu, lors d'une notation par produit de cycles, les points fixes sont omis. De plus,
		l'ordre d'expression des cycles n'est pas important. \end{rmq}

		\begin{déf} Soit $\sigma \in \mathfrak{S}_n$ une permutation. Si $\sigma$ ne comporte qu'un seul cycle de longueur $k \geq 2$, alors $\sigma$ est dit
		\textbf{cycle de longueur $k$}. Un cycle de longueur deux est appelé \textbf{transposition}. \end{déf}

		\begin{thm} Le nombre de permutations possibles d'ordre $n$ est $n!$. \end{thm}

		\begin{proof} Soit $\Perm_n$, le groupe symétrique de degré $n$. Soit $\sigma \in \Perm_n$. Déterminons $\sigma(i)$ pour $i \in \{1, 2, \ldots n\}$~:

		\begin{itemize}
			\item il y a $n$ choix pour $\sigma(1)$ (toute valeur dans $\{1, 2, 3, \dotsc, n\}$)~;
			\item il y a $(n-1)$ choix pour $\sigma(2)$ (toute valeur dans $\{1, 2, \dotsc, n\} \setminus \{\sigma(1)\}$)~;
			\item il y a $(n-2)$ choix pour $\sigma(3)$ (toute valeur dans $\{1, 2, \dotsc, n\} \setminus \{\sigma(1), \sigma(2)\}$)~:
			\item $\ldots$~;
			\item il y a 1 choix pour $\sigma(n)$.
		\end{itemize}

		Le nombre total de choix possibles est donc $n(n-1)(n-2) \ldots 1 = n!$. \end{proof}

		\begin{rmq} Les permutations sont des fonctions, on peut donc les composer. La composition $\sigma \circ \tau$ se note également $\sigma\tau$. La composition $k$
		fois de la permutation $\sigma$ avec elle-même se note $\sigma^k$. \end{rmq}

		\begin{déf} Soit $\sigma \in \Perm_n$. On appelle le \textbf{degré} de $\sigma$ le plus petit nombre naturel non nul $k$ tel que $\sigma^k = \Id_{\Perm_n}$.
		\end{déf}

		\begin{déf} Soit $\sigma \in \Perm_n$. Comme $\sigma$ est une bijection, elle admet un inverse noté $\sigma^{-1}$ tel que
		$\sigma \circ \sigma^{-1} = \Id_{\Perm_n} = \sigma^{-1} \circ \sigma$. \end{déf}

		\begin{lem}\label{TranspositionEstSonInverse} Soit $\tau \in \Perm_n$ une transposition. L'inverse $\tau^{-1}$ de $\tau$ est $\tau$. \end{lem}

		\begin{proof} Soit $\tau = (i, j) \in \Perm_n$ où $1 \leq i < j \leq n$, montrons que $\tau^{-1} = \tau$. En effet, $\tau^2 = \tau\tau = \tau \circ \tau$
		revient à permuter deux fois les positions $i$ et $j$ de la permutation, et donc de remettre $i$ et $j$ à leur place initiale~:
		\[(\tau\tau)(n) = \tau\left(\left\{\begin{aligned}j &\text{ si } n = i\\i &\text{ si } n = j\\n &\text{ sinon}\end{aligned}\right.\right) =
		\left\{\begin{aligned}i &\text { si } n = i\\j &\text{ si } n = j\\n &\text{ sinon}\end{aligned}\right. = \Id_{\Perm_n}\] \end{proof}

		\begin{lem}\label{InverseProduitPermutations} Soient $\tau_1, \tau_2, \ldots \tau_n$, $n$ transpositions et $\sigma \in \Perm_n$ une permutation telle que
		$\sigma = \tau_1\tau_2\ldots\tau_n$. Alors $\sigma^{-1} = \tau_n\tau_{n-1}\ldots\tau_1$. \end{lem}

		\begin{proof} Soient $\sigma \in \Perm_n$ et $\tau_1, \dotsc, \tau_n$ des transpositions telles que $\sigma = \tau_1\ldots\tau_n$.
		Montrons que $\sigma\sigma^{-1} = \sigma^{-1}\sigma = \Id_{\Perm_n}$.

		On sait par le lemme~\ref{TranspositionEstSonInverse} que $\tau_i\tau_i = \Id_{\Perm_n}$ pour tout $i$, et on calcule~:
		\[\begin{aligned}
			\tau_1\tau_2\ldots\tau_{n-1}\tau_n\tau_n\tau_{n-1}\ldots\tau_2\tau_1 &= \tau_1\tau_2\ldots\tau_{n-1}(\tau_n\tau_n)\tau_{n-1}\ldots\tau_2\tau_1
		= \tau_1\tau_2\ldots\tau_{n-1}\Id_{\Perm_n}\tau_{n-1}\ldots\tau_2\tau_1 \\
			&= \tau_1\tau_2\ldots(\tau_{n-1}\tau_{n-1})\ldots\tau2\tau_1 = \tau_1\tau_2\ldots\Id_{\Perm_n}\ldots\tau_2\tau_1 = \ldots = \tau_1\tau_2\tau_2\tau_1 = \tau_1\Id_{\Perm_n}\tau_1 \\
			&= \Id_{\Perm_n}.
		\end{aligned}\]

		Par un raisonnement similaire, on montre que~:
		\[\prod_{j=1}^n\tau_{n+1-j}\prod_{j=1}^n\tau_j = \Id_{\Perm_n}.\]

		Et puisque $\sigma = \tau_1\ldots\tau_n$, on sait que $\tau_n\ldots\tau_1$ est son inverse, $\sigma^{-1}$. \end{proof}

		\begin{prp}\label{PermPuissanceId} Soit $\sigma \in \Perm_n$ un cycle de longueur $k$. Alors $\forall m \in \N^* : \sigma^{mk} = \Id_{\Perm_n}$. \end{prp}

		\begin{proof} Montrons que $\sigma^k = \Id_{\Perm_n}$. Par définition, on sait qu'il existe $k$ nombres deux à deux distincts
		$n_1, \dotsc, n_k \in \{1, \dotsc, n\}$ tels que $\forall i \in \{1, \dotsc, n\} : n_i = \sigma^{i-1}(n_1)$ et $n-k$ nombres restants $n_{k+i}$
		(avec $i \in \{1, \dotsc, n-k\}$) tels que $\sigma(n_{k+i}) = n_{k+i}$. Dès lors, pour tout $j \in \{1, \dotsc, k\}$, on a
		$\sigma^k(n_j) = \sigma^k\left(\sigma^{j-1}(n_1)\right) = \sigma^{k+j-1}(n_1) = \sigma^{j-1}\left(\sigma\left(\sigma^{k-1}(n_1)\right)\right)
		= \sigma^{j-1}\left(\sigma(n_k)\right) = \sigma^{j-1}(n_1) \eqqcolon n_j$. On a donc bien $\sigma^k = \Id_{\Perm_n}$.

		Maintenant montrons que $\forall m \geq 1 : \sigma^{mk} = \sigma^{(m-1)k}$. En effet,
		$\sigma^{mk}(n) = \sigma^{(m-1)k}\left(\sigma^k(n)\right) = \sigma^{(m-1)k}(n)$.

		Dès lors, on sait que $\sigma^{mk}(n) = \sigma^{(m-1)k}(n) = \sigma^{(m-2)k}(n) = \ldots = \sigma{2k}(n) = \sigma^k(n) = n$. \end{proof}

		\begin{rmq} Le lemme~\ref{TranspositionEstSonInverse} est un cas particulier de cette proposition pour un cycle de longueur $k = 2$. \end{rmq}

		\begin{cor} Soit $\sigma \in \Perm_n$ une permutation admettant $n$ cycles de longueur respective $k_1, k_2, \dotsc, k_n$. Le degré de $\sigma$ est le plus
		petit commun multiple des longueurs des cycles. Donc $\deg \sigma = LCM(k_1, k_2, \dotsc, k_n)$. \end{cor}

		\begin{proof} On sait par la proposition~\ref{PermPuissanceId} qu'un cycle mis à la puissance de sa longueur donne l'identité. Soit
		$K \coloneqq LCM(k_1, \dotsc, k_n)$, le plus petit commun multiple des longueurs des cycles de la permutation $\sigma$. Par définition, $K$ est un multiple de
		tous les nombres $k_i$ pour $i \in \{1, \dotsc, n\}$, donc pour chaque cycle individuel, la puissance $K$ donne l'identité. On a donc
		$\sigma^{K} = \Id_{\Perm_n}$ et on sait par définition que $K$ est le plus petit nombre naturel $\geq 1$ satisfaisant cette propriété. $K$ est donc le degré de
		la permutation $\sigma$. \end{proof}

		\begin{déf} Soient $\sigma, \tau \in \Perm_n$ deux permutations. La \textbf{permutation conjuguée} de $\sigma$ par $\tau^{-1}$ est la bijection donnée par~:
		\[\tau\sigma\tau^{-1} \coloneqq \tau \circ \sigma \circ \tau{-1}.\] \end{déf}

		\begin{thm} La permutation conjuguée conserve les cycles telle que si $\sigma = (a_1, a_2, \dotsc, a_n)\ldots(b_1, b_2, \dotsc, b_m)$, alors~:
		\[\tau\sigma\tau^{-1} = (\tau(a_1), \tau(a_2), \dotsc, \tau(a_n))\ldots(\tau(b_1), \tau(b_2), \dotsc, \tau(b_m)).\] \end{thm}

		\begin{proof} Montrons que $(\tau(a_1), \dotsc, \tau(a_m))$ est une permutation de $\gamma \coloneqq \tau\sigma\tau^{-1}$. Soit $i \in \{1, \dotsc, m\}$.
		On sait $\gamma(\tau(a_i)) = (\gamma\tau)(a_i) = (\tau\sigma\tau^{-1}\tau)(a_i) = (\tau\sigma)(a_i) = \tau(\sigma(a_i))$. Si $i < m$, alors
		$\gamma(\tau(a_i)) = \tau(a_{i+1})$, et si $i = m$, alors $\gamma(\tau(a_i)) = \tau(\sigma(a_m)) = \tau(a_1)$. En appliquant le même raisonnement à tous les
		cycles de $\sigma$, on montre que les cycles $c$ sont conservés et qu'ils sont donnés par $\tau(c)$. \end{proof}

	\subsection{Signe d'une permutation}
		\begin{déf} Soit $\sigma \in \Perm_n$, une permutation. Une inversion dans $\sigma$ est un couple $(i, j) \in \{1, 2, \dotsc, n\}^2$ tel que $i < j$ et
		$\sigma(i) > \sigma(j)$. \end{déf}

		\begin{déf} Soit $\sigma \in \Perm_n$ une permutation. On définit $\sign(\sigma)$ le signe de $\sigma$ par~: \[\sign(\sigma) = (-1)^{N(\sigma)},\] où
		$N(\sigma)$ est le plus petit nombre d'inversions possible pour transformer $\Id_{\Perm_n}$ en $\sigma$. Si $\sign(\sigma) = -1$, on dit que $\sigma$ est
		impaire, sinon $\sigma$ est paire. \end{déf}

		\begin{prp} Soit $\sigma \in \Perm_n$ une permutation. Supposons qu'il existe $m+n$ transpositions $\tau_1, \dotsc, \tau_m, \tau'_1, \dotsc, \tau'_n$ telles que~:
		\[\sigma = \prod_{i=1}^m\tau_i = \prod_{j=1}^n\tau'_j.\] Alors $m$ et $n$ sont de même parité ($m \mod 2 = n \mod 2 = p \in \{0, 1\}$). \end{prp}

		\begin{proof} Toute inversion peut s'écrire comme un produit (une composition) de transpositions adjacentes~: soient $1 \leq i < j \leq n$. L'inversion $(i, j)$
		peut s'écrire comme suit~: \[\tau = (j, j-1)(j-1, j-2)\ldots(i+1, i+2)(i, i+1)(i+1, i+2)\ldots(j-2, j-1)(j-1, j).\]
		Le nombre de transpositions adjacentes est donc $2(j-i-2) + 1$ qui est impair. Dès lors, prenons $\hat{\tau_1}, \dotsc, \hat{\tau_{m'}}$ l'ensemble des
		transpositions adjacentes permettant de réécrire $\tau_1\ldots\tau_m$ et $\hat{\tau'_1}, \ldots \hat{\tau_{n'}}$ l'ensemble des transpositions adjacentes
		permettant de réécrire $\tau'_1\ldots\tau'_n$. On a donc les égalités suivantes~:
		\[\begin{aligned}
			\sigma &= \tau_1\ldots\tau_m &= \tau'_1\ldots\tau'_n \\
			       &= \hat{\tau_1}\ldots\hat{\tau_{m'}} &= \hat{\tau'_1}\ldots\hat{\tau'_{n'}}.
		\end{aligned}\]

		Soient $\nu_1, \dotsc, \nu_m$ et $\nu'_1, \dotsc, \nu'_n$ le nombre de transpositions adjacentes de $\hat{\tau_i}$ et $\hat{\tau'_i}$. On sait que
		$\sum_i\nu_i = m'$ et $\sum_i\nu'_i = n'$. Dès lors, on sait calculer~:
		\[m'-m = \sum_{i=1}^m\nu_i - m = \sum_{i=1}^m(\nu_i-1).\]
		De plus, étant donné que $\nu_i$ est impair pour tout $i$, la valeur $(\nu_i-1)$ est paire. La quantité $m'-m$ est donc une somme de valeurs paires et est
		donc paire également. Par un raisonnement similaire, on obtient $n'-n$ pair. Par le lemme~\ref{InverseProduitPermutations}, on sait que
		$\sigma^{-1} = \hat{\tau_m}\ldots\hat{\tau_1}$. En composant $\sigma$ et son inverse $\sigma^{-1}$, on obtient l'identité $\Id_{\Perm_n}$.

		Soit $N(\sigma)$ le nombre d'inversions de $\sigma$. En évaluant
		$N(\sigma) - m' \mod 2 \equiv N(\sigma\hat{\tau_k}\ldots\hat{\tau_1}) \mod 2 = N(\Id_{\Perm_n}) \mod 2 = 0 \mod 2 = 0$,
		on trouve que $N(\sigma) - m'$ est pair et donc $N(\sigma) - m' + (m' - m) = N(\sigma) - m$ est pair également car une somme de nombres pairs est paire.
		De manière similaire, on trouve $N(\sigma) - n$ pair également. La quantité $\left(N(\sigma) - n\right) - \left(N(\sigma) - m\right) = m - n$ est une
		différence de valeurs paires et est donc paire également. Or $m$ et $n$ sont de même parité si et seulement si $m - n$ est pair. \end{proof}

		\begin{prp}\label{PropsSign} Soient $\sigma, \tau \in \Perm_n$. L'opérateur $\sign$ respecte les propriétés suivantes~:

		\begin{enumerate}
			\item $\sign(\sigma\tau) = \sign(\sigma)\sign(\tau)$~;
			\item $\sign(\sigma^{-1}) = \sign(\sigma)$.
		\end{enumerate}
		\end{prp}

		\begin{proof} Soient $\sigma, \tau \in \Perm_n$.

		\begin{enumerate}
			\item Supposons $\sigma = \tau_1\ldots\tau_n$ et $\tau = \tau'_1\ldots\tau'_m$. On sait
			      $\sign(\sigma\tau) = (-1)^{m+n} = (-1)^m(-1)^n = \sign(\tau)\sign(\sigma)$.
			\item Supposons $\sigma = \tau_1\ldots\tau_n$. On sait $\sigma^{-1} = \tau_n\ldots\tau_1$. $\sigma$ et son inverse $\sigma^{-1}$ ont donc la même parité,
			      et alors le même signe.
		\end{enumerate}
		\end{proof}

		\begin{rmq} Si $\sigma$ est une permutation, alors $\sign(\sigma) = 1 \iff \sigma$ est paire. \end{rmq}

		\begin{lem}\label{DécompositionCycleEnTranspositions} Soit $\sigma \in \Perm_n$ un cycle de longueur $k$. Si $\sigma = (n_1, n_2, \dotsc, n_k)$ et
		$\tau_i = (n_i, n_{i+1})$ pour $1 \leq i < k$, alors $\sigma = \tau_1\tau_2\ldots\tau_{k-1}$. \end{lem}

		\begin{proof} Soient $k-1$ transpositions $\tau_i = (n_i, n_{i+1})$. La composition s'évaluant de droite à gauche, on a:
		\[\left(\tau_1\ldots\tau_{k-1}\right)(n_i) = \left(\tau_1\ldots\tau_{i-1}\right)\left(\tau_i\left[\left(\tau_{i+1}\ldots\tau_{k-1}\right)(n_i)\right]\right)
		= \left(\tau1\ldots\tau_{i-1}\right)\left(\tau_i[n_i]\right) = \left(\tau_1\ldots\tau_{i-1}\right)(n_{i+1}) = n_{i+1}.\]

		En prenant $m \not \in \{n_1, \dotsc, n_{k-1}\}$, alors aucune des $k-1$ transpositions ne modifie $m$, donc $\left(\tau_1\ldots\tau_{k-1}\right)$ agit comme
		l'identité $\Id_{\Perm_n}$. On a donc bien $\sigma = \tau_1\ldots\tau_{k-1}$. \end{proof}

		\begin{cor} Soit $\sigma \in \Perm_n$ un cycle de longueur $k$. Alors $\sigma$ est pair si et seulement si $k$ est impair. \end{cor}

		\begin{proof} On sait par le lemme~\ref{DécompositionCycleEnTranspositions} qu'il existe $k-1$ transpositions $\tau_i$ telles que
		$\sigma = \tau_1\ldots\tau_{k-1}$. Dès lors, $\sign(\sigma) = (-1)^{k-1}$. Si $k$ est pair, alors $\sign(\sigma) = -1$ et donc $\sigma$ est impair, et si $k$
		est impair, alors $\sign(\sigma) = 1$ et donc $\sigma$ est pair. \end{proof}

	\subsection{Déterminants}
		\begin{déf} Soient $x, y, z \in \R^2$ des points du plan, $\lambda \in \R$. On définit \textbf{l'aire orientée} $A : \R^2 \times \R^2 \to \R$ par les propriétés
		suivantes~:

		\begin{enumerate}
			\item $\Aor(x, y+z) = \Aor(x, y) + \Aor(x, z)$~;
			\item $\Aor(x+y, z) = \Aor(x, z) + \Aor(y, z)$~;
			\item $\Aor(\lambda x, y) = \lambda\Aor(x, y) = \Aor(x, \lambda y)$~;
			\item $\Aor(x, x) = 0$.
		\end{enumerate}

		Si $\{e_1, e_2\}$ est une base de $\R^2$, alors on définit l'unité d'aire par $A(e_1, e_2) = 1$. \end{déf}

		\begin{rmq} L'aire est dite \textit{orientée} car $A(x, y) = -A(y, x)$. En effet~:
		\[0 = \Aor(x+y, x+y) = \Aor(x, x) + \Aor(x, y) + \Aor(y, x) + \Aor(y, y) = \Aor(x, y) + \Aor(y, x).\] \end{rmq}

		\begin{déf} Soit $V$ un espace réel. Soit $f : V^n \to \R$ une fonction à valeurs réelles. On dit que $f$ est une
		\textbf{forme multilinéaire} si $f$ est linéaire pour toute composante $v_i$ de $(v_1, \dotsc, v_n) \in V^n$. Si lorsque $a_i = a_j$ pour $i \neq j$,
		alors on dit que $f$ est une \textbf{forme multilinéaire alternée}. \end{déf}

		\begin{prp}\label{ChangementDeSigneSwapDeParams} Soient $V$ un espace réel, $f : V^n \to \R$ une forme multilinéaire variée et $1 \leq i < j \leq n$. Alors
		$f(v_1, \dotsc, v_i, \dotsc, v_j, \dotsc, n_n) = -f(v_1, \dotsc, v_j, \dotsc, v_i, \dotsc, v_n)$. \end{prp}

		\begin{proof} Par définition dune forme multilinéaire variée, si deux paramètres $v_i$ et $v_j$ sont égaux pour $i \neq j$, alors
		la fonction s'annule en ce point. Dès lors~:
		\[\begin{aligned}
			0 = f(v_1, \dotsc, v_i+v_j, \dotsc, v_i+v_j, \dotsc, v_n) &= f(v_1, \dotsc, v_i, \dotsc, v_j, \dotsc, v_n) + f(v_1, \dotsc, v_i, \dotsc, v_i, \dotsc, v_n) \\
			                                                          &+ f(v_1, \dotsc, v_j, \dotsc, v_j, \dotsc, v_n) + f(v_1, \dotsc, v_j, \dotsc, v_i, \dotsc, v_n) \\
																	  &= f(v_1, \dotsc, v_i, \dotsc, v_j, \dotsc, v_n) + 0 + 0 + f(v_1, \dotsc, v_j, \dotsc, v_i, \dotsc, v_n).
		\end{aligned}\] \end{proof}

		\begin{prp}\label{f(0,)=0} Soient $V$ un espace vectoriel réel de dimension $n$ et $f : V^n \to \R$ une forme multilinéaire alternée. Soient
		$v_1, \dotsc, v_n \in V$ tels qu'il existe $1 \leq i \leq n$ tel que $v_i = 0$. Alors \[f(v_1, \dotsc, v_n) = 0.\] \end{prp}

		\begin{proof} Soit un tel $i$ et soit $j \neq i$. On pose $v_i \coloneqq v_j - v_j$, ce qui est admissible car $v_i = 0$. Dès lors, par linéarité, on a~:
		\[\begin{aligned}
			f(v_1, \dotsc, v_i, \dotsc, v_j, \dotsc, v_n) &= f(v_1, \dotsc, v_j-v_j, \dotsc, v_j, \dotsc, v_n) \\
				&= f(v_1, \dotsc, v_j, \dotsc, v_j, \dotsc, v_n) - f(v_1, \dotsc, v_j, \dotsc, v_j, \dotsc, v_n) = 0.
		\end{aligned}\]
		\end{proof}

		\begin{prp} Soient $V$ un espace vectoriel réel de dimension $n$ et $E = \{e_1, \dotsc, e_n\}$ une base de $V$. Alors il existe une et une seule forme
		multilinéaire alternée $f : V^n \to \R$ telle que~:
		\[f(e_1, \dotsc, e_n) = 1.\] \end{prp}

		\begin{proof} Supposons qu'une telle fonction existe, et montrons que~:
		\[f(v_1, \ldots v_n) = \sum_{\sigma \in \Perm_n}\left(\prod_{j=1}^na_{j\,\sigma(j)}\right)f(e_{1\,\sigma(1)}, \dotsc, e_{n\,\sigma(n)}).\]

		En effet, en posant pour tout $1 \leq i \leq n$~:
		\[v_i = \sum_{j=1}^nv_{i\,j}e_j,\]
		on peut déterminer, par (multi)linéarité de $f$~:
		\[\begin{aligned}
			f(v_1, \dotsc, v_n) &= f\left(\sum_{j=1}^nv_{1\,j}e_j, \dotsc, \sum_{j=1}^nv_{n\,j}e_j\right) = \sum_{j=1}^nv_{1\,j}f\left(e_i, \dotsc, \sum_{k=1}^nv_{n\,k}e_k\right) \\
			                    &= \sum_{k_1}^n\sum_{k_2}^n\ldots\sum_{k_n}^nv_{1\,k_1}v_{2\,k_2}\ldots v_{n\,k_n}f\left(e_{k_1}, \dotsc, e_{k_n}\right).
		\end{aligned}\]

		Or, si deux indices $k_\alpha$ sont égaux, la fonction s'annule. Dès lors, $k_2$ doit être différent de $k_1$, $k_3$ doit être différent de $k_1$ et de $k_2$,
		etc. Dès lors, l'ensemble des $k_i$ forme l'ensemble $\{1, 2, \dotsc, n\}$ et alors $i \mapsto k_i$ est une bijection. Comme une bijection d'un ensemble discret
		est une permutation, on peut sommer sur l'ensemble des permutations~:
		\[f(v_1, \dotsc, v_n) = \sum_{\sigma \in \Perm_n}v_{1\,\sigma(1)}\ldots v_{n\,\sigma(n)}f(e_{\sigma(1)}, \dotsc, e_{\sigma(n)}).\]

		Par la proposition~\ref{ChangementDeSigneSwapDeParams}, on sait que $f(e_{\sigma(1)}, \dotsc, e_{\sigma(n)}) = \sign(\sigma)f(e_1, \dotsc, e_n)$.
		Dès lors, on a bien~:
		\[f(v_1, \dotsc, v_n) = \sum_{\sigma \in \Perm_n}\sign(\sigma)\left(\prod_{i=1}^nv_{i\,\sigma(i)}\right)f(e_1, \dotsc, e_n).\]

		De plus, en imposant $f(e_1, \dotsc, e_n) = 1$, on peut poser~:
		\[\det(v_1, \dotsc, v_n) \coloneqq f(v_1, \dotsc, v_n) = \sum_{\sigma \in \Perm_n}\sign(\sigma)\left(\prod_{i=1}^nv_{i\,\sigma(i)}\right).\]

		Maintenant, montrons que $\det$ est une forme multilinéaire alternée. Soient $\lambda \in \R, a_1, \dotsc, a_{i-1}, a_i^{(1)}, a_i^{(2)}, a_{i+1}, \dotsc, a_n \in V$.
		\[\begin{aligned}
			\det&(a_1, \dotsc, a_{i-1}, a_i^{(1)} + \lambda a_i^{(2)}, a_{i+1}, \dotsc, a_n) \coloneqq
				\sum_{\sigma \in \Perm_n}\sign(\sigma)\left(\prod_{k=1, k \neq i}^na_{k\,\sigma(k)}\right)(a_{i\,\sigma(i)}^{(1)} + a_{i\,\sigma(i)}^{(2)}) \\
			&= \sum_{\sigma \in \Perm_n}\left[\sign(\sigma)\left(\prod_{k=1, k \neq i}^na_{k\,\sigma(k)}\right)a_i^{(1)} + \sign(\sigma)\left(\prod_{k=1, k \neq i}^na_{k\,\sigma(k)}\right)\lambda a_i^{(2)}\right].
		\end{aligned}\]

		En posant~:
		\[\begin{aligned}
			&a_i \coloneqq a_i^{(1)} + \lambda a_i^{(2)}, \\
			&b_\alpha \coloneqq \begin{dcases}a_\alpha \; &\text{ si } \alpha \neq i\\a_i^{(1)} \; &\text{ si } \alpha = i\end{dcases}, \\
			&c_\beta  \coloneqq \begin{dcases}a_\beta  \; &\text{ si } \beta  \neq i\\a_i^{(2)} \; &\text{ si } \beta  = i\end{dcases},
		\end{aligned}\]

		on obtient~:
		\[\begin{aligned}
			\det(a_1, \dotsc, a_n) &= \sum_{\sigma\in\Perm_n}\sign(\sigma)\prod_{k=1}^nb_{k\,\sigma(k)} + \lambda\sum_{\sigma\in\Perm_n}\sign(\sigma)\prod_{k=1}^nc_{k\,\sigma(k)} \\
			                       &= \det(a_1, \dotsc, a_{i-1}, a_i^{(1)}, a_{i+1}, \dotsc, a_n) + \lambda\det(a_1, \dotsc, a_{i-1}, a_i^{(2)}, a_{i+1}, \dotsc, a_n).
		\end{aligned}\]

		$\det$ est donc bien une forme multilinéaire car linéaire en chacun de ses paramètres. Il reste à montrer que $\det$ est une forme multilinéaire \textit{alternée}.
		Pour ce faire, supposons qu'il existe $1 \leq i < j \leq n$ tels que $a_i = a_j$. Dès lors, on sait~:
		\[\det(a_1, \dotsc, a_i, \dotsc, a_j, \dotsc, a_n) = \det(a_1, \dotsc, a_j, \dotsc, a_i, \dotsc, a_n).\]
		De plus, $\forall t \in \{1, 2, \dotsc, n\} : a_{i\,t} = a_{j\,t}$. Soit $\tau_{ij}$ la transposition $(i, j)$. Intervertir les coordonnées $i$ et $j$ n'a pas
		d'influence sur le résultat car $a_{\tau_{ij}(j)} = a_i = a_j = a_{\tau_{ij}(i)}$. Dès lors~:
		\[\begin{aligned}
			\det(a_1, \dotsc, a_i, \dotsc, a_j, \dotsc, a_n) &= \sum_{\sigma \in \Perm_n}\sign(\sigma\tau_{ij})\left(\prod_{k=1}^na_{k\,(\sigma\tau_{ij})(k)}\right) \\
				&= \sum_{\sigma\in\Perm_n}\sign(\sigma\tau_{ij})\left(\prod_{k=1, k \not \in \{i, j\}}^na_{k\,(\sigma\tau_{ij})(k)}\right)a_{i\,(\sigma\tau_{ij})(i)}a_{j\,(\sigma\tau_{ij})(j)} \\
				&= \sum_{\sigma\in\Perm_n}-\sign(\sigma)\left(\prod_{k=1, k\not\in\{i, j\}}^na_{k\,\sigma(k)}\right)a_{i\,\sigma(j)}a_{j\,\sigma(i)} \\
				&= -\sum_{\sigma\in\Perm_n}\sign(\sigma)\left(\prod_{k=1, k\not\in\{i, j\}}^na_{k\,\sigma(k)}\right)a_{j\,\sigma(j)}a_{i\,\sigma(j)} \\
				&= -\sum_{\sigma\in\Perm_n}\sign(\sigma)\left(\prod_{k=1}^na_{k\,\sigma(k)}\right) \\
				&= -\det(a_1, \dotsc, a_i, \dotsc, a_j, \dotsc, a_n).
		\end{aligned}\]

		Dès lors, on a $\det(a_1, \dotsc, a_i, \dotsc, a_j, \dotsc, a_n) = -\det(a_1, \dotsc, a_i, \dotsc, a_j, \dotsc, a_n)$, ce qui implique~:
		\[\det(a_1, \dotsc, a_i, \dotsc, a_j, \dotsc, a_n) = 0.\] \end{proof}

	\subsection{Matrices et applications linéaires}
		\begin{déf} Soit $a \in \M nn\R$ une matrice carrée à valeurs réelles. On appelle \textbf{le déterminant de la matrice $a$} la valeur~:
		\[\det(a) = \sum_{\sigma\in\Perm_n}\sign(\sigma)\prod_{k=1}^na_{k\,\sigma(k)}.\] \end{déf}

		\begin{prp}\label{detTransposéeEstdetMatrice} Soit $a \in \M nn\R$ une matrice. Alors $\det(a) = \det(a^T)$. \end{prp}

		\begin{proof} Soit $a = [a_{ij}] \in \M nn\R$. On sait que $a^T = [a_{ji}] \in \M nn\R$. Dès lors, le déterminant de la matrice transposée vaut~:
		\[\begin{aligned}
			\det(a^T) &= \sum_{\sigma\in\Perm_n}\sign(\sigma)\prod_{k=1}^n(a^T)_{k\,\sigma(k)} = \sum_{\sigma\in\Perm_n}\sign(\sigma)\prod_{k=1}^na_{\sigma(k)\,k} \\
			          &= \sum_{\sigma\in\Perm_n}\sign(\sigma)\prod_{k=1}^na_{(\sigma^{-1}\sigma)(k)\,\sigma^{-1}(k)}.
		\end{aligned}\]

		On peut permuter tous les indices car $\sigma^{-1}$ est également une permutation et donc une bijection. Les couples d'indices sont donc conservés. Dès lors~:
		\[\det(a^T) = \sum_{\sigma\in\Perm_n}\sign(\sigma)\prod_{k=1}^na_{k\,\sigma^{-1}(k)}.\]

		Or, par la proposition~\ref{PropsSign}, on sait que $\sign(\sigma) = \sign(\sigma^{-1})$. Dès lors~:
		\[\det(a^T) = \sum_{\sigma\in\Perm_n}\sign(\sigma^{-1})\prod_{k=1}^na_{k\,\sigma^{-1}}(k).\]

		De plus, $\Perm_n$ est un groupe, donc toute permutations $\sigma \in \Perm_n$ admet une permutation inverse $\sigma^{-1} \in \Perm_n$.
		On peut donc conclure par~:
		\[\det(a^T) = \sum_{\sigma\in\Perm_n}\sign(\sigma)\prod_{k=1}^na_{k\,\sigma(k)} = \det(a).\] \end{proof}

		\begin{rmq} L'application $\det$ étant une forme multilinéaire, l'inversion de deux lignes (ou colonnes) d'une matrice change le signe de son déterminant
		(par la proposition~\ref{ChangementDeSigneSwapDeParams}).

		De plus, on sait que si une matrice a deux lignes (ou colonnes) égales, alors le déterminant s'annule car c'est une forme multilinéaire \textbf{variée}.
		\end{rmq}

		\begin{prp}\label{detMatriceTriangulaire} Soit $a \in \M nn\R$ une matrice triangulaire. Alors~:
		\[\det(a) = \prod_{i=1}^na_{ii}.\] \end{prp}

		\begin{proof} Supposons que $a \in \M nn\R$ soit triangulaire. On sait donc que si $i < j$, alors $a_{ij} = 0$. Le déterminant est défini par la somme des produits
		des \textit{diagonales} $a_{i\,\sigma(i)}$. Or, si $\sigma(i) > i$, on sait que $a_{i\,\sigma(i)} = 0$. Dès lors, la seule permutation telle que
		$\forall i : i \geq \sigma(i)$ est $\Id_{\Perm_n}$. Cela peut se construire comme suit~: il faut que $1 \leq \sigma(1)$, donc on impose $\sigma(1) = 1$. Il faut
		également $\sigma(2) \leq 2$, ce qui impose que $\sigma(2) \in \{1, 2\}$, or $\sigma$ doit être une bijection, et $1$ est déjà l'image de $\sigma(1)$. En
		réitérant jusque $n$, on construit la permutation identité qui est la seule à respecter cette propriété. On a donc~:
		\[\begin{aligned}
			\det(a) &= \sum_{\sigma\in\Perm_n}\sign(\sigma)\prod_{k=1}^na_{k\,\sigma(k)} \\
			&= \sign(\Id_{\Perm_n})\prod_{k=1}^na_{k\,\Id_{\Perm_n}(k)} + \sum_{\sigma \in \Perm_n \setminus \{\Id_{\Perm_n}\}}\sign(\sigma)\prod_{k=1}^na_{k\,\sigma(k)} \\
			&= \sign(\Id_{\Perm_n})\prod_{k=1}^na_{k\,\Id_{\Perm_n}(k)} + 0 = \sign(\Id_{\Perm_n})\prod_{k=1}^na_{k\,\Id_{\Perm_n}(k)} = 1 \times \prod_{k=1}^na_{kk}.
		\end{aligned}\] \end{proof}

		\begin{rmq} Afin de calculer le déterminant de la matrice, on peut utiliser la méthode de Gauss pour trianguler la matrice. De là, il ne rester plus qu'à
		évaluer le produit de la diagonale~; les opérations élémentaires de Gauss ne modifiant pas le déterminant. \end{rmq}

		\begin{prp}\label{Binet} Soient $a, b \in \M nn\R$ deux matrices carrées. Alors~:
		\[\det(ab)= \det(a)\det(b).\] \end{prp}

		\begin{proof} Soient $a, b \in \M nn\R$ et $A, B \in \Hom(V, V)$ les fonctions linéaires associées aux matrices. On sait que~:
		\[\det(B(A(e_1)), \dotsc, B(A(e_n))) = \det(a)\det(B(e_1), \dotsc, B(e_n)) = \det(a)\det(b)\det(e_1, \dotsc, e_n) = \det(a)\det(b).\]

		Cependant, on sait également que~:
		\[\det(B(A(e_1)), \dotsc, B(A(e_n))) = \det((B \circ A)(e_1), \dotsc, (B \circ A)(e_n)) = \det(ba)\det(e_1, \dotsc, e_n) = \det(ba).\]

		Dès lors, on a bien~: \[\det(ab) = \det(a)\det(b),\] ce qui veut dire que le déterminant d'un produit est égal au produit des déterminants. \end{proof}

		\begin{prp} Soit $b \in \M nn\R$, une matrice carrée inversible. Alors $\det(b^{-1}) = \det(b)^{-1}$. \end{prp}

		\begin{proof} On sait que l'identité est une matrice triangulaire. Dès lors, par la proposition~\ref{detMatriceTriangulaire}, on sait que son déterminant
		est égal au produit des éléments de sa diagonale. Donc $\det(\Id_{\M nn\R}) = 1$.

		Cependant, par hypothèse, $b$ est inversible. Dès lors, $\Id_{\M nn \R} = b^{-1}b = bb^{-1}$. Et par la proposition~\ref{Binet}, on sait que
		$\det(bb^{-1}) = \det(b)\det(b^{-1})$. On a alors~:
		\[1 = \det(\Id_{\M nn\R}) = \det(b^{-1}b) = \det(b^{-1})\det(b).\]
		D'où on déduit $\det(b^{-1}) = \det(b)^{-1}$. \end{proof}

		\begin{rmq} En ayant supposé $b$ inversible, on a obtenu $\det(b) \neq 0$. Dès lors, par contraposée, on a également $\det(b) = 0 \Rightarrow b$ pas inversible.
		\end{rmq}

		\begin{cor} Soient $a, b \in \M nn\R$ deux matrices carrées inversibles. $\det(bab^{-1}) = \det(a)$. \end{cor}

		\begin{proof} On sait que $\det(bab^{-1}) = \det(b)\det(a)\det(b^{-1})$, et donc on a~: \[\det(bab^{-1}) = \det(b)\det(a)\det(b)^{-1} = \det(a).\] \end{proof}

		\begin{rmq} Ce corollaire veut dire que si $V$ est un espace vectoriel de dimension $n$, et $A \in \Hom(V, V)$, alors le déterminant de la matrice associée à
		$A$ ne dépend pas de la base choisie.

		Soient $E$ et $F$, deux bases de $V$. Si $m_{F, E}$ est la matrice de changement de base de $E$ vers $F$, alors~:
		\[\det(m_{E, E}(A)) = \det(m_{E, F}(A)m_{F, F}(A)m_{F, E}(A)) = \det(m_{F, F}(A)).\] \end{rmq}

		\begin{déf} On peut définir le \textbf{déterminant de l'opérateur linéaire} $A \in \Hom(V, V)$ par~: \[\det(A) \coloneqq \det(m_{E, E}(A))\] où $E$ est
		une base quelconque de $V$. La valeur est bien définie car le déterminant ne dépend pas de la base. \end{déf}

		\begin{thm}\label{inversiblessidet=0} Soient $V$ un espace vectoriel réel de dimension $n$ et $A \in \Hom(V, V)$ un opérateur linéaire. Alors $\det(A) \neq 0$
		si et seulement si $A$ est inversible. \end{thm}

		\begin{proof} En supposant $A$ inversible, supposons également, par l'absurde, $\det(A) = 0$. On a alors~:
		\[1 = \det(\Id_{\Hom(V, V)}) = \det(A \circ A^{-1}) = \det(A)\det(A^{-1}) = 0,\] ce qui est une contradiction. L'hypothèse est donc fausse et $\det(A) \neq 0$.

		Supposons maintenant $A$ non inversible et montrons que $\det(A) = 0$. Soit $E$ une base de $V$. Si $A$ n'est pas inversible, alors il existe $i$ tel que
		$A(e_i) = 0$ où $e_i \neq 0$ car c'est un vecteur de la base. Soit $f$, la seule forme multilinéaire alternée telle que $f(e_1, \dotsc, e_n) = 1$. On a donc~:
		\[\det(A) = \det(A)f(e_1, \dotsc, e_i, \dotsc, e_n) = f(A(e_1), \dotsc, A(e_i), \dotsc, A(e_n)) = f(A(e_1), \dotsc, 0, \dotsc, A(e_n)) = 0,\]
		par la proposition~\ref{f(0,)=0}. \end{proof}

		\begin{rmq} Soient $A \in \Hom(\R^n, \R^n)$ et $a \coloneqq m_{E, E}(A) \in \M nn\R$ sa matrice associée dans la base canonique de $\R^n$. L'application $A$
		envoie le cube unitaire de dimension $n$ sur le parallélogramme de dimension $n$ dont les coordonnées des points $A(e_i)$ sont les colonnes de $a$.
		Le volume orienté de l'image de $P \subset \R^n$ par $A$ est donné par~: \[\Vor(A(P)) = \det(A)\Vor(P).\]

		Dès lors, si $P$ est le cube unitaire de dimension $n$ et $Q$ est un parallélogramme de dimension $n$, alors en construisant la matrice $a \in \M nn\R$ en
		y mettant colonne par colonne les coordonnées des points de $Q$, alors $\Vor(Q) = \det(a)$.

		Si le déterminant est négatif, c'est que le parallélogramme a été \textit{retourné} et donc son volume orienté change de signe.
		De là, on voit bien que $\det(A) = 0$ si $\rang(A) \coloneqq \dim_\R\Im(A) < n$ car $A(P)$ n'est plus un parallélogramme à $n$ dimensions, et donc son volume en
		dimension $n$ est nul également. \end{rmq}

	\subsection{Cofacteurs, matrices adjointes et inverses}
		\begin{déf} Soit $a \coloneqq [a_{i\,j}] \in \M nn\R$ une matrice. Le cofacteur $A_i^j$ est défini par~:
		\[A_i^j \coloneqq (-1)^{i+j}\det_{(n-1)\times(n-1)}
		\begin{pmatrix}
			a_{1\,1}   & \ldots  & a_{1\,j-1}   & a_{1\,j+1}   & \ldots  & a_{1\,n}   \\
			  \vdots   & \ddots  &    \vdots    &   \vdots     & \iddots & \vdots     \\
			a_{i-1\,1} & \ldots  & a_{i-1\,j-1} & a_{i-1\,j+1} & \ldots  & a_{i-1\,n} \\
			a_{i+1\,1} & \ldots  & a_{i+1\,j-1} & a_{i+1\,j+1} & \ldots  & a_{i+1\,n} \\
			  \vdots   & \iddots &     \vdots   &    \vdots    & \ddots  & \vdots     \\
			a_{n\,1}   & \ldots  & a_{n\,j-1}   & a_{n\,j+1}   & \ldots  & a_{n\,n}
		\end{pmatrix}
		\] \end{déf}

		\begin{thm}[Théorème de Laplace]\label{thmLaplace} Soit $a \in \M nn\R$. Pour $1 \leq i, j \leq n$, on a~:
		\[\det(a) = \sum_{k=1}^na_{ik}A_i^k = \sum_{k=1}^na_{kj}A_k^j.\] \end{thm}

		\begin{proof} Prouvons d'abord la première égalité pour $i=n$~:
		\[\begin{aligned}
			\det(a) &= \sum_{\sigma\in\Perm_n}\sign(\sigma)\prod_{k=1}^na_{k\,\sigma(k)} = \sum_{j=1}^n\sum_{\sigma\in\Perm_n, \sigma(n)=j}\sign(\sigma)\prod_{k=1}^na_{k\,\sigma(k)} \\
			        &= \sum_{j=1}^na_{n\,j}\sum_{\sigma\in\Perm_n, \sigma(n)=j}\sign(\sigma)\prod_{k=1}^{n-1}a_{k\,\sigma(k)} \\
					&= \sum_{j=1}^na_{n\,j}\det
						\begin{pmatrix}
							a_{1\,1}   & \ldots  & a_{1\,j-1}   & a_{1\,j}   & a_{1\,j+1}   & \ldots  & a_{1\,n}   \\
							  \vdots   & \ddots  &   \vdots     &   \vdots   &   \vdots     & \iddots &   \vdots   \\
							a_{n-1\,1} & \ldots  & a_{n-1\,j-1} & a_{n-1\,j} & a_{n-1\,j+1} & \ldots  & a_{n-1\,n} \\
							    0      &     0   &      0       &     1      &     0        &    0    &    0
						\end{pmatrix} \\
					&= \sum_{j=1}^na_{n\,j}(-1)^{j-1}\det
						\begin{pmatrix}
							a_{1\,j}   & a_{1\,1}   & \ldots & a_{1\,j-1}   & a_{1\,j+1}   & \ldots  & a_{1\,n}   \\
							  \vdots   &   \vdots   & \ddots &   \vdots     &   \vdots     & \iddots &   \vdots   \\
							a_{n-1\,j} & a_{n-1\,1} & \ldots & a_{n-1\,j-1} & a_{n-1\,j+1} & \ldots  & a_{n-1\,n} \\
							     1     &     0      &     0  &     0        &      0       &    0    &       0
						\end{pmatrix} \\
					&= \sum_{j=1}^na_{n\,j}(-1)^{j-1}(-1)^{n-1}\det
						\begin{pmatrix}
							     1     &     0      &     0  &     0        &      0       &    0    &       0    \\
							a_{1\,j}   & a_{1\,1}   & \ldots & a_{1\,j-1}   & a_{1\,j+1}   & \ldots  & a_{1\,n}   \\
							  \vdots   &   \vdots   & \ddots &   \vdots     &   \vdots     & \iddots &   \vdots   \\
							a_{n-1\,j} & a_{n-1\,1} & \ldots & a_{n-1\,j-1} & a_{n-1\,j+1} & \ldots  & a_{n-1\,n}
						\end{pmatrix} \\
					&= \sum_{j=1}^na_{n\,j}(-1)^{j+n}\sum_{\gamma=1}^nb_{1\,\gamma}\sum_{\sigma\in\Perm_n, \sigma(1)=\gamma}\sign(\sigma)\prod_{k=2}^nb_{k\,\sigma(k)} \\
					&= \sum_{j=1}^na_{n\,j}(-1)^{j+n}\sum_{\sigma\in\Perm_n, \sigma(1)=1}\sign(\sigma)\prod_{k=2}^nb_{k\,\sigma(k)}.
		\end{aligned}\]

		Ici, on a posé~:
		\[b \coloneqq [b_{i\,j}] \coloneqq
			\begin{pmatrix}
				     1     &     0      &     0  &     0        &      0       &    0    &       0    \\
				a_{1\,j}   & a_{1\,1}   & \ldots & a_{1\,j-1}   & a_{1\,j+1}   & \ldots  & a_{1\,n}   \\
				  \vdots   &   \vdots   & \ddots &   \vdots     &   \vdots     & \iddots &   \vdots   \\
				a_{n-1\,j} & a_{n-1\,1} & \ldots & a_{n-1\,j-1} & a_{n-1\,j+1} & \ldots  & a_{n-1\,n}
			\end{pmatrix}
		\in \M nn\R.\]

		De plus, en imposant $\sigma(1) = 1$, pour tout $n \geq i > 1$, on a $\sigma(i) > 1$ car $\sigma$ est une bijection. Donc avec cette contrainte, et en
		itérant sur $k \in \{2, 3, \dotsc, n\}$, les éléments de la première ligne ou de la première colonne ne sont pas accédés. En posant~:
		\[c \coloneqq [c_{i\,j}] \coloneqq
			\begin{pmatrix}
				a_{1\,1}   & \ldots  & a_{1\,j-1}   & a_{1\,j+1}   & \ldots  & a_{1\,n}  \\
				  \vdots   & \ddots  &   \vdots     &   \vdots     & \iddots &   \vdots  \\
				a_{n-1\,1} & \ldots  & a_{n-1\,j-1} & a_{n-1\,j+1} & \ldots  & a_{n-1\,n} \\
			\end{pmatrix}
		\in \M {(n-1)}{(n-1)}\R,\]
		à savoir $c$ est la matrice $b$ sans la première ligne et la première colonne, on a alors~:
		\[\begin{aligned}
			\det(a) &= \sum_{j=1}^na_{n\,j}(-1)^{j+n}\sum_{\sigma\in\Perm_n, \sigma(1)=1}\sign(\sigma)\prod_{k=2}^nb_{k\,\sigma(k)} \\
			        &= \sum_{j=1}^na_{n\,j}(-1)^{j+n}\sum_{\sigma\in\Perm_n, \sigma(1)=1}\sign(\sigma)\prod_{k=2}^nc_{k-1\,\sigma(k-1)} \\
					&= \sum_{j=1}^na_{n\,j}(-1)^{j+n}\sum_{\sigma \in \Perm_{n-1}}\sign(\sigma)\prod_{k=1}^{n-1}c_{k\,\sigma(k)} \\
					&= \sum_{j=1}^na_{n\,j}A_n^j.
		\end{aligned}\]

		Afin de prouver le théorème pour tout $i$, il suffit de passer la proposition~\ref{ChangementDeSigneSwapDeParams} en échangeant les colonnes, ce qui change
		le signe (à cause de l'échange de deux colonnes), mais qui change également les signes des cofacteurs. Afin de prouver également le théorème,
		on pose $\gamma \in \{1, \dotsc, n\}$, et on part de la proposition~\ref{detTransposéeEstdetMatrice} qui dit que $\det(a) = \det(a^T)$ et on obtient~:
		\[\det(a) = \det(A^T) = \sum_{j=1}^n(a^T)_{\gamma\,j}(A^T)_\gamma^j = \sum_{j=1}^na_{j\,\gamma}A_j^\gamma.\] \end{proof}

		\begin{déf} Soit $a \coloneqq [a_{i\,j}] \in \M nn\R$ une matrice carrée. On définit sa matrice adjacente par~: \[\adj a = [A_j^i] \in \M nn\R.\] \end{déf}

		\begin{prp} Soit $a = [a_{i\,j}] \in \M nn\R$. Le produit matriciel $a \adj(A)$ donne une matrice diagonale telle que $(a \adj(a))_{ii} = \det(a)$ pour tout $i$.
		\end{prp}

		\begin{proof} En procédant au produit de $a$ et $\adj(a)$, on obtient~:
		\[(a\adj(a))_{i\,j} \coloneqq \sum_{k=1}^na_{i\,k}(\adj(a))_{k\,j} = \sum_{k=1}^na_{i\,k}A_j^k = \det(a)\delta_{ij}.\]

		Le facteur $\delta_{ij}$ impose que tout élément $(a\adj(a))_{i\,j}$ où $i \neq j$ soit nul. Il ne reste dès lors que la diagonale où chaque $(a\adj(a))_{i\,i}$
		vaut $\det(a)$. \end{proof}

		\begin{cor} Soit $a \in \M nn\R$ une matrice carrée inversible. Son inverse $a^{-1}$ vaut~:
		\[a^{-1} = \frac 1{\det(a)}\adj(a)\] \end{cor}

		\begin{proof} On sait que $a\adj(a) = \det(a)\Id_{\M nn\R}$. Dès lors, en divisant de part et d'autre par $\det(a)$ (qui est non nul car $a$ est inversible),
		et en multipliant par $a^{-1}$ à gauche, on obtient~:
		\[\begin{aligned}
			a^{-1}a\adj(a)\det(a)^{-1} &= a^{-1}\det(a)\Id_{\M nn\R}\det(a)^{-1} \\
			\adj(a)\frac 1{\det(a)} &= a^{-1}.
		\end{aligned}\]
		\end{proof}

		\begin{déf} Soient $a = [a_{i\,j}] \in \M nn\R$ et $b = [b_i], x = [x_i] \in \M n1\R$ tels que~: \[ax = b.\] On pose~:
		\[S_j(a, b) \coloneqq
		\begin{pmatrix}
			a_{1\,1} & \ldots & a_{1\,j-1} &   b_1  & a_{1\,j+1} & \ldots  & a_{1\,n} \\
			  \vdots & \ddots &   \vdots   & \vdots &   \vdots   & \iddots &   \vdots \\
			a_{n\,1} & \ldots & a_{n\,j-1} &   b_n  & a_{n\,j+1} & \ldots  & a_{n\,n}
		\end{pmatrix}
		\]

		La valeur $S_j(a, b)$ est donc la matrice $a$ où la $j$ème colonne a été remplacée par $b$. \end{déf}

		\begin{thm}[Règle de Cramer]\label{règleCramer} Soit $(S) : ax = b$ un système linéaire d'équations où $a \in \M nn\R, x, b \in \M n1\R$. Le système $(S)$
		admet une et une seule solution si et seulement si $\det(a) \neq 0$. De plus, cette solution est donnée par~:
		\[x_j = \frac {\det\left(S_j(a, b)\right)}{\det(a)}.\] \end{thm}

		\begin{proof} Soient $a \in \M nn\R, b, x \in \M n1\R$ tels que $ax = b$. On peut déterminer~:
		\[\begin{aligned}
			\det(S_j(a, b)) &= \det
				\begin{pmatrix}
					a_{1\,1} & \ldots & a_{1\,j-1} &   b_1  & a_{1\,j+1} & \ldots  & a_{1\,n} \\
					  \vdots & \ddots &   \vdots   & \vdots &   \vdots   & \iddots &   \vdots \\
					a_{n\,1} & \ldots & a_{n\,j-1} &   b_n  & a_{n\,j+1} & \ldots  & a_{n\,n}
				\end{pmatrix} \\
			&= \det
				\begin{pmatrix}
					a_{1\,1} & \ldots & a_{1\,j-1} &   \sum_{k=1}^na_{1\,k}x_k  & a_{1\,j+1} & \ldots  & a_{1\,n} \\
					  \vdots & \ddots &   \vdots   &           \vdots           &   \vdots   & \iddots &   \vdots \\
					a_{n\,1} & \ldots & a_{n\,j-1} &   \sum_{k=1}^na_{n\,k}x_k  & a_{n\,j+1} & \ldots  & a_{n\,n}
				\end{pmatrix} \\
			&= \sum_{k=1}^n
				\det
					\begin{pmatrix}
						a_{1\,1} & \ldots & a_{1\,j-1} &   a_{1\,k}x_k  & a_{1\,j+1} & \ldots  & a_{1\,n} \\
						  \vdots & \ddots &   \vdots   &     \vdots     &   \vdots   & \iddots &   \vdots \\
						a_{n\,1} & \ldots & a_{n\,j-1} &   a_{n\,k}x_k  & a_{n\,j+1} & \ldots  & a_{n\,n}
					\end{pmatrix} \\
			&= \sum_{k=1}^n
				x_k\det
					\begin{pmatrix}
						a_{1\,1} & \ldots & a_{1\,j-1} &   a_{1\,k}  & a_{1\,j+1} & \ldots  & a_{1\,n} \\
						  \vdots & \ddots &   \vdots   &   \vdots    &   \vdots   & \iddots &   \vdots \\
						a_{n\,1} & \ldots & a_{n\,j-1} &   a_{n\,k}  & a_{n\,j+1} & \ldots  & a_{n\,n}
					\end{pmatrix}
		\end{aligned}\]

		Ici, pour toute valeur de $k$ différente de $j$, la matrice aura deux colonnes identiques (à savoir la colonne $j$ et la colonne $k$). Dès lors, le déterminant
		s'annule, et il ne reste plus que~: \[\det(S_j(a, b)) = x_j\det(a).\] On en déduit que~: \[x_j = \frac {\det(S_j(a, b))}{\det(a)}.\] \end{proof}

\newpage
\section{Algèbre d'ensembles et résultats sur les espaces vectoriels}
	\subsection{Définitions}
		\begin{déf} Soit $\K$ un ensemble et soit $*$ une opération interne telle que $* : \K \times \K : (a, b) \mapsto a*b \eqqcolon ab$. Si $\K$ respecte les
		trois propriétés suivantes~:

		\begin{enumerate}
			\item $\exists 0 \in \K$ tel que $\forall a \in \K : a * 0 = 0 * a = a$~;
			\item $\forall a, b, c \in \K : a * (b*c) = (a*b) * c$~;
			\item $\forall a \in \K : \exists b \in \K \tq a + b = b + a = 0$ (on note $b$ $-a$ et on l'appelle \textbf{l'inverse de $a$}),
		\end{enumerate}
		on appelle $(K, *)$ un \textbf{groupe}.\end{déf}

		\begin{déf} Soit $(\K, *)$ un groupe. Si $*$ est une opération commutative ($\forall a, b \in \K : a * b = b * a$), on appelle $(\K, *)$ un
		\textbf{groupe abélien}. \end{déf}

		\begin{rmq} L'opération $*$ d'un groupe abélien est souvent notée $+$ pour montrer son caractère commutatif. \end{rmq}

		\begin{déf} Soit $(\K, +)$ un groupe abélien. On définit une opération $\cdot : \K \times \K \to \K : (a, b) \mapsto a \cdot b \eqqcolon ab$.
		Si $(\K, +, \cdot)$ respecte les propriétés suivantes~:

		\begin{enumerate}
			\item $\exists 1 \in \K \tq \forall a \in \K : a \cdot 1 = 1 \cdot a = a$~;
			\item $\forall a, b, c \in \K : a \cdot (b \cdot c) = (a \cdot b) \cdot c$~;
			\item $\forall a, b, c \in \K : (a \cdot (b + c) = a \cdot b + a \cdot c) \land ((a + b) \cdot c = a \cdot c + b \cdot c)$,
		\end{enumerate}
		on appelle $(\K, +, \cdot)$ un \textbf{anneau}. \end{déf}

		\begin{déf} Soit $(K, +, \cdot)$ un anneau. Si $\forall a \neq 0 \in \K : \exists b \in \K \tq a \cdot b = b \cdot a = 1$ (où $b$ est noté $a^{-1}$), alors
		on appelle $(K, +, \cdot)$ un corps. \end{déf}

		\begin{déf} Soit $(\K, +, \cdot)$ un corps. Si l'opération $\cdot$ est commutative, alors on appelle $(\K, +, \cdot)$ un \textbf{corps commutatif}. \end{déf}

	\subsection{Sous-espaces propres}
		\begin{déf} Soient $\K$ un corps commutatif, $V$ un espace vectoriel sur $\K$, et $A \in \Hom(V, V)$ un opérateur linéaire.

		\begin{itemize}
			\item On appelle \textbf{vecteur propre} de $A$ tout vecteur $v \in V \tq \exists \lambda \in \K \tq A(v) = \lambda v$~;
			\item on appelle \textbf{valeur propre } de $A$ toute valeur $\lambda \in \K \tq \exists v \neq 0 \in V$ et $v$ est une valeur propre de $A$~;
			\item l'ensemble des valeurs propres de $A$ est appelé le \textbf{spectre} de $A$.
		\end{itemize}
		\end{déf}

		\begin{déf} Soit $V$ un espace vectoriel sur un corps $\K$. Soit $A \in \Hom(V, )$ un opérateur linéaire. Soit $\lambda \in \K$une valeur propre de $A$.
		On appelle \textbf{sous-esapce propre} relatif à $\lambda$ l'ensemble~: \[V_\lambda \coloneqq \{v \in V \tq A(v) = \lambda v\}.\] \end{déf}

		\begin{prp} Soit $V$ un espace vectoriel sur un corps $\K$ et $V_\lambda$ un sous-espace propre relatif à $\lambda$, une valeur propre de $A \in \Hom(V, V)$.
		Alors $V_\lambda$ muni des opérateurs $+_V$ et $\cdot_V$ est un sous-espace vectoriel de $V$. \end{prp}

		\begin{proof} Pour montrer que $V_\lambda$ est un sous-espace de $V$, il faut montrer que~:
		\begin{enumerate}
			\item $0_V \in V_\lambda$~;
			\item $\forall w_1, w_2 \in V_\lambda : w_1 + w_2 \in V_\lambda$~;
			\item $\forall \mu \in \K, w \in V_\lambda : \mu w \in V_\lambda$.
		\end{enumerate}

		Montrons donc chacune de ces trois propriétés~:
		\begin{enumerate}
			\item On sait que $A(0) = 0 = \lambda 0$ pour tout $\lambda \in \K$. Dès lors $0 \in V_\lambda$.
			\item Soient $w_1, w_2 \in V_\lambda$. Alors $A(w_1 + w_2) = A(w_1) + A(w_2) = \lambda v_1 + \lambda v_2 = \lambda(v_1 + v_2)$.
			      Dès lors $w_1+w_2 \in V_\lambda$.
			\item Soient $\mu \in \K, w \in V_\lambda$. Alors $A(\mu w) = \mu A(w) = \mu (\lambda v) = \lambda(\mu v)$. Dès lors $\mu w \in V_\lambda$.
		\end{enumerate}

		On a donc bien $V_\lambda$ est un sous-espace de $V$ par la proposition~\ref{conditionsSousEspace}. \end{proof}

		\begin{lem} Soit $\K$ un corps commutatif et $V$ un espace vectoriel sur $\K$. Si $A \in \Hom(V, V)$, alors $\Ker(A) = V_0$. \end{lem}

		\begin{proof} On sait que $\Ker(A) \coloneqq \{v \in V \tq A(v) = 0\}$. Également, $V_0 \coloneqq \{v \in V \tq A(v) = 0 \cdot v = 0\} \eqqcolon \Ker(A)$.
		\end{proof}

		\begin{prp} Soient $a \coloneqq [a_{i\,j}] \in \M nn\K$ un matrice diagonale et $A \in \Hom(\K^n, \K^n)$ son opérateur linéaire associé.
		Soit $E \coloneqq \{e_1, \dotsc, e_n\}$ la base canonique de $\K^n$. Alors~: \[V_\lambda = \eng {\{e_i \in E \tq a_{i\,i} = \lambda\}}.\] \end{prp}

		\begin{proof} Montrons que $V_\lambda \subseteq \eng {\{e_i \in E \tq a_{i\,i} = \lambda\}} \subseteq V_\lambda$.

		Soit $v \in V_\lambda$. On sait que $A(v) = \lambda v$. Or, on sait également que $A(v) = av$. On a donc~:
		\[A(v) = \left[\sum_{k=1}^na_{i\,k}v_k\right]_i= \sum_{i=1}^ne_i\sum_{k=1}^na_{i\,k}v_k.\]
		Cependant, $a$ est une matrice diagonale. Donc $\forall i \neq k : a_{i\,k} = 0$. On trouve donc~:
		\[A(v) = \sum_{i=1}^ne_ia_{i\,i}v_i.\]
		En reprenant le fait que $A(v) = \lambda v$, on obtient~:
		\[\sum_{i=1}^n\lambda v_ie_i = \lambda v = A(v) = \sum_{i=1}^na_{i\,i}v_ie_i.\]
		Que l'on peut transformer en~:
		\[0 = \sum_{i=1}^na_{i\,i}v_ie_i - \lambda v_ie_i = \sum_{i=1}^n(\lambda - a_{i\,i})v_ie_i.\]

		Or, $e_i$ est un vecteur de la base. On sait donc que $\forall 1 \leq i \leq n : e_i \neq 0$. Ce qui laisse donc soit $v_i = 0$, soit $a_{i\,i} = \lambda$.
		On a alors~:
		\[\begin{aligned}
			A(v) &= \sum_{i=1}^na_{i\,i}v_ie_i = \sum_{i \tq a_{i\,i} = \lambda}a_{i\,i}v_ie_i + \sum_{i \tq a_{i\,i} \neq \lambda}a_{i\,i}v_ie_i \\
			     &= \sum_{i \tq a_{i\,i} = \lambda}a_{i\,i}v_ie_i + \sum_{i \tq a_{i\,i} \neq \lambda}a_{i\,i} \cdot 0 \cdot e_i \\
				 &= \sum_{i \tq a_{i\,i} = \lambda}a_{i\,i}v_ie_i \in \eng {\{e_i \in E \tq a_{i\,i} = \lambda\}}.
		\end{aligned}\]

		Soit $v \in \eng {\{e_i \in E \tq a_{i\,i} = \lambda\}}$. On sait donc que~:
		\[v = \sum_{i \tq a_{i\,i} = \lambda}\mu_ie_i.\]
		De plus, en évaluant l'image de $v$ par $A$, on obtient~:
		\[\begin{aligned}
			A(v) &= A\left(\sum{i \tq a_{i\,i} = \lambda}\mu_ie_i\right) = \sum_{i \tq a_{i\,i}=\lambda}\mu_iA(e_i)= \sum_{i \tq a_{i\,i}=\lambda}\mu_iae_i \\
			     &= \sum_{i \tq a_{i\,i}=\lambda}\mu_i\sum_{j=1}^ne_j\sum_{k=1}^na_{j\,k}\left(e_i\right)_k.
		\end{aligned}\]

		Or, $e_i$ est un vecteur de la base canonique. Ce qui veut dire que $\left(e_i\right)_k = \delta_{jk}$. De plus, $a$ est une matrice diagonale, et donc
		pour tout $i, j$, on a~: $a_{i\,j} = a_{i\,j}\delta_{ij}$. Dès lors~:
		\[\begin{aligned}
			A(v) &=\sum_{i \tq a_{i\,i}=\lambda}\mu_i\sum_{j=1}^ne_j\sum_{k=1}^n\left(a_{j\,k}\delta_{jk}\right)\left(\left(e_i\right)_k\delta_{ik}\right) \\
			     &= \sum_{i \tq a_{i\,i}=\lambda}\mu_i\sum_{j=1}^ne_ja_{j\,j}\left(e_i\right)_j\delta_{ij} = \sum_{i \tq a_{i\,i}=\lambda}\mu_ie_ia_{i\,i} \\
				 &= \sum_{i \tq a_{i\,i}=\lambda}\mu_ie_i\lambda = \lambda\sum_{i \tq a_{i\,i}=\lambda}\mu_ie_i = \lambda v \in V_\lambda.
		\end{aligned}\]
		\end{proof}

		\begin{lem}\label{combiliSousEspacesPropresLibre} Soient $V$ un espace vectoriel sur un corps commutatif $\K$, $A \in \Hom(V, V)$ un opérateur linéaire,
		$\lambda_1, \dotsc, \lambda_n$ des valeurs propres de $A$ deux à deux distinctes et
		$(v_1, \dotsc, v_n) \in V_{\lambda_1} \times V_{\lambda_2} \times \dotsb \times V_{\lambda_n}$ tels que $\forall 1 \leq i \leq n : v_i \neq 0$. Alors
		$\{v_1, \dotsc, v_m\}$ est une partie libre. \end{lem}

		\begin{proof} Montrons cela par récurrence sur $n$.

		\underline{Cas de base~:} $n = 1$. Soit $W \coloneqq \{v_1\} \in V_{\lambda_1}$. $W$ est libre car $v_1 \neq 0$ donc la seule combinaison linéaire nulle
		possible est la combinaison linéaire nulle triviale.

		\underline{Pas de récurrence~:} supposons que $\{v_1, \dotsc, v_n\}$ est une partie libre et montrons que $\{v_1, \dotsc, v_{n+1}\}$ est une partie libre,
		où $v_{n+1} \in V_{\lambda_{n+1}}$. Soient $\mu_1, \dotsc, \mu_{n+1} \in \K$ tels que~: \[\omega \coloneqq \sum_{i=1}^{n+1}\mu_iv_i = 0.\]
		On sait~: \[0 = A(0) = A\left(\sum_{i=1}^{n+1}\mu_iv_i\right) = \sum_{i=1}^{n+1}\mu_i A(v_i) = \sum_{i=1}^{n+1}\mu_i\lambda_iv_i.\]
		De plus, on peut ajouter $\omega$ sans changer la valeur car $\omega = 0$. Dès lors~:
		\[0 = \sum_{i=1}^{n+1}\mu_i\lambda_iv_i - \lambda_{n+1}\omega = \sum_{i=1}^{n+1}\mu_iv_i(\lambda_i - \lambda_{n+1}).\]
		Pour $1 \leq i \leq n, (\lambda_i - \lambda_{n+1}) \neq 0$. De plus, par hypothèse de récurrence, $\{v_1, \dotsc, v_n\}$ forme une partie libre. La seule
		combinaison linéaire nulle est formée par $\mu_i = 0 \forall 1 \leq i \leq n$. On a donc~: \[0 = \sum_{i=1}^{n}\mu_iv_i + \mu_{n+1}v_{n+1} = \mu_{n+1}v_{n+1}.\]
		Or, par hypothèse, on sait $v_{n+1} \neq 0$. On a donc $\mu_{n+1} = 0$. Dès lors, $\{v_1, \dotsc, v_{n+1}\}$ est une partie libre. \end{proof}

		\begin{prp}\label{unionVlambdaLibre} Soit $V$ un espace vectoriel sur un corps commutatif $\K$. Soient $A \in \Hom(V, V)$ un opérateur linéaire et
		$\lambda_1, \dotsc, \lambda_n$ des valeurs propres de $A$ deux à deux distinctes. Pour $1 \leq i \leq n$, on définit $B_i \subseteq V_{\lambda_i}$ des parties
		libres. Alors, l'ensemble~: \[W \coloneqq \bigcup_{i=1}^nB_i\] est également une partie libre. \end{prp}

		\begin{proof} Montrons à nouveau cela par récurrence sur $n$.

		\underline{Cas de base~:} $n = 1$. Soit $W \coloneqq B_1 \subseteq V_{\lambda_1}$. $W$ est une partie libre car $B_1$ est libre par hypothèse.

		\underline{Pas de récurrence~:} supposons que $B_1 \cup \dotsb \cup B_n$ soit une partie libre et montrons que $B_1 \cup \dotsb B_{n+1}$ est également
		une partie libre. Pour tout $1 \leq i \leq n+1$, on définit $B_i \coloneqq \{x_1^{(i)}, \dotsc, x_{n_i}^{(i)}\} \subseteq V_{\lambda_i}$.

		Prenons $\omega$ défini par~: \[\omega \coloneqq \sum_{i=1}^{n+1}\sum_{j=1}^{n_i}\mu_j^{(i)}x_j^{(i)} = 0.\]
		On veut ensuite montrer que $\forall 1 \leq i \leq n : \forall 1 \leq j \leq n_i : \mu_j^{(i)} = 0$. On définit également~:
		\[\forall 1 \leq i \leq n : \quad v_i \coloneqq \sum_{j=1}^{n_i}\mu_j^{(i)}x_j^{(i)}.\]
		On sait que $v_i \in V_{\lambda_i}$. Dès lors, on sait~: \[\sum_{i=1}^nv_i = 0.\]
		Par le lemme~\ref{combiliSousEspacesPropresLibre}, on sait que~: \[0 = v_{n+1} = \sum_{j=1}^{n_{n+1}}\mu_j^{(n+1)}x_j^{(n+1)}.\]
		Cependant, on sait que $B_{n+1}$ est une partie libre. Dès lors, la seule combinaison linéaire nulle est $\mu_j^{(n+1)} = 0$ pour tout $j$.
		Il reste~: \[0 = \sum_{i=1}^nv_i.\] Par hypothèse de récurrence, on sait que $B_1 \cup \dotsb \cup B_n$ est une partie libre. Dès lors, la seule
		combinaison linéaire nulle possible pour $0 = \sum_iv_i$ est $\mu_j^{(i)} = 0$ pour tout $1 \leq i \leq n$ et $1 \leq j \leq n_i$.

		La seule combinaison linéaire possible au final pour~: \[0 = \sum_{i=1}^{n+1}\sum_{j=1}^{n_i}\mu_j^{(i)}x_j^{(i)}\] est la combinaison linéaire nulle
		triviale. L'ensemble $W \coloneqq B_1 \cup \dotsb \cup B_{n+1}$ est donc bien une partie libre. \end{proof}

	\subsection{Équation caractéristique}
		\begin{déf} Soit $A$ un opérateur linéaire sur $V$, un $\K$-espace vectoriel où $\K$ est un corps commutatif. Alors~:
		\[\pi_A(\lambda) \coloneqq \det(A - \lambda\Id_{Hom(V, V)})\] est appelé le \textbf{polynôme caractéristique} de $A$. \end{déf}

		\begin{lem}\label{valeurPropreiffVlambda} Soit $A$ un opérateur linéaire sur un espace vectoriel $V$ sur un corps commutatif $\K$. Alors $\lambda \in \K$
		est une valeur propre de $A$ si et seulement si $V_\lambda \neq \{0\}$. \end{lem}

		\begin{proof} Supposons que $V_\lambda = \{0\}$. Alors seul $0$ est tel que $A(0) = \lambda 0 = 0$. Dès lors, $\lambda$ n'est pas une valeur propre.

		Supposons maintenant que $\lambda$ est une valeur propre de $A$. Alors, il existe $v \in V$ tel que $A(v) = \lambda v$. Donc $v \in V_\lambda$. On
		a alors bien $V_\lambda \neq \{0\}$. \end{proof}

		\begin{thm} Soit $V$ un espace vectoriel de dimension finie $n$ sur un corps commutatif $\K$ et soit $A \in \Hom(V, V)$. Alors~:

		\begin{enumerate}
			\item $\lambda \in \K$ est une valeur propre de $A$ si et seulement si $\det(A - \lambda\Id_{Hom(V, V)}) = 0$~;
			\item $V_\lambda = \Ker(A - \lambda\Id_{Hom(V, V)})$~;
			\item $\pi_A(\lambda)$ est un polynôme de degré $n$ en $\lambda$~;
			\item l'opérateur $A$ a au plus $n$ valeurs propres distinctes qui sont les racines de l'équation caractéristique.
		\end{enumerate}
		\end{thm}

		\begin{proof} Soient $\K$ un corps commutatif, $V$ un espace vectoriel sur $\K$, et $A \in \Hom(V, V)$, un opérateur linéaire sur $V$.

		\begin{enumerate}
			\item Par définition, $\lambda \in \K$ est une valeur propre de $A$ si et seulement si il existe $v \neq 0 \in V \tq A(v) = \lambda v$, donc
			      $A(v) - \lambda v = 0$, ou encore $(A - \lambda\Id_{Hom(V, V)})(v) = 0$.
			\item Par le point 1, on peut réécrire $\Ker(A - \lambda\Id_{Hom(V, V)}) \neq \{0\} \iff \det(A - \lambda \Id_{\Hom(V, V)}) = 0$, ce qui a déjà été montré
			      par le théorème~\ref{inversiblessidet=0}.
			\item Montrons cela par récurrence sur $n$.

			      \underline{Cas de base~:} $n = 1$. On sait que
				  $\det\left(\begin{pmatrix}a_{1\,1}\end{pmatrix} - \lambda\begin{pmatrix}1\end{pmatrix}\right) = a_11 - \lambda$, qui est effectivement un polynôme
				  de degré 1 en $\lambda$.

				  \underline{Pas de récurrence~:} supposons que pour $V$ de dimension $n$, $\det(a - \lambda \Id_{\Hom(V, V)})$ est un polynôme de degré $n$
				  en $\lambda$, et montrons que pour $V$ de dimension $n+1$, le déterminant est un polynôme de degré $n+1$.

				  Soient $a \in \M {(n+1)}{(n+1)}\K$ et $\gamma \in \N$ tel que $1 \leq \gamma \leq (n+1)$. Alors, par le théorème de Laplace~:
				  \[\det(a - \lambda \Id_{\M {(n+1)}{(n+1)}\K}) = \sum_{i=1}^{n+1}(a_{i\,\gamma} - \lambda\delta_{i\gamma})A_i^\gamma
				    = \sum_{i=1}^{n+1}a_{i\,\gamma}A_i^\gamma - \lambda A_\gamma^\gamma.\]
				  Or, $A_i^\gamma$ est, par hypothèse de récurrence, un polynôme de degré $n$. On a donc la somme de $n$ polynômes de degré $n$ et un polynôme
				  de degré $n+1$. Le degré étant la plus grande puissance à laquelle est montée une inconnue, le degré final du déterminant est $n+1$.
			\item La démonstration du point 4 dépasse l'objectif de ce cours. La démonstration est donc omise.
		\end{enumerate}
		\end{proof}

		\begin{lem}\label{polcarnorm} Soient $\K$ un corps commutatif, $V$ un espace vectoriel  de dimension finie $n$ sur $\K$ et $A \in \Hom(V, V)$. Alors
		$(-1)^n\pi_A(\lambda) \in \K[\lambda]$ est un polynôme normalisé. À savoir, si~:
		\[\pi_A(\lambda) = \sum_{i=0}^n\lambda^i\pi_i,\]
		alors $\pi_n = (-1)^n$. \end{lem}

		\begin{proof} Soit $a \coloneqq m_{E, E}(A)$, la matrice de $A$ dans une base $E$ de $V$. Alors, $\det(a - \lambda\Id_{\M nn\K})$ ne peut avoir comme seul
		monôme de degré $n$ celui compris dans, en prenant $\sigma = \Id_{\Perm_n}$~:
		\[\sign(\sigma)\prod_{i=1}^n(a_{i\,\sigma(i)} - \lambda) = \prod_{i=1}^n(a_{i\,i} - \lambda),\]
		à savoir~: \[\prod_{i=1}^n(-\lambda) = (-1)^n\lambda^n.\] \end{proof}

		\begin{thm}[Théorème de Cayley-Hamilton]\label{Cayley-Hamilton} Soit $V$ un espace vectoriel de dimension finie $n$ su un corps commutatif $\K$. Soit
		$A \in \Hom(V, V)$. Alors $\pi_A(A) = 0$. \end{thm}

		\begin{proof} Soit $\pi_A(\lambda) = \det(a - \lambda\Id_{\M nn\K})$, le polynôme caractéristique de l'opérateur linéaire $A$. On sait que $\pi_A$ est un
		polynôme de degré $n$ en $\lambda$. Dès lors~:
		\[\pi_A = \sum_{i=0}^n\pi_i\lambda^i,\]
		où $\pi_i \in \K$ pour $i \in \{1, \dotsc, n\}$. Prenons $E$ une base de $V$, et $a \coloneqq m_{E, E}(A) \in \M nn\K$ la matrice associée à $A$.

		Posons $b \coloneqq \adj(a - \lambda\Id_{\M nn\K})$. Soient $b_0, \dotsc, b_{n-1} \in \M nn\K$ tels que~: \[b = \sum_{i=0}^{n-1}\lambda^ib_i.\]
		Alors~:
		\[\begin{aligned}
			\pi_A(\lambda)\Id_{\M nn\K} &= \det(a - \lambda\Id_{\M nn\K})\Id_{\M nn\K} = (a - \lambda\Id_{\M nn\K})\adj(a - \lambda\Id_{\M nn\K}) \\
			                            &= (a - \lambda\Id_{\M nn\K})b = (a - \lambda\Id_{\M nn\K})\sum_{i=0}^{n-1}\lambda^ib_i \\
			                            &= \sum_{i=0}^{n-1}a\lambda^ib_i - \sum_{i=0}^n\lambda\Id_{\M nn\K}\lambda^ib_i
			                            = \sum_{i=0}^{n-1}a\lambda^ib_i - \sum_{i=0}^{n-1}\lambda^{i+1}b_i \\
										&= ab_0 + \sum_{i=1}^{n-1}\lambda^i(ab_i - b_{i-1}) - \lambda^nb_{n-1}
		\end{aligned}\]

		On sait du lemme~\ref{polcarnorm} que~: \[\pi_A(\lambda) = (-1)^n\lambda^n + \sum_{i=0}^{n-1}c_i\lambda^i.\]
		Dès lors, on doit avoir $b_{n-1} = (-1^n)\Id_{\M nn\K}$, $ab_0 = c_0$, et $ab_i - b_{i-1} = c_i\Id_{\M nn\K}$ pour $1 \leq i \leq n-1$. Dès lors, on a~:
		\[\begin{aligned}
			\pi_A(a) &= ab_0 + \sum_{i=1}^{n-1}a^i(ab_i-b_{i-1}) - a^nb_{b-1} = ab_0 + \sum_{i=1}^{n-1}a^{i+1}b_i - \sum_{i=1}^{n-1}a^ib_{i-1} \\
			         &= ab_0 + \sum_{i=1}^{n-2}a^{i+1}b_i + a^{n}b_{n-1} - \sum_{i=2}^{n-1}a^ib_{i-1} - a^1b_0 - a^nb_{n-1} = 0.
		\end{aligned}\]
		\end{proof}

		\subsubsection{Opérateurs diagonalisables}
		\begin{déf} Soient $V$ un espace vectoriel de dimension $n$ sur un corps commutatif $\K$ et $A \in \Hom(V, V)$. On dit que $A$ est diagonalisable
		s'il existe une base $E$ de $V$ telle que $m_{E, E}(A) = [\delta_{i\,j}\lambda_i]$. Donc $A$ est diagonalisable si et seulement s'il existe une base $E$ de
		$V$ composée de vecteurs propres de $A$. \end{déf}

		\begin{prp}\label{nvalpropresdiagonalisable} Soient $V$ un espace vectoriel de dimension finie $n$ sur un corps commutatif $\K$ et $A \in \Hom(V, V)$. Si $A$
		possède $n$ valeurs propres distinctes, alors $A$ est diagonalisable. \end{prp}

		\begin{proof} Soient $\lambda_1, \dotsc, \lambda_n$ les valeurs propres de $A$ telles que pour tout $i \neq j$, on a $\lambda_i \neq \lambda_j$.
		Soient $e_i \in V_{\lambda_i}$ pour $1 \leq i \leq n$. Par la proposition~\ref{unionVlambdaLibre}, on sait que $E \coloneqq \{e_1, \dotsc, e_n\}$ est une base de
		$V$. On a donc~:
		\[\forall 1 \leq i \leq n : \quad A(e_i) = \lambda_ie_i.\]
		Dès lors~:
		\[m_{E, E}(A) = [\lambda_i\delta_{i\,j}] \in \M nn\K.\] \end{proof}

		\begin{prp}\label{diagonalisablepolcarprod} Soient $V$ un espace vectoriel de dimension finie $n$ sur un corps commutatif $\K$ et $A \in \Hom(V, V)$.
		Si $A$ est diagonalisable, alors $\pi_A(\lambda)$ peut s'exprimer comme un produit de $n$ polynômes linéaires à coefficients dans $\K$. \end{prp}

		\begin{proof} Le polynôme caractéristique est un polynôme de degré $n$ en $\lambda$ ayant pour zéros les valeurs propres de $A$. Dès lors~:
		\[P(\lambda) \coloneqq (-1)^n\prod_{i=1}^n(\lambda - \lambda_i) = \prod_{i=1}^n(\lambda_i - \lambda)\]
		est le polynôme caractéristique de $A$ dans une base de vecteurs propres. \end{proof}

		\begin{rmq} Il y a deux raisons pour lesquelles un opérateur linéaire $A$ ne serait pas diagonalisable~:
		\begin{enumerate}
			\item un problème algébrique~: les racines de $\pi_A$ ne sont pas dans $\K$~;
			\item un problème géométrique~: il n'y a pas suffisamment de vecteurs propres pour former une base de $V$.
		\end{enumerate}
		\end{rmq}

		\subsubsection{Opérateurs triangulables}
		\begin{déf} Soient $V$ un espace vectoriel de dimension finie $n$ sur un corps commutatif $\K$ et $A \in \Hom(V, V)$. On dit que $A$ est triangulable
		s'il existe une base $E$ de $V$ telle que $m_{E, E}(A)$ est triangulaire. \end{déf}

		\begin{déf} Soient $V$ un $\K$-espace vectoriel où $\K$ est un corps commutatif et ${W_i \tq 0 \leq i \leq n}$ un ensemble de sous-espaces de $V$.
		Si les $W_i$ sont tels que~:
		\[\{0\} = W_0 \subset W_2 \subset \dotsb \subset W_n = V,\]
		et pour tout $i$, on a $A(W_i) \subseteq W_i$, alors on dit que $W_0, \dotsc, W_n$ forment un \textbf{drapeau maximal invariant par $A$}. \end{déf}

		\begin{prp} Soient $V$ un espace vectoriel sur un corps commutatif $\K$ et $A \in \Hom(V, V)$. Alors $A$ est triangulable si et seulement si
		$A$ possède un drapeau maximal invariant. \end{prp}

		\begin{proof} Soit $\{W_i\}_{0 \leq i \leq n}$ un drapeau maximal invariant par $A$. Montrons que $A$ est triangulable. Pour tout $i \in \{1, \dotsc, n\}$,
		on prend $e_i \in W_i \setminus W_{i-1}$. Posons $E \coloneqq \{e_1, \dotsc, e_n\}$. Montrons que $E$ est une base de $V$ et que $m_{E, E}(A)$ est une
		matrice triangulaire.

		Montrons que $E$ est libre. Soit $1 \leq i \leq n$. Posons $E_i \coloneqq \{e_1, \dotsc, e_i\}$. Supposons que $E_i$ n'est pas libre. Alors il
		existe $\mu_1, \dots, \mu_{i-1} \in \K$ tels que $\mu_j \neq 0$ pour tout $j < i$ et~:
		\[e_i = \sum_{j=1}^{i-1}\mu_je_j.\]
		Puisque $W_i$ est un espace vectoriel pour tout $i$, on a $e_i \in W_{i-1}$. Or par définition, $e_i \in W_i \setminus W_{i-1}$. Il y a donc contradiction
		et $E_i$ est libre. Dès lors, $E = E_n$ est libre.

		Montrons que $E$ est génératrice. Pour cela, montrons par récurrence que $E_i$ est génératrice de $W_i$. Le cas de base est $i=1$ et $\{e_1\}$ est
		une base de $W_1$.

		Le pas de récurrence suppose que $E_{i-1}$ est génératrice de $W_{i-1}$. Montrons que $E_i$ est génératrice de $W_i$. Soit $v_i \in W_i$. On
		distingue deux cas : soit $v_i \in W_i \setminus W_{i-1}$, soit $v_i \in W_{i-1}$. Dans le second cas, par hypothèse de récurrence, on prend
		$\mu_1, \dotsc, \mu_{i-1} \in \K$ tels que $\mu_1e_1 + \dotsb + \mu_{i-1}e_{i-1} = v_i$ et prenons $\mu_i = 0$.

		Dans le premier cas, prenons $F = \{f_1, \dotsc, f_{i-1}\}$ une base de $W_{i-1}$, $F \cup \{v_i\}$ une base de $W_i$. On a alors $\mu_j = 0$
		pour $j \in \{1, \dotsc, i-1\}$ et $\mu_i = 1$.

		On a donc bien $E_i$ est génératrice de $W_i$. En particulier $E_n$ est génératrice de $W_n = V$. Donc $E$ est une base de $V$.

		Montrons maintenant que $m_{E, E}(A)$ est triangulaire.

		Soit $a \coloneqq [a_{i\,j}] = m_{E, E}(A) \in \M nn\K$. Soit $w_i = \sum_j\alpha_je_j \in W_i$. On sait que $A(w_i) \in W_i$ et donc~:
		\[W_i \ni \sum_{j=1}^i\alpha_iA(e_j).\]
		Ce qui implique $A(e_j) \in W_i$ pour tout $j \leq i$. On peut donc écrire~:
		\[W_i \ni A(e_j) = \sum_{k=1}^ia_{k\,j}e_k.\]
		Or, on sait également que $A(e_j) \in V$. Donc~:
		\[V \ni A(e_j) = \sum_{k=1}^na_{k\,j}e_k.\]
		On a alors l'égalité~:
		\[\begin{aligned}
			\sum_{k=1}^ia_{k\,j}e_k &= \sum_{k=1}^na_{k\,j}e_k \\
			\sum_{k=1}^ia_{k\,j}e_k &= \sum_{k=1}^ia_{k\,j}e_k + \sum_{k=i+1}^na_{k\,j}e_k \\
			0 &= \sum_{k=i+1}a_{k\,j}e_k.
		\end{aligned}\]

		De plus, on sait que $E = \{e_1, \dotsc, e_n\}$ est libre, donc $\{e_{i+1}, e_{i+2}, \dotsc, e_n\}$ est libre également. Alors la seule combinaison linéaire
		nulle possible est $a_{k\,j} = 0$ pour tout $j \in \{i+1, i+2, \ldots, n\}$. Dès lors, on sait que $m_{E, E}(A)_{i\,j} = 0$ pour tout $j > i$.
		La matrice est bien triangulaire.

		Maintenant soit $E = \{e_1, \dotsc, e_n\}$ une base de $V$ telle que $m_{E, E}(A) \eqqcolon [a_{i\,j}]$ est une matrice triangulaire. Montrons que
		$\{0\}  \subset \eng {e_1} \subset \eng {\{e_1, e_2\}} \subset \dotsb \subset \eng {\{e_1, \dotsc, e_n\}} = V$ est un drapeau maximal invariant par $A$.

		En posant $W_i \coloneqq \eng {\{e_1, \dotsc, e_i\}}$, on voit bien que $W_n = V$, et que $W_i \subset W_{i+1}$ pour tout $0 \leq i < n$. Et on sait que
		$W_i$ est un espace vectoriel par définition de l'espace engendré $\eng X$.
		\end{proof}

		\begin{prp} Soit $V$ un espace vectoriel de dimension finie $n$ sur $\C$. Soit $A \in \Hom(V,V)$. Alors $A$ est triangulable. \end{prp}

		\begin{rmq} La preuve est omise. \end{rmq}

	\subsection{Formule canonique de Jordan}
		\begin{thm}[Théorème de Jordan] Soit $V$ un espace vectoriel de dimension finie $n$ sur $\C$. Soit $A \in \Hom(V, V)$. Il existe $E$, une base de $V$
		telle que $a \coloneqq m_{E, E}(A)$ soit donnée par~:
		\[a = \begin{pmatrix}J_1 &  & 0 \\ & \ddots & \\0 & & J_k\end{pmatrix},\]
		avec $k \in \N^*$, et où, pour $\gamma \in \{1, 2, \ldots, k\}$, on a~:
		\[J_\gamma = \left[\begin{cases}\lambda_\gamma & \text{ si } i = j \\ 1 & \text{ si } j-i = 1 \\0 & \text{ sinon}\end{cases}\right] \in \M{n_\gamma}{n_\gamma}\C,\]
		où $\lambda_\gamma \in \C$ est une valeur propre de $A$ et $n_\gamma \in \N^*$.
		\end{thm}

		\begin{rmq} Les blocs $J_i$ d'une telle matrice sont appelés ses \textit{blocs de Jordan}. Lorsque la matrice $m_{E, E}(A)$ est exprimée dans une telle
		base, on dit qu'elle est dans sa \textit{forme (canonique) de Jordan}. \end{rmq}

		\begin{rmq}[bis] Le théorème de Jordan dit également que s'il existe une autre base $F \neq E$ respectant cette propriété, alors les blocs de Jordan de
		$m_{F, F}(A)$ sont une permutation des blocs de Jordan de $m_{E, E}(A)$. \end{rmq}

		\begin{cor} Soit $V$, un espace vectoriel de dimension finie $n$ sur $\C$. Soit $A \in \Hom(V, V)$. La matrice $A$ est inversible si et seulement si 
		sa forme de Jordan ne contient que des blocs de Jordan de taille 1. \end{cor}

		\begin{déf} Soit $V$ un espace vectoriel de dimension finie $n$ sur un corps commutatif $\K$. Soit $A \in \Hom(V, V)$. Une valeur propre $\lambda$
		de $A$ est dite de \textbf{multiplicité algébrique} $m$ si $(x-\lambda)^m | \pi_A(x) = \det(A - x\Id_{V^V})$ et $(x-\lambda)^{m+1} \not | \pi_A(x)$
		et est dit de \textbf{multiplicité géométrique} $m$ si $\dim_\K V_\lambda = m$. \end{déf}

		\begin{prp} Soient $V$ un espace vectoriel de dimension $n$ sur $\K$ un corps commutatif et $A \in \Hom(V, V)$. Si $\lambda \in \K$ est une valeur propre
		de $A$, alors la multiplicité géométrique de $\lambda$ est inférieure ou égale à sa multiplicité algébrique. \end{prp}

		\begin{proof} Soit $m$, la dimension de $V_\lambda$, et spot $E$ une base de $V$ telle que les $m$ premiers vecteurs sont dans $V_\lambda$.
		On sait alors que les $m$ première colonnes de $m_{E, E}(A)$ correspondent à $\lambda e_i$ pour $1 \leq i \leq m$ et qu'il existe deux matrices B, C
		de dimensions respectives $m \times (n-m)$ et $(n-m) \times (n-m)$. On a alors $\pi_A(x) = (x-\lambda)^m\pi_C(x)$. On sait que $m$ est la multiplicité
		géométrique de $\lambda$ par $A$. Soit $\beta$ la multiplicité arithmétique de $\lambda$ par $C$. On sait $\beta \in \N^*$. Donc on a bien que
		$(x-\lambda)^{m+\beta} | \pi_A(x)$ mais $(x-\lambda)^{m+\beta+1} \not | \pi_A(x)$. On a bien $m \leq m + \beta$. Or $m + \beta$ représente la multiplicité
		arithmétique de $\lambda$ par $A$. \end{proof}

	\subsection{Polynôme minimal d'un opérateur et propriétés des polynômes}
		\begin{déf} Soit $V$ un espace vectoriel de dimension $n$ sur un corps commutatif $\K$. Soit $A \in \Hom(V, V)$. Le \textbf{polynôme minimal de $A$}
		est l'unique polynôme monique de degré minimal $\mu_A(x) \in \K[x]$ tel que $\mu_A(A) = 0_{V^V} \in \Hom(V, V)$. \end{déf}

		\begin{prp} Tout endomorphisme sur un espace vectoriel admet un polynôme minimal. \end{prp}

		\begin{proof} Par le théorème de Cayley-Hamilton (théorème~\ref{Cayley-Hamilton}), on sait que $\pi_A(A) = 0 \in \Hom(V, V)$. On en déduit donc l'existence
		de $\mu_A(x)$. Son unicité se montre en supposant $m_1(x), m_2(x) \in \K[x]$ deux polynômes moniques distincts de degré minimal tels que
		$m_1(A) = m_2(A) = 0_{V^V}$. On a alors $\deg((m_1-m_2)(x)) < \deg(m_1(x)) = \deg(m_2(x))$. On a également~:
		\[m_1(A) - m_2(A) = 0_{V^V} - 0_{V^V} = 0_{V^V}.\]
		Donc $(m_1-m_2)(x)$ est un polynôme minimal de $A$. Or $m_1$ et $m_2$ sont supposés de degré minimal. Il y a donc contradiction et $m_1(x) = m_2(x)$.
		\end{proof}

		\begin{prp} Le polynôme minimal d'un endomorphisme divise son polynôme caractéristique. \end{prp}

		\begin{proof} Soit $A$, un endomorphisme sur un espace vectoriel $V$ de dimension $n$ sur un corps commutatif $\K$. Par Euclide, on sait qu'il existe
		$q(x), r(x) \in \K[x]$ tels que $\pi_A(x) = q(x)\mu_A(x) + r(x)$. Or, on sait que~:
		\[r(A) = \pi_A(A) q(A)\mu_A(A) = 0_{V^V} - q(A)0_{V^V} = 0_{V^V}.\]
		On a donc le fait que le reste est nul, et donc $\mu_A(x) | \pi_A(x)$. \end{proof}

		\begin{lem}\label{valPropreRacineMu} Les valeurs propres de $A \in \Hom(V, V)$ pour $V$ un espace vectoriel complexe de dimension $n$ sont les racines
		de son polynôme minimal $\mu_A(x) \in \K[x]$. \end{lem}

		\begin{proof} Supposons par l'absurde qu'il existe $\lambda$ une valeur propre de $A$ telle que $\mu_A(\lambda) \neq 0$. On prend $v \neq 0 \in V_\lambda$.
		On a alors~:
		\[0 \neq \mu_A(\lambda) = \sum_{i=0}^m\mu_i\lambda^i,\]
		et donc également~:
		\[\mu_A(A)(v) = \sum_{i=0}^m\mu_iA^i(v) = \sum_{i=0}^m\lambda^iv = \mu_A(\lambda)v \neq 0,\]
		où $A^i(v)$ représente la composée de $A$ avec lui-même $i$ fois. Et comme $v$ est un vecteur propre de $A$ dans le sous-espace propre $V_\lambda$, on
		sait que $A^i(v) = A^{i-1}(\lambda v) = A^{i-2}(\lambda^2v) = \ldots = \lambda^iv$. On a également~:
		\[\mu_A(A) = 0_{V^V},\]
		et donc~:
		\[\mu_A(A)(v) = 0_{V^V}(v) = 0.\]

		Donc on a une contradiction $0 \neq 0$. Il n'y a donc pas de tel $\lambda$. \end{proof}

		\begin{prp} Soit $A \in \Hom(V, V)$ un endomorphisme sur un espace vectoriel $V$ complexe de dimension $n$. Alors le polynôme minimal $m_A(x)$ est donné par~:
		\[m_A(x) = \prod_{i=1}^k(x-\lambda_i)^{m_i},\]
		où $k$ est le nombre de valeurs propres de $A$, et pour $i \in \{1, \dotsc, k\}$, on a $m_i \leq n_i$, la multiplicité algébrique de la valeur propre
		$\lambda_i$ dans $\mu_A$ telle que~: \[\sum_{i=1}^km_i = \deg(\mu_A(x)).\] \end{prp}

		\begin{proof} On sait par le lemme~\ref{valPropreRacineMu} que toutes les valeurs propres de $A$ sont des racines de $\mu_A$. Si on note $m_i \geq 1$ la
		multiplicité algébrique de la valeur propre $\lambda_i$ dans $\mu_A$, on a bien~:
		\[\mu_A(x) = \prod_{i=1}^k(x-\lambda_i)^{m_i}.\]

		Il faut également $m_i \leq n_i$ si $n_i$ est la multiplicité de $\lambda_i$ dans $\pi_A$ car il faut que $\mu_A$ divise $\pi_A$. Pour cette même raison, on
		ne peut avoir une racine de $\mu_A$ n'étant pas une racine de $\pi$. Et par définition, les racines de $\pi_A$ sont les valeurs propres.
		On a donc bien $m_1 + m_2 + \dotsb + m_k = \deg(\mu_A(x))$. \end{proof}

		\begin{déf} Soit $\K$ un corps commutatif. On dit que $f(x) \in \K[x]$ est un polynôme irréductible si $f$ n'est pas constant et si
		$\exists a(x), b(x) \in \K[x] \Rightarrow a(x)$ est constant ou $b(x)$ est constant. \end{déf}

		\begin{lem} Soient $p(x), q(x) \in \K[x]$ deux polynômes non nuls. Soient $k \in \N^*, w(x), f(x) \in \K$ tels que $q(x) = w(x)f(x)^k$ et $\GCD(w(x), f(x)) = 1$.
		Alors il existe $s(x), t(x) \in \K[x]$ uniques  tels que $\deg(s(x)) < \deg(f(x))$ et~:
		\[\frac {p(x)}{q(x)} = \frac {s(x)}{f(x)^k} + \frac {t(x)}{f(x)^{k-1}}.\] \end{lem}

		\begin{proof} En multipliant par $q(x)$ de part et d'autre, on obtient~:
		\begin{equation}\label{p(x)}
			p(x) = s(x)w(x) + t(x)f(x).
		\end{equation}
		Par Euclide, on sait qu'il existe $y(x), z(x) \in \K[x]$ tels que~:
		\begin{equation}\label{GCDwfEuclide}
			1 = \GCD(w(x), f(x)) = y(x)w(x) + z(x)f(x).
		\end{equation}
		En multipliant \eqref{GCDwfEuclide} par $p(x)$, on obtient~:
		\begin{equation}\label{p(x)Subst}
			p(x) = \left(p(x)y(x)\right)w(x) + \left(p(x)z(x)\right)f(x).
		\end{equation}
		Soient $Q(x), R(x) \in \K[x]$ avec $\deg(r(x)) < \deg(f(x))$ tels que~:
		\begin{equation}\label{euclide(py)(x)}
			p(x)y(x) = Q(x)f(x) + R(x).
		\end{equation}
		En réinsufflant \eqref{euclide(py)(x)} dans \eqref{p(x)Subst}, on obtient~:
		\begin{align*}
			p(x) &= \left(p(x)y(x)\right)w(x) + \left(p(x)z(x)\right)f(x) \\
			     &= \left(Q(x)f(x) + R(x)\right)w(x) + \left(p(x)z(x)\right)f(x) \\
				 &= \left(Q(x)w(x)\right)f(x) + R(x)w(x) + \left(p(x)z(x)\right)f(x) \\
				 &= R(x)w(x) + \left(Q(x)w(x) + p(x)z(x)\right)f(x),
		\end{align*}
		ce qui correspond bien à \eqref{p(x)}.

		La démonstration d'unicité se fait par l'absurde en supposant qu'il existe $s_1(x), s_2(x), t_1(x), t_2(x) \in \K[x]$ et avec le fait que
		$\GCD(f(x), w(x)) = 1$. \end{proof}

		\begin{prp}[Décomposition en fractions partielles] Soit $\K$, un corps commutatif. Soient $\lambda_1, \dotsc, \lambda_k \in \K$ et
		$p(x) \in \K[x]$ tels que~:
		\[p(x) = \prod_{j=1}^k(x-\lambda_j)^{m_j}\]
		pour $m_j \in \N^*$. Alors il existe $a \in \M n{m_i}\K$ tel que~:
		\[\frac 1{p(x)} = \sum_{i=1}^k\sum_{j=1}^{m_i}\frac {a_{i\,j}}{(x-\lambda_i)^j}.\] \end{prp}

		\begin{rmq} La démonstration est omise. \end{rmq}

	\subsection{Sous-espaces généralisés}
		\begin{prp} Soit $V$ un espace vectoriel de dimension $n$ sur un corps $\K$ et soit $A \in \Hom(V, V)$. Soient $\lambda_1, \dotsc, \lambda_k \in \K$
		les valeurs propres de $\mu_A$, le polynôme minimal de $A$ et $a_i] \in \K^{m_i}$ pour $i \in \{1, \dotsc, k\}$, les coefficients de sa décomposition en
		fractions partielles. En posant~:
		\[h_i(x) \coloneqq \sum_{j=1}^{m_i}a_{i\,j}(x-\lambda_i)^{m_i-j},\]
		et~:
		\[\psi_i(x) \coloneqq \prod_{j \neq i}(x-\lambda_j)^{m_j},\]
		on a~:
		\begin{equation}\label{1SurMu_A}
			\frac 1{\mu_A(x)} = \sum_{i=1}^k\frac {h_i(x)\psi_i(x)}{\mu_A(x)}.
		\end{equation}
		\end{prp}

		\begin{proof} En développant le calcul, on a~:
		\[
			\begin{aligned}
			\frac 1{\mu_A(x)} &= \sum_{i=1}^k\sum_{j=1}^{m_i}\frac {a_{i\,j}}{(x-\lambda_i)^j} \\
			                  &= \sum_{i=1}^k\sum_{j=1}^{m_i}\frac {a_{i\,j}(x-\lambda_i)^{m_i}}{(x-\lambda_i)^j(x-\lambda_i)^{m_i}} \\
			                  &= \sum_{i=1}^k\frac 1{(x-\lambda_i)^{m_i}}\sum_{j=1}^{m_i}a_{i\,j}(x-\lambda_i)^{m_i-j} \\
			                  &= \sum_{i=1}^k\frac 1{(x-\lambda_i)^{m_i}}h_i(x) \\
			                  &= \sum_{i=1}^k\frac 1{(x-\lambda_i)^{m_i}}h_i(x)\frac {\prod_{j \neq i}(x-\lambda_j)^{m_j}}{\prod_{j \neq i}(x-\lambda_j)^{m_j}} \\
			                  &= \sum_{i=1}^k\frac {h_i(x)\psi_i(x)}{\mu_A(x)}.
			\end{aligned}
		\]
		\end{proof}

		\begin{rmq} En multipliant \eqref{1SurMu_A} par $\mu_A(x)$ de part et d'autre, on obtient~:
		\begin{equation}\label{h_iPsi_i=1}
			1 = \sum_{i=1}^kh_i(x)\psi_i(x).
		\end{equation}
		\end{rmq}

		\begin{cor} En définissant $\C[x] \ni E_i \coloneqq h_i(x)\psi_i(x)$, on obtient~:
		\begin{equation}
			\sum_{i=1}^kE_i(A) = \Id_{\C[x]}.
		\end{equation}
		\end{cor}

		\begin{proof} On remarque~:
		\[
			0 = h_i(A)0_{\C[x]}(v) = h_i(A)\mu_A(A)(v) = h_i(A)\left(A - \lambda_i\Id_{\C[x]}\right)^{m_i}\psi_i(A)(v)
			  = \left(A - \lambda_i\Id_{\C[x]}\right)^{m_i}E_i(A)(v).
		\]

		Et donc, puisque $\left(A - \lambda_i\Id_{\C[x]}\right)^{m_i}$ divise $\psi_j(A)$ pour tout $j \neq i$, on sait également que
		$\left(A-\lambda_i\Id_{\C[x]}\right)^{m_i}$ divise $E_j(A) \coloneqq h_j(A)\psi_j(A)$. Dès lors, pour tout $i \neq j$, on a~:
		\begin{equation}
			E_i(A)E_j(A)(v) = 0.
		\end{equation}

		Dès lors, on sait, par \eqref{h_iPsi_i=1}~:
		\[E_i(A)(v) = E_i(A)(\Id_{\C[x]}(v)) = E_i(A)\sum_{j=1}^kE_j(A)(v) = E_i(A)^2(v).\]

		On a donc bien~:
		\[\Id_{\C[x]}(v) = \sum_{j=1}^kE_j(A)(v).\]
		\end{proof}

		\begin{prp} Soient $V$ un $\C$-espace vectoriel de dimension finie $n$ et $A \in \Hom(V, V)$. Alors~:
		\begin{equation}\label{V=bigoplusE_i}
			V = \bigoplus_{i=1}^kE_i(A)(V).
		\end{equation}
		\end{prp}

		\begin{proof} On remarque que~: \[V = \sum_{i=1}^kE_i(A)(V).\]
		Prenons $v \in E_i(A)(V) \cap \eng {\cup_{j \neq i}E_j(A)(V)}$. Alors pour $j \in \{1, \dotsc, k\}$, il existe $w_j \in V$ tels que~:
		\[v = E_i(A)(w_i) = \sum_{j \neq i}E_j(A)(w_j).\]

		Dès lors, pour $h \neq i$, on trouve~:
		\[0 = E_h(A)E_i(A)(v) = E_h(A)\sum_{j \neq i}E_j(A)(w_j) = E_h(A)^2(w_h) = E_h(A)(w_h).\]
		Par conséquent, on trouve également $v = E_i(A)(w_i) = 0$. \end{proof}

		\begin{prp} Si $V$ est un $\C$-espace vectoriel de dimension finie $n$ et $A \in \Hom(V, V)$, alors pour tout $i \in \{1, \dotsc, k\} :
		E_i(A)(V) \in \Ker\left((A-\lambda_i\Id_{\C[x]})^{m_i}\right)$. \end{prp}

		\begin{proof} On prend $v \in V$. On calcule ensuite~:
		\[0 = \mu_A(A)(v) = h_i\mu_A(A)(v) = h_i(A - \lambda_i)^{m_i}\psi_i(A)(v) = (A-\lambda_i)^{m_i}E_i(A)(v).\]

		On a donc $E_i(A)(v) \in \Ker(A - \lambda_i)^{m_i}$.

		Prenons maintenant $v \in \Ker(A - \lambda_i)^{m_i}$. On sait que $(x-\lambda_i)^{m_i}$ divise $E_j(x)$ pour $j \neq i$.
		Et donc, on trouve~: \[v = \sum_{j=1}^kE_j(A)(v) = E_i(A)(v).\]
		On trouve donc bien $E_i(A)(V) = \Ker(A-\lambda_i\Id_{\C[x]})^{m_i}$. \end{proof}

		\begin{rmq} On observe alors que l'écriture~\eqref{V=bigoplusE_i} est équivalente à~:
		\[V = \bigoplus_{i=1}^k\Ker(A - \lambda_i\Id_{\C[x]})^{m_i}.\]
		\end{rmq}

		\begin{déf} Soit $V$ un espace vectoriel complexe de dimension finie $n$. Soit $A \in \Hom(V, V)$ et $\lambda_1, \dotsc, \lambda_k$, les valeurs propres
		distinctes de $A$. On appelle le \textbf{sous-espace propre généralisé de $A$ associé à la valeur propre $\lambda_i$} la valeur~:
		\[V(\lambda_i) \coloneqq \Ker(A - \lambda_i \Id_{\C[x]})^{m_i}.\] \end{déf}

		\begin{rmq} On remarque que $V(\lambda)$ est un sous-espace de $V$ qui contient le sous-espace propre $V_\lambda$ de $A$. \end{rmq}

		\begin{prp} Soit $A \in \Hom(V, V)$ un opérateur linéaire sur un espace vectoriel complexe $V$ de dimension finie $n$. Soient $\{\lambda_i\}_{1 \leq i \leq k}$
		les valeurs propres distinctes de $A$. Alors~:
		\begin{equation}
			V = \bigoplus_{i=1}^kV(\lambda_i),
		\end{equation}
		et pour tout $i \in \{1, \dotsc, k\} :$
		\[\restr f{V(\lambda_i)} \in \Hom(V(\lambda_i), V(\lambda_i)).\]
		\end{prp}

	\subsection{Existence de la forme canonique de Jordan}
		

\end{document}
