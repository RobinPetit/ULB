\documentclass[10pt, mathserif]{beamer}

\usepackage[utf8]{inputenc}
\usepackage[T1]{fontenc}
\usepackage{amsmath, amsthm, amssymb}
\usepackage{mathtools}
\usepackage{mathdots}
\usepackage{commath}

\usepackage{helvet}

\usetheme{JuanLesPins}

\title{Aperçu de l'étude des nombres premiers}
\author{Bodart Corentin %
	\and Jonathan Dauwe %
	\and Azalais Davin %
	\and Thomas Lemaire %
	\and Robin Petit %
	\and Alex Ternes}
\date{4 mai 2016}
\institute{Université Libre de Bruxelles}

\DeclareMathOperator{\ord}{ord}

\newcommand{\tq}{\text{ t.q. }}
\newcommand{\Z}{\mathbb Z}
\newcommand{\N}{\mathbb N}
\newcommand{\eng}[1]{\left\langle#1\right\rangle}
\renewcommand{\mod}{\mbox{ mod }}

% amsmthm
\newtheorem{thm}{Théorème}[section]
\newtheorem{lem}[thm]{Lemme}
\renewcommand{\proofname}{\textit{Preuve}}
\theoremstyle{definition}
\newtheorem{déf}[thm]{Définition}
\theoremstyle{remark}
\newtheorem*{rmq}{Remarque}
\newtheorem*{rap}{Rappel}

\setbeamertemplate{navigation symbols}{\insertframenumber/\inserttotalframenumber}

\begin{document}

	\frame{\titlepage}

	\begin{frame}
		\frametitle{Plan de la présentation}
		\tableofcontents
	\end{frame}

\section{Existence d'une infinité de nombres premiers}
	\begin{frame}
		Euclide
	\end{frame}

\section{Nombres de Wilson}  % ajouter Mersenne ?
	\begin{frame}
		les nbres de Wilson
	\end{frame}
	
	%\subsection{Nombres premiers de Mersenne}
	%\begin{frame}
	%\end{frame}
	
\section{Théorèmes de Fermat et Euler}
	\begin{frame}
		\begin{block}{Objectif}
			Prouver le théorème suivant grâce à la théorie des groupes~:
		\end{block}

		\begin{thm}[Petit théorème de Fermat (1640)]
			Soient $a, p \in\Z$ tels que $p$ est premier et $a$ n'est pas divisible par $p$. Alors $a^{p-1} - 1$ est un multiple de $p$, c'est-à-dire~:
			\[a^{p-1}-1 \equiv 0 (\mod p).\]
		\end{thm}
	\end{frame}

	\begin{frame}
		\frametitle{Préliminaires à la preuve}
		\begin{déf}
			Soient $G$ un groupe et $a \in G$. L'\textbf{ordre} de $a$ est~:
			\begin{itemize}
				\item le plus petit $m \in \N^*$ tel que $a^m = e$, où $e$ est le neutre de $G$~;
				\item le cardinal du sous-groupe engendré par $a$, c-à-d~: $\eng a \coloneqq \{e = a^0, a^1, \dotsc, a^{m-1}\}$.
			\end{itemize}

			L'ordre de $a$ est noté $\ord(a)$.
		\end{déf}

		\begin{thm}[Thorème de Lagrange]
			Si $G$ est un groupe fini et $H \subseteq G$ un sous-groupe, alors $\abs H$ divise $\abs G$.
		\end{thm}
	\end{frame}

	\begin{frame}
		\begin{rap}[Petit théorème de Fermat]
			$\forall a, p \in \Z : (p \text{ premier } \land p \not | a) \Rightarrow (a^{p-1} -1 \equiv 0 (\mod p))$
		\end{rap}

		\begin{block}{Preuve du petit théorème de Fermat - partie 1/2}
			On considère le groupe $\left(\Z/p\Z^*, \cdot, 1\right)$. En effet, $\Z/p\Z^* = \{[1], [2], \dotsc, [p-1]\}$ car $p$ est premier. \\
			Soit maintenant $[a] \in \Z/p\Z^*$. $a$ n'est pas divisible par $p$ car $[0] = [p] = [kp] \not \in \Z/p\Z^*$ pour $k \in \Z$. \\
			Prenons $m \coloneqq \ord([a])$. Alors on sait~:
			\begin{itemize}
				\item $[a]^m = [1]$~;
				\item $\abs{\eng a} = m$.
			\end{itemize}

			Puisque $\eng a$ est un sous-groupe de $\Z/p\Z^*$, on sait par Lagrange que $m = \abs{\eng a}$ divise $\abs{\Z/p\Z^*} = p-1$.
		\end{block}
	\end{frame}

	\begin{frame}
		\begin{rap}[Petit théorème de Fermat]
			$\forall a, p \in \Z : (p \text{ premier } \land p \not | a) \Rightarrow (a^{p-1} -1 \equiv 0 (\mod p))$
		\end{rap}

		\begin{proof}[Preuve du petit théorème de fermat - partie 2/2]
			On sait $m | p-1$, que l'on peut réécrire comme suit~:
			\[\exists k \in \N \tq m \cdot k = p-1.\]

			Finalement, on a~:
			\[\left[a^{p-1}\right]\ = [a]^{p-1} = [a]^{mk} = \left([a]^m\right)^k = [1]^k = [1].\]
			On a effectivement $\left[a^{p-1}\right] = [1]$, ce qui signifie~:
			\[a^{p-1} \equiv 1 (\mod p).\]
		\end{proof}
	\end{frame}

	\begin{frame}
		\frametitle{Généralisation du petit théorème de Fermat}

		\begin{thm}[Théorème d'Euler (1761)]
			Soient $n \in \N^*$ et $a \in \N$ tels que $a, n$ soient premiers entre eux. Alors~:
			\[a^{\phi(n)} \equiv 1 (\mod n)\]
		\end{thm}

		\begin{proof}
			La démonstration est assez similaire à celle du théorème de Fermat : on prend le groupe $(\Z/n\Z^*, \cdot, 1)$ qui correspond aux classes
			d'entiers inversibles $\mod n$ et dont le cardinal vaut $\phi(n)$.
		\end{proof}
	\end{frame}

\section{Nombres de Carmichaël}
	\begin{frame}
		Carmichaël
	\end{frame}

\section{Les racines primitives}
	\begin{frame}
		racines primitives.
	\end{frame}

\section{Test de Lucas-Lehmer}
	\begin{frame}
		\frametitle{Nombres de Mersenne}
		\begin{déf}
			Un nombre de Mersenne (nommé selon Marin Mersenne, 16-17e siècle) est nombre sous la forme $M_n = 2^n-1$.
		\end{déf}

		\begin{lem}
			Soit $p \in \N^*$. Si $p$ est divisible par $m \in \N$, alors le nombre de Mersenne $M_m$ divise $M_p$.
		\end{lem}

		\begin{rmq}
			Ce lemme veut dire qu'il n'est pas nécessaire de tester la primalité de $M_n$ pour $n$ non premier car si $n$ n'est pas premier, alors $M_n$ ne
			l'est pas non plus. La réciproque n'est pas vraie. Exemple~: $p = 11$ est premier, or $M_p = 2^{11}-1 = 2047 = 23 \times 89$.
		\end{rmq}
	\end{frame}

	\begin{frame}
		\begin{proof}
			La preuve est uniquement calculatoire. Supposons qu'il existe $m, t \in \N \setminus \{1, p\}$ tels que $p = mt$. On a alors~:
			\[\begin{aligned}
				M_n &= 2^n-1 = 2^{mt}-1 = \left(2^m\right)^t - 1 = \sum_{i=0}^{t-1}\left(2^m\right)^i = \sum_{i=1}^t\left(2^{mi} - 2^{m(i-1)}\right) \\
				    &= 2^m\sum_{i=1}^t2^{m(i-1)}-\sum_{i=1}^t2^{m(i-1)} \\
				    &= (2^m-1)\sum_{i=1}^t2^{m(i-1)} = M_m\sum_{i=1}^t2^{m(i-1)}.\end{aligned}\]
		\end{proof}
	\end{frame}

	\begin{frame}
		\frametitle{Teste Lucas-Lehmer}
		Le test de Lucas-Lehmer permet de déterminer si un nombre de Mersenne est premier ou non. Il est basé sur la suite naturelle~:
		\[\begin{cases}
			L_0 &= 4 \\
			L_n &= (L_{n-1})^2 - 2 \;\text{ si } n \geq 1
		\end{cases}\]
		dont les premiers termes sont les suivants~:
		\[4, 14, 194, 37\,634, 1\,416\,317\,954, \ldots\]
	\end{frame}

	\begin{frame}
		\begin{thm}[Test de Lucas-Lehmer]
			Soit $p \in \N^*$. $M_p$ est premier si et seulement si $M_p$ divise $L_{p-2}$.
		\end{thm}

		\begin{rmq}
			Le théorème est une double implication. Il faut donc montrer les deux pour démontrer le théorème. Nous ne montrerons ici pas le fait que si $M_p$
			est premier, alors $M_p$ divise $L_{p-2}$.
		\end{rmq}

		\begin{lem}[Lemme préliminaire]
			Soit $G$ un groupe. Soit $a \in G$ un élément. Alors $\ord(a) \leq \abs G$.
		\end{lem}
	\end{frame}

	\begin{frame}
		\frametitle{Preuve - partie 1/3}
		Montrons que $L_{p-2} = kM_p \Rightarrow M_p$ premier. Une manière d'exprimer la divisibilité de $L_{p-2}$ par $M_p$ est de dire que
		$L_{p-2} \equiv 0 (\mod M_p)$.

		Premièrement, on remarque que l'on peut exprimer la suite $(L_n)$ définie récursivement comme une suite directe. Posons
		$\omega = 2 + \sqrt 3, \bar \omega = 2 - \sqrt 3$. On trouve dès lors $L_n = \omega^{2^n} + \bar \omega^{2^n}$. \\

		On suppose qu'il existe $k \in \N$ tel que~:
		\[L_{p-2} = \omega^{2^{p-2}} + \bar \omega^{2^{p-2}} = kM_p.\]
		En multipliant par $\omega^{2^{p-2}}$ des deux côtés et en réarrangeant les termes, on obtient~:
		\[\left(\omega^{2^{p-2}}\right)^2 \omega^{2^{p-1}} = kM_p\omega^{2^{p-2}} - \bar \omega^{2^{p-2}}\omega^{2^{p-2}} = 1.\]
	\end{frame}

	\begin{frame}
		\frametitle{Preuve - partie 2/3}
		Supposons par l'absurde que $M_p$ est composite (n'est pas premier). On prend donc $2 < q < M_p$ le plus petit diviseur premier de $M_p$. On prend
		alors $\Z_q \coloneqq \Z/q\Z$ l'ensemble des entiers modulo $q$, et on pose~:
		\[X \coloneqq \{a + b\sqrt 3 \tq a, b \in \Z_q\},\]
		où $\forall x = (a+b\sqrt 3), y = (c+d\sqrt 3) \in X : x \cdot y = (ac+3bd) + (ac+bd)\sqrt 3$. On pose $X^* = \{x \in X \tq \exists x^{-1} \in X\}$
		le groupe des éléments de $X$ admettant un inverse (preuve que $X^*$ est un groupe est omise). On sait que $0 \not \in X^*$, donc $\abs {X^*} \leq \abs X -1=q^2 - 1$.

		De plus, on sait $q > 2$, donc $\omega, \bar \omega \in X^*$. Également, $M_p \equiv 0 (\mod q)$, donc, dans $X$, on a~:
		\[kM_p\omega^{2^{p-2}} = 0.\]
	\end{frame}
	
	\begin{frame}
		\frametitle{Preuve - partie 3/3}
		On a vu que $\omega^{2^{p-1}} = kM_p\omega^{2^{p-2}} - 1 = 0 - 1 = -1$ dans $X$. En mettant au carré l'équation, on obtient~:
		\[\left(\omega^{2^{p-1}}\right)^2 = \omega^{2^p} = 1.\]
		Dès lors, on sait que $\omega \in X^*$ et est d'un ordre qui divise $2^p$. Or, $\ord(\omega)$ ne divise pas $\omega^{2^{p-1}}$. Donc $\ord(\omega) = 2^p$.
		Par le lemme préliminaire, on a~:
		\[2^p \leq \abs {X^*} \leq q^2 - 1.\]
		Et comme $q$ est un diviseur de $M_p$, on a $q^2 \leq M_p = 2^-1$. On a alors $2^p \leq 2^p - 1$, ce qui est une contradiction. Notre hypothèse disant que
		$M_p$ est composite est donc fausse. $M_p$ est bien premier.
	\end{frame}

\end{document}
