\documentclass[10pt, mathserif]{beamer}

\usepackage[utf8]{inputenc}
\usepackage[T1]{fontenc}
\usepackage{amsmath, amsthm, amssymb}
\usepackage{mathtools}
\usepackage{mathdots}
\usepackage{commath}

\usepackage{helvet}

\usetheme{JuanLesPins}

\title{Aperçu de l'étude des nombres premiers}
\author{ Corentin Bodart%
	\and Jonathan Dauwe %
	\and Azalais Davin %
	\and Thomas Lemaire %
	\and Robin Petit %
	\and Alex Ternes}
\date{4 mai 2016}
\institute{Université Libre de Bruxelles}

\DeclareMathOperator{\ord}{ord}

\newcommand{\tq}{\text{ t.q. }}
\newcommand{\Z}{\mathbb Z}
\newcommand{\N}{\mathbb N}
\renewcommand{\P}{\mathbb P}
\newcommand{\eng}[1]{\left\langle#1\right\rangle}

% amsmthm
\newtheorem{thm}{Théorème}[section]
\newtheorem{lem}[thm]{Lemme}
\renewcommand{\proofname}{\textit{Preuve}}
\theoremstyle{definition}
\newtheorem{déf}[thm]{Définition}
\theoremstyle{remark}
\newtheorem*{rmq}{Remarque}
\newtheorem*{rap}{Rappel}

\setbeamertemplate{navigation symbols}{\insertframenumber/\inserttotalframenumber}
\AtBeginSection{\frame{\sectionpage}}

\begin{document}

	\frame{\titlepage}

	\begin{frame}
		\frametitle{Plan de la présentation}
		\tableofcontents
	\end{frame}

\section{Théorèmes fondamentaux d'Euclide, de Wilson, d'Euler}
	\subsection{Existence d'une infinité de nombres premiers}
	\begin{frame}
		\begin{déf}
			Soient $a, b \in \N$. On dit que $a$ est divisible par $b$ si $\exists c \in \N$ tel que $a = b \cdot c$.
		\end{déf}

		\begin{block}{Propriétés}
			\begin{enumerate}
				\item si $n > 0$, il n'admet pas de diviseur supérieur à lui-même~;
				\item tout nombre naturel admet au moins un diviseur~;
				\item $2$ est le premier nombre premier.
			\end{enumerate}
		\end{block}
	\end{frame}

	\begin{frame}
		\frametitle{Infinité de nombres premiers}
		\begin{thm}[Théorème d'Euclide]
			Il existe une infinité de nombres premiers distincts.
		\end{thm}

		\begin{proof}
			Soit $\P \subset \N^*$ l'ensemble des nombres premiers. Supposons par l'absurde que $\P$ est fini et de cardinalité $n$.
			On note $N = \prod_{p \in \P}p$ le produit de tous les nombres de $\P$. Il y a dès lors deux cas distincts concernant $N+1$ :
			soit $N+1$ est premier, soit $N+1$ est composé. Dans le premier cas, il y a contradiction car $N+1$ est un nombre premier n'appartenant pas à $\P$. \\
			Dans le second cas, on prend $1 < Q < N+1$, un diviseur premier de $N+1$. Or, $Q \not \in \P$ car $\forall p \in \P : p | N$ et donc
			$p \not | N+1$. Il y a contradiction car $Q$ est premier mais n'est pas dans $\P$.

			Ces deux contradictions impliquent que la supposition est fausse, et donc $\P$ est infini.
		\end{proof}
	\end{frame}

	\begin{frame}
		\frametitle{Bézout et restes chinois}
		\begin{thm}[Théorème de Bachet-Bézout]
			Soient $a, b, x, y \in \Z$. L'équation suivante (en $x, y$)~: \[ax + by = 1\]
			admet des solutions si et seulement si $a$ et $b$ sont premiers entre eux.
		\end{thm}

		\begin{thm}[Théorème des restes chinois]
			Soient $m_1, \dotsc, m_k \in \N$ premiers deux à deux et $a_1, \dotsc, a_k \in \N$. Le système~:
			\[x \equiv a_i \pmod {m_i}\hspace{2cm}\text{ pour } i = 1, 2,  \dotsc, k\]
			possède une solution unique modulo $n \coloneqq \prod_{i=1}^km_i$.
		\end{thm}
	\end{frame}

	\begin{frame}
		\frametitle{Preuve du théorème des restes chinois}
		Preuve par récurrence. Commençons par le pas initial avec $k = 2$. \\
		Par Bézout, on sait qu'il existe $u_1, u_2 \in \Z \tq m_1u_1 + m_2u_2 = 1$. Donc il existe
		\[k_1, k_2 \in \Z \tq :\begin{cases}(m_2u_2)x &= (m_2u_2)m_1k_1 + (m_2u_2)a_1 \\ (m_1u_1)x &= (m_1u_1)m_2k_2 + (m_1u_1)a_2\end{cases}.\]
		Ce qui est équivalent à~:
		\[\begin{aligned}
			(m_1u_1 + m_2u_2)x &= m_1m_2(u_1k_2 + u_2k_1) + m_1u_1a_2 + m_2u_2a_1 \\
			1 \cdot x &\equiv m_1u_1a_2 + m_2u_2a_1 \pmod {m_1m_2}
		\end{aligned}\]
		
		Pour le pas de récurrence, prenons $k \gneqq 2$. Notons $M \coloneqq \prod_{i=1}^{k-1}m_i$. On sait que $M$ et $m_n$ sont premiers entre eux.
		On peut donc dire $x \equiv a_n \pmod {m_n}$ et $x \equiv a_M \pmod {M}$ si et seulement si $x$ a une solution unique modulo $n \coloneqq m_nM$.
	\end{frame}

	\subsection{Nombres de Wilson}
	\begin{frame}
		\begin{thm}[Théorème de Wilson]
			Soit $p > 1$ un nombre entier. $p$ est premier si et seulement si $(p-1)! + 1 \equiv 0 \pmod p$
		\end{thm}

		\begin{block}{Propriétés préparatoires}
			\begin{enumerate}
				\item si $p$ est un nombre premier et $a \in \{1, \dotsc, p-1\}$, alors il existe un unique $b$ (modulo $p$) tel que $a \cdot b \equiv 1 \pmod p$~;
				\item si $p$ est premier, alors $a \times a \equiv 1 \pmod d$ si et seulement si $a \equiv 1 \pmod p$ ou $a \equiv p-1 \pmod p$
			\end{enumerate}
		\end{block}

		\begin{block}{Preuve du sens direct du théorème}
			Par les propriétés ci-dessus, on a immédiatement~:
			\[(p-1)! \equiv (p-1) \pmod p \equiv -1 \pmod p.\]
		\end{block}
	\end{frame}

	\begin{frame}
		\begin{proof}[Preuve du sens indirect du théorème]
			Faisons la preuve par contraposée~: montrons que si $p$ n'est pas premier, alors on a $(p-1)! \not \equiv -1 \pmod p$.

			\underline{Un cas particulier~:} prenons $p = 4$. On trouve alors~:
			\[(p-1)! = 3! = 6 \equiv 2 \pmod 4.\]

			\underline{pas de récurrence~:} soit $p > 4$, un nombre composé. Soient $a, b \in \N^* \tq p = ab$. Deux cas sont à distinguer~:
			\begin{itemize}
				\item si $a \neq b$, en calculant $(p-1)!$, on peut regrouper $a$ et $b$ (en les mettant en évidence). On a alors~:
				$(p-1)! = a \cdot b \cdot 1 \dotsm (p-1) \equiv 0 \pmod p$~;
				\item si $a = b$, alors $p$ est un carré parfait tel que $p = a^2 > 2a$. Par conséquent, $a$ et $2a$ apparaissent tous deux dans $(p-1)!$
				et on peut mettre en évidence~: $(p-1)! = a \cdot 2a \cdot 1 \dotsm (p-1) = 2(a \cdot b)1 \dotsm (p-1) \equiv 0 \pmod p$.
			\end{itemize}
		\end{proof}
	\end{frame}
	
	\subsection{Fonction indicatrice d'Euler}
	\begin{frame}
		\begin{déf}
			Soit $\varphi : \N \to \N^* : n \mapsto \varphi(n) = \abs {\left\{m \in \N \tq (m < n) \land (\gcd(m, n) = 1)\right\}}$. On appelle $\varphi$ la fonction
			indicatrice d'Euler.
		\end{déf}

		%\begin{lem}
		%\end{lem}
	\end{frame}
	
\section{Théorèmes de Fermat et Euler}
	\begin{frame}
		\begin{block}{Objectif}
			Prouver le théorème suivant grâce à la théorie des groupes~:
		\end{block}

		\begin{thm}[Petit théorème de Fermat (1640)]
			Soient $a, p \in\Z$ tels que $p$ est premier et $a$ n'est pas divisible par $p$. Alors $a^{p-1} - 1$ est un multiple de $p$, c'est-à-dire~:
			\[a^{p-1}-1 \equiv 0 \pmod p.\]
		\end{thm}
	\end{frame}

	\begin{frame}
		\frametitle{Préliminaires à la preuve}
		\begin{déf}
			Soient $G$ un groupe et $a \in G$. L'\textbf{ordre} de $a$ est~:
			\begin{itemize}
				\item le plus petit $m \in \N^*$ tel que $a^m = e$, où $e$ est le neutre de $G$~;
				\item le cardinal du sous-groupe engendré par $a$, c-à-d~: $\eng a \coloneqq \{e = a^0, a^1, \dotsc, a^{m-1}\}$.
			\end{itemize}

			L'ordre de $a$ est noté $\ord(a)$.
		\end{déf}

		\begin{thm}[Thorème de Lagrange]
			Si $G$ est un groupe fini et $H \subseteq G$ un sous-groupe, alors $\abs H$ divise $\abs G$.
		\end{thm}
	\end{frame}

	\begin{frame}
		\begin{rap}[Petit théorème de Fermat]
			$\forall a, p \in \Z : (p \text{ premier } \land p \not | a) \Rightarrow \left(a^{p-1} -1 \equiv 0 \pmod p\right)$
		\end{rap}

		\begin{block}{Preuve du petit théorème de Fermat - partie 1/2}
			On considère le groupe $\left(\Z/p\Z^*, \cdot, 1\right)$. En effet, $\Z/p\Z^* = \{[1], [2], \dotsc, [p-1]\}$ car $p$ est premier. \\
			Soit maintenant $[a] \in \Z/p\Z^*$. $a$ n'est pas divisible par $p$ car $[0] = [p] = [kp] \not \in \Z/p\Z^*$ pour $k \in \Z$. \\
			Prenons $m \coloneqq \ord([a])$. Alors on sait~:
			\begin{itemize}
				\item $[a]^m = [1]$~;
				\item $\abs{\eng a} = m$.
			\end{itemize}

			Puisque $\eng a$ est un sous-groupe de $\Z/p\Z^*$, on sait par Lagrange que $m = \abs{\eng a}$ divise $\abs{\Z/p\Z^*} = p-1$.
		\end{block}
	\end{frame}

	\begin{frame}
		\begin{rap}[Petit théorème de Fermat]
			$\forall a, p \in \Z : (p \text{ premier } \land p \not | a) \Rightarrow \left(a^{p-1} -1 \equiv 0 \pmod p\right)$
		\end{rap}

		\begin{proof}[Preuve du petit théorème de fermat - partie 2/2]
			On sait $m | p-1$, que l'on peut réécrire comme suit~:
			\[\exists k \in \N \tq m \cdot k = p-1.\]

			Finalement, on a~:
			\[\left[a^{p-1}\right]\ = [a]^{p-1} = [a]^{mk} = \left([a]^m\right)^k = [1]^k = [1].\]
			On a effectivement $\left[a^{p-1}\right] = [1]$, ce qui signifie~:
			\[a^{p-1} \equiv 1 \pmod p.\]
		\end{proof}
	\end{frame}

	\begin{frame}
		\frametitle{Généralisation du petit théorème de Fermat}

		\begin{thm}[Théorème d'Euler (1761)]
			Soient $n \in \N^*$ et $a \in \N$ tels que $a, n$ soient premiers entre eux. Alors~:
			\[a^{\phi(n)} \equiv 1 \pmod n\]
		\end{thm}

		\begin{proof}
			La démonstration est assez similaire à celle du théorème de Fermat : on prend le groupe $(\Z/n\Z^*, \cdot, 1)$ qui correspond aux classes
			d'entiers inversibles $\pmod n$ et dont le cardinal vaut $\varphi(n)$.
		\end{proof}
	\end{frame}

\section{Nombres de Carmichaël}
	\begin{frame}
		\begin{déf}
			Un nombre de Carmichaël est un nombre  qui ne respecte pas la réciproque du petit théorème de Fermat
		\end{déf}

		\begin{block}{Réciproque du petit théorème de Fermat}
			$\left(\forall a \tq \gcd(a, p) = 1 : a^{p-1}-1 \equiv \pmod p\right) \Rightarrow p$ est premier
		\end{block}
	\end{frame}

	\begin{frame}
		\frametitle{Exemple de nombre de Carmichaël - partie 1/2}
		Exemple d'un nombre de Carmichaël~: prenons $p = 561$. On sait que $p = 3 \times 11 \times 17$ est un nombre composé. On remarque que $3-1$, $11-1$,
		et $17-1$ divisent $p-1$~:
		\[\frac {560}{3-1} = \frac {560}2 = 280 \;;\; = \frac {560}{11-1} = \frac {560}{10} = 56 \;;\; \frac {560}{17-1} = \frac {560}{16} = 35.\]
		Et donc~:
		\[561 \equiv 1 \begin{cases}\pmod {3-1} \\ \pmod {11-1} \\ \pmod {17-1}.\end{cases}\]
	\end{frame}

	\begin{frame}
		\frametitle{Exemple de nombre de Carmichaël - partie 2/2}
		Prenons maintenant $a$ non-divisible par $3$, $11$, et $17$. Dès lors, par le petit théorème de Fermat, on a~:
		\begin{enumerate}
			\item $a^{561-1} = \left(a^{280}\right)^{3-1} = (a^{(1)})^{3 -1} \equiv 1 \pmod {3}$~;
			\item $a^{561-1} = \left(a^{56}\right)^{11-1} = (a^{(2)})^{11-1} \equiv 1 \pmod {11}$~;
			\item $a^{561-1} = \left(a^{35}\right)^{17-1} = (a^{(3)})^{17-1} \equiv 1 \pmod {17}$.
		\end{enumerate}

		Et par les règles de calculs de congruence, on obtient finalement~:
		\[a^{561-1} \equiv 1 \pmod {(3 \times 11 \times 17)} = 1 \pmod {561}.\]
	\end{frame}

\section{Les racines primitives}
	\subsection{Lemmes préliminaires et démonstrations}
	\begin{frame}
		\frametitle{Lemme 1}
		\begin{lem}[1]
			Soit $n$, un entier naturel. $\sum_{d \mid n} \varphi(d) = n$.
		\end{lem}

		\begin{block}{Preuve du lemme (1) - partie 1/2}
			On effectue un double comptage~:
			\[\begin{aligned}
			n &= \#\left\{1,\,2,\ldots,\,n\right\} = \#\bigcup_{d\mid n}\left\{1\le x\le n \mid \gcd(n,x)=d\right\} \\
			  &= \sum_{d\mid n}\#\left\{1\le x\le n\mid \gcd(n,x)=d\right\} \\
			  &= \sum_{d\mid n}\#\left\{1\le \frac{x}{d}\le \frac{n}{d}\mid \gcd\left(\frac{n}{d},\frac{x}{d}\right)=1 \right\} \\
			  &= \sum_{d\mid n}\varphi\left(\frac{n}{d}\right) = \sum_{d'\mid n}\varphi(d')
			\end{aligned}\]
		\end{block}
	\end{frame}

	\begin{frame}
		\begin{proof}[Preuve du lemme (1) - partie 2/2]
			En effet,
			\begin{itemize}
				\item le plus grand diviseur commun de $n$ et $x$ est un diviseur de $n$~;
				\item puisque $\gcd(n,x)$ est bien défini, ces ensembles sont disjoints~;
				\item $\gcd(n,x)=d\Longleftrightarrow \gcd\left(\frac{n}{d},\frac{x}{d}\right)=1$~;
				\item par définition, $\varphi(d)=\#\{1\le x\le d\mid \gcd(d,x)=1\}$~;
				\item en prenant $d\cdot d'=n$.
			\end{itemize}
		\end{proof}
	\end{frame}

	\begin{frame}
		\frametitle{Lemme 2}
		\begin{lem}[2]
			Soit $p$, un nombre premier et $d$, un diviseur de $p-1$. On prouve que
			\[X^d - 1 = 0 \pmod p\]
			a exactement $d$ racines (distinctes) dans $\Z/p\Z$.
		\end{lem}
	\end{frame}

	\begin{frame}
		\begin{block}{Preuve du lemme (2) - partie 1/2}
			On travaille sur le corps $\Z/p\Z$. On rappelle que $A \cdot B = 0\Longleftrightarrow A = 0 \lor B=0$.

			De plus, par la division euclidienne des polynômes, on sait qu'un polynôme de degré $n$ a au plus $n$ racines.

			Par le Petit Théorème de Fermat, on sait que~: \[X^{p-1}-1 = 0\] a exactement $p-1$ racines (les résidus inversibles/premier avec p). De plus,
			\[X^{p-1} - 1 = \left(X^{d}-1\right) \cdot \left(X^{\frac{p-1}{d} \cdot (d-1)} + X^{\frac{p-1}{d} \cdot (d-2)} + \dotsb + X^{\frac{p-1}{d}}+1\right).\]
			Ainsi, pour tout $x \neq 0$,
			\[p(x) = x^d-1 = 0 \lor q(x) = x^{\frac{p-1}{d} \cdot (d-1)} + \dotsb + 1 = 0.\]
		\end{block}
	\end{frame}

	\begin{frame}
		\begin{proof}[Preuve du lemme (2) - partie 2/2]
			On a donc~:
			\[\begin{aligned}
				p-1 &\leq \#\{x \tq p(x) = 0\} + \#\{x \tq q(x) = 0\} \\
				    & \leq d + \frac{p-1}{d} \cdot(d-1) \\
					&= p-1.
			\end{aligned}\]

			Finalement, les inégalités sont des égalités : $\#\{x \tq p(x) = 0\} = d$. On a donc exactement $d$ solutions distinctes à l'équation $X^d-1=0$.
		\end{proof}
	\end{frame}

	\begin{frame}
		\begin{thm}[Existence d'une racine primitive]
			Soit $p$, un nombre premier. Alors il existe un élément $g$ de $\Z/p\Z$ tel que l'ordre de $g$ est égal à $p-1$ c'est-à-dire~:
			\[g^{p-1}-1 \equiv 0 \land g^{d}-1 \not \equiv 0 \quad \forall d \mid p-1.\]
		\end{thm}

		\begin{block}{Preuve du théorème - partie 1/2}
			 On démontre un résultat plus général~: si $d \mid p-1$, il existe $\varphi(d)$ éléments de $\Z/p\Z$ dont l'ordre est égal à $d$. On résonne par
			 «~récurrence forte~» sur les diviseurs de $p-1$.
			 
			 \underline{Initiation~: $d = 1$}. On a bien $\varphi(1) = 1$ élément d'ordre 1~:
			 \[x^1 - 1 \equiv 0 \iff x \equiv 1.\]
			 
			 \underline{Récurrence~:} soit $d$, un diviseur de $p-1$. Supposons que le résultat soit vrai pour tout $d' < d$, diviseur de $p-1$.
		\end{block}
	\end{frame}

	\begin{frame}
		\begin{proof}[Preuve du théorème - partie 2/2]
			Par le second lemme, il existe $d$ solutions à l'équation $X^d-1\equiv 0$. De plus, comme vu dans la preuve du théorème de Euler/Fermat,
			si $x^d - 1 \equiv 0$, alors l'ordre de $x$ modulo $p$ divise $d$. Ainsi~:
			\[d = \sum_{d'\mid d}\left(\text{nombre de solution d'ordre d'}\right).\]
			
			Par hypothèse de récurrence, le nombre de solutions d'ordre $d'$ est égal à $\varphi(d')$ pour $d' < d$. Dès lors,
			\[\text{nombre de solutions d'ordre d} = d - \sum_{d' \mid d \land d' < d}\varphi(d').\]

			Enfin, puisque $\sum_{d'\mid d}\varphi(d') = d$, le nombre de solutions d'ordre $d$ est bien $\varphi(d)$.
			
			Finalement, en prenant $d = p-1$, on a bien $\varphi(p-1) \ge 1$ racines primitives dans $\Z/p\Z$.
		\end{proof}
	\end{frame}

\section{Mersenne et Lucas-Lehmer}
	\subsection{Introduction aux nombres de Mersenne}
	\begin{frame}
		\frametitle{Nombres de Mersenne}
		\begin{déf}
			Un nombre de Mersenne (nommé selon Marin Mersenne, 16-17e siècle) est nombre sous la forme $M_n = 2^n-1$.
		\end{déf}

		\begin{lem}
			Soit $p \in \N^*$. Si $p$ est divisible par $m \in \N$, alors le nombre de Mersenne $M_m$ divise $M_p$.
		\end{lem}

		\begin{rmq}
			Ce lemme veut dire qu'il n'est pas nécessaire de tester la primalité de $M_n$ pour $n$ non premier car si $n$ n'est pas premier, alors $M_n$ ne
			l'est pas non plus. La réciproque n'est pas vraie. Exemple~: $p = 11$ est premier, or $M_p = 2^{11}-1 = 2047 = 23 \times 89$.
		\end{rmq}
	\end{frame}

	\begin{frame}
		\begin{proof}
			La preuve est uniquement calculatoire. Supposons qu'il existe $m, t \in \N \setminus \{1, n\}$ tels que $n = mt$. On a alors~:
			\[\begin{aligned}
				M_n &= 2^n-1
				    = 2^{mt}-1
				    = \left(2^m\right)^t - 1
					= \frac {\left(2^m\right)^t - 1}{2^m - 1}\left(2^m - 1\right) \\
					&= \sum_{k=0}^{t-1}\left(2^m - 1\right)\left(2^m\right)^k
					= \left(2^m-1\right)\sum_{k=0}^{t-1}2^{mk}
					= M_m\sum_{k=0}^{t-1}2^{mk}
			\end{aligned}\]

			Où, par la formule de la somme d'une suite géométrique, on a~:
			\[\sum_{i=0}^{n}u_n = u_0\frac {q^{n+1}-1}{q-1},\]
			avec $u_k = u_0q^k$.
		\end{proof}
	\end{frame}

	\subsection{Test de Lucas-Lehmer}
	\begin{frame}
		\frametitle{Test Lucas-Lehmer}
		Le test de Lucas-Lehmer permet de déterminer si un nombre de Mersenne est premier ou non. Il est basé sur la suite naturelle~:
		\[\begin{cases}
			L_0 &= 4 \\
			L_n &= (L_{n-1})^2 - 2 \;\text{ si } n \geq 1
		\end{cases}\]
		dont les premiers termes sont les suivants~:
		\[4, 14, 194, 37\,634, 1\,416\,317\,954, \ldots\]
	\end{frame}

	\begin{frame}
		\begin{thm}[Test de Lucas-Lehmer]
			Soit $p \in \N^*$. $M_p$ est premier si et seulement si $M_p$ divise $L_{p-2}$.
		\end{thm}

		\begin{rmq}
			Le théorème est une double implication. Il faut donc montrer les deux pour démontrer le théorème. Nous ne montrerons ici pas le fait que si $M_p$
			est premier, alors $M_p$ divise $L_{p-2}$.
		\end{rmq}

		\begin{lem}[Lemme préliminaire]
			Soit $G$ un groupe. Soit $a \in G$ un élément. Alors $\ord(a) \leq \abs G$.
		\end{lem}
	\end{frame}

	\begin{frame}
		\frametitle{Preuve - partie 1/3}

		Montrons que $L_{p-2} = kM_p \Rightarrow M_p$ premier. Une manière d'exprimer la divisibilité de $L_{p-2}$ par $M_p$ est de dire que
		$L_{p-2} \equiv 0 \pmod {M_p}$.

		Premièrement, on remarque que l'on peut exprimer la suite $(L_n)$ définie récursivement comme une suite directe. Posons
		$\omega = 2 + \sqrt 3, \bar \omega = 2 - \sqrt 3$. On trouve dès lors $L_n = \omega^{2^n} + \bar \omega^{2^n}$. \\

		On suppose qu'il existe $k \in \N$ tel que~:
		\[L_{p-2} = \omega^{2^{p-2}} + \bar \omega^{2^{p-2}} = kM_p.\]
		En multipliant par $\omega^{2^{p-2}}$ des deux côtés et en réarrangeant les termes, on obtient~:
		\[\left(\omega^{2^{p-2}}\right)^2 = \omega^{2^{p-1}} = kM_p\omega^{2^{p-2}} - \bar \omega^{2^{p-2}}\omega^{2^{p-2}} = kM_p\omega^{2^{p-2}} - 1.\]
	\end{frame}

	\begin{frame}
		\frametitle{Preuve - partie 2/3}

		Supposons par l'absurde que $M_p$ est composé (n'est pas premier). On prend donc $2 < q < M_p$ le plus petit diviseur premier de $M_p$. On prend
		alors $\Z_q \coloneqq \Z/q\Z$ l'ensemble des entiers modulo $q$, et on pose~:
		\[X \coloneqq \{a + b\sqrt 3 \tq a, b \in \Z_q\},\]
		où $\forall x = (a+b\sqrt 3), y = (c+d\sqrt 3) \in X :$
		\[x \cdot y = \left((ac+3bd) \mod q\right) + \left((ac+bd) \mod q\right)\sqrt 3.\]
		On pose $X^* = \{x \in X \tq \exists x^{-1} \in X\}$ le groupe des éléments de $X$ admettant un inverse (preuve que $X^*$ est un groupe est omise).
		On sait que $0 \not \in X^*$, donc $\abs {X^*} \leq \abs X -1=q^2 - 1$.

		De plus, on sait $q > 2$, donc $\omega, \bar \omega \in X^*$. Également, $M_p \equiv 0 \pmod q$, donc, dans $X$, on a~:
		\[kM_p\omega^{2^{p-2}} = 0.\]
	\end{frame}
	
	\begin{frame}
		\frametitle{Preuve - partie 3/3}

		On a vu que $\omega^{2^{p-1}} = kM_p\omega^{2^{p-2}} - 1 = 0 - 1 = -1$ dans $X$. En mettant au carré l'équation, on obtient~:
		\[\left(\omega^{2^{p-1}}\right)^2 = \omega^{2^p} = 1.\]
		Dès lors, on sait que $\omega \in X^*$ et est d'un ordre qui divise $2^p$. Or, $\ord(\omega)$ ne divise pas $2^{p-1}$. Donc $\ord(\omega) = 2^p$.
		Par le lemme préliminaire, on a~:
		\[2^p \leq \abs {X^*} \leq q^2 - 1.\]
		Et comme $q$ est un diviseur de $M_p$, on a $q^2 \leq M_p = 2^p-1$. On a alors $2^p \leq 2^p - 2$, ce qui est une contradiction. Notre hypothèse disant que
		$M_p$ est composé est donc fausse. $M_p$ est bien premier.
		
		\begin{flushright}$\square$\end{flushright}
	\end{frame}

	\begin{frame}
		\frametitle{Remerciements et questions}
	\end{frame}

\end{document}
