\documentclass{report}

\usepackage[french]{babel}
\usepackage{commath}
\usepackage{palatino, eulervm}
\usepackage[utf8]{inputenc}
\usepackage[T1]{fontenc}
\usepackage{fullpage}
\usepackage{hyperref}

\usepackage{amssymb}
\usepackage{amsmath}
\usepackage{amsfonts}
\usepackage{amsthm}
\usepackage{mathtools}
\usepackage{stmaryrd}

\usepackage[parfill]{parskip}

\newcommand{\C}{{\mathbb C}}
\newcommand{\R}{{\mathbb R}}
\newcommand{\N}{{\mathbb N}}
\newcommand{\scpr}[2]{\left\langle#1, #2\right\rangle}
\newcommand{\tq}{\text{ t.q. }}
\newcommand{\pinfty}{{+\infty}}
\newcommand{\minfty}{{-\infty}}
\newcommand{\intint}[2]{\left\llbracket#1, #2\right\rrbracket}
\newcommand{\cste}{\text{c}^{\text{ste}}}
\newcommand{\dx}{\dif x}
\newcommand{\Id}{\mathrm {Id}}

\newcommand{\TODO}{TODO}

\newtheorem{thm}{Théorème}[chapter]
\newtheorem{prp}[thm]{Proposition}
\newtheorem{cor}[thm]{Corollaire}
\newtheorem{lem}[thm]{Lemme}
\addto\captionsfrench{\renewcommand\proofname{\underline{Démonstration}}}
\theoremstyle{definition}
\newtheorem{déf}[thm]{Définition}
\theoremstyle{remark}
\newtheorem*{rmq}{Remarque}
\newtheorem{ex}{Exemple}[chapter]

\title{Espaces fonctionnels et séries de Fourier --- Notes de cours de Pr. P. Godin}
\author{Robin Petit}
\date{Année académique 2017-2018}

\begin{document}

\pagenumbering{Roman}
\maketitle
\tableofcontents
\setcounter{page}{1}
\pagenumbering{arabic}

\chapter{Transformation de Fourier}

\section{Définitions}

On considère $\R^n$ à $n$ fixé en tant qu'espace de mesure $(\R^n, \mathcal M, \lambda)$ avec $\mathcal M$ la famille des ensembles Lebesgue-mesurables et $\lambda$
la mesure de Lebesgue sur $\R^n$.

\begin{déf} Pour $f \in L^1(\R^n)$, on définit sa \textit{transformée de Fourier} par~:
\begin{equation}
	\hat f : \R^n \to \R : \xi \mapsto \int e^{-i\scpr x\xi}f(x)\dif x.
\end{equation}
\end{déf}

Cette fonction est bien définie car $x \mapsto e^{-i\scpr x\xi}$ est continue, et $f$ est intégrable, donc $x \mapsto e^{-i\scpr x\xi}f(x)$ est intégrable

\begin{prp} Pour $f \in L^1$, $\hat f$ est continue.
\end{prp}

\begin{proof} Soient $\xi_0 \in \R^n$ et $(h_k)_{k \in \N} \in {\R^n}^\N \tq h_k \xrightarrow[k \to \pinfty]{} 0$.
\[\hat f(\xi_0 + h_k) = \int e^{-i\scpr x{\xi_0}}e^{-i\scpr x{h_k}}f(x)\dif x.\]

Puisque $\abs {e^{-i\scpr x{\xi_0}}e^{-i\scpr x{h_k}}f(x)} = \abs {f(x)}$ et $x \mapsto e^{-i\scpr x{\xi_0}}e^{-i\scpr x{h_k}}f(x)$ converge partout
(en particulier presque partout), par le théorème de la convergence dominée~:
\[\hat f(\xi_0 + h_k) \xrightarrow[k \to \pinfty]{} \int e^{-i\scpr x{\xi_0}}f(x)\dif x = \hat f(\xi_0).\]
\end{proof}

\begin{prp} Pour $f \in L^1 \tq \forall j \in \intint 1n : x_jf \in L^1$~: $\hat f \in C^1$ et~:
\begin{equation}\label{eq:pd transformée Fourier}
	\forall j \in \intint 1n : \pd {\hat f}{\xi_j}\sVert[3]_\xi = \int e^{-i\scpr x\xi}(-ix_j)f(x)\dif x.
\end{equation}
\end{prp}

\begin{proof} Soit $(h_k)_{k \in \N} \in \R^\N \tq h_k \xrightarrow[k \to \pinfty]{} 0$, et prenons $\{e_j\}_{j=1}^n$ la base canonique de $\R^n$.
\[\frac {\hat f(\xi + h_ke_j) - \hat f(\xi)}{h_k} = \int e^{-i\scpr x\xi}\underbrace {\frac {e^{-ih_kx_j}-1}{h_k}}_{\xrightarrow[k \to \pinfty]{} -ix_j}\dif x.\]

En module~:
\[\abs {e^{-i\scpr x\xi}\frac {e^{-ih_kx_j}-1}{h_k}f(x)} = \abs {e^{-i\scpr x\xi}}\abs {\frac {e^{-ih_kx_j}-1}{h_k}}\abs {f(x)} \leq C\abs {x_j}\abs {f(x)},\]
qui est intégrable par hypothèse.

En effet, si $x_j = 0$, alors tout est nul et l'inégalité devient une égalité~; et si $x_j \neq 0$, alors $\frac {\abs {e^{-ih_kx_j}-1}}{\abs {x_jh_k}}$ est borné.

Dès lors, par le théorème de convergence dominée, la limite passe sous l'intégrale et on a~\eqref{eq:pd transformée Fourier}.
\end{proof}

\begin{cor} Pour $m \in \N^*$, si $(1+\abs x)^mf \in L^1$, alors $f \in C^m$ et on peut dériver $m$ fois sous le signe.
\end{cor}

\begin{proof} Exercice (récurrence sur $m$).
\end{proof}

\begin{déf} On définit l'ensemble de Schwartz~:
\begin{equation}
	\mathcal S(\R^n) \coloneqq \left\{u \in C^\infty(\R^n) \tq \forall \alpha, \beta \in \N^n : x^\alpha\partial^\beta u \text{ est borné dans } \R^n\right\}.
\end{equation}
\end{déf}

\begin{prp} $\mathcal S(\R^n)$ est un $\R$-espace vectoriel.
\end{prp}

\begin{proof} \TODO
\end{proof}

\begin{prp} $C^\infty_0(\R^n) \subseteq \mathcal S(\R^n) \subseteq \bigcap_{1 \leq p < \pinfty} L^p(\R^n)$.
\end{prp}

\begin{proof} Pour $u \in \mathcal S$ et $\pinfty > p \geq 1$~:
\[\int \abs u^p\dif x = \int \Big(\underbrace{\abs u(1+\abs x)^N}_{\text{borné pour tout $N$}}\Big)^p(1+\abs x)^{-Np}\dif x
	\leq \int (C_N)^p\underbrace {(1+\abs x)^{-Np}}_{\text{intégrable pour $Np > n$}}\dif x.\]

Dès lors, pour $N$ suffisamment grand ($Np>n$), on a $\int \abs u\dif x \leq \cste$
\end{proof}

\begin{prp} Pour $u \in \mathcal S$ et $\alpha, \beta \in \N^n$, alors~: $x^\alpha\partial^\beta u \in \mathcal S$.
\end{prp}

\begin{proof} Soient $\lambda, \mu \in \N^n$. Par Leibniz~:
\[\partial^\mu(fg) = \sum_{\sigma \leq \mu}\binom \mu\sigma\partial^\sigma f \partial^{\mu-\sigma} g.\]

Donc~:
\[x^\lambda\partial^\mu(x^\alpha\partial^\beta u) = \sum_{\sigma \leq \mu}\binom \mu\sigma x^\lambda\partial^\sigma(x^\alpha)\partial^{\mu-\sigma}u,\]
où $x^\lambda\partial^\sigma(x^\alpha) \leq \cste x^\gamma$. On en déduit que $x^\lambda\partial^\mu(x^\alpha\partial^\beta u)$ est une somme finie de termes
bornés et est donc bornée.
\end{proof}

À défaut de définir une topologie sur $\mathcal S(\R^n)$, on définit uniquement une notion de convergence.

\begin{déf} Soit $(u_k)_{k \in \N} \in \mathcal S^\N$, $u \in \mathcal S$, on dit que $u_k$ converge vers $u$ lorsque $k \to \pinfty$
(noté $u_k \xrightarrow[k \to \pinfty]{\mathcal S} u$) lorsque~:
\begin{equation}
	\forall \alpha, \beta \in \N^n : \sup_{x \in \R^n}\abs {x^\alpha\partial^\beta(u-u_k)} \xrightarrow[k \to \pinfty]{} 0.
\end{equation}
\end{déf}

\begin{thm} Soit $u \in \mathcal S$. Alors~:
\begin{enumerate}
	\item $\hat u \in \mathcal S$. De plus si $u_k \xrightarrow[k \to \pinfty]{\mathcal S} u$, alors $\hat {u_k} \xrightarrow[k \to \pinfty]{\mathcal S} \hat u$.
	\item $\widehat {D_ju}(\xi) = \xi_j\hat u(\xi)\quad$ (de plus $\widehat {x_ju} = D_j \hat u$).
\end{enumerate}
\end{thm}

\begin{proof} Pour le premier point, on calcule~:
\[D^\alpha_\xi\hat u(\xi) = \int D^\alpha_\xi(e^{-i\scpr x\xi})u(x)\dif x = \int e^{-i\scpr x\xi}(-x)^\alpha u(x)\dif x.\]

Donc~:
\[\xi^\beta D^\alpha_\xi\hat u(\xi) = \int \xi^\beta e^{-i\scpr x\xi}(-x)^\alpha u(x)\dif x = \int (-D_x)^\beta(e^{-i\scpr x\xi})(-x)^\alpha u(x)\dif x
	= \int e^{-i\scpr x\xi} D^\beta_x\left((-x)^\alpha u(x)\right)\dif x.\]

Pour montrer cette dernière égalité, intégrons par partie. D'abord observons pour $\phi \in \mathcal S$ et $j \in \intint 1n$~:
\[\int \partial_j\phi\dx = \int\ldots\int\left(\int \partial_j\phi\dx_j\right)\dx_1\ldots\dx_{j-1}\dx_{j+1}\ldots\dx_n.\]
Or~:
\[\int \partial_j\phi\dx_j = \lim_{N \to \pinfty}\int_{-N}^N\partial_j\phi(x)\dx_j
	= \lim_{N \to \pinfty}\left(\underbrace {\phi(x_1, \ldots, x_{j-1}, N, x_{j+1}, \ldots, x_n)}_{\xrightarrow[N \to \pinfty]{} 0} - \underbrace {\phi(x_1, \ldots, x_{j-1}, N, x_{j+1}, \ldots, x_n)}_{\xrightarrow[N \to \pinfty]{} 0}\right) = 0,\]
puisque $\phi \in \mathcal S$.

On en déduit donc que $\int\partial_j\phi\dx = 0$.

Dès lors, puisque $e^{-i\scpr x\xi}(-x)^\alpha u(x) \in \mathcal S$ et par récurrence~:
\[\int (-D_x)^\beta(e^{-i\scpr x\xi})(-x)^\alpha u(x)\dif x	= \int e^{-i\scpr x\xi} D^\beta_x\left((-x)^\alpha u(x)\right)\dif x.\]

Montrons alors que $\forall N \in \N : \exists C_N \geq 0 \tq \abs {D_x^\beta((-x)^\alpha u(x))} \leq C_N(1 + \abs x)^{-N}$. Par Leibniz~:
\[(1 + \abs x)^N\partial^\beta (x^\alpha u(x)) = (1 + \abs x)^N\sum_{\gamma \leq \beta} \binom \beta\gamma\partial^\gamma x^\alpha\partial^{\beta-\gamma}u(x)\]
est borné car $u \in \mathcal S$. Dès lors~:
\[\abs {\xi^\beta D_\xi^\alpha \hat u(x)} \leq C_N\int(1 + \abs x)^{-N}\dx.\]

Pour $N$ suffisamment grand ($N > n$), on a $\abs {\xi^\beta D_\xi^\alpha \hat u(x)} \leq \cste$.

Dès lors, on trouve~:
\[\abs {\xi^\beta D_\xi^\alpha\hat u(x)} \leq \sup_{x \in \R^n}(1 + \abs x)^{-N}\abs {D_x^\beta((-x)^\alpha(u - u_k))} \xrightarrow[k \to \pinfty]{} 0.\]
On en déduit donc $\hat {u_k} \xrightarrow[k \to \pinfty]{\mathcal S} \hat u$.

Pour le second point, la seconde formule découle directement du premier pour $\alpha = e_j$~:
\[D_j\hat u(\xi) = \int e^{-i\scpr x\xi}(-x_j)u(x)\dx = \widehat {x_ju}(\xi).\]

La première égalité se démontre par~:
\[\widehat {D_ju}(\xi) = \int e^{-i\scpr x\xi} D_ju(x)\dx = -\int D_{x,j}\left(e^{-i\scpr x\xi}\right)u(x)\dx = \xi_j\int e^{-i\scpr x\xi}u(x)\dx = \xi_j\hat u(\xi).\]
\end{proof}

\section{Formule d'inversion}

\begin{thm}\label{thm:inversion Fourier} Soit $u \in \mathcal S$. Alors~:
\begin{equation}\label{eq:inversion Fourier}
	u(x) = (2\pi)^{-n}\int e^{i\scpr x\xi}\hat u(\xi)\dif\xi.
\end{equation}
\end{thm}

La fonction $(y, \xi) \mapsto e^{i\scpr x\xi}e^{-i\scpr y\xi}u(y)$ n'est pas intégrable pour $(y, \xi)$. On ne va donc pas pouvoir appliquer Fubini.

\begin{proof} Pour $\chi \in \mathcal S$, $(y, \xi) \mapsto e^{-i\scpr y\xi}e^{i\scpr x\xi}\chi(\xi)u(y)$ est intégrable. Donc par Fubini~:
\[\int e^{i\scpr x\xi}\chi(\xi)\hat u(\xi)\dif\xi = \int e^{i\scpr x\xi}\chi(\xi)\int e^{-i\scpr y\xi}u(y)\dif y\dif\xi
	= \int u(y)\int e^{-i\scpr {y-x}\xi}\chi(\xi)\dif\xi \dif y = \int u(y)\hat \chi(y-x)\dif y.\]

Pour $\psi \in \mathcal S$, $\delta > 0$ tels que $\chi(\xi) = \psi(\delta\xi)$~:
\[\hat \chi(\xi) = \int e^{-i\scpr y\xi}\psi(\delta\xi)\dif\xi = \delta^{-n}\hat \psi(\xi/\delta).\]

Alors~:
\[\int e^{i\scpr x\xi}\psi(\delta\xi)\hat u(\xi)\dif\xi = \int u(x+y)\delta^{-n}\psi(y/\delta)\dif y = \int u(x+\delta y)\hat \psi(y)\dif y.\]

Par le théorème de convergence dominée~:
\[\int u(x+\delta y)\hat \psi(y)\dif y \xrightarrow[\delta \to \pinfty]{} u(x)\int \hat \psi(y)\dif y,\]
or~:
\[\int e^{i\scpr x\xi}\psi(\delta \xi)\hat u(\xi)\dif\xi \xrightarrow[\delta \to \pinfty]{} \psi(0)\int e^{i\scpr x\xi}\hat u(\xi)\dif\xi.\]

Par unicité de la limite, si $\int \hat \psi \dif y\neq 0$~:
\[u(x) = \frac {\psi(0)}{\int\hat \psi(y)\dif y}\int e^{i\scpr x\xi}\hat u(\xi)\dif\xi\]

Dans le cas $n=1$, on prend $\psi_1 : x \mapsto e^{-x^2/2}$. En intégrant $z \mapsto e^{-z^2/2}$ sur un chemin rectangulaire $[a,b,c,d] \subset \C$, on trouve~:
\[\int_a^b e^{-x^2/2}\dx + \int_b^c e^{-z^2/2}\dif z + \int_c^d e^{-z^2/2}\dif z + \int_d^a e^{-z^2/2}\dif z = 0\]
par Cauchy. Pour $(a, b) \to (\minfty, \pinfty)$, on trouve que $\int_b^c e^{-z^2/2}$ et $\int_d^a e^{-z^2/2}$ tendent vers $0$. Donc à la limite~:
\[\int_a^b e^{-x^2/2}\dx = \int_{\Im z = t}e^{-z^2/2}\dif z = \int e^{(x^2-t^2)/2}e^{-itx}\dx.\]

Donc $\hat \psi(t) = \psi(t)\int\psi\dx$. On en déduit~:
\[\int \hat \psi(t)\dif t = \left(\int \psi(x)\dx\right)^2 = \left(\int e^{-x^2/2}\right)^2 = 2\pi.\]

Dès lors $\psi(0) = 1$ et $\int\hat\psi\dx = 2\pi$, qui donne bien la formule.

Dans le cas général $n > 1$, on prend $\psi(x) = e^{-\abs x^2/2} = \prod_{j=1}^n\psi_1(x_j)$. Donc~:
\[\hat\psi(\xi) = \int e^{-i\scpr x\xi}\psi(x)\dx = \int e^{-i\scpr x\xi}\prod_{j=1}^ne^{-x_j^2/2}\dx = \int\prod_{j=1}^ne^{-ix_j\xi_j}\prod_{j=1}^ne^{-x_j^2/2}\dx
	= \int\prod_{j=1}^n\left(e^{-ix_j\xi_j}e^{-x_k^2/2}\right)\dx.\]

Par Fubini~:
\[\hat\psi(\xi) = \prod_{j=1}^n\int e^{-ix_j\xi_j}e^{-x_j^2/2}\dx = \prod_{j=1}^n\hat \psi_1(\xi_j).\]

On trouve alors~:
\[\int \hat \psi(\xi)\dif\xi = \int \prod_{j=1}^n\hat\psi_j(\xi_j)\dif\xi = \prod_{j=1}^n\int\hat \psi_1(\xi_j)\dif\xi_j = (2\pi)^{-n},\]
où l'avant dernière égalité s'obtient en appliquant Fubini.

Puisque $\hat \psi(0) = 1$, on a bien~\eqref{eq:inversion Fourier}.
\end{proof}

On définit une application \textit{transformée de Fourier} $\mathcal F : \mathcal S \to \mathcal S : u \mapsto \mathcal Fu \coloneqq \hat u$.

\begin{prp} $\mathcal F$ est une bijection linéaire.
\end{prp}

\begin{proof} Par la formule d'inversion, $\mathcal F$ est injective~: si $\mathcal Fu = 0$, alors $u = 0$.

De plus, $\mathcal F$ est surjective. Pour $f \in \mathcal S$, montrons qu'il existe $u \in \mathcal S \tq \mathcal Fu = f$.
Prenons $u(x) = (2\pi)^{-n}\int e^{i\scpr x\xi}f(\xi)\dif\xi$. Alors~:
\[\mathcal Ff(x) = \int e^{-i\scpr x\xi}f(\xi)\dif\xi = (2\pi)^nu(-x).\]

De plus~:
\[\hat {\hat u}(\xi) = \int e^{-i\scpr x\xi}\hat u(x)\dx = (2\pi)^n(2\pi)^{-n}\int e^{-i\scpr x\xi}\hat u(x)\dx = (2\pi)^nu(-\xi),\]
donc $\hat f = \hat {\hat u}$ pour tout $x$, et puisque $\mathcal F$ est injective, $f = \hat u$. Donc $\mathcal F$ est surjective, et donc surjective.

La linéarité est triviale~:
\[\mathcal F(f+\lambda g)(\xi) = \int e^{-i\scpr x\xi}(f+\lambda g)(x)\dx = \int e^{-i\scpr x\xi}f(x)\dx + \lambda\int e^{-i\scpr x\xi}g(x)\dx
	= \left(\hat f + \lambda \hat g\right)(\xi).\]
\end{proof}

En posant $\tilde {\mathcal F} : \mathcal S \to \mathcal S : u \mapsto \tilde {\mathcal F}u$ où $\tilde {\mathcal F}u(x) = (2\pi)^{-n}\int e^{i\scpr x\xi}u(\xi)\dif\xi$.
Par un raisonnement similaire à la Proposition précédente, on trouve $\tilde {\mathcal F}$ est une bijection linéaire. De plus $\mathcal F^{-1} = \tilde {\mathcal F}$.

De plus, puisque $\mathcal F$ transforme des suites convergentes en suites convergentes sur $\mathcal S$, $\tilde {\mathcal F}$ fait de même.

Cela veut dire que $\mathcal F$ est une homéomorphisme linéaire de $\mathcal S$ dans $\mathcal S$ pour la topologie non définie ici.

\begin{prp} Pour $u, v \in \mathcal S$~:
\begin{enumerate}
	\item $\int u\hat v = \int\hat uv$~;
	\item $\int u\overline v = (2\pi)^{-n}\int \hat u\hat {\overline v}$. Cette égalité est appelée \textit{identité de Parseval}.
\end{enumerate}
\end{prp}

\begin{proof} Le premier point se montre par la formule de la preuve du Théorème~\ref{thm:inversion Fourier} pour $u,\chi \in \mathcal S$~:
\[\int e^{i\scpr x\xi}\chi(\xi)\hat u(\xi)\dif\xi = \int u(x+y)\hat \chi(y)\dif y\]
en $x=0$.

Le second point, prenons $u, w \in \mathcal S$ et posons $v \coloneqq (2\pi)^{-n}\overline {\hat w}$. Par le premier point~:
\[\int \hat uv = \int u\hat v = \int u(2\pi)^{-n}\hat {\overline {\hat w}}.\]

On peut voir que~:
\[(2\pi)^{-n}\hat {\overline {\hat w}}(\xi) = (2\pi)^{-n}\int e^{-i\scpr x\xi}\overline {\hat w}(x)\dx =
	(2\pi)^{-n}\overline {\int e^{i\scpr x\xi}\hat w(x)\dx} = \overline {w(\xi)}.\]

Dès lors~:
\[\int \hat u(2\pi)^{-n}\overline {\hat w} = \int u\overline w.\]
\end{proof}

\begin{cor}[Formule de Plancherel] Pour $u \in \mathcal S$, on a~:
\begin{equation}\label{eq:Plancherel}
	\int \abs u^2\dx = (2\pi)^{-n}\int\abs {\hat u}^2\dif\xi
\end{equation}
\end{cor}

\section{Discussion sur la définition de la transformée}

On peut définir la transformée de Fourier de plusieurs manières, paramétrisé par $a, b \in \R$~:
\[\mathcal F_{a,b}u(\xi) = a\int e^{-ib\scpr x\xi}u(x)\dx.\]

La théorie reste la même à homothétie près puisque~:
\[\mathcal Fu(\xi) = \frac 1a\mathcal F_{a,b}u(\xi/b).\]

\[(2\pi)^{-n}\int \hat u(\xi)\overline {\hat v}(\xi)\dif\xi = \frac {(2\pi)^{-n}b^n}{a^n}\int\mathcal F_{a,b}u(\eta)\overline {\mathcal F_{a,b}v(\eta)}\dif\eta.\]
Donc on peut choisir $a=1$ et $b=2\pi$ ou encore $a=(2\pi)^{n/2}$ et $b=1$ afin de simplifier la formule de Parseval qui devient~:
\[\int u\overline v = \int \mathcal Fu\overline {\mathcal Fv}.\]

Cependant le choix $a=b=1$ permet de ne pas avoir de terme $b^k$ lors des dérivations sous le signe intégral.

\section{Extension de la transformée à $L^1 \cap L^2$}

\begin{prp} Il existe une unique application linéaire continue $\mathbb F : L^2 \to L^2$ tel que $\mathbb F\sVert[2]_{\mathcal S} = \mathcal F$ et~:
\[\int u\overline v = (2\pi)^{-n}\int \mathbb Fu\overline {\mathbb Fv}\dif\xi,\]
i.e. $\mathbb F$ préserve l'identité de Parseval.
\end{prp}

\begin{proof} Admettons que $C_0^\infty(\R^n)$ est dense dans $L^2(\R^n)$. Puisque $C_0^\infty(\R^n) \subseteq \mathcal S(\R^n)$, on a $\mathcal S(\R^n)$ dense dans $L^2(\R^n)$.
Par cette densité, pour $u \in \mathcal S$, il existe $(u_k)_{k \in \N} \in \mathcal S^\N$ tel que $u_k \xrightarrow[k \to \pinfty]{L^2} u$, et donc $(u_k)$ est de Cauchy
pour cette norme. $(\hat {u_k})_{k \in \N}$ est également de Cauchy car~:
\[\norm {\hat {u_k} - \hat {u_m}} = (2\pi)^{n/2}\norm {u_k - u_m}.\]

Par cette complétude, il existe $z \in \mathcal S \tq \hat {u_k} \xrightarrow[k \to \pinfty]{L^2} z$. On pose alors $\mathbb Fu \coloneqq z$. Montrons que $z$ ne dépend pas
de la suite $(\hat {u_k})$ choisie pour montrer que $\mathbb F$ est bien définie.

Soit $(v_k)_{k \in \N} \tq v_k \xrightarrow[k \to \pinfty]{L^2} z$. Alors $v_k-u_k \xrightarrow[k \to \pinfty]{L^2} 0$. Par Plancherel,
$\hat {v_k}-\hat {u_k} \xrightarrow[k \to \pinfty]{L^2} 0$. Dès lors $\hat {v_k} \xrightarrow[k \to \pinfty]{L^2} z$.

Montrons alors que $\mathbb F$ est linéaire.

Soient $u, v \in L^2$. Soient $(u_k)_{k \in \N}, (v_k)_{k \in \N} \in \mathcal S^\N$ telles que $u_k \xrightarrow[k \to \pinfty]{L^2} u$ et $v_k \xrightarrow[k \to \pinfty]{L^2} v$.
Alors $u_k+v_k \xrightarrow[k \to \pinfty]{L^2} u+v$. Par linéarité de $\mathcal F$, $\hat {u_k} + \hat {v_k} = \widehat {u_k + v_k} \xrightarrow[k \to \pinfty]{L^2} \mathbb F(u+v)$.

Donc $\hat {u_k} + \hat {v_k} \xrightarrow[k \to \pinfty]{L^2} \mathbb Fu + \mathbb Fv$ et $\hat {u_k} + \hat {v_k} \xrightarrow[k \to \pinfty]{L^2} \mathbb F(u+v)$.
On en déduit $\mathbb F u + \mathbb F v = \mathbb F(u+v)$. Il est également trivial que pour $\lambda \in \R : \mathbb F(\lambda u) = \lambda \mathbb Fu$.

Pour montrer que $\mathbb F\sVert[2]_{\mathcal S} = \mathcal F$, prenons $u \in \mathcal S$, et la suite $(u_k)_{k \in \N}$ constante $u_k = u$. Par définition de $\mathbb F$,
on a $\mathbb Fu = \hat u$ car $\forall k \in \intint 1n : \hat {u_k} = \hat u$, donc $\hat {u_k} \xrightarrow[k \to \pinfty]{} \hat u$.

Montrons finalement que $\mathbb F$  vérifie Parseval.

Premier cas~: $u \in L^2$ et $v \in \mathcal S$. Il existe $\mathcal S^\N \ni (u_k)_{k \in \N} \xrightarrow[k \to \pinfty]{L^2} u$. Donc~:
\[\int u\overline v = \int u_k\overline v + \int (u-u_k)\overline v.\]

Puisque~:
\[\abs {\int (u-u_k)\overline v} \leq \underbrace {\norm {u-u_k}_{L^2}}_{\xrightarrow[k \to \pinfty]{} 0} \norm v \xrightarrow[k \to \pinfty]{} 0,\]
on sait~:
\[\int u_k\overline v \xrightarrow[k \to \pinfty]{} \int u\overline v.\]

Or $u_k, v \in \mathcal S$. Donc pour $k \to \pinfty$, par Cauchy-Schwartz et par Parseval pour $\mathcal F$~:
\[\int u\overline v = \int u_k\overline v \xrightarrow[k \to \pinfty]{L^2} (2\pi)^{-n}\int\mathbb Fu\overline {\hat v}.\]

Dans le cas général $u, v \in L^2$, par le premier point pour $(v_k)_{k \in \N} \tq v_k \xrightarrow[k \to \pinfty]{} v$~:
\[\int u\overline {v_k} = (2\pi)^{-n}\int \mathbb Fu\overline {\hat {v_k}}.\]

Or $v_k \xrightarrow[k \to \pinfty]{L^2} \mathbb Fv$. Par Cauchy-Schwartz, on a~:
\begin{enumerate}
	\item $\int u\overline {v_k} \xrightarrow[k \to \pinfty]{L^2} \int u\overline v$~;
	\item et $\int \mathbb Fu\overline {\hat {v_k}} \xrightarrow[k \to \pinfty]{L^2} \int \mathbb Fu\overline {\mathbb Fv}$.
\end{enumerate}

L'identité de Parseval est donc bien vérifiée pour $\mathbb F$. Il reste à vérifier que $\mathbb F$ est continue et qu'elle est unique.

La continuité découle de Parseval~:
\[\norm u_{L^2} = (2\pi)^{-n/2}\norm {\mathbb Fu}_{L^2},\]
donc pour $\varepsilon > 0$, pour $\delta = (2\pi)^{-n/2}\varepsilon$, on a que si $\norm {u-v}_{L^2} < \delta$, alors $\norm {\mathbb Fu - \mathbb Fv}_{L^2} < \varepsilon$.

Si il existe $\mathbb F_1 : L^2 \to L^2$ continue et linéaire telle que $\mathbb F_1\sVert[2]_{\mathcal S} = \mathcal F$, alors par densité, pour $u \in L^2$, il existe
$(u_k)_{k \in \N} \in \mathcal S^\N \tq u_k \xrightarrow[k \to \pinfty]{} u$, et donc, par continuité~:
\[\underbrace {\mathbb F(u_k)}_{\xrightarrow[k \to \pinfty]{} \mathbb F(u)} = \underbrace {\mathbb F_1(u_k)}_{\xrightarrow[k \to \pinfty]{} \mathbb F_1(u)}.\]

Donc puisque deux application continues qui coïncident sur une sous-ensemble dense coïncident partout, on a bien que $\mathbb F = \mathbb F_1$.
\end{proof}

De la même manière, $\tilde {\mathcal F}$ se prolonge sur $L^2$ en $\tilde {\mathbb F}$

\begin{prp} $\mathbb F \circ \tilde {\mathbb F} = \Id_{L^2} = \tilde {\mathbb F} \circ \mathbb F$.
\end{prp}

\begin{proof} Ceci vient directement de la même propriété sur $\mathcal F$ et $\tilde {\mathbb F}$. Soit $u \in L^2$ et soit
$\mathcal S^\N \ni (u_k)_{k \in \N} \xrightarrow[k \to \pinfty]{L^2} u$. On sait~:
\[\mathcal S \ni \tilde {\mathcal F}(u_k) = \tilde {\mathbb F}(u_k) \xrightarrow[k \to \pinfty]{L^2} \tilde {\mathbb F}(u)\]
par continuité de $\tilde {\mathbb F}$. Par continuité de $\mathbb F$, on a~:
\[\mathcal F \circ \tilde {\mathcal F}(u_k) = \mathbb F \circ \tilde {\mathbb F}(u_k) \xrightarrow[k \to \pinfty]{L^2} \mathbb F \circ \tilde {\mathbb F}(u).\]

Or $\mathcal F \circ \tilde {\mathcal F}(u_k) = u_k \xrightarrow[k \to \pinfty]{L^2} u$. Par unicité de la limite, on a $\mathbb F \circ \tilde {\mathbb F}(u)$.
L'autre égalité se démontre de la même manière.
\end{proof}

À ce stade, il est légitime de se demander si les définitions que l'on a sur $L^1$ (la formule intégrale définie depuis $\mathcal S$) et sur $L^2$ (la définition de $\mathbb F$)
sont compatibles, i.e. si pour $u \in L^1 \cap L^2$ on a bien $\hat u = \mathbb Fu$. Cette égalité tient bien (démonstration à venir).

\section{Exemple d'application de la théorie de Fourier}

Pour $\Delta = \sum_{j=1}^n\partial_j^2$ le Laplacien sur $\R^n$ et $f \in \mathcal S$, soit la PDE suivante~:
\begin{equation}
	(1+\sum_{j=1}^nD_j^2)u = u-\Delta u = f,
\end{equation}
ou plus généralement, pour des $a_\alpha \in \C$~:
\begin{equation}\label{eq:PDE générale Fourier}
	\underbrace {\sum_{\abs \alpha \leq m}a_\alpha D^\alpha}_{P(D) \text{ polynôme}} u = f,
\end{equation}
dans le cas du Laplacien, ce polynôme est $P(\xi) = 1+\abs \xi^2$.

Sous l'hypothèse $\inf_{\xi \in \R^n}\abs {P(\xi)} \gneqq 0$, trouvons $u \tq P(D)u = f$.

Formellement~:
\begin{align*}
	\widehat {P(D)u}(\xi) &= \hat f(\xi) \\
	P(\xi)\hat u(\xi) &= \hat f(\xi) \\
	\hat u(\xi) &= \frac {\hat f(\xi)}{P(\xi)} \\
	u(x) &= (2\pi)^{-n}\int e^{i\scpr x\xi}\frac {\hat f(\xi)}{P(\xi)}\dif\xi.
\end{align*}

Plus rigoureusement, puisque $f \in \mathcal S$, on sait $\hat f \in \mathcal S$. De plus, $P$ est borné par dessous. Donc $\abs {\hat f/P} \leq C_N(1+\abs \xi)^{-N}$,
et du coup la fonction sous l'intégrale ($\xi  \mapsto e^{i\scpr x\xi}\frac {\hat f(\xi)}{P(\xi)}$) est $L^1$, et cette intégrale est bien définie pour $N > n$.

De plus, puisque la dérivation selon $x$ sur $u$ fait juste descendre du $\xi$ de l'exponentielle, par récurrence avec le théorème de convergence dominée et par la borne supérieure
ci-dessus, on trouve que $u \in C^\infty(\R^n)$. On peut alors vérifier que la fonction $u$ ainsi trouvée est bien une solution de~\eqref{eq:PDE générale Fourier}~:
\[\sum_{\abs \alpha \leq m}a_\alpha D^\alpha u = (2\pi)^{-n}\int e^{i\scpr x\xi}\underbrace {\sum_{\abs \alpha \leq m}a_\alpha \xi^\alpha}_{=P(\xi)} \frac {\hat f(\xi)}{P(\xi)}\dif\xi
	= (2\pi)^{-n} \int e^{i\scpr x\xi}\hat f(\xi)\dif\xi = f(x).\]

\chapter{Espaces de Hilbert}

\begin{déf} Soit $H$ un $\C$-espace vectoriel. Un produit scalaire (forme hermitienne définie positive) sur $H$ est une application $\scpr \cdot\cdot : H \times H \to \C \tq$~:
\begin{itemize}
	\item[$(i)$]   à $y \in \C$ fixé~: $x \mapsto \scpr xy$ est une application linéaire de $H$ dans $\C$~;
	\item[$(ii)$]  pour $x, y \in \C$~: $\scpr xy = \overline {\scpr yx}$~;
	\item[$(iii)$] pour $x \in \C$~: $\scpr xx \geq 0$ où $\scpr xx = 0 \iff x = 0$.
\end{itemize}

Sur un produit scalaire, on peut définir une norme $\norm x \coloneqq \scpr xx^{1/2}$.
\end{déf}

\begin{prp} $\norm \cdot : H \to \R^+$ est une norme.
\end{prp}

\begin{prp} $\norm \cdot$ vérifie Cauchy-Schwartz, i.e.~:
\[\forall x, y \in H : \abs {\scpr xy} \leq \norm x\norm y.\]
\end{prp}

\begin{proof} Soit $\alpha \in \C \tq \abs \alpha = 1$ et $\alpha\scpr xy \in \R^+$ ($\alpha\scpr xy = \abs {\scpr xy}$). Soit $r \in \R$.

\[0 \leq \scpr {x-r\alpha y}{x-r\alpha y} = \scpr xx - r\alpha\scpr yx - r\overline \alpha\scpr xy + r^2\scpr yy = A - 2Br + Cr^2,\]
pour $A = \scpr xx \in \R^+$, $B = \alpha\scpr xy = \abs {\scpr xy} \in \R^+$, $C = \scpr yy \in \R^+$.

Si $C = 0$, alors $B = 0$, et donc $\scpr yx = 0$ et Cauchy-Schwartz est vérifié.

Si $C \gneqq 0$, alors pour $r = B/C$~: $0 \leq A - 2Br + Cr^2 = \frac {AC-B^2}C$, donc $B^2 \leq AC$, donc Cauchy-Schwartz est vérifié.
\end{proof}

\begin{prp} $\norm \cdot$ vérifie l'inégalité triangulaire, i.e.~:
\[\norm {x+y}^2 \leq \norm x^2 + \norm y^2.\]
\end{prp}

\begin{proof} $\norm {x+y}^2 = \scpr xx + \scpr xy + \scpr yx + \scpr yy \leq \norm x^2 + 2\norm x\norm y + \norm y^2 = (\norm x + \norm y)^2$.
\end{proof}

On a donc $(H, \norm \cdot)$ un e.v. normé, depuis lequel on peut alors définir une distance~: $d(x, y) \coloneqq \norm {x-y}$.

\begin{déf} Si $H$ est complet pour $d$, on dit que $H$ est un espace de Hilbert.
\end{déf}

Quelques exemples d'espaces de Hilbert~:
\begin{itemize}
	\item[(0)] $\C^n$ pour $\scpr xy \coloneqq \sum_{j=1}^nx_j\overline {y_j}$~;
	\item[(1)] Pour $(\Omega, \mathcal A, \mu)$ un espace de mesure, $L^2(\Omega, \mathcal A, \mu)$ muni du produit scalaire $\scpr fg \coloneqq \int f\overline g\dif\mu$~;
	\item[(2)] Pour $(\N, \mathcal P(\N), \abs \cdot)$ comme espace de mesure, on a l'équivalent dénombrable de l'exemple (0)~: $\int f\overline g = \sum_{k \geq 1}f_k\overline {g_k}$.
	On note $\ell^2(\N) \coloneqq L^2(\N, \mathcal P(\N), \abs \cdot)$.
\end{itemize}

Un dernière exemple bien moins trivial~: les espaces de Sobolev.

\begin{déf} Soit $s \geq 0$ un paramètre, on définit l'espace de Sobolev d'ordre $s$ sur $\R^n$ par~:
\begin{equation}
	H^s(\R^n) \coloneqq \left\{u \in L^2(\R^n) \tq (2\pi)^{-n}\int \abs {\mathbb Fu(\xi)}^2(1+\abs \xi)^s\dif\xi \lneqq \pinfty\right\}.
\end{equation}

On y définit le produit scalaire suivante pour $u, v \in H^s(\R^n)$~:
\begin{equation}
	\scpr uv_s \coloneqq (2\pi)^{-n}\int\mathbb Fu\overline {\mathbb Fv}(1+\abs\xi)^s\dif\xi.
\end{equation}
\end{déf}

% Montrer que cette intégrale est bien définie

\begin{prp} $(u, v) \mapsto \scpr uv_s$ est un produit scalaire.
\end{prp}

\begin{proof} À $v$ fixé, $u \mapsto \scpr uv_s$ est linéaire par linéarité de $\mathbb F$ et par linéarité de l'intégrale.

Soient $u, v \in H^s$.
\[\scpr uv_s = (2\pi)^{-n}\int \mathbb Fu\overline {\mathbb Fv}\underbrace {(1+\abs \xi)^s}_{\in \R^+}\dif\xi
	= (2\pi)^{-n}\overline {\int \mathbb Fv\overline {\mathbb Fu}(1+\abs \xi)^s\dif\xi} = \overline {\scpr vu_s}.\]

Finalement, pour $u \in H^s$~:
\[\scpr uu_s = (2\pi)^{-n}\int\underbrace {\abs {\mathbb Fu}^2}_{\geq 0}\underbrace {(1+\abs\xi)^s}_{\geq 0}\dif\xi \geq 0,\]
et de plus, il est évident que $\scpr uu_s = 0 \iff u = 0$.
\end{proof}

Par linéarité de Fourier, $H^s(\R^n)$ est un espace vectoriel, et de plus il est normé par le produit scalaire défini ci-dessus. C'est donc un espace de Hilbert.

\end{document}
