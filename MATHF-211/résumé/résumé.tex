\documentclass{report}

\usepackage{commath}
\usepackage[french]{babel}
\usepackage{palatino, eulervm}
\usepackage[utf8]{inputenc}
\usepackage[T1]{fontenc}
\usepackage{amsmath, amsthm, amsfonts, amssymb}
\usepackage{mathtools}
\usepackage{fullpage}
\usepackage[bottom]{footmisc}
\usepackage[parfill]{parskip}
\usepackage{upgreek}
\usepackage{stmaryrd}

\title{MATH-F-211~: Topologie}
\author{R. Petit}
\date{année académique 2016 - 2017}

% amsthm
\newtheorem{thm}{Théorème}[section]
\newtheorem{prp}[thm]{Proposition}
\newtheorem{lem}[thm]{Lemme}
\newtheorem{cor}[thm]{Corollaire}
\theoremstyle{definition}
\newtheorem{déf}[thm]{Définition}
\theoremstyle{remark}
\newtheorem*{rmq}{Remarque}
\newtheorem{ex}{Exemple}[section]

\DeclareMathOperator{\tq}{\text{ t.q. }}
\DeclareMathOperator{\Id}{Id}
\DeclareMathOperator{\adh}{adh}

\renewcommand{\proofname}{\it{Preuve}}

\newcommand{\cont}[2]{\mathcal C\left(\left[#1, #2\right]\right)}
\newcommand{\restr}[2]{\left.#1\right|_{#2}}
\newcommand{\intint}[2]{\left\llbracket#1, #2\right\rrbracket}

\newcommand{\R}{\mathbb R}
\newcommand{\Rp}{{\R^+}}
\newcommand{\Rm}{{\R^-}}
\newcommand{\Q}{\mathbb Q}
\newcommand{\N}{\mathbb N}

\newcommand{\pinfty}{{+\infty}}
\newcommand{\minfty}{{-\infty}}

\newcommand{\mconv}[2]{\xrightarrow[#1 \to #2]{}}
\renewcommand{\top}{\mathcal T}

\begin{document}
\pagenumbering{Roman}
\maketitle
\tableofcontents
\newpage
\setcounter{page}{1}
\pagenumbering{arabic}

\chapter{Topologie générale}
	\section{Espaces métriques}
		\begin{déf} Soit $M$ un ensemble non vide. Une fonction $d : M \times M \to \R$ est un e \textit{métrique} si $d$ satisfait~:
		\begin{itemize}
			\item[M1.] $\forall x, y \in M : d(x, y) \geq 0$ avec $d(x, y) = 0 \iff x = y$~;
			\item[M2.] $\forall x, y \in M : d(x, y) = d(y, x)$~;
			\item[M3.] $\forall x, y, z \in M : d(x, z) \leq d(x, y) + d(y, z)$.
		\end{itemize}

		Le couple $(M, d)$ est appelé \textit{espace métrique}.
		\end{déf}

		\begin{ex} La métrique euclidienne sur $\R$ est définie par~:
		\[d_E : \R \times \R \to \R : (x, y) \mapsto \abs {x-y}.\]
		\end{ex}

		\begin{proof} EXERCICE.
		\end{proof}

		\begin{ex}\label{ex:metriqueeuclidiennern} La métrique euclidienne sur $\R^n$ est définie par~:
		\[d_E : \R^n \times \R^n \to \R : (x, y) \mapsto \norm {x-y} = \sqrt {\sum_{i=1}^n (x_i-y_i)^2}.\]
		\end{ex}

		\begin{lem}[Inégalité de Cauchy-Schwartz] Soient $r, s \in \R^n$. Alors~:
		\[\left(\sum_{i=1}^n r_is_i\right)^2 \leq \left(\sum_{i=1}^nr_i^2\right)\left(\sum_{i=1}^ns_i^2\right).\]
		\end{lem}

		\begin{proof} Soit la fonction~:
		\[F(t) = \sum_{j=1}^n(r_j + ts_j)^2 = \left(\sum_{j=1}^ns_j^2\right)t^2 + \left(2\sum_{j=1}^nr_js_j\right)t + \sum_{j=1}^nr_j^2.\]

		La fonction $F(t)$ est positive pour tout $t$ car c'est une somme de valeurs positives. Dès lors, son discriminant est négatif. On a alors~:
		\[\left(2\sum_{j=1}^nr_js_j\right)^2 - 4\left(\sum_{j=1}^nr_j^2\right)\left(\sum_{j=1}^ns_j^2\right) \leq 0.\]
		
		En divisant par 4 de part et d'autre et en réarrangeant l'inégalité, on obtient~:
		\[\left(\sum_{j=1}^nr_js_j\right)^2 \leq \left(\sum_{j=1}^nr_j^2\right)\left(\sum_{j=1}^ns_j^2\right).\]
		\end{proof}

		\begin{proof}[Preuve de l'exemple~\ref{ex:metriqueeuclidiennern}] M1 et M2 sont triviaux.

		Pour M3, posons pour $1 \leq i \leq n$~: $r_i \coloneqq x_i - y_i$ et $s_i \coloneqq y_i - z_i$.

		Par l'inégalité de Cauchy-Schwartz, on peut écrire~:
		\begin{align*}
			2\sum_{i=1}^nr_is_i &\leq 2\sqrt {\left(\sum_{i=1}^nr_i^2\right)\left(\sum_{i=1}^ns_i^2\right)} \\
			2\sum_{i=1}^nr_is_i + \sum_{i=1}^n(r_i^2+s_i^2) &\leq 2\sqrt{\left(\sum_{i=1}^nr_i^2\right)\left(\sum_{i=1}^ns_i^2\right)} + \sum_{i=1}^n(r_i^2+s_i^2) \\
			\sum_{i=1}^n(r_i+s_i)^2 &\leq \left(\sqrt {\sum_{i=1}^ns_i^2} + \sqrt {\sum_{i=1}^nr_i^2}\right)^2 \\
			\sqrt {\sum_{i=1}^n(r_i+s_i)^2} &\leq \sqrt {\sum_{i=1}^nr_i^2} + \sqrt {\sum_{i=1}^ns_i^2} \\
			\sqrt {\sum_{i=1}^n(x_i-z_i)^2} &\leq \sqrt {\sum_{i=1}^n(x_i-y_i)^2} + \sqrt {\sum_{i=1}^n(y_i-z_i)^2} \\
			d_E(x, z) &\leq d_E(y, z) + d_E(x, y).
		\end{align*}
		\end{proof}

		\begin{déf}\label{déf:métriquediscrète} Soit $M \neq \emptyset$. On définit la \textit{métrique discrète} sur $M$ par~:
		\[d : M \times M \to \R : (x, y) \mapsto \begin{cases} 1 &\text{ si } x \neq y \\ 0 &\text{ si } x = y \end{cases}.\]
		\end{déf}

		\begin{proof} M1 et M2 sont triviaux.

		Pour M3~:
		\begin{itemize}
			\item soit $x \neq z$, et donc $x \neq y$ ou $y \neq z$, ce qui implique $d(x, y) + d(y, z) \geq 1 = d(x, z)$~;
			\item soit $x = z$, et donc $0 = d(x, z) \leq d(x, y) + d(y, z)$.
		\end{itemize}
		\end{proof}

		\begin{déf} La métrique de Manhattan est définie par~:
		\[d_{\mathcal M} : \R^2 \times \R^2 \to \R : (x, y) \mapsto \abs {x_1 - y_1} + \abs {x_2 - y_2}.\]
		\end{déf}

		\begin{proof} M1 et M2 sont triviaux.

		Pour M3, on pose $x, y, z \in \R^2$. On a alors~:
		\begin{align*}
			d_{\mathcal M}(x, z) &= d_E(x_1, z_1) + d_E(x_2, z_2) \leq d_E(x_1, y_1) + d_E(y_1, z_1) + d_E(x_2, y_2) + d_E(y_2, z_2) \\
			&= \left(d_E(x_1, y_1) + d_E(x_2, y_2)\right) + \left(d_E(y_1, z_1) + d_E(y_2, z_2)\right) = d_{\mathcal M}(x, y) + d_{\mathcal M}(y, z).
		\end{align*}
		\end{proof}

		\begin{déf} Soit $\cont ab$, l'ensemble des fonctions continues sur $[a, b]$ à valeurs dans $\R$. Soient $f, g \in \cont ab$, et on définit~:
		\begin{itemize}
			\item $d_1 : \cont ab \times \cont ab \to \R : (f, g) \mapsto \displaystyle\int_a^b\abs {(f-g)(x)}\dif x$~;
			\item $d_2 : \cont ab \times \cont ab \to \R : (f, g) \mapsto \displaystyle\sqrt{\int_a^b\left((f-g)(x)\right)^2\dif x}$~;
			\item $d_\infty : \cont ab \times \cont ab \to \R : (f, g) \mapsto \sup\left\{\abs {(f-g)(x)} \tq x \in [a, b]\right\}$.
		\end{itemize}
		\end{déf}

		\begin{déf} Soit $\mathcal C^1([a, b])$, l'ensemble des fonctions continument différentiables sur $[a, b]$ à valeurs dans $\R$. On définit~:
		\begin{align*}
			d : \mathcal C^1([a, b]) \times \mathcal C^1([a, b]) \to \R : (f, g) &\mapsto
			\sup\left\{\abs {(f-g)(x)} \tq x \in [a, b]\right\} + \sup\left\{\abs {(f'-g')(x)} \tq x \in [a, b]\right\} \\
			&= d_\infty(f, g) + d_\infty(f', g').
		\end{align*}
		\end{déf}

		\begin{rmq} Si $f$ et $g$ sont $k$ fois continument dérivables, alors on définit~:
		\[d(f, g) = \sum_{i=0}^kd_\infty(f^{(i)}, g^{(i)}).\]
		\end{rmq}

		\subsection{Sous-espaces métriques}
			\begin{prp} Soit $(M, d)$ un espace métrique. Soit $A \subset M$, non vide. Alors $(A, d_A)$ est un espace métrique, où~:
			\[d_A = \restr d{A \times A}.\]
			\end{prp}

			\begin{déf} Soient $(M, d_M)$ et $(N, d_N)$ deux espaces métriques. Soit $A \subset M$ non-vide. On définit trois métriques distinctes~:
			\begin{align*}
				&d_1 : (M \times N)^2 \to \R : ((x, y), (x', y')) \mapsto d_M(x, x') + d_N(y, y') \\
				&d_2 : (M \times N)^2 \to \R : ((x, y), (x', y')) \mapsto \sqrt {d_M(x, x') + d_N(y, y')} \\
				&d_\infty : (M \times N)^2 \to \R : ((x, y), (x', y')) \mapsto \max\left\{d_M(x, x'), d_N(y, y')\right\}
			\end{align*}
			\end{déf}

			\begin{proof} EXERCICE.
			\end{proof}

			\begin{déf} Soit $(M, d)$ un espace métrique et soient $a \in M, r \in \R^+_0$. On définit la \textit{boule ouverte} centrée en $a$ de rayon $r$
			par~:
			\[B(a, r) \coloneqq \left\{x \in M \tq d(x, a) \lneqq r\right\}.\]
			\end{déf}

			\begin{déf} Soit $f : (M, d_M) \times (N, d_N)$, une application entre deux espaces métriques. Si $f$ est une bijection et~:
			\[\forall x, y \in M : d_N(f(x), f(y)) = d_M(x, y),\]
			alors on dit que $f$ est une \textit{isométrie}. \end{déf}

			\begin{rmq} L'ensemble des isométries d'un espace métrique dans lui-même forme un groupe pour la composition. \end{rmq}

	\section{Suites et limites}
		\begin{déf} Une suite $(x_n)$ dans un espace métrique $(M, d)$ converge vers un point $a \in M$ si et seulement si~:
		\[\forall \varepsilon > 0 : \exists N \in \N \tq \forall n \geq N : d(x_n, a) < \varepsilon.\]
		\end{déf}

		\begin{lem} Soit $(x_n)$ une suite dans un espace métrique $(M, d)$. S'il existe $a$ et $b$ dans $M$ tels que $x_n \to a$ et $x_n \to b$, alors $a=b$.
		\end{lem}

		\begin{lem} Soient $(M, d_M)$ et $(N, d_N)$ deux espaces métriques. Soient $(x_n)$ une suite dans $M$ et $(y_n)$ une suite dans $N$. Alors la suite
		$(x_n, y_n)_n$ dans $M \times N$ converge par $d$ en $(a, b) \in M \times N$ si et seulement si $x_n \to a$ et $y_n \to b$, où
		$d \in \{d_1, d_2, d_\infty\}$. \end{lem}

		\begin{rmq} Ici, la convergence est assurée par les trois métriques si elle est constatée par une seule. En réalité, de manière générale, la convergence
		dépend de la métrique. \end{rmq}

		\begin{ex} La fonction~:
		\[f_n : [0, 1] \to [0, 1] : x \mapsto
			\begin{cases}nx &\text{ si } 0 \leq x < \frac 1n \\2-nx &\text{ si } \frac 1n \leq x < \frac 2n \\0 &\text{ sinon}\end{cases}.\]

		On observe que~:
		\begin{align*}
			d_1(f_n, 0) &= \int_0^1\abs{f_n(x) - 0(x)}\dif x = \int_0^1\abs {f_n(x)}\dif x = \frac 1n \mconv n\pinfty 0 \\
			d_2(f_n, 0) &= \int_0^1\abs{f_n(x)}^2 \leq \sqrt{\frac 1n} \mconv n\pinfty 0 \\
			d_\infty(f_n, 0) &= 1 \quad \forall n \in \N.
		\end{align*}

		Il y a donc convergence vers 0 (la fonction nulle) pour $d_1$ et $d_2$ dans $\cont 01$ mais vers 1 (la fonction constante valant 1)
		pour $d_\infty$. \end{ex}

		\begin{déf} Une suite $(x_n)$ dans un espace métrique $(M, d)$ est dite \textit{de Cauchy} si~:
		\[\forall \varepsilon > 0 : \exists N \in \N \tq \forall m, n \geq N : d(x_m, x_n) < \varepsilon.\]
		\end{déf}

		\begin{lem} Soit $(x_n)$ une suite convergente dans un espace métrique $(M, d)$. Alors $(x_n)$ est de Cauchy. \end{lem}

		\begin{rmq} On ne peut pas cependant dire que la réciproque est vraie~: le cas est trop général. \end{rmq}

		\begin{ex} Si $(x_n) \subset \Q$ est une suite convergente en $\sqrt 2$ dans $\R$, alors $(x_n)$ est de Cauchy. Or $(x_n)$ ne converge pas dans $\Q$.
		\end{ex}

		\begin{déf} Soient $(M, d)$, un espace métrique et $A \subseteq M$. $A$ est dit borné lorsque~:
		\[\exists L \in \Rp_0 \tq \forall x, y \in M : d(x, y) \leq L.\]

		De plus, la suite $(x_n) \subset M$ est dite bornée lorsque le sous-ensemble $\{x_n \tq n \in \N\} \subset M$ est borné.
		\end{déf}

		\begin{prp} Un sous-ensemble $A$ de $M$, où $(M, d)$ est un espace métrique, est borné si et seulement si~:
		\[\exists x_0 \in M, R \in \Rp_0 \tq A \subset B(x_0, R).\]
		\end{prp}

		\begin{proof}  % TODO
		\end{proof}

		\begin{thm}[Bolzanno-Weierstrass] Soit $(x_n) \subset \R^m$. Si $(x_n)$ est bornée, alors il existe une sous-suite de $(x_n)$ convergente.
		\end{thm}
	
	\section{Fonctions et applications continues}
		\begin{déf} Soit $f : (M, d_M) \to (N, d_N)$, une application entre deux espaces métriques. On dit que $f$ est continue en $a \in M$ lorsque~:
		\[\forall \varepsilon > 0 : \exists \delta > 0 \tq d_M(x, a) < \delta \Rightarrow d_N(f(x), f(a)) < \varepsilon.\]

		On dit que $f$ est continue lorsqu'elle est continue en tout point $a$ de $M$.
		\end{déf}

		\begin{lem} Soit $f : M \to N$, une application entre deux espaces métriques. $f$ est continue en $a \in M$ si et seulement si~:
		\[\forall (x_n) \subset M : \left(x \to a\right) \Rightarrow \left(f(x_n) \to f(a)\right).\]
		\end{lem}

		\begin{prp} Soient $(M, d_M), (N, d_N), (P, d_P)$ trois espaces métriques. Soient $f : (M, d_M) \to (N, d_N)$ et $g : (N, d_N) \to (P, d_P)$ continues.
		Alors la fonction $g \circ f$ est également continue.
		\end{prp}

	\section{Ensembles ouverts et fermés}
		\begin{déf} Soit $(M, d)$ un espace métrique. Un sous-ensemble $U \subseteq M$ est dit~:
		\begin{itemize}
			\item ouvert si $\forall x \in U : \exists \varepsilon > 0 \tq B(x, \varepsilon) \subseteq U$~;
			\item fermé si son complémentaire $(M \setminus U)$ est ouvert.
		\end{itemize}
		\end{déf}
		
		\begin{lem}\label{lem:suiteconvouverts} Soit $(x_n) \subset M$. La suite $(x_n)$ converge en $a \in M$ lorsque $n \to \pinfty$ si et seulement si pour
		tout ouvert $U \subseteq M$~: si $a \in U$, alors il existe $N \in \N$ tel que $\forall n > N : x_n \in U$.
		\end{lem}
		
		\begin{lem}\label{lem:continuitéouverts} Soit $f : M \to N$ une application allant d'un espace métrique dans un autre. L'application $f$ est continue si
		et seulement si pour tout ouvert $u \subseteq M : f^{-1}(U)$ est un ouvert.
		\end{lem}
		
		\begin{proof} Supposons d'abord $f$ continue et prenons $U \subset N$ un ouvert et $a \in f^{-1}(U)$. Par ouverture de $U$, on a~:
		\[\exists \varepsilon > 0 \tq B(f(a), \varepsilon) \subseteq U.\]
		Également, par continuité de $f$, on sait que~
		\[\exists \delta > 0 \tq \forall x \in f^{-1}(U) : d_M(x, a) < \delta \Rightarrow d_N(f(x), f(a)) < \varepsilon.\]
		Autrement dit, si $x \in B(a, \delta)$, alors $f(x) \in B(f(a), \varepsilon)$. Et donc $f(x) \in U$, ou encore $B(a, \delta) \subseteq f^{-1}(U)$ qui est
		donc ouvert.
		
		Supposons alors que pour tout ouvert $U \subseteq N$, $f^{-1}(U)$ est ouvert. Soient $a \in M$ et $\varepsilon > 0$.
		On sait que $B(f(a), \varepsilon)$ est un ouvert, et donc $f^{-1}\left(B(f(a), \varepsilon\right)$ est également un ouvert. Or on sait que
		$a \in f^{-1}\left(B(f(a), \varepsilon\right)$. Dès lors, il existe $\delta > 0$ tel que $B(a, \delta) \subseteq f^{-1}\left(B(f(a), \varepsilon\right)$.
		Ou encore, de manière équivalente, si $d_M(x, a) < \delta$, alors $d_N(f(x), f(a)) < \varepsilon$. La fonction $f$ est donc continue.
		\end{proof}
		
		\begin{rmq} L'intérêt de ces lemmes est d'avoir une caractérisation de la convergence et de la continuité ne dépendant pas de la métrique mais uniquement
		de la notion d'ouvert.
		\end{rmq}
		
		\begin{lem}\label{lem:espmetraxiomestop} Soit $(M, d)$ un espace métrique. Alors~:
		\begin{enumerate}
			\item $M$ et $\emptyset$ sont des ouverts~;
			\item si $U_1, \ldots, U_k$ sont des ouverts de $M$, alors $\bigcap_{i=1}^kU_i$ est un ouvert de $M$~;
			\item si $\{U_i \tq i \in I\}$ est une collection quelconque d'ouverts de $M$, alors $\bigcup_{i\in I}U_i$ est un ouvert de $M$.
		\end{enumerate}
		\end{lem}
		
		\begin{proof} Le point 1 est trivial.
		
		Pour le point 2, prenons $a \in \bigcap_{i=1}^nU_i$. On sait que pour tout $1 \leq i \leq n$, il existe $\varepsilon_i > 0$ tel que
		$B(a, \varepsilon_i) \subseteq U_i$. Prenons donc $\varepsilon \coloneqq \min_i\{\varepsilon_i\}$, on sait donc que~:
		\[\forall i \in \{1, \ldots, n\} : B(a, \varepsilon) \subseteq U_i.\]
		On peut donc dire que $B(a, \varepsilon) \subseteq \bigcap_{i=1}^nU_i$.
		
		Pour le point 3, prenons $a \in \bigcup_{i\in I}U_i$. On sait donc qu'il existe $j \in I$ tel que $a \in U_j$, et donc, par ouverture de $U_j$, il existe
		$\varepsilon > 0$ tel que $B(a, \varepsilon) \subseteq U_j \subseteq \bigcup_{i\in I}U_i$.
		\end{proof}
		
	\section{Métriques équivalentes}
		\begin{déf} Soit $M$ un ensemble non-vide et soient $d$ et $d'$, deux métriques sur $M$. On dit qu'elles sont \textit{topologiquement équivalentes} lorsqu'elles
		déterminent les mêmes ouverts.
		\end{déf}
		
		\begin{cor} Soit $M$ un ensemble non-vide et soient $d, d'$ deux métriques topologiquement équivalentes sur $M$. Une suite $(x_n) \subset M$ converge par
		rapport à $d$ si et seulement si elle converge par rapport à $d'$.
		\end{cor}
		
		\begin{proof} Les deux métriques $d$ et $d'$ déterminent les mêmes ouverts. Dès lors, par le Lemme~\ref{lem:suiteconvouverts}, on a la double implication
		de la convergence.
		\end{proof}
		
		\begin{thm} Les trois métriques $d_1, d_2, d_\infty$ sont topologiquement équivalentes.
		\end{thm}
		
		\begin{proof} En notant $d$ l'une de ces métriques et $d'$ une autre, pour toute boule ouverte $B_d$ pour la métrique $d$, il est possible de déterminer une
		boule ouverte $B_{d'}$ pour la métrique $d'$ telle que $B_{d'} \subseteq B_d$.
		\end{proof}
		
		\begin{thm}\label{thm:d1d2dinflipschitzeq} Soit $M$ un espace métrique. Pour tout $x, y \in M$, on a~:
		\[\frac 12 d_1(x, y) \leq \frac 1{\sqrt 2}d_2(x, y) \leq d_\infty(x, y) \leq d_2(x, y) \leq d_1(x, y).\]
		\end{thm}
		
		\begin{déf} Soit $M$ un ensemble non-vide et soient $d, d'$ deux métriques sur $M$. Ces métriques sont dites \textit{Lipschtiz-équivalentes} lorsque~:
		\[\exists A, B \gneqq 0 \tq \forall x, y \in M : Ad(x, y) \leq d'(x, y) \leq Bd(x, y).\]
		\end{déf}
		
		\begin{rmq} Par le Théorème~\ref{thm:d1d2dinflipschitzeq}, on sait que les métriques $d_1, d_2, d_\infty$ sont Lipschitz-équivalentes.
		\end{rmq}
		
		\begin{lem} Soit $M$ un ensemble non-vide et soient $d, d'$ deux métriques sur $M$. Si $d$ et $d'$ sont Lipschitz-équivalentes, alors elles sont topologiquement
		équivalentes.
		\end{lem}

\chapter{Topologie différentielle}
	\section{Définitions}
		\begin{déf} Soit $X$ un ensemble non-vide. Une collection $\top_X$ des sous-ensembles de $X$ est une topologie lorsque~:
		\begin{itemize}
			\item[T1] $\{X, \emptyset\} \subseteq \top_X$~;
			\item[T2] si $U_1, \ldots, U_k \in \top_X$, alors $\bigcap_{i=}^kU_i \in \top_X$~;
			\item[T3] si $\{U_i \tq i \in I\}$ est une collection quelconque d'éléments de $\top_X$, alors $\bigcup_{i=1}^kU_i \in \top_X$.
		\end{itemize}

		On appelle le couple $(X, \top_X)$ un \textit{espace topologique}. Les éléments de $\top_X$ sont appelés les \textit{ouverts} de $X$.
		\end{déf}

		\begin{déf} Soient $(X, \top_X), (Y, \top_Y)$ deux espaces topologiques et $f : X \to Y$. On dit que $f$ est $(\top_X, \top_Y)$-continue si~:
		\[\forall U \in \top_Y : f^{-1}(U) \in \top_X.\]
		\end{déf}

		\begin{ex} Si $(M, d_M)$ est un espace métrique, on définit $\top_{d_M}$ comme étant la collection de tous les ouverts de $(M, d_M)$. On a vérifié par
		le Lemme~\ref{lem:espmetraxiomestop} que $(M, \top_{d_M})$ est un espace topologique.
		\end{ex}

		\begin{rmq} On observe qu'une fonction est continue au sens topologique ssi elle est continue au sens précédent, par le Lemme~\ref{lem:continuitéouverts}.
		\end{rmq}

		\begin{déf} Soit $X$ un ensemble non-vide quelconque. On définit~:
		\begin{itemize}
			\item la \textit{topologie grossière sur $X$} par $\top_X = \{X, \emptyset\}$~;
			\item la \textit{topologie discrète sur $X$} par $\top_X = \mathcal P(X) = 2^X$.
		\end{itemize}
		\end{déf}

		\begin{rmq} La topologie discrète revient à la topologie induite par la métrique discrète (voir Définition~\ref{déf:métriquediscrète}) sur un ensemble.

		La topologie grossière par contre ne peut être issue d'une métrique lorsque $\abs X \geq 2$.
		\end{rmq}

		\begin{ex} Soient $\N = \{0, 4, \ldots\}$ et $U_n = \intint 0n$ pour $n \in \N$. On pose $\top \coloneqq \{U_n \tq n \in \N\} \cup \{\N, \emptyset\}$.

		L'axiome T1 est satisfait de manière triviale (par définition).

		Pour prouver que T2 est respecté, prenons $U_{n_1}, \ldots, U_{n_k} \in \top$. Posons $n \coloneqq \min_i\{n_i\}$. On a alors
		$\bigcap_{i=1}^kU_{n_i} = U_n \in \top$. Si dans l'intersection, il y a $\emptyset$, alors l'intersection est $\emptyset \in \top$. Également si
		l'intersection comporte au plus $k-1$ fois $\N$, on peut les \textit{retirer} et retomber sur le cas $U_n$. Si l'intersection est
		$\N \cap \N \cap \ldots \cap \N$, alors l'intersection vaut $\N \in \top$.

		Pour l'axiome T3, prenons $\{U_{n_i} \tq i \in I\}$. Si $\{n_i \tq i \in I\}$ est borné, alors $\bigcup_{i \in I}U_{n_i} = U_{\max_{i \in I}\{n_i\}}$,
		et sinon $\bigcup_{i \in I}U_i = \N$.

		On a donc bien une topologie sur $\N$, ce qui veut dire que $(\N, \top)$ est un espace topologique.
		\end{ex}

		\begin{rmq} Ici, $\top$ ne peut être issu d'une métrique car toute fonction continue de $(\N, \top)$ dans $(\R, \top_{d_E})$ est constante. En effet,
		soit $f : \N \to \R$, $(\top, \top_{d_E})$-continue. Soit $n_0 \in \N$. Posons $y \coloneqq f(n_0) \in \R$. Soit $\varepsilon > 0$. Puisque
		$(y-\varepsilon, y+\varepsilon)$ est un ouvert, par continuité, on sait que $f^{-1}\left((y-\varepsilon, y+\varepsilon)\right)$ est également un ouvert,
		et qui contient $n_0$, et donc qui inclut $U_{n_0}$.

		On en déduit $f(0) \in (y-\varepsilon, y+\varepsilon)$, or $\varepsilon$ est quelconque. On trouve donc~:
		\[f(0) \in \bigcap_{\varepsilon > 0}(y-\varepsilon, y+\varepsilon) = \{y\}.\]
		On a alors trouvé que $f(0) = y = f(n_0)$, et ce, pour tout $n_0 \in \N$. La fonction $f$ est donc bien constante.
		\end{rmq}

		\begin{déf} Soit $(X, \top)$ un espace topologique et soit $x \in X$. Un \textit{voisinage de $x$} est un sous-ensemble $V$ de $X$ tel que~·
		\[\exists O \in \top \tq x \in O \subseteq O.\]

		On note $\mathcal V_X(x)$ l'ensemble des voisinages de $x$ dans $X$.
		\end{déf}

		\begin{déf} Soit $f : (X, \top_X) \to (Y, \top_Y)$, une application et $a \in X$. On dit que $f$ est continue en $a$ lorsque~:
		\[\forall V \in \mathcal V_Y(f(a)) : f^{-1}(V) \in \mathcal V_X(a).\]
		\end{déf}

		\begin{rmq} On remarque alors qu'une fonction est continue si elle est continue en tous ses points.
		\end{rmq}

		\begin{déf} Soit $A \subseteq (X, \top)$. Munissons $A$ d'une topologie~:
		\[\restr \top A \coloneqq \left\{U \cap A \tq U \in \top\right\}.\]

		On appelle $\restr \top A$ la \textit{topologie induite par $A$}.
		\end{déf}

		\begin{rmq} Montrons que $\restr \top A$ est bien une topologie.

		Pour T1, on sait que $\emptyset, X \in \top$. Et donc $\restr \top A \supseteq \{\emptyset \cap A, X \cap A\} = \{\emptyset, A\}$.

		Pour T2, prenons $U_1, \ldots, U_k \in \restr \top A$. On en déduit que pour tout $i = 1, \ldots, k$, il existe $V_i \in \top$ tel que $V_i \cap A = U_i$.
		Et donc~:
		\[\bigcap_{i=1}^kU_i = \bigcap_{i=1}^k(V_i \cap A) = \left(\bigcap_{i=1}^kV_i\right) \cap A \in \restr \top A,\]
		car $\bigcap_{i=1}^kV_i \in \top$.

		Pour T3, à nouveau, pour tout $i \in I$, il existe $V_i \in \top$ tel que $U_i = V_i \cap A$. On a alors~:
		\[\bigcup_{i \in I}U_i = \bigcup_{i \in I}(V_i \cap A) = \left(\bigcup_{i \in I}V_i\right) \cap A \in \restr \top A,\]
		car $\bigcup_{i\in I}V_i \in \top$.
		\end{rmq}

		\begin{déf} On appelle \textit{inclusion} la fonction d'identité allant d'un ensemble $A \subset X$ dans $X$~:
		\[i : A \to X : x \mapsto x.\]
		\end{déf}

		\begin{lem} Soit $(X, \top_X)$ un espace topologique et soit $\emptyset \neq A \subset X$. La fonction d'inclusion $i : A \to X$ est
		$(\restr \top A, \top_X)$-continue.
		\end{lem}

		\begin{rmq} La topologie induite est la plus réduite (avec le moins d'ouverts) telle que l'inclusion est continue.
		\end{rmq}

		\begin{lem} Soit $f : (X, \top_X) \to (Y, \top_Y)$ continue et soit $\emptyset \neq A \subseteq X$. Alors $(f \circ i) : A \to (Y, \top_Y)$
		est $(\restr \top A, \top_Y)$-continue.
		\end{lem}

		\begin{rmq} Cela revient à dire que la restriction d'une fonction continue à un sous-ensemble de son domaine est toujours continue.
		\end{rmq}

		\begin{lem} Soient $f : (X, \top_X) \to (Y, \top_Y)$, une application telle que $f(X) \subseteq B \subseteq Y$. Alors $f$ est $(\top_X, \top_Y)$-continue
		si et seulement si $\tilde f : (X, \top_X) \to (B, \restr {\top_Y} B)$ est $(\top_X, \restr {\top_Y}B)$-continue.
		\end{lem}

		\begin{lem} Soient $f : (X, \top_X) \to (Y, \top_Y)$ et $g : (Y, \top_Y) \to (Z, \top_Z)$ deux applications continues. Alors $g \circ f$ est continue.
		\end{lem}

		\begin{proof} Soit $U \in \top_Z$. On sait que $(g \circ f)^{-1}(U) = (f^{-1} \circ g^{-1})(U) = f^{-1}(g^{-1}(U))$, où $g^{-1}(U)$ est un ouvert par
		continuité de $g$. Appelons-le $V$. On a alors $f^{-1}(V)$ un ouvert également par continuité de $f$. Donc $(g \circ f)^{-1}(U) \in \top_X$ et donc
		$(g \circ f)$ est continue.
		\end{proof}

		\begin{rmq} On peut remarquer l'efficacité de la topologie dans de telles démonstrations.
		\end{rmq}

	\section{Homéomorphismes}
		\begin{déf} Soit $f : (X, \top_X) \to (Y, \top_Y)$ une application entre deux espaces topologiques. $f$ est un \textit{homéomorphisme} lorsque~:
		\begin{itemize}
			\item[$(i)$]   $f$ est bijective~;
			\item[$(ii)$]  $f$ est continue~;
			\item[$(iii)$] $f^{-1}$ est continue.
		\end{itemize}
		\end{déf}

		\begin{rmq} Attention~: un hom\textbf{é}omorphisme n'est pas un homomorphisme~!

		Également, une bijection continue n'est pas forcément un homéomorphisme.\\
		Par exemple $\Id : (X, \top_1) \to (X, \top_2)$, avec $\top_2 \subsetneqq \top_1$.
		\end{rmq}

		\begin{ex} Tous les ouverts de $(\R, \top_{d_E})$ sont homéomorphes. En effet, l'application~:
		\[f : (a, b) \to (c, d) : x \mapsto c + \frac {(x-a)(d-c)}{(b-a)}\]
		est continue (et sa réciproque également) et bijective. La continuité est assurée par la composition d'applications continues (translations et
		homothéties).
		\end{ex}

		\begin{déf} On appelle \textit{propriété topologique} toute propriété résistante aux homéomorphismes.
		\end{déf}

		\begin{ex} Le fait qu'une fonction continue sur $(\N, \top_{d_D})$, avec $d_D$ la métrique discrète, est constante est une propriété topologique.

		À l'opposé, le fait d'être un ensemble borné n'est pas une propriété topologique. En effet, l'ensemble ouvert $(-1, 1)$ est borné, mais prenons
		l'application $f : (-1, 1) \to \R : x \mapsto \frac x{1 + \abs x}$. On voit bien que $f$ est un homéomorphisme mais que $f((-1, 1)) = \R$ n'est plus borné.
		\end{ex}

	\section{Topologie produit}
		\begin{déf} Étant donnés deux espaces topologiques $(X, \top_X), (Y, \top_Y)$, on définit la collection $\top$ de sous-ensembles de $X \times Y$ comme
		suit~:
		\[W \in \top \iff \forall x, y \in W : \exists U \in \top_X, V \in \top_Y \tq (x, y) \in U \times V \subseteq W.\]
		On appelle $\top$ la \textit{topologie produit} de $\top_X$ et $\top_Y$.
		\end{déf}

		\begin{lem} $\top$ est une topologie sur $X \times Y$.
		\end{lem}

		\begin{proof} Pour vérifier T1, on observe que $\emptyset \in \top$ car si $W = \emptyset$, la condition est vérifiée pour tout $(x, y) \in W$.
		Également, $X \times Y \in \top$ car si on prend $(x, y) \in W = X \times Y$, on peut prendre $X \in \top_X$, $Y \in \top_Y$ tels que~:
		\[(x, y) \in X \times Y \subseteq W = X \times Y.\]

		Pour vérifier T2, prenons $W_1, W_2 \in \top$. Soit $(x, y) \in W_1 \cap W_2$. On sait que~:
		\begin{align*}
			&\exists U_1 \in \top_X, V_1 \in \top_Y \tq (x, y) \in U_1 \times V_1 \subseteq W_1, \\
			&\exists U_2 \in \top_X, V_2 \in \top_Y \tq (x, y) \in U_2 \times V_2 \subseteq W_2.
		\end{align*}

		On peut alors dire $(x, y) \in (U_1 \times V_1) \cap (U_2 \times V_2) = (U_1 \cap U_2) \times (V_1 \cap V_2) \subseteq W_1 \cap W_2.$

		Pour vérifier T3, soit $\{W_i \tq i \in I\} \subseteq \top$. Alors $\bigcup_{i \in I}W_i \in \top$. En effet, soit $(x, y) \in \bigcup_{i \in I}W_i$.
		On sait qu'il existe $i \in I$ tel que $(x, y) \in W_i$. Donc $\exists U \in \top_X$ et $V \in \top_Y$ tels que
		$(x, y) \in U \times V \subseteq \bigcup_{i \in I}W_i$.
		\end{proof}

		\begin{rmq} Les produits d'ouverts sont des ouverts, mais tous les ouverts ne sont pas des produits d'ouverts~: il suffit de prendre les boules ouvertes.
		\end{rmq}

		\begin{prp}\label{prp:projcont} Les projections $\pi_1 : X \times Y \to X$ et $\pi_2 : X \times Y \to Y$ sont continues.
		\end{prp}

		\begin{proof} Soit $U \in \top_X$. Alors $\pi^{-1}(U) = U \times Y \in \top$. De même, soit $V \in \top_Y$. Alors~:
		\[\pi^{-1}(V) = X \times V \in \top.\]
		\end{proof}

		\begin{rmq} La topologie produit est la moins fine qui rende les $\pi_i$ continues.
		\end{rmq}

		\begin{prp} Soit $f : Z \to X \times Y$, une application continue, où $X, Y, Z$ sont des espaces topologiques. Alors $f$ est continue si et seulement si
		$(\pi_1 \circ f)$ et $(\pi_2 \circ f)$ sont continues.
		\end{prp}

		\begin{proof} Supposons $f$ continue. Par la Proposition~\ref{prp:projcont}, on sait que les projections sont continues, donc la composition
		d'applications continues est une application continue.

		Supposons maintenant les $(\pi_i \circ f)$ continues. Soit $W \in \top$. Soit $z \in f^{-1}(W)$. Puisque $W$ est un ouvert, on sait qu'il existe
		$(U, V) \in (\top_X, \top_Y)$ tels que $f(z) \in U \times V$. Par hypothèse, on sait $(\pi_1 \circ f)^{-1}(U), (\pi_2 \circ f)^{-1}(V) \in \top$. Par
		l'axiome d'intersection finie, on sait que $T_z \coloneqq (\pi_1 \circ f)^{-1}(U) \cap (\pi_2 \circ f)^{-1}(V) \in \top$. On en déduit également
		$f(z) \supseteq f^{-1}(W) \in T_z$.

		Puisque $T_z \subseteq f^{-1}(W)$ pour tout $z$, on peut dire que~:
		\[f^{-1}(W) = \bigcup_{z \in f^{-1}(W)}T_z \in \top,\]
		par l'axiome d'union quelconque. La préimage d'un ouvert $W$ est donc bien un ouvert $\in \top$. L'application $f$ est donc continue.
		\end{proof}

		\begin{rmq} Cette proposition assure la continuité composante par composante.
		\end{rmq}

	\section{Topologie quotient}
		\begin{déf} un quotient d'un espace topologique $(X, \top_X)$ est une application surjective $\pi : X \to Y$, où $Y$ est un ensemble quelconque.
		\end{déf}

		\begin{rmq} On parle de quotient car on peut aisément faire des classes d'équivalences contenant tous les $x \in X$ tels que $\pi(x)$ ont la même
		valeur. Par surjection de $\pi$, on sait que tous les $y \in Y$ admettent une préimage, et on peut donc identifier les classes $[\pi(x)]$ aux valeurs
		$y = \pi(x)$.
		\end{rmq}

		\begin{déf} La topologie quotient sur $Y$ est la collection $\top_\pi$ de tous les sous-ensembles $O$ de $Y$ tels que $\pi^{-1}(O) \in \top$.
		\end{déf}

		\begin{lem} $\top_\pi$ est une topologie sur $Y$. De plus, $\pi$ est $(\top, \top_\pi)$-continue.
		\end{lem}

		\begin{proof} Pour vérifier T1, on sait que $\emptyset \in \top_\pi$ car $\pi^{-1}(\emptyset) = \emptyset \in \top$. De plus, $\pi^{-1}(Y) = X \in \top$
		par surjection de $\pi$, et donc $Y \in \top_\pi$.

		Pour vérifier T2, si $O_1, \ldots, O_k \in \top_\pi$, alors~:
		\[\pi^{-1}\left(\bigcap_{i=1}^kO_i\right) = \bigcap_{i=1}^k\pi^{-1}(O_i) \in \top,\]
		par l'axiome d'intersection finie.

		Pour vérifier T3, si $\{O_i \tq i \in I\} \subset \top_\pi$, alors~:
		\[\pi^{-1}\left(\bigcup_{i \in I}O_i\right) = \bigcup_{i \in I}\pi^{-1}(O_i) \in \top,\]
		par l'axiome d'union quelconque.
		\end{proof}

		\begin{rmq} Bien qu'il soit correct d'écrire $\pi^{-1}(O_1 \cap O_2) = \pi^{-1}(O_1) \cap \pi^{-1}(O_2)$, il est généralement faux d'écrire
		$\pi(O_1 \cap O_2) = \pi(O_1) \cap \pi(O_2)$.
		\end{rmq}

		\begin{rmq} la topologie quotient est la topologie la plus fine (qui contient le plus d'ouverts) qui rende $\pi$ continue.
		\end{rmq}

		\begin{lem} Soit $\pi : (X, \top) \to Y$, surjective, et $f : Y \to (Z, \top')$. Alors $f$ est $(\top_\pi, \top')$-continue si et seulement si
		$(f \circ \pi)$ est $(\top, \top')$-continue.
		\end{lem}

		\begin{proof} Supposons $f$ continue. $(f \circ \pi)$ est alors une composition d'applications continues, et est donc continue.

		Supposons maintenant $(f \circ \pi)$ continue. Prenons $U \in \top'$. Alors $f^{-1}(U) \in \top_\pi \iff (\pi^{-1} \circ f^{-1})(U) \in \top$.
		Or $(\pi^{-1} \circ f^{-1}) = (f \circ \pi)^{-1}$. Donc, par hypothèse, $(f \circ \pi)^{-1}(U) \in \top$.
		\end{proof}

	\section{Sous-ensembles fermés et sous-ensembles denses}
		\begin{déf} Un sous-ensemble $F$ d'un espace topologique $X$ est fermé lorsque $X \setminus F$ est ouvert.
		\end{déf}

		\begin{lem} Soit $X$ un espace topologique. Alors~:
		\begin{itemize}
			\item[$(i)$]   $X, \emptyset$ sont fermés~;
			\item[$(ii)$]  si $F_1, \ldots, F_k$ sont fermés, alors $\bigcup_{i=1}^kF_i$ est fermé~;
			\item[$(iii)$] si $\{F_i \tq i \in I\}$ est une collection de fermés, alors $\bigcap_{i \in I}F_i$ est un ouvert.
		\end{itemize}
		\end{lem}

		\begin{lem} Une application $f : (X, \top_X) \to (Y, \top_Y)$ est continue si et seulement si~:
		\[\forall F \subset Y : F \text{ fermé } \Rightarrow f^{-1}(F) \text{ fermé } \subseteq X.\]
		\end{lem}

		\begin{rmq} Un sous-ensemble peut être à la fois ouvert et fermé, et peut n'être ni ouvert, ni fermé.
		\end{rmq}

		\begin{déf} Soit $A$ un sous-ensemble d'un espace topologique $(X, \top)$. L'adhérence de $A$, notée $\adh A$ est le sous-ensemble~:
		\[\adh A \coloneqq \{x \in X \tq \forall U \in \top : x \in U \Rightarrow U \cap A \neq \emptyset\}.\]
		\end{déf}

		\begin{déf} Un point limite de $A$ est un point dans $\adh A \setminus A$.
		\end{déf}

		\begin{ex} Soit l'intervalle $(a, b)$ avec $a < b \in \R$. $\adh\left((a, b)\right) = [a, b]$ si on se place dans $\R$, mais
		$\adh\left((a, b)\right) = (a, b)$ si on se place dans $(a, b)$.
		\end{ex}

		\begin{lem} Soient $(M, d)$ est un espace métrique, $A \subseteq M$, et $x \in M$. Alors $x \in \adh A$ si et seulement si~:
		\[\exists (x_n) \subset A \tq x_n \xrightarrow[n \to +\infty]{} x.\]
		\end{lem}

		\begin{lem} Soient $A, B \subseteq (X, \top_X)$, un espace topologique. Alors~:
		\begin{itemize}
			\item[$(i)$]   $A$ est fermé si et seulement si $A = \adh A$~;
			\item[$(ii)$]  $A \subseteq B \Rightarrow \adh A \subseteq \adh B$~;
			\item[$(iii)$] $\adh\left(\adh(A)\right) = \adh(A)$~;
			\item[$(iv)$]  $\adh A$ est fermé~;
			\item[$(v)$]   soit $F$, l'intersection de tous les fermés qui contiennent $A$~; alors $F = \adh A$.
		\end{itemize}
		\end{lem}

		\begin{proof}~
		\begin{itemize}
			\item[$(i)$] Supposons d'abord $A$ fermé. On a alors $A \subseteq \adh A$. Montrons que $\adh A \subseteq A$. Prenons $x \not \in A$. Puisque $A$
			est fermé, on sait que $U \coloneqq X \setminus A$ est ouvert et $x \in U$. Mais $U \cap A = (X \setminus A) \cap A = A \cap \emptyset = \emptyset$.
			Donc $x \not \in \adh A$, ce qui montre que $X \setminus A \subseteq X \setminus \adh A$, ou encore $\adh A \subseteq A$.

			Supposons ensuite $\adh A = A$. Montrons que $X \setminus A$ est ouvert. Soit $x \in X \setminus A = X \setminus \adh A$. Alors, comme
			$x \not \in \adh A$, il existe $U_x$ un ouvert tel que $x \in U_x$ et $U_x \cap A = \emptyset$. Donc~:
			\[U_x \subset X \setminus A = \bigcup_{x \in X \setminus A}U_x,\]
			on a donc bien $X \setminus A$ ouvert.
			
			\item[$(ii)$] Trivial.
			\item[$(iii)$] Par $(ii)$, on sait que $\adh A \subseteq \adh(\adh(A))$. Montrons que $\adh(\adh(A)) \subseteq \adh A$. Soit $x \in \adh(\adh(A))$,
			et soit $U$ ouvert tel que $x \in U$. Alors $U \cap \adh(A) \neq \emptyset$, par définition de l'adhérence. Soit $y \in U \cap \adh(A)$. On peut
			donc dire $y \in \adh(A)$ et $U$ est un ouvert tel que $U \ni x$. Donc $U \cap A \neq \emptyset$, ou encore $x \in \adh(A)$.

			\item[$(iv)$] Trivial par $(i)$ et $(iii)$.
			\item[$(v)$] Montrons que $F \subseteq \adh(A)$~: $\adh A$ est fermé, donc $\adh(A) \supseteq \bigcap_{C \text{ fermé } \supseteq A}C = F$.

			Montrons maintenant que $F \supseteq \adh A$~: si $C$ est un fermé qui continent $A$, alors $C = \adh C \supseteq \adh A$. Donc~:
			\[\bigcap_{C \text{ fermé } \supseteq A}C \supseteq \adh A.\]
		\end{itemize}
		\end{proof}
\end{document}
