\documentclass{article}

\usepackage[french]{babel}
\usepackage{commath}
\usepackage{palatino, eulervm}
\usepackage[T1]{fontenc}
\usepackage[utf8]{inputenc}
\usepackage{fullpage}
\usepackage{amsmath, amsthm, amssymb, amsfonts}
\usepackage{mathtools}
\usepackage{cases}
\usepackage{stmaryrd}
\usepackage[bottom]{footmisc}
\usepackage[parfill]{parskip}
\usepackage[framemethod=tikz]{mdframed}
\usepackage{hyperref}

\title{MATHF-3001 --- Théorie de la mesure \\Résolution des TPs}
\author{R. Petit}
\date{Année académique 2018 - 2019}

\newtheorem{ex}{Exercice}[section]
\theoremstyle{definition}
%\newtheorem{proof}{Résolution}[section]
\addto\captionsfrench{\renewcommand\proofname{\underline{Résolution}}}

\newcommand{\TODO}[1]{\textcolor{red}{TODO: #1}}

% link amsthm and mdframed
\iftrue
%\iffalse
	% pre-amsthm
	\mdfdefinestyle{resultstyle}{%
		hidealllines=true,%
		leftline=true,%
		rightline=true,%
		innerleftmargin=10pt,%
		innerrightmargin=10pt,%
		innertopmargin=10pt,%
		innerbottommargin=8pt,%
	}

	\surroundwithmdframed[style=resultstyle]{ex}
\fi

\newcommand{\st}{\text{ s.t. }}
\newcommand{\C}{\complement}
\newcommand{\N}{{\mathbb N}}

\newcommand{\intint}[2]{\left\llbracket#1, #2\right\rrbracket}

\begin{document}
\maketitle

\section{Séance 1}

\begin{ex} Soient $(X, \mathcal F)$ un espace mesurable et $Y \subset X$. Mq $\mathcal F_Y \coloneqq \mathcal F \cap Y$ est une $\sigma$-algèbre sur $Y$.
\end{ex}

\begin{proof}~
\begin{enumerate}
	\item $\emptyset \in \mathcal F$, donc $\emptyset \cap Y = \emptyset \in \mathcal F_Y$.
	\item Soit $F \in \mathcal F$. $F \cap Y \in \mathcal F_Y$ et donc~:
	\[Y \setminus (F \cap Y) = Y \setminus F \cup \emptyset = Y \cap F^\C \in \mathcal F_Y\]
	car $F^\C \in \mathcal F$.
	\item Soit $(F_n)_{n \geq 0} \in \mathcal F^\N$. On sait que $\bigcup_{n \geq 0}F_n \in \mathcal F$. De plus $(F_n \cap Y)_{n \geq 0} \in {\mathcal F_Y}^\N$. Donc~:
	\[\bigcup_{n \geq 0}(F_n \cap Y) = \bigcup_{n \geq 0}F_n \cap Y \in \mathcal F_Y.\]
\end{enumerate}
\end{proof}

\begin{ex}~
\begin{enumerate}
	\item Soit $X$ un ensemble fini. Décrire la $\sigma$-algèbre engendrée par la classe des parties finies de $X$. Que peut-on dire si $X$ est fini~?
	\item Dans $X = \intint 0n$, on considère $\mathcal A = \{0\}$ et $\mathcal B = \{\{0\}, \{1, 2\}\}$. Décrire $\sigma(\mathcal A)$ et $\sigma(\mathcal B)$.
\end{enumerate}
\end{ex}

\begin{proof}~
\begin{enumerate}
	\item Soit $\mathcal F = \sigma(\{Y \in \mathcal P(X) \st \text{ $Y$ est fini }\})$. Alors~:
	\[\mathcal F = \{Y \in \mathcal P(X) \st \text{ $Y$ est au plus dénombrable ou $Y^\C$ est au plus dénombrable}\}\]
	car la famille doit être stable par complémentaire (d'où la définition symétrique par complémentarité) et par union dénombrable (d'où le fait
	que $Y$ ou $Y^\C$ soit \text{au plus dénombrable}). Si $X$ est fini, alors l'ensemble des parties finies de $X$ est exactement $\mathcal P(X)$
	qui est une $\sigma$-algèbre. Donc $\mathcal F = \sigma(\mathcal P(X)) = \mathcal P(X)$.
	\item $\sigma(\mathcal A)$ est la $\sigma$-algèbre engendrée par un unique élément donc~: $\sigma(\mathcal A) = \{\emptyset, \{0\}, \{0\}^\C, \intint 0n\}$
	où $\{0\}^\C = \intint 1n$.

	$\sigma(\mathcal B) = \{\emptyset, \{0\}, \{1, 2\}, \{0, 1, 2\}, \intint 3n, \intint 1n, \{0\} \cup \intint 3n, \intint 0n\}$.
\end{enumerate}
\end{proof}

\begin{ex} Soient $X, Y$ deux ensembles, et $f : X \to Y$.
\begin{enumerate}
	\item Si $\mathcal F$ est une $\sigma$-algèbre sur $Y$, mq $\mathcal A \coloneqq f^{-1}(\mathcal F)$ est une $\sigma$-algèbre sur $X$.
	\item Soit $\mathcal A$ une $\sigma$-algèbre sur $X$.
	\begin{enumerate}
		\item Mq $\mathcal F \coloneqq \{B \in \mathcal P(Y) \st f^{-1}(B) \in \mathcal A\}$ est une $\sigma$-algèbre sur $Y$.
		\item Que peut-on dire de $f(\mathcal A)$~?
	\end{enumerate}
\end{enumerate}
\end{ex}

\begin{proof}~
\begin{enumerate}
	\item~
	\begin{itemize}
		\item $\emptyset \in \mathcal F$ donc $\emptyset = f^{-1}(\emptyset) \in f^{-1}(\mathcal F)$.
		\item Soit $A \in \mathcal A$. Il existe $B \in \mathcal F \st f^{-1}(B) = A$. $f^{-1}(Y \setminus B) = X \setminus A \in \mathcal A$.
		\item Soit $(A_n)_{n \geq 0} \in \mathcal A^\N$. Il existe $(B_n)_{n \geq 0} \in \mathcal F^\N \st \forall n \geq 0 : A_n = f^{-1}(B_n)$.
		$\bigcup_{n \geq 0}A_n = \bigcup_{n \geq 0}f^{-1}(B_n) = f^{-1}\left(\bigcup_{n \geq 0}B_n\right) \in f^{-1}(\mathcal F)$.
	\end{itemize}
	\item~
	\begin{enumerate}
		\item~
		\begin{itemize}
			\item $\emptyset \in \mathcal A$ donc $\emptyset \in \mathcal F$.
			\item Soient $B \in \mathcal F$, $A \coloneqq f^{-1}(B)$. $f^{-1}(B^\C) = f^{-1}(Y) \setminus f^{-1}(B) = f^{-1}(B)^\C \in \mathcal A$.
			\item Soit $(B_n)_{n \geq 0} \in \mathcal F^\N$. On pose $B \coloneqq \bigcup_{n \geq 0}B_n$.
			\[f^{-1}(B) = \bigcup_{n \geq 0}f^{-1}(B_n) = \bigcup_{n \geq 0}A_n \in \mathcal A\]
			où $\forall n \geq 0 : A_n = f^{-1}(B_n) \in \mathcal A$. Donc $B \in \mathcal F$.
		\end{itemize}
		\item $f(\mathcal A)$ n'est pas nécessairement une $\sigma$-algèbre~: l'égalité $f(A^\C) = f(A)^\C$ n'est pas vraie en général. Par exemple pour
		$f : [\pm\varepsilon] \to [0, \varepsilon^2] : x \mapsto x^2$, on a~:
		\[[0, \varepsilon^2] = f([-\varepsilon, 0]) = f([\pm\varepsilon] \setminus [0, +\varepsilon])
		  \neq f([\pm\varepsilon]) \setminus f([0, \varepsilon]) = [0, \varepsilon^2] \setminus [0, \varepsilon^2] = \emptyset.\]
		Donc rien ne garantit que $f(\mathcal A)$ est stable par passage au complémentaire.

		\TODO{Donner un contre-exemple avec des $\sigma$-algèbres finies sur de petits ensembles.}
	\end{enumerate}
\end{enumerate}
\end{proof}

\begin{ex} Soient $(X, \mathcal A), (Y, \mathcal B)$ espaces mesurables. Soit $\mathcal F \subset \mathcal P(Y)$. Si $\mathcal B = \sigma(\mathcal F)$, mq $f : X \to Y$
est mesurable ssi $f^{-1}(\mathcal F) \subseteq \mathcal A$.
\end{ex}

\begin{proof} $\underline \Rightarrow$: en supposant $f$ mesurable, si $B \in \mathcal F$, alors $B \in \mathcal B$ et donc $f^{-1}(B) \in \mathcal A$.

$\underline \Leftarrow$: on pose $\mathcal B' \coloneqq \{B \in \mathcal B \st f^{-1}(B) \in \mathcal A\}$. Par le point précédent, $\mathcal B'$ est une $\sigma$-algèbre.
Par hypothèse~: $\mathcal F \subset \mathcal B'$, et donc $\sigma(\mathcal F) \subset \sigma(\mathcal B') = \mathcal B'$. Or $\mathcal B = \sigma(\mathcal F)$. De plus,
puisque $\mathcal B' \subset \mathcal B$, on a $\mathcal B \subset \mathcal B' \subset \mathcal B$, ce qui implique $\mathcal B = \mathcal B'$, i.e.~:
\[\forall B \in \mathcal B : f^{-1}(B) \in \mathcal A.\]
\end{proof}

\begin{ex}~
\begin{enumerate}
	\item Mq toute intersection (non-vide) de classes de Dynkin est une classe de Dynkin.
	\item Mq pour tout $\mathcal F \subset \mathcal P(X)$ il existe une plus petite classe de Dynkin au sens de l'inclusion (notée $\lambda(\mathcal F)$).
	\item Mq si $\mathcal D$ est une classe de Dynkin stable par intersections finies, alors $\mathcal D$ est une $\sigma$-algèbre.
	\item Mq si $\mathcal F \subset \mathcal P(X)$ est stable par intersections finies, alors $\lambda(\mathcal F) = \sigma(\mathcal F)$.
\end{enumerate}
\end{ex}

\begin{proof}~
\begin{enumerate}
	\item~[Exactement même raisonement que pour les $\sigma$-algèbres] Soit $(\mathcal D_i)_{i \in I}$ une famille non-vide de classes de Dynkin et soit
	$\mathcal D \coloneqq \bigcap_{i \in I}\mathcal D_i$.

	\begin{itemize}
		\item $\forall i \in I : \emptyset \in \mathcal D_i$ donc $\emptyset \in \mathcal D$.
		\item Soit $D \in \mathcal D$. Puisque $\forall i \in I : D \in \mathcal D_i$ et que les $\mathcal D_i$ sont des classes de Dynkin, on a $\forall i \in I : D^\C \in \mathcal D_i$
		et donc $D^\C \in \mathcal D$.
		\item Soit $(D_n)_{n \geq 0} \in \mathcal D^\N$. On sait que $\forall i \in I : \bigsqcup_{n \geq 0}D_n \in \mathcal D_i$ et donc $\bigsqcup_{n \geq 0}D_n \in \mathcal D$.
	\end{itemize}

	\item Comme pour les $\sigma$-algèbres, on peut définir~:
	\[\lambda(\mathcal F) \coloneqq \bigcap_{\overset{\mathcal D \text{ Dynkin}}{\mathcal F \subset \mathcal D}}\mathcal D.\]
	Par le point ci-dessus, $\lambda(\mathcal F)$ est une classe de Dynkin et toute classe de Dynkin $\mathcal D' \supset \mathcal F$ contient $\lambda(\mathcal F)$ par définition.

	\item Soit $\mathcal D$ une classe de Dynkin stable par intersections finies et soit $(D_n)_{n \geq 0} \in \mathcal D^\N$. Montrons donc que $\bigcup_{n \geq 0}D_n \in \mathcal D$.
	On pose $B_0 \coloneqq D_0$ et pour $n > 0$, on pose $B_n \coloneqq A_n \cap (\bigcap_{j=1}^{n-1}B_j^\C)$. Par récurrence, on observe que les $B_n$ sont dans $\mathcal D$ par
	stabilité sous intersections finies. De plus les $B_n$ sont disjoints deux à deux et leur union est égale à l'union des $D_n$. Donc $\bigcup_{n \geq 0}D_n \in \mathcal D$.

	\item Soit $D \in \lambda(\mathcal F)$. On pose $\mathcal D_D \coloneqq \{Q \in \lambda(\mathcal F) \st Q \cap D \in \lambda(\mathcal F)\} \subset \lambda(\mathcal F)$.
	Montrons que $\mathcal D_D$ est une classe de Dynkin.
	\begin{itemize}
		\item $\emptyset \in \mathcal D_D$ puisque $\lambda(\mathcal F) \ni \emptyset = \emptyset \cap D$.
		\item Soit $Q \in \mathcal D_D$. $Q^\C \cap D = (D^\C \cup Q)^\C = (D^\C \sqcup (\underbrace {Q \cap D}_{\in \lambda(\mathcal F)}))^\C \in \lambda(\mathcal F)$ par
		stabilité par passage au complément, stabilité par union disjointe.
		\item Soit $(Q_n)_{n \geq 0} \in {\mathcal D_D}^\N$ deux à deux disjoints. On a~:
		\[\bigsqcup_{n \geq 0}Q_n \cap D = \bigsqcup_{n \geq 0}(\underbrace {Q_n \cap D}_{\in \lambda(\mathcal F)}) \in \lambda(\mathcal F).\]
	\end{itemize}

	On remarque également que si $D \in \mathcal F$~: $\mathcal F \subset \mathcal D_D \subset \lambda(\mathcal F)$, ce qui implique $\lambda(\mathcal F) = \mathcal D_D$.

	Or par symétrie de l'intersection, pour $D, Q \in \lambda(\mathcal F)$ on a~: $Q \in \mathcal D_D \iff D \in \mathcal D_Q$. Dès lors on a une équivalence entre les deux
	assertions suivantes~:
	\begin{itemize}
		\item $\forall (D, Q) \in \mathcal F \times \lambda(\mathcal F) : Q \in \mathcal D_D$ (autrement dit $\forall D \in \mathcal F : \lambda(\mathcal F) = \mathcal D_D$)~;
		\item $\forall (D, Q) \in \mathcal F \times \lambda(\mathcal F) : D \in \mathcal D_Q$ (autrement dit $\forall Q \in \lambda(\mathcal F) : \mathcal F \subset \mathcal D_Q$).
	\end{itemize}

	On peut alors en déduire que $\forall Q \in \lambda(\mathcal F) : \lambda(\mathcal F) = \mathcal D_Q$. Dès lors, montrer que $\lambda(\mathcal F)$ est stable par
	instersections finies revient à montrer que $\forall D, Q \in \lambda(\mathcal F) : D \cap Q \in \lambda(\mathcal F)$, i.e. $D \in \mathcal D_Q = \lambda(\mathcal F)$.
	On a donc bien la stabilité de $\lambda(\mathcal F)$ sous intersections finies, on peut donc déduire que $\lambda(\mathcal F)$ est une $\sigma$-algèbre qui contient $\mathcal F$,
	donc $\sigma(\mathcal F) \subset \lambda(\mathcal F)$. Or toute $\sigma$-algèbre est une classe de Dynkin, donc $\lambda(\mathcal F) \subset \sigma(\mathcal F)$, ce qui permet
	de conclure.
\end{enumerate}
\end{proof}

\end{document}
