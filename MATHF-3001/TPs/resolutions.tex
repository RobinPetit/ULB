\documentclass{article}

\usepackage[french]{babel}
\usepackage{commath}
\usepackage{palatino, eulervm}
\usepackage[T1]{fontenc}
\usepackage[utf8]{inputenc}
\usepackage{fullpage}
\usepackage{amsmath, amsthm, amssymb, amsfonts}
\usepackage{mathtools}
\usepackage{cases}
\usepackage{stmaryrd}
\usepackage[bottom]{footmisc}
\usepackage[parfill]{parskip}
\usepackage[framemethod=tikz]{mdframed}
\usepackage{hyperref}

\title{MATHF-3001 --- Théorie de la mesure \\Résolution des TPs}
\author{R. Petit}
\date{Année académique 2018 - 2019}

\newtheorem{ex}{Exercice}[section]
\theoremstyle{definition}
%\newtheorem{proof}{Résolution}[section]
\addto\captionsfrench{\renewcommand\proofname{\underline{Résolution}}}

\newcommand{\TODO}[1]{\textcolor{red}{TODO: #1}}

% link amsthm and mdframed
\iftrue
%\iffalse
	% pre-amsthm
	\mdfdefinestyle{resultstyle}{%
		hidealllines=true,%
		leftline=true,%
		rightline=true,%
		innerleftmargin=10pt,%
		innerrightmargin=10pt,%
		innertopmargin=10pt,%
		innerbottommargin=8pt,%
	}

	\surroundwithmdframed[style=resultstyle]{ex}
\fi

\newcommand{\pinfty}{{+\infty}}
\newcommand{\minfty}{{-\infty}}

\newcommand{\st}{\text{ s.t. }}
\newcommand{\C}{\complement}
\newcommand{\N}{{\mathbb N}}
\newcommand{\Q}{{\mathbb Q}}
\newcommand{\R}{{\mathbb R}}
\newcommand{\B}{{\mathbb B}}
\renewcommand{\P}{{\mathbb P}}

\newcommand{\intint}[2]{\left\llbracket#1, #2\right\rrbracket}

\DeclareMathOperator{\Vol}{Vol}

\begin{document}
\maketitle

\section{Séance 1}

\begin{ex} Soient $(X, \mathcal F)$ un espace mesurable et $Y \subset X$. Mq $\mathcal F_Y \coloneqq \mathcal F \cap Y$ est une $\sigma$-algèbre sur $Y$.
\end{ex}

\begin{proof}~
\begin{enumerate}
	\item $\emptyset \in \mathcal F$, donc $\emptyset \cap Y = \emptyset \in \mathcal F_Y$.
	\item Soit $F \in \mathcal F$. $F \cap Y \in \mathcal F_Y$ et donc~:
	\[Y \setminus (F \cap Y) = Y \setminus F \cup \emptyset = Y \cap F^\C \in \mathcal F_Y\]
	car $F^\C \in \mathcal F$.
	\item Soit $(F_n)_{n \geq 0} \in \mathcal F^\N$. On sait que $\bigcup_{n \geq 0}F_n \in \mathcal F$. De plus $(F_n \cap Y)_{n \geq 0} \in {\mathcal F_Y}^\N$. Donc~:
	\[\bigcup_{n \geq 0}(F_n \cap Y) = \bigcup_{n \geq 0}F_n \cap Y \in \mathcal F_Y.\]
\end{enumerate}
\end{proof}

\begin{ex}~
\begin{enumerate}
	\item Soit $X$ un ensemble fini. Décrire la $\sigma$-algèbre engendrée par la classe des parties finies de $X$. Que peut-on dire si $X$ est fini~?
	\item Dans $X = \intint 0n$, on considère $\mathcal A = \{0\}$ et $\mathcal B = \{\{0\}, \{1, 2\}\}$. Décrire $\sigma(\mathcal A)$ et $\sigma(\mathcal B)$.
\end{enumerate}
\end{ex}

\begin{proof}~
\begin{enumerate}
	\item Soit $\mathcal F = \sigma(\{Y \in \mathcal P(X) \st \text{ $Y$ est fini }\})$. Alors~:
	\[\mathcal F = \{Y \in \mathcal P(X) \st \text{ $Y$ est au plus dénombrable ou $Y^\C$ est au plus dénombrable}\}\]
	car la famille doit être stable par complémentaire (d'où la définition symétrique par complémentarité) et par union dénombrable (d'où le fait
	que $Y$ ou $Y^\C$ soit \text{au plus dénombrable}). Si $X$ est fini, alors l'ensemble des parties finies de $X$ est exactement $\mathcal P(X)$
	qui est une $\sigma$-algèbre. Donc $\mathcal F = \sigma(\mathcal P(X)) = \mathcal P(X)$.
	\item $\sigma(\mathcal A)$ est la $\sigma$-algèbre engendrée par un unique élément donc~: $\sigma(\mathcal A) = \{\emptyset, \{0\}, \{0\}^\C, \intint 0n\}$
	où $\{0\}^\C = \intint 1n$.

	$\sigma(\mathcal B) = \{\emptyset, \{0\}, \{1, 2\}, \{0, 1, 2\}, \intint 3n, \intint 1n, \{0\} \cup \intint 3n, \intint 0n\}$.
\end{enumerate}
\end{proof}

\begin{ex} Soient $X, Y$ deux ensembles, et $f : X \to Y$.
\begin{enumerate}
	\item Si $\mathcal F$ est une $\sigma$-algèbre sur $Y$, mq $\mathcal A \coloneqq f^{-1}(\mathcal F)$ est une $\sigma$-algèbre sur $X$.
	\item Soit $\mathcal A$ une $\sigma$-algèbre sur $X$.
	\begin{enumerate}
		\item Mq $\mathcal F \coloneqq \{B \in \mathcal P(Y) \st f^{-1}(B) \in \mathcal A\}$ est une $\sigma$-algèbre sur $Y$.
		\item Que peut-on dire de $f(\mathcal A)$~?
	\end{enumerate}
\end{enumerate}
\end{ex}

\begin{proof}~
\begin{enumerate}
	\item~
	\begin{itemize}
		\item $\emptyset \in \mathcal F$ donc $\emptyset = f^{-1}(\emptyset) \in f^{-1}(\mathcal F)$.
		\item Soit $A \in \mathcal A$. Il existe $B \in \mathcal F \st f^{-1}(B) = A$. $f^{-1}(Y \setminus B) = X \setminus A \in \mathcal A$.
		\item Soit $(A_n)_{n \geq 0} \in \mathcal A^\N$. Il existe $(B_n)_{n \geq 0} \in \mathcal F^\N \st \forall n \geq 0 : A_n = f^{-1}(B_n)$.
		$\bigcup_{n \geq 0}A_n = \bigcup_{n \geq 0}f^{-1}(B_n) = f^{-1}\left(\bigcup_{n \geq 0}B_n\right) \in f^{-1}(\mathcal F)$.
	\end{itemize}
	\item~
	\begin{enumerate}
		\item~
		\begin{itemize}
			\item $\emptyset \in \mathcal A$ donc $\emptyset \in \mathcal F$.
			\item Soient $B \in \mathcal F$, $A \coloneqq f^{-1}(B)$. $f^{-1}(B^\C) = f^{-1}(Y) \setminus f^{-1}(B) = f^{-1}(B)^\C \in \mathcal A$.
			\item Soit $(B_n)_{n \geq 0} \in \mathcal F^\N$. On pose $B \coloneqq \bigcup_{n \geq 0}B_n$.
			\[f^{-1}(B) = \bigcup_{n \geq 0}f^{-1}(B_n) = \bigcup_{n \geq 0}A_n \in \mathcal A\]
			où $\forall n \geq 0 : A_n = f^{-1}(B_n) \in \mathcal A$. Donc $B \in \mathcal F$.
		\end{itemize}
		\item $f(\mathcal A)$ n'est pas nécessairement une $\sigma$-algèbre~: l'égalité $f(A^\C) = f(A)^\C$ n'est pas vraie en général. Par exemple pour
		$f : [\pm\varepsilon] \to [0, \varepsilon^2] : x \mapsto x^2$, on a~:
		\[[0, \varepsilon^2] = f([-\varepsilon, 0]) = f([\pm\varepsilon] \setminus [0, +\varepsilon])
		  \neq f([\pm\varepsilon]) \setminus f([0, \varepsilon]) = [0, \varepsilon^2] \setminus [0, \varepsilon^2] = \emptyset.\]
		Donc rien ne garantit que $f(\mathcal A)$ est stable par passage au complémentaire.

		\TODO{Donner un contre-exemple avec des $\sigma$-algèbres finies sur de petits ensembles.}
	\end{enumerate}
\end{enumerate}
\end{proof}

\begin{ex} Soient $(X, \mathcal A), (Y, \mathcal B)$ espaces mesurables. Soit $\mathcal F \subset \mathcal P(Y)$. Si $\mathcal B = \sigma(\mathcal F)$, mq $f : X \to Y$
est mesurable ssi $f^{-1}(\mathcal F) \subseteq \mathcal A$.
\end{ex}

\begin{proof} $\underline \Rightarrow$: en supposant $f$ mesurable, si $B \in \mathcal F$, alors $B \in \mathcal B$ et donc $f^{-1}(B) \in \mathcal A$.

$\underline \Leftarrow$: on pose $\mathcal B' \coloneqq \{B \in \mathcal B \st f^{-1}(B) \in \mathcal A\}$. Par le point précédent, $\mathcal B'$ est une $\sigma$-algèbre.
Par hypothèse~: $\mathcal F \subset \mathcal B'$, et donc $\sigma(\mathcal F) \subset \sigma(\mathcal B') = \mathcal B'$. Or $\mathcal B = \sigma(\mathcal F)$. De plus,
puisque $\mathcal B' \subset \mathcal B$, on a $\mathcal B \subset \mathcal B' \subset \mathcal B$, ce qui implique $\mathcal B = \mathcal B'$, i.e.~:
\[\forall B \in \mathcal B : f^{-1}(B) \in \mathcal A.\]
\end{proof}

\begin{ex}\label{ex:1.5}~
\begin{enumerate}
	\item Mq toute intersection (non-vide) de classes de Dynkin est une classe de Dynkin.
	\item Mq pour tout $\mathcal F \subset \mathcal P(X)$ il existe une plus petite classe de Dynkin au sens de l'inclusion (notée $\lambda(\mathcal F)$).
	\item Mq si $\mathcal D$ est une classe de Dynkin stable par intersections finies, alors $\mathcal D$ est une $\sigma$-algèbre.
	\item Mq si $\mathcal F \subset \mathcal P(X)$ est stable par intersections finies, alors $\lambda(\mathcal F) = \sigma(\mathcal F)$.
\end{enumerate}
\end{ex}

\begin{proof}~
\begin{enumerate}
	\item~[Exactement même raisonement que pour les $\sigma$-algèbres] Soit $(\mathcal D_i)_{i \in I}$ une famille non-vide de classes de Dynkin et soit
	$\mathcal D \coloneqq \bigcap_{i \in I}\mathcal D_i$.

	\begin{itemize}
		\item $\forall i \in I : \emptyset \in \mathcal D_i$ donc $\emptyset \in \mathcal D$.
		\item Soit $D \in \mathcal D$. Puisque $\forall i \in I : D \in \mathcal D_i$ et que les $\mathcal D_i$ sont des classes de Dynkin, on a $\forall i \in I : D^\C \in \mathcal D_i$
		et donc $D^\C \in \mathcal D$.
		\item Soit $(D_n)_{n \geq 0} \in \mathcal D^\N$. On sait que $\forall i \in I : \bigsqcup_{n \geq 0}D_n \in \mathcal D_i$ et donc $\bigsqcup_{n \geq 0}D_n \in \mathcal D$.
	\end{itemize}

	\item Comme pour les $\sigma$-algèbres, on peut définir~:
	\[\lambda(\mathcal F) \coloneqq \bigcap_{\overset{\mathcal D \text{ Dynkin}}{\mathcal F \subset \mathcal D}}\mathcal D.\]
	Par le point ci-dessus, $\lambda(\mathcal F)$ est une classe de Dynkin et toute classe de Dynkin $\mathcal D' \supset \mathcal F$ contient $\lambda(\mathcal F)$ par définition.

	\item Soit $\mathcal D$ une classe de Dynkin stable par intersections finies et soit $(D_n)_{n \geq 0} \in \mathcal D^\N$. Montrons donc que $\bigcup_{n \geq 0}D_n \in \mathcal D$.
	On pose $B_0 \coloneqq D_0$ et pour $n > 0$, on pose $B_n \coloneqq A_n \cap (\bigcap_{j=1}^{n-1}B_j^\C)$. Par récurrence, on observe que les $B_n$ sont dans $\mathcal D$ par
	stabilité sous intersections finies. De plus les $B_n$ sont disjoints deux à deux et leur union est égale à l'union des $D_n$. Donc $\bigcup_{n \geq 0}D_n \in \mathcal D$.

	\item Soit $D \in \lambda(\mathcal F)$. On pose $\mathcal D_D \coloneqq \{Q \in \lambda(\mathcal F) \st Q \cap D \in \lambda(\mathcal F)\} \subset \lambda(\mathcal F)$.
	Montrons que $\mathcal D_D$ est une classe de Dynkin.
	\begin{itemize}
		\item $\emptyset \in \mathcal D_D$ puisque $\lambda(\mathcal F) \ni \emptyset = \emptyset \cap D$.
		\item Soit $Q \in \mathcal D_D$. $Q^\C \cap D = (D^\C \cup Q)^\C = (D^\C \sqcup (\underbrace {Q \cap D}_{\in \lambda(\mathcal F)}))^\C \in \lambda(\mathcal F)$ par
		stabilité par passage au complément, stabilité par union disjointe.
		\item Soit $(Q_n)_{n \geq 0} \in {\mathcal D_D}^\N$ deux à deux disjoints. On a~:
		\[\bigsqcup_{n \geq 0}Q_n \cap D = \bigsqcup_{n \geq 0}(\underbrace {Q_n \cap D}_{\in \lambda(\mathcal F)}) \in \lambda(\mathcal F).\]
	\end{itemize}

	On remarque également que si $D \in \mathcal F$~: $\mathcal F \subset \mathcal D_D \subset \lambda(\mathcal F)$, ce qui implique $\lambda(\mathcal F) = \mathcal D_D$.

	Or par symétrie de l'intersection, pour $D, Q \in \lambda(\mathcal F)$ on a~: $Q \in \mathcal D_D \iff D \in \mathcal D_Q$. Dès lors on a une équivalence entre les deux
	assertions suivantes~:
	\begin{itemize}
		\item $\forall (D, Q) \in \mathcal F \times \lambda(\mathcal F) : Q \in \mathcal D_D$ (autrement dit $\forall D \in \mathcal F : \lambda(\mathcal F) = \mathcal D_D$)~;
		\item $\forall (D, Q) \in \mathcal F \times \lambda(\mathcal F) : D \in \mathcal D_Q$ (autrement dit $\forall Q \in \lambda(\mathcal F) : \mathcal F \subset \mathcal D_Q$).
	\end{itemize}

	On peut alors en déduire que $\forall Q \in \lambda(\mathcal F) : \lambda(\mathcal F) = \mathcal D_Q$. Dès lors, montrer que $\lambda(\mathcal F)$ est stable par
	instersections finies revient à montrer que $\forall D, Q \in \lambda(\mathcal F) : D \cap Q \in \lambda(\mathcal F)$, i.e. $D \in \mathcal D_Q = \lambda(\mathcal F)$.
	On a donc bien la stabilité de $\lambda(\mathcal F)$ sous intersections finies, on peut donc déduire que $\lambda(\mathcal F)$ est une $\sigma$-algèbre qui contient $\mathcal F$,
	donc $\sigma(\mathcal F) \subset \lambda(\mathcal F)$. Or toute $\sigma$-algèbre est une classe de Dynkin, donc $\lambda(\mathcal F) \subset \sigma(\mathcal F)$, ce qui permet
	de conclure.
\end{enumerate}
\end{proof}

\newpage
\section{Séance 2}

\begin{ex} Soient $(X, \mathcal F)$ un espace mesurable et $\mu$ une fonction additive sur $\mathcal A$ à valeurs dans $\mathbb R^+$. Mq les conditions suivantes sont équivalentes~:
\begin{enumerate}
	\item $\mu$ est $\sigma$-additive~;
	\item $\mu$ est continue à gauche~;
	\item $\mu$ est continue à droite.
\end{enumerate}

Donner un exemple de mesure $\mu : \mathcal A \to [0, \pinfty]$ qui ne satisfait pas le point 3. Que faut-il ajouter comme hypothèse pour ce résultat~?
\end{ex}

\begin{proof}~
\begin{itemize}
	\item[\underline {$1. \Rightarrow 2.$}] Soit $(B_n)_{n \geq 0} \in \mathcal A^\N$. On pose $A_0 \coloneqq B_0$ et $\forall n > 0 : A_n \coloneqq B_n \setminus B_{n-1}$,
	ce qui donne (car les $A_n$ sont dans $\mathcal A$)~:
	\[\mu\left(\bigcup_{n \geq 0}B_n\right) = \mu\left(\bigsqcup_{n \geq 0}A_n\right) = \sum_{n \geq 0}\mu(A_n)
	  = \lim_{N \to \pinfty}\underbrace {\sum_{n=0}^N\mu(A_n)}_{= \mu(B_N)} = \lim_{N \to \pinfty}\mu(B_N).\]
	\item[\underline {$2. \Rightarrow 1.$}] Soit $(A_n)_{n \geq 0} \in \mathcal A^\N$ deux à deux disjoints.
	On pose $B_0 \coloneqq A_0$ et $\forall n > 0 : B_n \coloneqq A_n \cup B_{n-1}$. Les $B_n$ forment une suite croissante dans $\mathcal A$. On a alors~:
	\[\mu\left(\bigsqcup_{n \geq 0}A_n\right) = \mu\left(\bigcup_{n \geq 0}B_n\right) = \lim_{n \to \pinfty}\mu(B_n) = \lim_{n \to \pinfty}\mu\left(\bigsqcup_{j=0}^nA_j\right)
	  = \lim_{n \to \pinfty}\sum_{j=0}^n\mu(A_j) = \sum_{n \geq 0}\mu(A_n).\]
	\item[\underline {$2. \Rightarrow 3.$}] Soit $(C_n)_{n \geq 0} \in \mathcal A^\N$ une suite décroissante. On a alors que $({C_n}^\C)_{n \geq 0}$ est une suite croissante dans
	$\mathcal A$. Donc~:
	\[\mu\left(\bigcup_{n \geq 0}{C_n}^\C\right) = \lim_{n \to \pinfty}\mu({C_n}^\C) = \mu(X) - \lim_{n \to \pinfty}\mu({C_n}^\C)\]
	car $\mu(X) < \pinfty$. De plus~:
	\[\mu\left(\bigcup_{n \geq 0}{C_n}^\C\right) = \mu\left(\left(\bigcap_{n \geq 0}C_n\right)^\C\right) = \mu(X) - \mu\left(\bigcap_{n \geq 0}C_n\right).\]
	Par finitude de $\mu$, on conclut~:
	\[\mu\left(\bigcap_{n \geq 0}C_n\right) = \lim_{n \to \pinfty}\mu(C_n).\]
	\item[\underline {$3. \Rightarrow 2.$}] Exactement même raisonnement par passage au complémentaire.
\end{itemize}

Si la mesure n'est pas finie, on peut construire une suite $(C_n)_n$ telle que $\forall n \geq 0 : \mu(C_n) = \pinfty$ et $\bigcap_{n \geq 0}C_n = \emptyset$. Par exemple,
dans l'espace mesuré $(\N, \mathcal P(\N), \# = \abs \cdot)$~: $\forall n \geq 0 : C_n \coloneqq \{m \in \N \st m > n\}$ est de mesure $\pinfty$ et
$\bigcap_{n \geq 0}C_n = \emptyset$. On a donc~:
\[\mu\left(\bigcap_{n \geq 0}C_n\right) = \mu(\emptyset) = 0 \neq \pinfty = \lim_{n \to \pinfty}\pinfty = \lim_{n \to \pinfty}\mu(C_n).\]

Il faut donc supposer que pour la suite $(C_n)_n$, il existe $n_0 \in \N \st \mu(C_n) \lneqq \pinfty$ afin d'éviter le cas où $(\mu(C_n))_{n \geq 0}$ est infinie pour tous les termes.
\end{proof}

\begin{ex} Soit $X$ un ensemble non dénombrable et $\mathcal A = \{A \in \mathcal P(X) \st A \text{ ou } A^\C \text{ est dénombrable}\}$. Soit $\mu : \mathcal A \to \{0, 1\}$
où $\mu(A) = 0 \iff A$ est dénombrable. Mq $\mu$ est une mesure sur $(X, \mathcal A)$.
\end{ex}

\begin{proof} $\mathcal A$ est une $\sigma$-algèbre (voir cours).
\begin{itemize}
	\item $\mu(\emptyset) = 0$ car $\emptyset$ est fini.
	\item Soit $(A_n)_{n \geq 0} \in \mathcal A^\N$ deux à deux disjoints. On note $A \coloneqq \bigsqcup_{n \geq 0}A_n$. On a soit $\mu(A) = 0$ ou $\mu(A) = 1$, et~:
	\[\mu(A) = 0 \iff \underbrace {\forall n \geq 0 : \mu(A_n) = 0}_{\text{i.e. tous les $A_n$ dénombrables}},\]
	et donc~:
	\[\mu(A) = 0 = \sum_{n \geq 0}0 = \sum_{n \geq 0}\mu(A_n)\]
	pour le premier cas. Pour le second cas, si $\mu(A) = 1$, il existe $n_0 \in \N \st {A_{n_0}}^\C$ est dénombrable (ou $\mu(A_{n_0}) = 1$), donc $\mu(A) \leq \sum_{n \geq 0}\mu(A_n)$.
	Supposons par l'absurde qu'il existe $n_1 \neq n_0 \st {A_{n_1}}^\C$ est dénombrable ($\mu(A_{n_1}) = 1$). On a donc ${A_{n_0}}^\C$ et ${A_{n_1}}^\C$ dénombrables.
	Or $A_{n_0} \cap A_{n_1} = \emptyset$, donc 	$A_{n_0} \subseteq {A_{n_1}}^\C$ ou $A_{n_1} \subseteq {A_{n_0}}^\C$, ce qui implique $A_{n_0}$ ou $A_{n_1}$ dénombrable,
	ce qui est une contradiction car $X$ est non-dénombrable.
\end{itemize}
\end{proof}

\begin{ex} Soit $(X, \mathcal A, \P)$ un espace de probabilité. Mq $\mathcal T \coloneqq \{A \in \mathcal A \st \P(A) \in \{0, 1\}\}$ est une $\sigma$-algèbre.
\end{ex}

\begin{proof}~
\begin{itemize}
	\item $\P(\emptyset) = 0$ donc $\emptyset \in \mathcal T$.
	\item Soit $A \in \mathcal T$. En particulier $A \in \mathcal A$ et $\P(A) \in \{0, 1\}$. Puisque $\P$ est une mesure (finie), on a
	$\P(A^\C) = \P(X) - \P(A) = 1 - \P(A) \in \{0, 1\}$. Donc $A^\C \in \mathcal T$.
	\item Soit $(A_n)_{n \geq 0} \in \mathcal T^\N$. On note $A \coloneqq \bigcup_{n \geq 0}A_n$. Mq $\P(A) \in \{0, 1\}$.
	\begin{itemize}
		\item si $\forall n \geq 0 : \P(A_n) = 0$, alors par $\sigma$-sous-additivité $0 \leq \P(A) \leq \sum_{n \geq 0}P(A_n) = 0$.
		\item si $\exists n_0 \in \N \st \P(A_{n_0}) = 1$, alors par monotonie, puisque $A_{n_0} \subseteq A \subseteq X$~:
		\[1 = \P(A_{n_0}) \leq \P(A) \leq \P(X) = 1.\]
	\end{itemize}
\end{itemize}
\end{proof}

\begin{ex} Soient $(X, \mathcal A, \mu)$ un espace mesuré, $(Y, \mathcal B)$ un espace mesurable et $g : X \to Y$ une application mesurable. On pose~:
\[\nu : \mathcal B \to [0, \pinfty] : B \mapsto \mu(g^{-1}(B)).\]
Mq $\nu$ est une mesure sur $(Y, \mathcal B)$.
\end{ex}

\begin{proof} On sait que $\forall B \in \mathcal B : g^{-1}(B) \in \mathcal A$ puisque $g$ est mesurable. Donc $\nu$ est bien définie. Mq $\nu$ est une mesure.
\begin{itemize}
	\item $\nu(\emptyset) = \mu(g^{-1}(\emptyset)) = \mu(\emptyset) = 0$ car $\mu$ est une mesure.
	\item Soient $(B_n)_{n \geq 0} \in \mathcal B^\N$ deux à deux disjoints. Mq $\nu$ est $\sigma$-additive.
	\[\nu\left(\bigsqcup_{n \geq 0}B_n\right) = \mu\left(g^{-1}\left(\bigsqcup_{n \geq 0}B_n\right)\right) = \mu\left(\bigsqcup_{n \geq 0}g^{-1}(B_n)\right)
	  = \sum_{n \geq 0}\mu(g^{-1}(B_n)) = \sum_{n \geq 0}\nu(B_n).\]
\end{itemize}
\end{proof}

\begin{ex} Soit $(X, \mathcal A)$ un espace mesurable.
\begin{enumerate}
	\item Pour $x \in X$, mq $\delta_x$ est une mesure.
	\item Mq si $\mu$ est une mesure sur $(X, \mathcal A) \st \forall A \in \mathcal A : \mu(A) = 0 \iff x \not \in A$ alors $\exists C \gneqq 0 \st \mu = C\delta_x$.
\end{enumerate}
\end{ex}

\begin{proof}~
\begin{enumerate}
	\item Mq $\delta_x$ est une mesure.
	\begin{itemize}
		\item $\delta_x(\emptyset) = 0$ car $x \not \in \emptyset$.
		\item Soit $(A_n)_{n \geq 0} \in \mathcal A^\N$ 2 à 2 disjoints. Mq $\delta_x(\bigsqcup_{n \geq 0}A_n) = \sum_{n \geq 0}\delta_x(A_n)$.
		\begin{itemize}
			\item Si $\delta_x(\bigsqcup_{n \geq 0}A_n) = 0$, alors $\forall n \geq 0 : x \not \in A_n$, i.e. $\forall n \geq 0 : \delta_x(A_n) = 0$.
			\item Si $\delta_x(\bigsqcup_{n \geq 0}A_n) = 1$, alors $\exists n_0 \st x \in A_{n_0}$. Et puisque les $A_n$ sont disjoints, $\forall n \neq n_0 : x \not \in A_n$.
		\end{itemize}
	\end{itemize}
	\item Soient $B, C \in \mathcal A \st \mu(B) \neq 0 \neq \mu(C)$. Alors $\delta_x(B) = 1 = \delta_x(C)$. Mq $\mu(B) = \mu(C)$. $B \cap C \neq \emptyset$ puisque $x \in B \cap C$.
	On pose $\tilde C \coloneqq C \cap B^\C$ et $\tilde B \coloneqq C^\C \cap B$. On a alors que $B$ et $\tilde C$ sont disjoints ($C$ et $\tilde B$ également).
	De plus, $x \not \in \tilde B$ et $x \not \in \tilde C$, et donc $\mu(\tilde B) = \mu(\tilde C) = 0$. On a donc~:
	\[\mu(C) = \mu(C) + \mu(\tilde B) = \mu(C \sqcup \tilde B) = \mu(B \cup C) = \mu(B \sqcup \tilde C) = \mu(B) + \mu(\tilde C) = \mu(B).\]
	On a donc $\mu : \mathcal A \to \{0, C\}$ où $\mu(A) \iff \delta_x(A) = 1$.
\end{enumerate}
\end{proof}

\begin{ex} Soit $(X, \mathcal A)$ un espace mesurable. Mq la mesure de comptage est une mesure.
\end{ex}

\begin{proof}~
\begin{itemize}
	\item $\abs \emptyset = 0$.
	\item La $\sigma$-additivité est triviale~: $\abs {\bigsqcup_{n \geq 0}A_n} = \sum_{n \geq 0}\abs A_n$.
\end{itemize}
\end{proof}

\begin{ex} Soit $X$ un ensemble fini non-vide. Mq $\mu = \frac {\abs \cdot}{\abs X}$ est une mesure de proba sur $(X, \mathcal P(X))$.
\end{ex}

\begin{proof}~
\begin{itemize}
	\item $\mu(\emptyset) = 0/\abs X = 0$.
	\item Soient $(A_n)_{n \geq 0} \in \mathcal P(X)^\N$ 2 à 2 disjoints.
	\[\mu(\bigsqcup_{n \geq 0}A_n) = \frac {\sum_{n \geq 0}\abs {A_n}}{\abs X} = \sum_{n \geq 0}\frac {\abs {A_n}}{\abs X}.\]
\end{itemize}

\underline {\textit{Note:}} si $(X, \mathcal A, \mu)$ est un espace mesuré, alors $\forall \alpha > 0 : \alpha\mu : \mathcal A \to [0, \pinfty] : A \mapsto \alpha \cdot \mu(A)$
est une mesure sur $(X, \mathcal A)$. Donc l'exercice peut être simplement résolu par le fait que $\mu$ est la mesure de comptage normalisée par $\abs X \in {\R^+}^*$
\end{proof}

\begin{ex} Soit $(X, \mathcal A)$ un espace de mesure.
\begin{enumerate}
	\item Soit $(\mu_n)_{n \geq 0}$ une suite croissante de mesures sur $(X, \mathcal A)$. Mq $\mu \coloneqq \lim_{n \to \pinfty}\mu_n$ est une mesure.
	\item Soit $(\mu_n)_{n \geq 0}$ une suite de mesures. Est-ce que $\mu \coloneqq \sum_{n \geq 0}\mu_n$ est une mesure~?
	\item Pour $n \geq 0$, on définit la mesure $\mu_n$ sur $(\N, \mathcal P(\N))$ par $\mu_n(A) = \abs {A \cap [n, \pinfty)}$.
	\begin{itemize}
		\item Mq $\forall n \geq 0 : \mu_n$ est bien une mesure et que la suite $(\mu_n)_n$ est décroissante.
		\item Est-ce que $\mu = \lim_{n \to \pinfty}\mu_n$ est une mesure sur $(\N, \mathcal P(\N))$~? Caractériser entièrement $\mu$.
	\end{itemize}
\end{enumerate}
\end{ex}

\begin{proof}~
\begin{enumerate}
	\item On note que puisque la suite des $\mu_n$ est croissante, pour tout $A \in \mathcal A$, $\mu(A)$ est bien définie car soit la suite $(\mu_n(A))_n$ converge vers une
	valeur réelle, soit elle diverge vers $\pinfty$.
	\begin{itemize}
		\item $\mu(\emptyset) = \lim_{n \to \pinfty}\mu_n(\emptyset) = 0$.
		\item Soient $(A_n)_{n \geq 0}$ 2 à 2 disjoints.
		\begin{align*}
			\mu\left(\bigsqcup_{n \geq 0}A_n\right) &= \lim_{k \to \pinfty}\mu_k\left(\bigsqcup_{n \geq 0}A_n\right) = \lim_{k \to \pinfty}\lim_{N \to \pinfty}\sum_{n=0}^N\mu_k(A_n) \\
				&= \lim_{N \to \pinfty}\sum_{n=0}^N\lim_{k \to \pinfty}\mu_k(A_n) = \sum_{n \geq 0}\lim_{k \to \pinfty}\mu_k(A_n) = \sum_{n \geq 0}\mu(A_n).
		\end{align*}
	\end{itemize}
	\item~
	\begin{itemize}
		\item $\mu(\emptyset) = \sum_{n \geq 0}\mu_n(\emptyset) = 0$.
		\item Soient $(A_n)_{n \geq 0} \in \mathcal A^\N$ 2 à 2 disjoints. On note $A \coloneqq \bigsqcup_{n \geq 0}A_n$. Par non-négativité des $(\mu_k(A_n))_{n,k}$,
		on a que les sommes sur $k$ et $n$ commutent, i.e.~:
		\[\sum_{k \geq 0}\sum_{n \geq 0}\mu_k(A_n) = \sum_{n \geq 0}\sum_{k \geq 0}\mu_k(A_n),\]
		et donc $\mu(A) = \sum_{n \geq 0}\mu(A_n)$.
	\end{itemize}
	On en déduit donc que $\mu = \sum_{k \geq 0}\mu_k$ est une mesure sur $(X, \mathcal A)$. De plus, puisque $\alpha \cdot \mu$ (pour $\alpha > 0$, $\mu$ mesure sur $(X, \mathcal A)$)
	est également une mesure sur $(X, \mathcal A)$, on a que pour $(\alpha_n)_{n \geq 0} \in \left({\R^+}^*\right)^\N$~: $\mu = \sum_{n \geq 0}\alpha_n\mu_n$ est une mesure également.
	\item~
	\begin{itemize}
		\item Soit $n \in \N$.
		\begin{itemize}
			\item $\mu_n(\emptyset) = \abs \emptyset = 0$.
			\item La $\sigma$-additivité est triviale par la $\sigma$-additivité de la mesure de comptage.
		\end{itemize}
		De plus, pour $A \in \mathcal P(\N)$ et $n \in \N$: $\mu_n(A) = \big|\underbrace  {A \cap [n, \pinfty)}_{\supseteq A \cap [n+1, \pinfty)}\big| \geq \mu_{n+1}(A)$.
		\item Soit $A \in \mathcal P(\N)$. Deux cas sont à distinguer~:
		\begin{enumerate}
			\item Soit $A$ est fini, en quel cas $\max A$ est fini et donc $\forall n > \max A : \mu_n(A) = 0$, et donc $\mu(A) = 0$.
			\item Soit $A$ est infini, et donc dénombrable. On a alors $\forall n \geq 0 : A \cap [n, \pinfty) \neq \emptyset$ car si il existe un $n \geq 0$ tel que
			$A \cap [n, \pinfty) = \emptyset$, alors $A \subset [0, n) \cap \N$, et donc $A$ est fini. Dès lors $\mu(A) > 0$.

			De plus~: $\forall n \geq 0 : \mu_n(A) = \pinfty$. Car si $\exists n \geq 0 \st \mu_n(A) \lneq \pinfty$, alors $\mu_n(A) = \abs {A \cap [n, \pinfty)} = k \in \N$
			et donc $A \cap [n, \pinfty) = \{m_1, \ldots, m_k\}$. Dans ce cas~: $\mu_{m_k+1}(A) = 0$, ce qui est une contradiction.

			On en déduit que si $A$ est infini (dénombrable), alors $\mu(A) = \pinfty$.
		\end{enumerate}
		$\mu$ vaut donc $0$ sur les parties finies de $\N$ et $\pinfty$ sur les parties dénombrables. $\mu$ n'est donc pas une mesure car~:
		$\mu(\N) = \sum_{n \in \N}\mu(\{n\}) = \sum_{n \geq 0}0 = 0 \neq \pinfty$.
	\end{itemize}
\end{enumerate}
\end{proof}

\begin{ex} Soient $(X, \mathcal A, \mu)$ un espace mesuré et $(A_n)_{n \geq 0} \in {\mathcal A}^\N$.
\begin{enumerate}
	\item Mq~:
	\[\mu\left(\liminf_{n \to \pinfty}A_n\right) \leq \liminf_{n \to \pinfty}\mu(A_n).\]
	\item Si $\exists n_0 \in \N \st \mu\left(\bigcup_{n \geq n_0}A_n\right) \leq \pinfty$, mq~:
	\[\mu\left(\limsup_{n \to \pinfty}A_n\right) \geq \limsup_{n \to \pinfty}\mu(A_n).\]
\end{enumerate}
\end{ex}

\begin{proof}~
\begin{enumerate}
	\item Pour $n \geq 0$~: on pose $B_n \coloneqq \cap_{m \geq n}A_m$. La suite $(B_n)_{n \geq 0}$ est trivialement croissante. On a donc~:
	\[\mu\left(\liminf_{n \to \pinfty}A_n\right) = \mu\left(\bigcup_{n \geq 0}B_n\right) = \lim_{n \to \pinfty}\mu(B_n),\]
	et~:
	\[\liminf_{n \to \pinfty}\mu(A_n) = \lim_{n \to \pinfty}\inf_{k \geq n}\mu(A_k).\]

	De plus~: $\mu(B_n) = \mu\left(\bigcap_{k \geq n}A_k\right) \leq \mu(A_m)$ pour $m \geq n$ par monotonie de $\mu$, et donc en particulier $\mu(B_n) \leq \inf_{k \geq n}\mu(A_k)$.
	Dès lors la suite $\mu(B_n)_n$ est dominée par $(\inf_{k \geq n}\mu(A_k))_n$. Dès lors~:
	\[\mu\left(\liminf_{n \to \pinfty}A_n\right) = \lim_{n \to \pinfty}\mu(B_n) \leq \lim_{n \to \pinfty}\inf_{k \geq n}\mu(A_k) = \liminf_{n \to \pinfty}\mu(A_n).\]
	\item On pose $C_n \coloneqq \bigcup_{m \geq n}A_m$. On sait qu'il existe $n_0 \in \N \st \mu\left(C_{n_0}\right) \lneqq \pinfty$. Les $C_n$ forment une suite décroissante.
	On a donc~:
	\[\mu\left(\limsup_{n \to \pinfty}A_n\right) = \mu\left(\bigcap_{n \geq 0}C_n\right) = \lim_{n \to \pinfty}\mu(C_n).\]
	De plus~:
	\[\limsup_{n \to \pinfty}\mu(A_n) = \lim_{n \to \pinfty}\sup_{k \geq n}\mu(A_k).\]
	Or $\forall k \geq n : C_n \supseteq A_k$ et donc $\forall k \geq n : \mu(C_n) \geq \mu(A_k)$, et en prticulier $\mu(C_n) \geq \sup_{k \geq n}\mu(A_k)$. Dès lors on conclut~:
	\[\mu\left(\limsup_{n \to \pinfty}A_n\right) = \lim_{n \to \pinfty}\mu(C_n) \geq \lim_{n \to \pinfty}\sup_{k \geq n}\mu(A_k) = \limsup_{n \to \pinfty}\mu(A_n).\]
\end{enumerate}
\end{proof}

\begin{ex} Soit $(X, \mathcal A)$ un espace mesurable. Soient $\mu, \nu$ deux mesures finies sur $(X, \mathcal A)$ telles que
$\forall A \in \mathcal A : \mu(A) \leq \frac 12 \Rightarrow \mu(A) = \nu(A)$.
\begin{enumerate}
	\item Mq $\mu = \nu$.
	\item Mq le résultat est faux si l'inégalité est changée en inégalité stricte.
\end{enumerate}
\end{ex}

\begin{proof}~
\begin{enumerate}
	\item Soit $B \in \mathcal A \st \mu(B) \gneqq \frac 12$. Alors $\mu(B^\C) = \mu(X) - \mu(B) = 1 - \mu(B) < \frac 12$. Dès lors $\mu(B^\C) = \nu(B^\C)$ par hypothèse,
	et on en déduit $\mu(B) = 1 - \mu(B^\C) = 1 - \nu(B^\C) = \nu(B)$, et donc $\mu = \nu$.
	\item Si l'inégalité devient stricte, on peut choisir, sur l'espace mesurable $(\{0, 1\}, \mathcal P(\{0, 1\}))$, $\mu$ la mesure d'une Bernoulli de proba
	$\frac 12$ et $\nu$ la mesure d'une Bernoulli de proba $\frac 13$. On a alors~:

	\begin{tabular}{c|c|c}
		$A$ & $\mu(A)$ & $\nu(A)$ \\
		\hline
		$\emptyset$ & $0$ & $0$ \\
		$\{0\}$ & $\frac 12$ & $\frac 23$ \\
		$\{1\}$ & $\frac 12$ & $\frac 13$ \\
		$\{0, 1\}$ & $1$ & $1$
	\end{tabular}

	Puisque $\mathcal B \coloneqq \{A \in \mathcal P(\{0, 1\}) \st \mu(A) \lneqq \frac 12\} = \{\emptyset\}$, on a bien $\mu = \nu$ sur $\mathcal B$, mais $\mu \neq \nu$.
\end{enumerate}
\end{proof}

\begin{ex} Soient $(X, \mathcal A)$ un espace mesurable et une partie stable par intersections finies $\mathcal F \subset \mathcal P(X) \st \sigma(\mathcal F) = \mathcal A$.
Si $\mu$ et $\nu$ sont deux mesures finies sur $(X, \mathcal A)$ telles que $\nu(X) = \mu(X)$ et $\mu = \nu$ sur $\mathcal F$. Mq $\mu = \nu$.
\end{ex}

\begin{proof} On pose $\mathcal D \coloneqq \{A \in \mathcal A \st \mu(A) = \nu(A)\}$. Mq $\mathcal D$ est une classe de Dynkin~:
\begin{itemize}
	\item $\emptyset \in \mathcal D$ car $\mu(\emptyset) = 0 = \nu(\emptyset)$.
	\item Soit $A \in \mathcal D$. $\mu(A^\C) = \mu(X) - \mu(A) = \nu(X) - \nu(A) = \nu(A^\C)$ et donc $A^\C \in \mathcal D$.
	\item Soient $(A_n)_{n \geq 0} \in {\mathcal A}^\N$ 2 à 2 disjoints et $A \coloneqq \bigsqcup_{n \geq 0}A_n$.
	\[\mu(A) = \sum_{n \geq 0}\mu(A_n) = \sum_{n \geq 0}\nu(A_n) = \nu(A).\]
	On en conclut $A \in \mathcal D$.
\end{itemize}

De plus par hypothèse $\mathcal F \subset \mathcal D$, et donc par l'exercice~\ref{ex:1.5} on a $\mathcal D = \sigma(\mathcal F) = \mathcal A$. Dès lors $\mu = \nu$ sur $\mathcal A$,
et donc $\mu = \nu$.
\end{proof}

\newpage
\section{Séance 3}
\begin{ex} Soient $\B$ la tribu borélienne sur $\R$ et $\mathcal L$ la mesure de Lesbesgue sur $\B$.
\begin{enumerate}
	\item Mq $\forall x \in \R : \{x\} \in \B$.
	\item Mq $\Q \in \B$ et $\mathcal L(\Q) = 0$.
	\item Mq une union non-dénombrable d'ensembles négligeables n'est pas nécessairement négligeable.
	\item Mq $N \in \B$ est un ensemble négligeable ssi $\forall \varepsilon > 0 : \exists U_\varepsilon \st N \subseteq U_\varepsilon$ et $\mathcal L(U_\varepsilon) < \varepsilon$.
\end{enumerate}
\end{ex}

\begin{proof}~
\begin{enumerate}
	\item $\{x\} = [x, x]$ est fermé dans $\R$, et $\mathcal L(\{x\}) = x-x = 0$.
	\item $\mathcal L(\Q) = \mathcal L(\bigsqcup_{q \in \Q}\{q\}) = \sum_{q \in \Q}\mathcal L(\{q\}) = 0$.
	\item $\mathcal L(\R) = \pinfty$, or~: $\bigsqcup_{x \in \R}\{x\}$.
	\item $\underline {\Leftarrow}$~: par monotonie, si $\forall \varepsilon > 0 : N \subseteq U_\varepsilon$, alors $\mathcal L(N) \leq \mathcal L(U_\varepsilon) < \varepsilon$.
	On en déduit $\mathcal L(N) = 0$, et donc $N$ est négligeable.

	$\underline {\Rightarrow}$~: Soit $N \in \B \st \mathcal L(N) = 0$. Pour $\varepsilon > 0$~: mq $\exists U_\varepsilon$ ouvert $\st \mathcal L(U_\varepsilon) < \varepsilon$.
	Rappelons la mesure extérieure de Lebesgue~: $\mathcal L^*(A) \coloneqq \inf_{(I_n)_{n \geq 0} \in \mathcal C_A}\sum_{n \geq 0}\Vol(I_n)$.

	Soit $(I_n)_{n \geq 0} \in \mathcal C_N$. Les $I_n$ sont compacts. On peut prendre une nouvelle suite $(J_n)_{n \geq 0} \st \forall n \geq 0 :
	\Vol(\overline {J_n}) < \Vol(I_n) + \frac \varepsilon{2^{n+1}}$ et $I_n \subseteq J_n$. Par $\sigma$-sous-additivité (parce que les $J_n$ ne sont pas forcément
	mutuellement disjoints), pour $J = \bigcup_{n \geq 0}J_n \supseteq N$~:
	\[\mathcal L^*(N) \leq \mathcal L^*(\bigcup_{n \geq 0}J_n) \leq \sum_{n \geq 0}\mathcal L^*(J_n).\]

	Or, pour $n \geq 0$~: $\mathcal L^*(J_n) \leq \Vol(I_n) + \frac \varepsilon{2^{n+1}}$. Finalement~:
	\[\mathcal L^*(N) < \sum_{n \geq 0}\left(\Vol(I_n) + \frac \varepsilon{2^{n+1}}\right) = 0 + \varepsilon\]
\end{enumerate}
\end{proof}

\begin{ex} Montrer qu'une droite $E$ dans $\R^2$ est de mesure nulle pour $\mathcal L$.
\end{ex}

\begin{proof} À $x = (x_1, x_2), y = (y_1, y_2) \in \R^2$ fixés, $E = \{x + ty\}_{t \in \R}$. Pour $\alpha < \beta \in \R$, on définit
$E_\alpha^\beta \coloneqq \{x+ty\}_{t \in [\alpha, \beta]}$. Mq $\mathcal L(E_\alpha^\beta) = 0$.

Si $y = (0, \lambda)$ (ou si $y = (\lambda, 0)$ par symétrie), on peut recouvrir $E_\alpha^\beta$ par l'intervalle compact
$[x_1 \pm \varepsilon] \times [x_2+\alpha\lambda, x_2+\beta\lambda]$ de volume arbitrairement petit (pour $\varepsilon$ aussi petit que nécessaire), et donc
$\mathcal L(E_\alpha^\beta) = 0$.

Sinon, soit $(I_n)_{n \geq 0}$ un recouvrement de $E_\alpha^\beta$ par des intervalles compacts. Pour $n \geq 0 : I_n = [a_n^1, b_n^1] \times [a_n^2, b_n^2]$ et
$\Vol(I_n) = (b_n^1-a_n^1)(b_n^2-a_n^2)$.

Montrons qu'il existe $(J_n)_{n \geq 0} \st \sum_{n \geq 0}\Vol(J_n) < \frac 12\sum_{n \geq 0}\Vol(I_n)$.

Soit $n \geq 0$. $I_n = [a_n^1, b_n^1] \times [a_n^2, b_n^2]$ où les 4 \textit{coins} sont $C_1 = (a_n^1, a_n^2), C_2 = (a_n^1, b_n^2), C_3 = (b_n^1, a_n^2), C_4 = (b_n^1, b_n^2)$.
WLOG supposons $\abs {E_\alpha^\beta \cap \{C_i\}_{i=1}^4} = 2$, i.e. $E$ passe par deux coins de $I_n$ (soit $C_1$ et $C_3$, soit $C_2$ et $C_4$)\footnote{En effet, si ce n'est pas
le cas, on peut "réduire" $I_n$ afin que ce soit le cas (et qui est donc de volume strictement inférieur).}. On définit alors~:
\[\begin{cases}
	J_{2n}   \coloneqq [a_n^1, \frac 12(b_n^1 + a_n^1)] \times [a_n^2, \frac 12(b_n^2 + a_n^2)]
	J_{2n+1} \coloneqq [\frac 12(b_n^1 + a_n^1), b_n^1] \times [\frac 12(b_n^2 + a_n^2), b_n^2]
\end{cases}\]
si $E \cap \{C_1, C_3\}$~; et~:
\[\begin{cases}
	J_{2n}   \coloneqq [a_n^1, \frac 12(b_n^1 + a_n^1)] \times [\frac 12(b_n^2 + a_n^2), b_n^2]
	J_{2n+1} \coloneqq [\frac 12(b_n^1 + a_n^1), b_n^1] \times [a_n^2, \frac 12(b_n^2 + a_n^2)]
\end{cases}\]
sinon.

On a bien $\Vol(I_n) = 2\left(\Vol(J_{2n} + \Vol(J_{2n+1}))\right)$, et donc $\sum_{n \geq 0}\Vol(I_n) = 2\sum_{n \geq 0}\Vol(J_n)$.

Et donc~:
\[\mathcal L^*(E_\alpha^\beta) = \inf_{(I_n)_n \in \mathcal C_{E_\alpha^\beta}}\sum_{n \geq 0}\Vol(I_n) = 0.\]

On a alors que $E_\alpha^\beta$ est mesurable et de mesure de Lebesgue nulle. Et on trouve que~:
\[\mathcal L^*(E) = \mathcal L^*(\bigcup_{n \geq 0}E_{-n}^{+n}) = \lim_{n \to \pinfty}\mathcal L^*(E_{-n}^{+n}) = 0.\]

Donc $E$ est également mesurable pour $\mathcal L$ et est de mesure nulle.
\end{proof}

\begin{ex} Pour $B \in \B^n$ et $\lambda > 0$, on définit $\lambda B = \{\lambda b\}_{b \in B}$.
\begin{enumerate}
	\item Mq $\forall \lambda > 0, B \in \B^n : \lambda B \in \B^n$.
	\item Mq $\mathcal L(\lambda B) = \lambda^n\mathcal L(B)$.
\end{enumerate}
\end{ex}

\begin{proof} On note $\mathcal B_\lambda \coloneqq \{B \in \B^n \st \lambda B \in \B^n\}$.
\begin{enumerate}
	\item~
	\begin{enumerate}
		\item Mq $\mathcal B_\lambda$ est une $\sigma$-algèbre.
		\begin{itemize}
			\item $\emptyset \in \mathcal B_\lambda$ car $\lambda \emptyset = \emptyset$.
			\item Soit $B \in \mathcal B_\lambda$. $\lambda B^\C = (\lambda B)^\C \in \B^n$ par stabilité par passage au complémentaire.
			\item Soit $(B_n)_{n \geq 0} \in {\B^n}^\N$. $\lambda\bigcup_{n \geq 0}B_n = \{\lambda b \st \exists n \geq 0, b \in B_n\} = \bigcup_{n \geq 0}\lambda B_n \in \B^n$ par
			stabilité d'unions dénombrables.
		\end{itemize}
		\item Mq $\left\{\prod_{k=1}^n(\minfty, b_k]\right\}_{(b_1, \ldots, b_n) \in \R^n} \subset \mathcal B_\lambda$. Soit $(b_1, \ldots, b_n) \in \R^n$.
		On sait que $\prod_{k=1}^n(\minfty, b_k] \in \B^n$ et donc $\lambda\prod_{k=1}^n(\minfty, b_k] = \prod_{k=1}^n(\minfty, \lambda b_k] \in \B^n$.
		On a donc $\left\{\prod_{k=1}^n(\minfty, b_k]\right\}_{(b_1, \ldots, b_n) \in \R^n} \subseteq \mathcal B_\lambda$.
		Et donc $\B^n \subseteq \sigma\left(\left\{\prod_{k=1}^n(\minfty, b_k]\right\}_{(b_1, \ldots, b_n) \in \R^n}\right) \subseteq \mathcal B_\lambda \subseteq \B^n$,
		et donc $\mathcal B_\lambda = \B^n$.
	\end{enumerate}
	\item On voit que $(I_n)_{n \geq 0}$ recouvre $B$ ssi $(\lambda I_n)_{n \geq 0}$ recouvre $\lambda B$. Et donc~:
	\[\mathcal L^*(\lambda B) = \inf_{(I_n)_n \in \mathcal C_B}\sum_{n \geq 0}\Vol(\lambda I_n) = \inf_{(I_n)_n \in \mathcal C_B}\sum_{n \geq 0}\lambda^n\Vol(I_n)
	= \lambda^n\inf_{(I_n)_n \in \mathcal C_B}\sum_{n \geq 0}\Vol(I_n) = \lambda^n\mathcal L^*(B).\]
\end{enumerate}
\end{proof}

\begin{ex}[Vrai ou Faux] Justifier les affirmations suivantes~:
\begin{enumerate}
	\item Si $E \subseteq \R^n$ est négligeable, alors $\overline E$ est négligeable.
	\item Il existe un ensemble non-mesurable sur $\R^n$ de complémentaire de mesure extérieure de Lebesgue nulle.
	\item Il existe des ensemble non-mesurables dont l'union est mesurable.
	\item Si $A \subset \R^n$ satisfait $\mathcal L(\mathring A) = \mathcal L(\overline A)$, alors $A$ est mesurable.
\end{enumerate}
\end{ex}

\begin{proof}~
\begin{enumerate}
	\item Faux~: $\Q$ est négligeable et $\overline \Q = \R$ n'est pas négligeable.
	\item Faux~: si $A^\C$ est de mesure extérieure de Lebesgue nulle, alors $A^\C$ est mesurable, et donc $A^\C \in \mathcal M_{\mathcal L^*}$. Or l'ensemble des mesurables
	est une $\sigma$-algèbre, et donc $A = {A^\C}^\C \in \mathcal M_{\mathcal L^*}$.
	\item Vrai~: si $A$ est non-mesurable, alors $A^\C$ ne l'est pas non plus. Or $\mathcal M_{\mathcal L^*} \ni X = A \cup A^\C$.
	\item Vrai~: par définition de complétion de mesure. Si $\mathring A$ et $\overline A$ sont mesurables et de même mesure, alors $\mathring A \subseteq A \subseteq \overline A$
	et $\mathcal L(\overline A \setminus \mathring A) = 0$. Dès lors $A \in \mathcal M_{\mathcal L^*}$ et par monotonie~:
	$\mathcal L(A) = \mathcal L(\mathring A) = \mathcal L(\overline A)$.
\end{enumerate}
\end{proof}

\end{document}
