\documentclass{article}

\usepackage[french]{babel}
\usepackage{commath}
\usepackage{palatino, eulervm}
\usepackage[T1]{fontenc}
\usepackage[utf8]{inputenc}
\usepackage{fullpage}
\usepackage{amsmath, amsthm, amssymb, amsfonts}
\usepackage{mathrsfs}
\usepackage{mathtools}
\usepackage{cases}
\usepackage{stmaryrd}
\usepackage[bottom]{footmisc}
\usepackage[parfill]{parskip}
\usepackage[framemethod=tikz]{mdframed}
\usepackage{hyperref}

\title{MATHF-3001 --- Théorie de la mesure \\Résolution des TPs}
\author{R. Petit}
\date{Année académique 2018 - 2019}

\newtheorem{ex}{Exercice}[section]
\theoremstyle{definition}
%\newtheorem{proof}{Résolution}[section]
\addto\captionsfrench{\renewcommand\proofname{\underline{Résolution}}}

\newcommand{\TODO}[1]{\textcolor{red}{TODO: #1}}

% link amsthm and mdframed
\iftrue
%\iffalse
	% pre-amsthm
	\mdfdefinestyle{resultstyle}{%
		hidealllines=true,%
		leftline=true,%
		rightline=true,%
		innerleftmargin=10pt,%
		innerrightmargin=10pt,%
		innertopmargin=10pt,%
		innerbottommargin=8pt,%
	}

	\surroundwithmdframed[style=resultstyle]{ex}
\fi

\newcommand{\restr}[2]{\left.#1\vphantom{\big|}\right|_{#2}}

\newcommand{\pinfty}{{+\infty}}
\newcommand{\minfty}{{-\infty}}

\newcommand{\st}{\text{ s.t. }}
\newcommand{\C}{\complement}
\newcommand{\N}{{\mathbb N}}
\newcommand{\Z}{{\mathbb Z}}
\newcommand{\Q}{{\mathbb Q}}
\newcommand{\R}{{\mathbb R}}
\newcommand{\B}{{\mathbb B}}
\renewcommand{\P}{{\mathbb P}}

\newcommand{\intint}[2]{\left\llbracket#1, #2\right\rrbracket}

\DeclareMathOperator{\Vol}{Vol}
\DeclareMathOperator{\Id}{Id}

\begin{document}
\maketitle

\section{Séance 1}

\begin{ex} Soient $(X, \mathcal F)$ un espace mesurable et $Y \subset X$. Mq $\mathcal F_Y \coloneqq \mathcal F \cap Y$ est une $\sigma$-algèbre sur $Y$.
\end{ex}

\begin{proof}~
\begin{enumerate}
	\item $\emptyset \in \mathcal F$, donc $\emptyset \cap Y = \emptyset \in \mathcal F_Y$.
	\item Soit $F \in \mathcal F$. $F \cap Y \in \mathcal F_Y$ et donc~:
	\[Y \setminus (F \cap Y) = Y \setminus F \cup \emptyset = Y \cap F^\C \in \mathcal F_Y\]
	car $F^\C \in \mathcal F$.
	\item Soit $(F_n)_{n \geq 0} \in \mathcal F^\N$. On sait que $\bigcup_{n \geq 0}F_n \in \mathcal F$. De plus $(F_n \cap Y)_{n \geq 0} \in {\mathcal F_Y}^\N$. Donc~:
	\[\bigcup_{n \geq 0}(F_n \cap Y) = \bigcup_{n \geq 0}F_n \cap Y \in \mathcal F_Y.\]
\end{enumerate}
\end{proof}

\begin{ex}~
\begin{enumerate}
	\item Soit $X$ un ensemble fini. Décrire la $\sigma$-algèbre engendrée par la classe des parties finies de $X$. Que peut-on dire si $X$ est fini~?
	\item Dans $X = \intint 0n$, on considère $\mathcal A = \{0\}$ et $\mathcal B = \{\{0\}, \{1, 2\}\}$. Décrire $\sigma(\mathcal A)$ et $\sigma(\mathcal B)$.
\end{enumerate}
\end{ex}

\begin{proof}~
\begin{enumerate}
	\item Soit $\mathcal F = \sigma(\{Y \in \mathcal P(X) \st \text{ $Y$ est fini }\})$. Alors~:
	\[\mathcal F = \{Y \in \mathcal P(X) \st \text{ $Y$ est au plus dénombrable ou $Y^\C$ est au plus dénombrable}\}\]
	car la famille doit être stable par complémentaire (d'où la définition symétrique par complémentarité) et par union dénombrable (d'où le fait
	que $Y$ ou $Y^\C$ soit \text{au plus dénombrable}). Si $X$ est fini, alors l'ensemble des parties finies de $X$ est exactement $\mathcal P(X)$
	qui est une $\sigma$-algèbre. Donc $\mathcal F = \sigma(\mathcal P(X)) = \mathcal P(X)$.
	\item $\sigma(\mathcal A)$ est la $\sigma$-algèbre engendrée par un unique élément donc~: $\sigma(\mathcal A) = \{\emptyset, \{0\}, \{0\}^\C, \intint 0n\}$
	où $\{0\}^\C = \intint 1n$.

	$\sigma(\mathcal B) = \{\emptyset, \{0\}, \{1, 2\}, \{0, 1, 2\}, \intint 3n, \intint 1n, \{0\} \cup \intint 3n, \intint 0n\}$.
\end{enumerate}
\end{proof}

\begin{ex} Soient $X, Y$ deux ensembles, et $f : X \to Y$.
\begin{enumerate}
	\item Si $\mathcal F$ est une $\sigma$-algèbre sur $Y$, mq $\mathcal A \coloneqq f^{-1}(\mathcal F)$ est une $\sigma$-algèbre sur $X$.
	\item Soit $\mathcal A$ une $\sigma$-algèbre sur $X$.
	\begin{enumerate}
		\item Mq $\mathcal F \coloneqq \{B \in \mathcal P(Y) \st f^{-1}(B) \in \mathcal A\}$ est une $\sigma$-algèbre sur $Y$.
		\item Que peut-on dire de $f(\mathcal A)$~?
	\end{enumerate}
\end{enumerate}
\end{ex}

\begin{proof}~
\begin{enumerate}
	\item~
	\begin{itemize}
		\item $\emptyset \in \mathcal F$ donc $\emptyset = f^{-1}(\emptyset) \in f^{-1}(\mathcal F)$.
		\item Soit $A \in \mathcal A$. Il existe $B \in \mathcal F \st f^{-1}(B) = A$. $f^{-1}(Y \setminus B) = X \setminus A \in \mathcal A$.
		\item Soit $(A_n)_{n \geq 0} \in \mathcal A^\N$. Il existe $(B_n)_{n \geq 0} \in \mathcal F^\N \st \forall n \geq 0 : A_n = f^{-1}(B_n)$.
		$\bigcup_{n \geq 0}A_n = \bigcup_{n \geq 0}f^{-1}(B_n) = f^{-1}\left(\bigcup_{n \geq 0}B_n\right) \in f^{-1}(\mathcal F)$.
	\end{itemize}
	\item~
	\begin{enumerate}
		\item~
		\begin{itemize}
			\item $\emptyset \in \mathcal A$ donc $\emptyset \in \mathcal F$.
			\item Soient $B \in \mathcal F$, $A \coloneqq f^{-1}(B)$. $f^{-1}(B^\C) = f^{-1}(Y) \setminus f^{-1}(B) = f^{-1}(B)^\C \in \mathcal A$.
			\item Soit $(B_n)_{n \geq 0} \in \mathcal F^\N$. On pose $B \coloneqq \bigcup_{n \geq 0}B_n$.
			\[f^{-1}(B) = \bigcup_{n \geq 0}f^{-1}(B_n) = \bigcup_{n \geq 0}A_n \in \mathcal A\]
			où $\forall n \geq 0 : A_n = f^{-1}(B_n) \in \mathcal A$. Donc $B \in \mathcal F$.
		\end{itemize}
		\item $f(\mathcal A)$ n'est pas nécessairement une $\sigma$-algèbre~: l'égalité $f(A^\C) = f(A)^\C$ n'est pas vraie en général. Par exemple pour
		$f : [\pm\varepsilon] \to [0, \varepsilon^2] : x \mapsto x^2$, on a~:
		\[[0, \varepsilon^2] = f([-\varepsilon, 0]) = f([\pm\varepsilon] \setminus [0, +\varepsilon])
		  \neq f([\pm\varepsilon]) \setminus f([0, \varepsilon]) = [0, \varepsilon^2] \setminus [0, \varepsilon^2] = \emptyset.\]
		Donc rien ne garantit que $f(\mathcal A)$ est stable par passage au complémentaire.

		\TODO{Donner un contre-exemple avec des $\sigma$-algèbres finies sur de petits ensembles.}
	\end{enumerate}
\end{enumerate}
\end{proof}

\begin{ex} Soient $(X, \mathcal A), (Y, \mathcal B)$ espaces mesurables. Soit $\mathcal F \subset \mathcal P(Y)$. Si $\mathcal B = \sigma(\mathcal F)$, mq $f : X \to Y$
est mesurable ssi $f^{-1}(\mathcal F) \subseteq \mathcal A$.
\end{ex}

\begin{proof} $\underline \Rightarrow$: en supposant $f$ mesurable, si $B \in \mathcal F$, alors $B \in \mathcal B$ et donc $f^{-1}(B) \in \mathcal A$.

$\underline \Leftarrow$: on pose $\mathcal B' \coloneqq \{B \in \mathcal B \st f^{-1}(B) \in \mathcal A\}$. Par le point précédent, $\mathcal B'$ est une $\sigma$-algèbre.
Par hypothèse~: $\mathcal F \subset \mathcal B'$, et donc $\sigma(\mathcal F) \subset \sigma(\mathcal B') = \mathcal B'$. Or $\mathcal B = \sigma(\mathcal F)$. De plus,
puisque $\mathcal B' \subset \mathcal B$, on a $\mathcal B \subset \mathcal B' \subset \mathcal B$, ce qui implique $\mathcal B = \mathcal B'$, i.e.~:
\[\forall B \in \mathcal B : f^{-1}(B) \in \mathcal A.\]
\end{proof}

\begin{ex}\label{ex:1.5}~
\begin{enumerate}
	\item Mq toute intersection (non-vide) de classes de Dynkin est une classe de Dynkin.
	\item Mq pour tout $\mathcal F \subset \mathcal P(X)$ il existe une plus petite classe de Dynkin au sens de l'inclusion (notée $\lambda(\mathcal F)$).
	\item Mq si $\mathcal D$ est une classe de Dynkin stable par intersections finies, alors $\mathcal D$ est une $\sigma$-algèbre.
	\item Mq si $\mathcal F \subset \mathcal P(X)$ est stable par intersections finies, alors $\lambda(\mathcal F) = \sigma(\mathcal F)$.
\end{enumerate}
\end{ex}

\begin{proof}~
\begin{enumerate}
	\item~[Exactement même raisonement que pour les $\sigma$-algèbres] Soit $(\mathcal D_i)_{i \in I}$ une famille non-vide de classes de Dynkin et soit
	$\mathcal D \coloneqq \bigcap_{i \in I}\mathcal D_i$.

	\begin{itemize}
		\item $\forall i \in I : \emptyset \in \mathcal D_i$ donc $\emptyset \in \mathcal D$.
		\item Soit $D \in \mathcal D$. Puisque $\forall i \in I : D \in \mathcal D_i$ et que les $\mathcal D_i$ sont des classes de Dynkin, on a $\forall i \in I : D^\C \in \mathcal D_i$
		et donc $D^\C \in \mathcal D$.
		\item Soit $(D_n)_{n \geq 0} \in \mathcal D^\N$. On sait que $\forall i \in I : \bigsqcup_{n \geq 0}D_n \in \mathcal D_i$ et donc $\bigsqcup_{n \geq 0}D_n \in \mathcal D$.
	\end{itemize}

	\item Comme pour les $\sigma$-algèbres, on peut définir~:
	\[\lambda(\mathcal F) \coloneqq \bigcap_{\overset{\mathcal D \text{ Dynkin}}{\mathcal F \subset \mathcal D}}\mathcal D.\]
	Par le point ci-dessus, $\lambda(\mathcal F)$ est une classe de Dynkin et toute classe de Dynkin $\mathcal D' \supset \mathcal F$ contient $\lambda(\mathcal F)$ par définition.

	\item Soit $\mathcal D$ une classe de Dynkin stable par intersections finies et soit $(D_n)_{n \geq 0} \in \mathcal D^\N$. Montrons donc que $\bigcup_{n \geq 0}D_n \in \mathcal D$.
	On pose $B_0 \coloneqq D_0$ et pour $n > 0$, on pose $B_n \coloneqq A_n \cap (\bigcap_{j=1}^{n-1}B_j^\C)$. Par récurrence, on observe que les $B_n$ sont dans $\mathcal D$ par
	stabilité sous intersections finies. De plus les $B_n$ sont disjoints deux à deux et leur union est égale à l'union des $D_n$. Donc $\bigcup_{n \geq 0}D_n \in \mathcal D$.

	\item Soit $D \in \lambda(\mathcal F)$. On pose $\mathcal D_D \coloneqq \{Q \in \lambda(\mathcal F) \st Q \cap D \in \lambda(\mathcal F)\} \subset \lambda(\mathcal F)$.
	Montrons que $\mathcal D_D$ est une classe de Dynkin.
	\begin{itemize}
		\item $\emptyset \in \mathcal D_D$ puisque $\lambda(\mathcal F) \ni \emptyset = \emptyset \cap D$.
		\item Soit $Q \in \mathcal D_D$. $Q^\C \cap D = (D^\C \cup Q)^\C = (D^\C \sqcup (\underbrace {Q \cap D}_{\in \lambda(\mathcal F)}))^\C \in \lambda(\mathcal F)$ par
		stabilité par passage au complément, stabilité par union disjointe.
		\item Soit $(Q_n)_{n \geq 0} \in {\mathcal D_D}^\N$ deux à deux disjoints. On a~:
		\[\bigsqcup_{n \geq 0}Q_n \cap D = \bigsqcup_{n \geq 0}(\underbrace {Q_n \cap D}_{\in \lambda(\mathcal F)}) \in \lambda(\mathcal F).\]
	\end{itemize}

	On remarque également que si $D \in \mathcal F$~: $\mathcal F \subset \mathcal D_D \subset \lambda(\mathcal F)$, ce qui implique $\lambda(\mathcal F) = \mathcal D_D$.

	Or par symétrie de l'intersection, pour $D, Q \in \lambda(\mathcal F)$ on a~: $Q \in \mathcal D_D \iff D \in \mathcal D_Q$. Dès lors on a une équivalence entre les deux
	assertions suivantes~:
	\begin{itemize}
		\item $\forall (D, Q) \in \mathcal F \times \lambda(\mathcal F) : Q \in \mathcal D_D$ (autrement dit $\forall D \in \mathcal F : \lambda(\mathcal F) = \mathcal D_D$)~;
		\item $\forall (D, Q) \in \mathcal F \times \lambda(\mathcal F) : D \in \mathcal D_Q$ (autrement dit $\forall Q \in \lambda(\mathcal F) : \mathcal F \subset \mathcal D_Q$).
	\end{itemize}

	On peut alors en déduire que $\forall Q \in \lambda(\mathcal F) : \lambda(\mathcal F) = \mathcal D_Q$. Dès lors, montrer que $\lambda(\mathcal F)$ est stable par
	instersections finies revient à montrer que $\forall D, Q \in \lambda(\mathcal F) : D \cap Q \in \lambda(\mathcal F)$, i.e. $D \in \mathcal D_Q = \lambda(\mathcal F)$.
	On a donc bien la stabilité de $\lambda(\mathcal F)$ sous intersections finies, on peut donc déduire que $\lambda(\mathcal F)$ est une $\sigma$-algèbre qui contient $\mathcal F$,
	donc $\sigma(\mathcal F) \subset \lambda(\mathcal F)$. Or toute $\sigma$-algèbre est une classe de Dynkin, donc $\lambda(\mathcal F) \subset \sigma(\mathcal F)$, ce qui permet
	de conclure.
\end{enumerate}
\end{proof}

\newpage
\section{Séance 2}

\begin{ex} Soient $(X, \mathcal F)$ un espace mesurable et $\mu$ une fonction additive sur $\mathcal A$ à valeurs dans $\mathbb R^+$. Mq les conditions suivantes sont équivalentes~:
\begin{enumerate}
	\item $\mu$ est $\sigma$-additive~;
	\item $\mu$ est continue à gauche~;
	\item $\mu$ est continue à droite.
\end{enumerate}

Donner un exemple de mesure $\mu : \mathcal A \to [0, \pinfty]$ qui ne satisfait pas le point 3. Que faut-il ajouter comme hypothèse pour ce résultat~?
\end{ex}

\begin{proof}~
\begin{itemize}
	\item[\underline {$1. \Rightarrow 2.$}] Soit $(B_n)_{n \geq 0} \in \mathcal A^\N$. On pose $A_0 \coloneqq B_0$ et $\forall n > 0 : A_n \coloneqq B_n \setminus B_{n-1}$,
	ce qui donne (car les $A_n$ sont dans $\mathcal A$)~:
	\[\mu\left(\bigcup_{n \geq 0}B_n\right) = \mu\left(\bigsqcup_{n \geq 0}A_n\right) = \sum_{n \geq 0}\mu(A_n)
	  = \lim_{N \to \pinfty}\underbrace {\sum_{n=0}^N\mu(A_n)}_{= \mu(B_N)} = \lim_{N \to \pinfty}\mu(B_N).\]
	\item[\underline {$2. \Rightarrow 1.$}] Soit $(A_n)_{n \geq 0} \in \mathcal A^\N$ deux à deux disjoints.
	On pose $B_0 \coloneqq A_0$ et $\forall n > 0 : B_n \coloneqq A_n \cup B_{n-1}$. Les $B_n$ forment une suite croissante dans $\mathcal A$. On a alors~:
	\[\mu\left(\bigsqcup_{n \geq 0}A_n\right) = \mu\left(\bigcup_{n \geq 0}B_n\right) = \lim_{n \to \pinfty}\mu(B_n) = \lim_{n \to \pinfty}\mu\left(\bigsqcup_{j=0}^nA_j\right)
	  = \lim_{n \to \pinfty}\sum_{j=0}^n\mu(A_j) = \sum_{n \geq 0}\mu(A_n).\]
	\item[\underline {$2. \Rightarrow 3.$}] Soit $(C_n)_{n \geq 0} \in \mathcal A^\N$ une suite décroissante. On a alors que $({C_n}^\C)_{n \geq 0}$ est une suite croissante dans
	$\mathcal A$. Donc~:
	\[\mu\left(\bigcup_{n \geq 0}{C_n}^\C\right) = \lim_{n \to \pinfty}\mu({C_n}^\C) = \mu(X) - \lim_{n \to \pinfty}\mu({C_n}^\C)\]
	car $\mu(X) < \pinfty$. De plus~:
	\[\mu\left(\bigcup_{n \geq 0}{C_n}^\C\right) = \mu\left(\left(\bigcap_{n \geq 0}C_n\right)^\C\right) = \mu(X) - \mu\left(\bigcap_{n \geq 0}C_n\right).\]
	Par finitude de $\mu$, on conclut~:
	\[\mu\left(\bigcap_{n \geq 0}C_n\right) = \lim_{n \to \pinfty}\mu(C_n).\]
	\item[\underline {$3. \Rightarrow 2.$}] Exactement même raisonnement par passage au complémentaire.
\end{itemize}

Si la mesure n'est pas finie, on peut construire une suite $(C_n)_n$ telle que $\forall n \geq 0 : \mu(C_n) = \pinfty$ et $\bigcap_{n \geq 0}C_n = \emptyset$. Par exemple,
dans l'espace mesuré $(\N, \mathcal P(\N), \# = \abs \cdot)$~: $\forall n \geq 0 : C_n \coloneqq \{m \in \N \st m > n\}$ est de mesure $\pinfty$ et
$\bigcap_{n \geq 0}C_n = \emptyset$. On a donc~:
\[\mu\left(\bigcap_{n \geq 0}C_n\right) = \mu(\emptyset) = 0 \neq \pinfty = \lim_{n \to \pinfty}\pinfty = \lim_{n \to \pinfty}\mu(C_n).\]

Il faut donc supposer que pour la suite $(C_n)_n$, il existe $n_0 \in \N \st \mu(C_n) \lneqq \pinfty$ afin d'éviter le cas où $(\mu(C_n))_{n \geq 0}$ est infinie pour tous les termes.
\end{proof}

\begin{ex} Soit $X$ un ensemble non dénombrable et $\mathcal A = \{A \in \mathcal P(X) \st A \text{ ou } A^\C \text{ est dénombrable}\}$. Soit $\mu : \mathcal A \to \{0, 1\}$
où $\mu(A) = 0 \iff A$ est dénombrable. Mq $\mu$ est une mesure sur $(X, \mathcal A)$.
\end{ex}

\begin{proof} $\mathcal A$ est une $\sigma$-algèbre (voir cours).
\begin{itemize}
	\item $\mu(\emptyset) = 0$ car $\emptyset$ est fini.
	\item Soit $(A_n)_{n \geq 0} \in \mathcal A^\N$ deux à deux disjoints. On note $A \coloneqq \bigsqcup_{n \geq 0}A_n$. On a soit $\mu(A) = 0$ ou $\mu(A) = 1$, et~:
	\[\mu(A) = 0 \iff \underbrace {\forall n \geq 0 : \mu(A_n) = 0}_{\text{i.e. tous les $A_n$ dénombrables}},\]
	et donc~:
	\[\mu(A) = 0 = \sum_{n \geq 0}0 = \sum_{n \geq 0}\mu(A_n)\]
	pour le premier cas. Pour le second cas, si $\mu(A) = 1$, il existe $n_0 \in \N \st {A_{n_0}}^\C$ est dénombrable (ou $\mu(A_{n_0}) = 1$), donc $\mu(A) \leq \sum_{n \geq 0}\mu(A_n)$.
	Supposons par l'absurde qu'il existe $n_1 \neq n_0 \st {A_{n_1}}^\C$ est dénombrable ($\mu(A_{n_1}) = 1$). On a donc ${A_{n_0}}^\C$ et ${A_{n_1}}^\C$ dénombrables.
	Or $A_{n_0} \cap A_{n_1} = \emptyset$, donc 	$A_{n_0} \subseteq {A_{n_1}}^\C$ ou $A_{n_1} \subseteq {A_{n_0}}^\C$, ce qui implique $A_{n_0}$ ou $A_{n_1}$ dénombrable,
	ce qui est une contradiction car $X$ est non-dénombrable.
\end{itemize}
\end{proof}

\begin{ex} Soit $(X, \mathcal A, \P)$ un espace de probabilité. Mq $\mathcal T \coloneqq \{A \in \mathcal A \st \P(A) \in \{0, 1\}\}$ est une $\sigma$-algèbre.
\end{ex}

\begin{proof}~
\begin{itemize}
	\item $\P(\emptyset) = 0$ donc $\emptyset \in \mathcal T$.
	\item Soit $A \in \mathcal T$. En particulier $A \in \mathcal A$ et $\P(A) \in \{0, 1\}$. Puisque $\P$ est une mesure (finie), on a
	$\P(A^\C) = \P(X) - \P(A) = 1 - \P(A) \in \{0, 1\}$. Donc $A^\C \in \mathcal T$.
	\item Soit $(A_n)_{n \geq 0} \in \mathcal T^\N$. On note $A \coloneqq \bigcup_{n \geq 0}A_n$. Mq $\P(A) \in \{0, 1\}$.
	\begin{itemize}
		\item si $\forall n \geq 0 : \P(A_n) = 0$, alors par $\sigma$-sous-additivité $0 \leq \P(A) \leq \sum_{n \geq 0}P(A_n) = 0$.
		\item si $\exists n_0 \in \N \st \P(A_{n_0}) = 1$, alors par monotonie, puisque $A_{n_0} \subseteq A \subseteq X$~:
		\[1 = \P(A_{n_0}) \leq \P(A) \leq \P(X) = 1.\]
	\end{itemize}
\end{itemize}
\end{proof}

\begin{ex} Soient $(X, \mathcal A, \mu)$ un espace mesuré, $(Y, \mathcal B)$ un espace mesurable et $g : X \to Y$ une application mesurable. On pose~:
\[\nu : \mathcal B \to [0, \pinfty] : B \mapsto \mu(g^{-1}(B)).\]
Mq $\nu$ est une mesure sur $(Y, \mathcal B)$.
\end{ex}

\begin{proof} On sait que $\forall B \in \mathcal B : g^{-1}(B) \in \mathcal A$ puisque $g$ est mesurable. Donc $\nu$ est bien définie. Mq $\nu$ est une mesure.
\begin{itemize}
	\item $\nu(\emptyset) = \mu(g^{-1}(\emptyset)) = \mu(\emptyset) = 0$ car $\mu$ est une mesure.
	\item Soient $(B_n)_{n \geq 0} \in \mathcal B^\N$ deux à deux disjoints. Mq $\nu$ est $\sigma$-additive.
	\[\nu\left(\bigsqcup_{n \geq 0}B_n\right) = \mu\left(g^{-1}\left(\bigsqcup_{n \geq 0}B_n\right)\right) = \mu\left(\bigsqcup_{n \geq 0}g^{-1}(B_n)\right)
	  = \sum_{n \geq 0}\mu(g^{-1}(B_n)) = \sum_{n \geq 0}\nu(B_n).\]
\end{itemize}
\end{proof}

\begin{ex} Soit $(X, \mathcal A)$ un espace mesurable.
\begin{enumerate}
	\item Pour $x \in X$, mq $\delta_x$ est une mesure.
	\item Mq si $\mu$ est une mesure sur $(X, \mathcal A) \st \forall A \in \mathcal A : \mu(A) = 0 \iff x \not \in A$ alors $\exists C \gneqq 0 \st \mu = C\delta_x$.
\end{enumerate}
\end{ex}

\begin{proof}~
\begin{enumerate}
	\item Mq $\delta_x$ est une mesure.
	\begin{itemize}
		\item $\delta_x(\emptyset) = 0$ car $x \not \in \emptyset$.
		\item Soit $(A_n)_{n \geq 0} \in \mathcal A^\N$ 2 à 2 disjoints. Mq $\delta_x(\bigsqcup_{n \geq 0}A_n) = \sum_{n \geq 0}\delta_x(A_n)$.
		\begin{itemize}
			\item Si $\delta_x(\bigsqcup_{n \geq 0}A_n) = 0$, alors $\forall n \geq 0 : x \not \in A_n$, i.e. $\forall n \geq 0 : \delta_x(A_n) = 0$.
			\item Si $\delta_x(\bigsqcup_{n \geq 0}A_n) = 1$, alors $\exists n_0 \st x \in A_{n_0}$. Et puisque les $A_n$ sont disjoints, $\forall n \neq n_0 : x \not \in A_n$.
		\end{itemize}
	\end{itemize}
	\item Soient $B, C \in \mathcal A \st \mu(B) \neq 0 \neq \mu(C)$. Alors $\delta_x(B) = 1 = \delta_x(C)$. Mq $\mu(B) = \mu(C)$. $B \cap C \neq \emptyset$ puisque $x \in B \cap C$.
	On pose $\tilde C \coloneqq C \cap B^\C$ et $\tilde B \coloneqq C^\C \cap B$. On a alors que $B$ et $\tilde C$ sont disjoints ($C$ et $\tilde B$ également).
	De plus, $x \not \in \tilde B$ et $x \not \in \tilde C$, et donc $\mu(\tilde B) = \mu(\tilde C) = 0$. On a donc~:
	\[\mu(C) = \mu(C) + \mu(\tilde B) = \mu(C \sqcup \tilde B) = \mu(B \cup C) = \mu(B \sqcup \tilde C) = \mu(B) + \mu(\tilde C) = \mu(B).\]
	On a donc $\mu : \mathcal A \to \{0, C\}$ où $\mu(A) \iff \delta_x(A) = 1$.
\end{enumerate}
\end{proof}

\begin{ex} Soit $(X, \mathcal A)$ un espace mesurable. Mq la mesure de comptage est une mesure.
\end{ex}

\begin{proof}~
\begin{itemize}
	\item $\abs \emptyset = 0$.
	\item La $\sigma$-additivité est triviale~: $\abs {\bigsqcup_{n \geq 0}A_n} = \sum_{n \geq 0}\abs A_n$.
\end{itemize}
\end{proof}

\begin{ex} Soit $X$ un ensemble fini non-vide. Mq $\mu = \frac {\abs \cdot}{\abs X}$ est une mesure de proba sur $(X, \mathcal P(X))$.
\end{ex}

\begin{proof}~
\begin{itemize}
	\item $\mu(\emptyset) = 0/\abs X = 0$.
	\item Soient $(A_n)_{n \geq 0} \in \mathcal P(X)^\N$ 2 à 2 disjoints.
	\[\mu(\bigsqcup_{n \geq 0}A_n) = \frac {\sum_{n \geq 0}\abs {A_n}}{\abs X} = \sum_{n \geq 0}\frac {\abs {A_n}}{\abs X}.\]
\end{itemize}

\underline {\textit{Note:}} si $(X, \mathcal A, \mu)$ est un espace mesuré, alors $\forall \alpha > 0 : \alpha\mu : \mathcal A \to [0, \pinfty] : A \mapsto \alpha \cdot \mu(A)$
est une mesure sur $(X, \mathcal A)$. Donc l'exercice peut être simplement résolu par le fait que $\mu$ est la mesure de comptage normalisée par $\abs X \in {\R^+}^*$
\end{proof}

\begin{ex} Soit $(X, \mathcal A)$ un espace de mesure.
\begin{enumerate}
	\item Soit $(\mu_n)_{n \geq 0}$ une suite croissante de mesures sur $(X, \mathcal A)$. Mq $\mu \coloneqq \lim_{n \to \pinfty}\mu_n$ est une mesure.
	\item Soit $(\mu_n)_{n \geq 0}$ une suite de mesures. Est-ce que $\mu \coloneqq \sum_{n \geq 0}\mu_n$ est une mesure~?
	\item Pour $n \geq 0$, on définit la mesure $\mu_n$ sur $(\N, \mathcal P(\N))$ par $\mu_n(A) = \abs {A \cap [n, \pinfty)}$.
	\begin{itemize}
		\item Mq $\forall n \geq 0 : \mu_n$ est bien une mesure et que la suite $(\mu_n)_n$ est décroissante.
		\item Est-ce que $\mu = \lim_{n \to \pinfty}\mu_n$ est une mesure sur $(\N, \mathcal P(\N))$~? Caractériser entièrement $\mu$.
	\end{itemize}
\end{enumerate}
\end{ex}

\begin{proof}~
\begin{enumerate}
	\item On note que puisque la suite des $\mu_n$ est croissante, pour tout $A \in \mathcal A$, $\mu(A)$ est bien définie car soit la suite $(\mu_n(A))_n$ converge vers une
	valeur réelle, soit elle diverge vers $\pinfty$.
	\begin{itemize}
		\item $\mu(\emptyset) = \lim_{n \to \pinfty}\mu_n(\emptyset) = 0$.
		\item Soient $(A_n)_{n \geq 0}$ 2 à 2 disjoints.
		\begin{align*}
			\mu\left(\bigsqcup_{n \geq 0}A_n\right) &= \lim_{k \to \pinfty}\mu_k\left(\bigsqcup_{n \geq 0}A_n\right) = \lim_{k \to \pinfty}\lim_{N \to \pinfty}\sum_{n=0}^N\mu_k(A_n) \\
				&= \lim_{N \to \pinfty}\sum_{n=0}^N\lim_{k \to \pinfty}\mu_k(A_n) = \sum_{n \geq 0}\lim_{k \to \pinfty}\mu_k(A_n) = \sum_{n \geq 0}\mu(A_n).
		\end{align*}
	\end{itemize}
	\item~
	\begin{itemize}
		\item $\mu(\emptyset) = \sum_{n \geq 0}\mu_n(\emptyset) = 0$.
		\item Soient $(A_n)_{n \geq 0} \in \mathcal A^\N$ 2 à 2 disjoints. On note $A \coloneqq \bigsqcup_{n \geq 0}A_n$. Par non-négativité des $(\mu_k(A_n))_{n,k}$,
		on a que les sommes sur $k$ et $n$ commutent, i.e.~:
		\[\sum_{k \geq 0}\sum_{n \geq 0}\mu_k(A_n) = \sum_{n \geq 0}\sum_{k \geq 0}\mu_k(A_n),\]
		et donc $\mu(A) = \sum_{n \geq 0}\mu(A_n)$.
	\end{itemize}
	On en déduit donc que $\mu = \sum_{k \geq 0}\mu_k$ est une mesure sur $(X, \mathcal A)$. De plus, puisque $\alpha \cdot \mu$ (pour $\alpha > 0$, $\mu$ mesure sur $(X, \mathcal A)$)
	est également une mesure sur $(X, \mathcal A)$, on a que pour $(\alpha_n)_{n \geq 0} \in \left({\R^+}^*\right)^\N$~: $\mu = \sum_{n \geq 0}\alpha_n\mu_n$ est une mesure également.
	\item~
	\begin{itemize}
		\item Soit $n \in \N$.
		\begin{itemize}
			\item $\mu_n(\emptyset) = \abs \emptyset = 0$.
			\item La $\sigma$-additivité est triviale par la $\sigma$-additivité de la mesure de comptage.
		\end{itemize}
		De plus, pour $A \in \mathcal P(\N)$ et $n \in \N$: $\mu_n(A) = \big|\underbrace  {A \cap [n, \pinfty)}_{\supseteq A \cap [n+1, \pinfty)}\big| \geq \mu_{n+1}(A)$.
		\item Soit $A \in \mathcal P(\N)$. Deux cas sont à distinguer~:
		\begin{enumerate}
			\item Soit $A$ est fini, en quel cas $\max A$ est fini et donc $\forall n > \max A : \mu_n(A) = 0$, et donc $\mu(A) = 0$.
			\item Soit $A$ est infini, et donc dénombrable. On a alors $\forall n \geq 0 : A \cap [n, \pinfty) \neq \emptyset$ car si il existe un $n \geq 0$ tel que
			$A \cap [n, \pinfty) = \emptyset$, alors $A \subset [0, n) \cap \N$, et donc $A$ est fini. Dès lors $\mu(A) > 0$.

			De plus~: $\forall n \geq 0 : \mu_n(A) = \pinfty$. Car si $\exists n \geq 0 \st \mu_n(A) \lneq \pinfty$, alors $\mu_n(A) = \abs {A \cap [n, \pinfty)} = k \in \N$
			et donc $A \cap [n, \pinfty) = \{m_1, \ldots, m_k\}$. Dans ce cas~: $\mu_{m_k+1}(A) = 0$, ce qui est une contradiction.

			On en déduit que si $A$ est infini (dénombrable), alors $\mu(A) = \pinfty$.
		\end{enumerate}
		$\mu$ vaut donc $0$ sur les parties finies de $\N$ et $\pinfty$ sur les parties dénombrables. $\mu$ n'est donc pas une mesure car~:
		$\mu(\N) = \sum_{n \in \N}\mu(\{n\}) = \sum_{n \geq 0}0 = 0 \neq \pinfty$.
	\end{itemize}
\end{enumerate}
\end{proof}

\begin{ex} Soient $(X, \mathcal A, \mu)$ un espace mesuré et $(A_n)_{n \geq 0} \in {\mathcal A}^\N$.
\begin{enumerate}
	\item Mq~:
	\[\mu\left(\liminf_{n \to \pinfty}A_n\right) \leq \liminf_{n \to \pinfty}\mu(A_n).\]
	\item Si $\exists n_0 \in \N \st \mu\left(\bigcup_{n \geq n_0}A_n\right) \leq \pinfty$, mq~:
	\[\mu\left(\limsup_{n \to \pinfty}A_n\right) \geq \limsup_{n \to \pinfty}\mu(A_n).\]
\end{enumerate}
\end{ex}

\begin{proof}~
\begin{enumerate}
	\item Pour $n \geq 0$~: on pose $B_n \coloneqq \cap_{m \geq n}A_m$. La suite $(B_n)_{n \geq 0}$ est trivialement croissante. On a donc~:
	\[\mu\left(\liminf_{n \to \pinfty}A_n\right) = \mu\left(\bigcup_{n \geq 0}B_n\right) = \lim_{n \to \pinfty}\mu(B_n),\]
	et~:
	\[\liminf_{n \to \pinfty}\mu(A_n) = \lim_{n \to \pinfty}\inf_{k \geq n}\mu(A_k).\]

	De plus~: $\mu(B_n) = \mu\left(\bigcap_{k \geq n}A_k\right) \leq \mu(A_m)$ pour $m \geq n$ par monotonie de $\mu$, et donc en particulier $\mu(B_n) \leq \inf_{k \geq n}\mu(A_k)$.
	Dès lors la suite $\mu(B_n)_n$ est dominée par $(\inf_{k \geq n}\mu(A_k))_n$. Dès lors~:
	\[\mu\left(\liminf_{n \to \pinfty}A_n\right) = \lim_{n \to \pinfty}\mu(B_n) \leq \lim_{n \to \pinfty}\inf_{k \geq n}\mu(A_k) = \liminf_{n \to \pinfty}\mu(A_n).\]
	\item On pose $C_n \coloneqq \bigcup_{m \geq n}A_m$. On sait qu'il existe $n_0 \in \N \st \mu\left(C_{n_0}\right) \lneqq \pinfty$. Les $C_n$ forment une suite décroissante.
	On a donc~:
	\[\mu\left(\limsup_{n \to \pinfty}A_n\right) = \mu\left(\bigcap_{n \geq 0}C_n\right) = \lim_{n \to \pinfty}\mu(C_n).\]
	De plus~:
	\[\limsup_{n \to \pinfty}\mu(A_n) = \lim_{n \to \pinfty}\sup_{k \geq n}\mu(A_k).\]
	Or $\forall k \geq n : C_n \supseteq A_k$ et donc $\forall k \geq n : \mu(C_n) \geq \mu(A_k)$, et en prticulier $\mu(C_n) \geq \sup_{k \geq n}\mu(A_k)$. Dès lors on conclut~:
	\[\mu\left(\limsup_{n \to \pinfty}A_n\right) = \lim_{n \to \pinfty}\mu(C_n) \geq \lim_{n \to \pinfty}\sup_{k \geq n}\mu(A_k) = \limsup_{n \to \pinfty}\mu(A_n).\]
\end{enumerate}
\end{proof}

\begin{ex} Soit $(X, \mathcal A)$ un espace mesurable. Soient $\mu, \nu$ deux mesures finies sur $(X, \mathcal A)$ telles que
$\forall A \in \mathcal A : \mu(A) \leq \frac 12 \Rightarrow \mu(A) = \nu(A)$.
\begin{enumerate}
	\item Mq $\mu = \nu$.
	\item Mq le résultat est faux si l'inégalité est changée en inégalité stricte.
\end{enumerate}
\end{ex}

\begin{proof}~
\begin{enumerate}
	\item Soit $B \in \mathcal A \st \mu(B) \gneqq \frac 12$. Alors $\mu(B^\C) = \mu(X) - \mu(B) = 1 - \mu(B) < \frac 12$. Dès lors $\mu(B^\C) = \nu(B^\C)$ par hypothèse,
	et on en déduit $\mu(B) = 1 - \mu(B^\C) = 1 - \nu(B^\C) = \nu(B)$, et donc $\mu = \nu$.
	\item Si l'inégalité devient stricte, on peut choisir, sur l'espace mesurable $(\{0, 1\}, \mathcal P(\{0, 1\}))$, $\mu$ la mesure d'une Bernoulli de proba
	$\frac 12$ et $\nu$ la mesure d'une Bernoulli de proba $\frac 13$. On a alors~:

	\begin{tabular}{c|c|c}
		$A$ & $\mu(A)$ & $\nu(A)$ \\
		\hline
		$\emptyset$ & $0$ & $0$ \\
		$\{0\}$ & $\frac 12$ & $\frac 23$ \\
		$\{1\}$ & $\frac 12$ & $\frac 13$ \\
		$\{0, 1\}$ & $1$ & $1$
	\end{tabular}

	Puisque $\mathcal B \coloneqq \{A \in \mathcal P(\{0, 1\}) \st \mu(A) \lneqq \frac 12\} = \{\emptyset\}$, on a bien $\mu = \nu$ sur $\mathcal B$, mais $\mu \neq \nu$.
\end{enumerate}
\end{proof}

\begin{ex} Soient $(X, \mathcal A)$ un espace mesurable et une partie stable par intersections finies $\mathcal F \subset \mathcal P(X) \st \sigma(\mathcal F) = \mathcal A$.
Si $\mu$ et $\nu$ sont deux mesures finies sur $(X, \mathcal A)$ telles que $\nu(X) = \mu(X)$ et $\mu = \nu$ sur $\mathcal F$. Mq $\mu = \nu$.
\end{ex}

\begin{proof} On pose $\mathcal D \coloneqq \{A \in \mathcal A \st \mu(A) = \nu(A)\}$. Mq $\mathcal D$ est une classe de Dynkin~:
\begin{itemize}
	\item $\emptyset \in \mathcal D$ car $\mu(\emptyset) = 0 = \nu(\emptyset)$.
	\item Soit $A \in \mathcal D$. $\mu(A^\C) = \mu(X) - \mu(A) = \nu(X) - \nu(A) = \nu(A^\C)$ et donc $A^\C \in \mathcal D$.
	\item Soient $(A_n)_{n \geq 0} \in {\mathcal A}^\N$ 2 à 2 disjoints et $A \coloneqq \bigsqcup_{n \geq 0}A_n$.
	\[\mu(A) = \sum_{n \geq 0}\mu(A_n) = \sum_{n \geq 0}\nu(A_n) = \nu(A).\]
	On en conclut $A \in \mathcal D$.
\end{itemize}

De plus par hypothèse $\mathcal F \subset \mathcal D$, et donc par l'exercice~\ref{ex:1.5} on a $\mathcal D = \sigma(\mathcal F) = \mathcal A$. Dès lors $\mu = \nu$ sur $\mathcal A$,
et donc $\mu = \nu$.
\end{proof}

\newpage
\section{Séance 3}
\begin{ex} Soient $\B$ la tribu borélienne sur $\R$ et $\mathcal L$ la mesure de Lesbesgue sur $\B$.
\begin{enumerate}
	\item Mq $\forall x \in \R : \{x\} \in \B$.
	\item Mq $\Q \in \B$ et $\mathcal L(\Q) = 0$.
	\item Mq une union non-dénombrable d'ensembles négligeables n'est pas nécessairement négligeable.
	\item Mq $N \in \B$ est un ensemble négligeable ssi $\forall \varepsilon > 0 : \exists U_\varepsilon \st N \subseteq U_\varepsilon$ et $\mathcal L(U_\varepsilon) < \varepsilon$.
\end{enumerate}
\end{ex}

\begin{proof}~
\begin{enumerate}
	\item $\{x\} = [x, x]$ est fermé dans $\R$, et $\mathcal L(\{x\}) = x-x = 0$.
	\item $\mathcal L(\Q) = \mathcal L(\bigsqcup_{q \in \Q}\{q\}) = \sum_{q \in \Q}\mathcal L(\{q\}) = 0$.
	\item $\mathcal L(\R) = \pinfty$, or~: $\bigsqcup_{x \in \R}\{x\}$.
	\item $\underline {\Leftarrow}$~: par monotonie, si $\forall \varepsilon > 0 : N \subseteq U_\varepsilon$, alors $\mathcal L(N) \leq \mathcal L(U_\varepsilon) < \varepsilon$.
	On en déduit $\mathcal L(N) = 0$, et donc $N$ est négligeable.

	$\underline {\Rightarrow}$~: Soit $N \in \B \st \mathcal L(N) = 0$. Pour $\varepsilon > 0$~: mq $\exists U_\varepsilon$ ouvert $\st \mathcal L(U_\varepsilon) < \varepsilon$
	et $N \subset U_\varepsilon$. Rappelons la mesure extérieure de Lebesgue~:
	\[\mathcal L^*(A) \coloneqq \inf_{(I_n)_{n \geq 0} \in \mathcal C_A}\sum_{n \geq 0}\Vol(I_n).\]

	Soit $(I_n)_{n \geq 0} \in \mathcal C_N$. Les $I_n$ sont compacts. On peut prendre une nouvelle suite $(J_n)_{n \geq 0} \st \forall n \geq 0 :
	\Vol(\overline {J_n}) < \Vol(I_n) + \frac \varepsilon{2^{n+1}}$\footnote{Par exemple en prenant l'intervalle ouvert $J_n = (a-\varepsilon/2^{n+2}, b+\varepsilon/2^{n+2})$
	pour $I_n = [a, b]$.} et $I_n \subseteq J_n$. Par $\sigma$-sous-additivité (parce que les $J_n$ ne sont pas forcément mutuellement disjoints),
	pour $J = \bigcup_{n \geq 0}J_n \supseteq N$~:
	\[\mathcal L^*(N) \leq \mathcal L^*(\bigcup_{n \geq 0}J_n) \leq \sum_{n \geq 0}\mathcal L^*(J_n).\]

	Or, pour $n \geq 0$~: $\mathcal L^*(J_n) \leq \Vol(I_n) + \frac \varepsilon{2^{n+1}}$. Finalement~:
	\[\mathcal L^*(J) < \sum_{n \geq 0}\left(\Vol(I_n) + \frac \varepsilon{2^{n+1}}\right) = 0 + \varepsilon.\]

	\underline {Note~:} si on définit un ensemble négligeable comme étant inclus dans un ensemble de mesure nulle (et pas comme étant un ensemble de mesure nulle, comme considéré
	ci-dessus), l'implication $\Leftarrow$ est triviale.
\end{enumerate}
\end{proof}

\begin{ex} Montrer qu'une droite $E$ dans $\R^2$ est de mesure nulle pour $\mathcal L$.
\end{ex}

\begin{proof} À $x = (x_1, x_2), y = (y_1, y_2) \in \R^2$ fixés, $E = \{x + ty\}_{t \in \R}$. Pour $\alpha < \beta \in \R$, on définit
$E_\alpha^\beta \coloneqq \{x+ty\}_{t \in [\alpha, \beta]}$. Mq $\mathcal L(E_\alpha^\beta) = 0$.

Si $y = (0, \lambda)$ (ou si $y = (\lambda, 0)$ par symétrie), on peut recouvrir $E_\alpha^\beta$ par l'intervalle compact
$[x_1 \pm \varepsilon] \times [x_2+\alpha\lambda, x_2+\beta\lambda]$ de volume arbitrairement petit (pour $\varepsilon$ aussi petit que nécessaire), et donc
$\mathcal L(E_\alpha^\beta) = 0$.

Sinon, soit $(I_n)_{n \geq 0}$ un recouvrement de $E_\alpha^\beta$ par des intervalles compacts. Pour $n \geq 0 : I_n = [a_n^1, b_n^1] \times [a_n^2, b_n^2]$ et
$\Vol(I_n) = (b_n^1-a_n^1)(b_n^2-a_n^2)$.

Montrons qu'il existe $(J_n)_{n \geq 0} \st \sum_{n \geq 0}\Vol(J_n) < \frac 12\sum_{n \geq 0}\Vol(I_n)$.

Soit $n \geq 0$. $I_n = [a_n^1, b_n^1] \times [a_n^2, b_n^2]$ où les 4 \textit{coins} sont $C_1 = (a_n^1, a_n^2), C_2 = (a_n^1, b_n^2), C_3 = (b_n^1, a_n^2), C_4 = (b_n^1, b_n^2)$.
WLOG supposons $\abs {E_\alpha^\beta \cap \{C_i\}_{i=1}^4} = 2$, i.e. $E$ passe par deux coins de $I_n$ (soit $C_1$ et $C_3$, soit $C_2$ et $C_4$)\footnote{En effet, si ce n'est pas
le cas, on peut "réduire" $I_n$ afin que ce soit le cas (et qui est donc de volume strictement inférieur).}. On définit alors~:
\[\begin{cases}
	J_{2n}   &\coloneqq [a_n^1, \frac 12(b_n^1 + a_n^1)] \times [a_n^2, \frac 12(b_n^2 + a_n^2)] \\
	J_{2n+1} &\coloneqq [\frac 12(b_n^1 + a_n^1), b_n^1] \times [\frac 12(b_n^2 + a_n^2), b_n^2]
\end{cases}\]
si $E \cap \{C_1, C_3\}$~; et~:
\[\begin{cases}
	J_{2n}   &\coloneqq [a_n^1, \frac 12(b_n^1 + a_n^1)] \times [\frac 12(b_n^2 + a_n^2), b_n^2] \\
	J_{2n+1} &\coloneqq [\frac 12(b_n^1 + a_n^1), b_n^1] \times [a_n^2, \frac 12(b_n^2 + a_n^2)]
\end{cases}\]
sinon.

On a bien $\Vol(I_n) = 2\left(\Vol(J_{2n}) + \Vol(J_{2n+1})\right)$, et donc $\sum_{n \geq 0}\Vol(I_n) = 2\sum_{n \geq 0}\Vol(J_n)$.

Et donc~:
\[\mathcal L^*(E_\alpha^\beta) = \inf_{(I_n)_n \in \mathcal C_{E_\alpha^\beta}}\sum_{n \geq 0}\Vol(I_n) = 0.\]

On a alors que $E_\alpha^\beta$ est mesurable et de mesure de Lebesgue nulle. Et on trouve que~:
\[\mathcal L^*(E) = \mathcal L^*(\bigcup_{n \geq 0}E_{-n}^{+n}) = \lim_{n \to \pinfty}\mathcal L^*(E_{-n}^{+n}) = 0.\]

Donc $E$ est également mesurable pour $\mathcal L$ et est de mesure nulle.
\end{proof}

\begin{ex} Pour $B \in \B^n$ et $\lambda > 0$, on définit $\lambda B = \{\lambda b\}_{b \in B}$.
\begin{enumerate}
	\item Mq $\forall \lambda > 0, B \in \B^n : \lambda B \in \B^n$.
	\item Mq $\mathcal L(\lambda B) = \lambda^n\mathcal L(B)$.
\end{enumerate}
\end{ex}

\begin{proof} On note $\mathcal B_\lambda \coloneqq \{B \in \B^n \st \lambda B \in \B^n\}$.
\begin{enumerate}
	\item~
	\begin{enumerate}
		\item Mq $\mathcal B_\lambda$ est une $\sigma$-algèbre.
		\begin{itemize}
			\item $\emptyset \in \mathcal B_\lambda$ car $\lambda \emptyset = \emptyset$.
			\item Soit $B \in \mathcal B_\lambda$. $\lambda B^\C = (\lambda B)^\C \in \B^n$ par stabilité par passage au complémentaire.
			\item Soit $(B_n)_{n \geq 0} \in {\B^n}^\N$. $\lambda\bigcup_{n \geq 0}B_n = \{\lambda b \st \exists n \geq 0, b \in B_n\} = \bigcup_{n \geq 0}\lambda B_n \in \B^n$ par
			stabilité d'unions dénombrables.
		\end{itemize}
		\item Mq $\left\{\prod_{k=1}^n(\minfty, b_k]\right\}_{(b_1, \ldots, b_n) \in \R^n} \subset \mathcal B_\lambda$. Soit $(b_1, \ldots, b_n) \in \R^n$.
		On sait que $\prod_{k=1}^n(\minfty, b_k] \in \B^n$ et donc $\lambda\prod_{k=1}^n(\minfty, b_k] = \prod_{k=1}^n(\minfty, \lambda b_k] \in \B^n$.
		On a donc $\left\{\prod_{k=1}^n(\minfty, b_k]\right\}_{(b_1, \ldots, b_n) \in \R^n} \subseteq \mathcal B_\lambda$.
		Et donc $\B^n \subseteq \sigma\left(\left\{\prod_{k=1}^n(\minfty, b_k]\right\}_{(b_1, \ldots, b_n) \in \R^n}\right) \subseteq \mathcal B_\lambda \subseteq \B^n$,
		et donc $\mathcal B_\lambda = \B^n$.
	\end{enumerate}
	\item On voit que $(I_n)_{n \geq 0}$ recouvre $B$ ssi $(\lambda I_n)_{n \geq 0}$ recouvre $\lambda B$. Et donc~:
	\[\mathcal L^*(\lambda B) = \inf_{(I_n)_n \in \mathcal C_B}\sum_{n \geq 0}\Vol(\lambda I_n) = \inf_{(I_n)_n \in \mathcal C_B}\sum_{n \geq 0}\lambda^n\Vol(I_n)
	= \lambda^n\inf_{(I_n)_n \in \mathcal C_B}\sum_{n \geq 0}\Vol(I_n) = \lambda^n\mathcal L^*(B).\]
\end{enumerate}
\end{proof}

\begin{ex}[Vrai ou Faux] Justifier les affirmations suivantes~:
\begin{enumerate}
	\item Si $E \subseteq \R^n$ est négligeable, alors $\overline E$ est négligeable.
	\item Il existe un ensemble non-mesurable sur $\R^n$ de complémentaire de mesure extérieure de Lebesgue nulle.
	\item Il existe des ensemble non-mesurables dont l'union est mesurable.
	\item Si $A \subset \R^n$ satisfait $\mathcal L(\mathring A) = \mathcal L(\overline A)$, alors $A$ est mesurable.
\end{enumerate}
\end{ex}

\begin{proof}~
\begin{enumerate}
	\item Faux~: $\Q$ est négligeable et $\overline \Q = \R$ n'est pas négligeable.
	\item Faux~: si $A^\C$ est de mesure extérieure de Lebesgue nulle, alors $A^\C$ est mesurable, et donc $A^\C \in \mathcal M_{\mathcal L^*}$. Or l'ensemble des mesurables
	est une $\sigma$-algèbre, et donc $A = {A^\C}^\C \in \mathcal M_{\mathcal L^*}$.
	\item Vrai~: si $A$ est non-mesurable, alors $A^\C$ ne l'est pas non plus. Or $\mathcal M_{\mathcal L^*} \ni X = A \cup A^\C$.
	\item Vrai~: par définition de complétion de mesure. Si $\mathring A$ et $\overline A$ sont mesurables et de même mesure, alors $\mathring A \subseteq A \subseteq \overline A$
	et $\mathcal L(\overline A \setminus \mathring A) = 0$. Dès lors $A \in \mathcal M_{\mathcal L^*}$ et par monotonie~:
	$\mathcal L(A) = \mathcal L(\mathring A) = \mathcal L(\overline A)$.
\end{enumerate}
\end{proof}

\newpage
\section{Séance 4}

\begin{ex} Soit $f : (X, \mathcal A) \to \R$. Mq si $\forall q \in \Q : f^{-1}((q, \pinfty)) \in \mathcal A$, alors $f$ est mesurable.
\end{ex}

\begin{proof} Pour cela, montrons que $f^{-1}((r, \pinfty)) \in \mathcal A$ pour $r \in \R \setminus \Q$. Par densité de $\Q$ dans $\R$, on sait que pour $r \in \R$~:
\[(r, \pinftyà) = \bigcup_{q \in (r, \pinfty) \cap \Q}(q, \pinfty).\]

Dès lors, pour $r \in \R$~: $f^{-1}((r, \pinfty)) = \bigcup_{q \in (r, \pinfty) \cap \Q}\underbrace {f^{-1}((q, \pinfty))}_{\in \mathcal A} \in \mathcal A$.
\end{proof}

\begin{ex} Mq les fonctions $f$ et $g$ sont mesurables sur $(\R, \B)$.
\end{ex}

\begin{proof} Pour cela, on utilise le fait qu'un produit de fonctions mesurables est mesurable et que~:
\begin{itemize}
	\item les fonctions caractéristiques sur des boréliens sont mesurables~;
	\item $\alpha : \R \to \R : x \mapsto x^2$ est mesurable car $\alpha = \Id \cdot \Id$~;
	\item $(\R \setminus \Q) \cap [0, 1]$ est mesurable car $[0, 1]$ est mesurable et $\Q^\C$ est mesurable aussi.
\end{itemize}

Donc puisque $f = \alpha\chi_{[0, 1]} + \chi_{(1, 2)}$ et $g = \alpha\chi_{\Q^\C \cap [0, 1]}$, on a $f$ et $g$ mesurables.
\end{proof}

\begin{ex} Soient $(X, \mathcal A)$ un espace mesurable et $(f_k)_{k \geq 0}$ une suite de fonctions mesurables de $X$ dans $\R$. Mq
l'ensemble $A \coloneqq \{x \in X \st \lim_{k \to \pinfty}f_k(x) \textit{ existe}\}$ est mesurable.
\end{ex}

\begin{proof} La limite de la suite $(f_k(x))_{k \geq 0}$ existe ssi $f_1 \coloneqq \limsup_{k \to \pinfty}f_k$ et $f_2 \coloneqq \liminf_{k \to \pinfty}f_k$ existent en $x$
et sont identiques. On sait que $f_1$ et $f_2$ sont mesurables, et donc que $f_1 - f_2$ l'est également. Or $A = (f_1-f_2)^{-1}(\{0\})$, donc $A \in \mathcal A$.
\end{proof}

\begin{proof}[Résolution alternative par les suites de Cauchy] À $x \in X$ fixé, la suite $(f_k(x))_{k \geq 0}$ est une suite réelle et donc converge ssi elle est de Cauchy,
i.e. $\lim_{k \to \pinfty}f_k(x)$ existe ssi
\[\forall \varepsilon > 0 : \exists N \in \N \st \forall m, n \geq N : \abs {f_m(x) - f_n(x)} < \varepsilon.\]

Donc $A$ peut s'écrire comme union/intersection de Boréliens. Pour $m, n \in \N$, posons $f_{m,n} \coloneqq \abs {f_m-f_n}$. Par mesurabilité des $f_{m,n}$, on observe~:
\[A = \left\{x \in X \st \forall \varepsilon > 0 : \exists N \in \N \st \forall m, n \geq N : f_{m,n}(x) < \varepsilon\right\}
	= \bigcap_{\varepsilon \in {\Q^*}^+}\bigcup_{N \in \N}\bigcap_{m,n \geq N}f_{m,n}^{-1}((\pm \varepsilon)) \in \B.\]
\end{proof}

\begin{ex} Soient $f : \R \to \R$ Borel-mesurable et $g : \R \to \R \st g \neq f$ sur un ensemble $D$ au plus dénombrable. Mq $g$ est Borel-mesurable.
\end{ex}

\begin{proof} Rappelons d'abord que les singletons sont des fermés et donc des Boréliens. Dès lors, $D \in \B$ et $\mathcal L(D) = 0$. De plus, par passage au complémentaire,
on sait que $D^\C \in \B$ également. Pour $b \in \R$~:
\[g^{-1}((\minfty, b]) = \left[\underbrace {f^{-1}((\minfty, b]) \cap D^\C}_{\eqqcolon A}\right] \sqcup \left[\underbrace {g^{-1}((\minfty, b]) \cap D}_{\eqqcolon B}\right].\]
Puisque $B \subseteq D$, on a $B$ au plus dénombrable, et en particulier $B \in \B$. Par mesurabilité de $f$, on sait que $f^{-1}((\minfty, b]) \in \B$, et puisque $D^\C \in \B$,
on déduit que $A \in \B$. Dès lors $g^{-1}((\minfty, b])$ est union de deux Boréliens, et est donc un Borélien.
\end{proof}

\begin{ex} Sur un espace mesurable $(X, \mathcal A)$, mq $\chi_A$ est mesurable ssi $A \in \mathcal A$.
\end{ex}

\begin{proof} \underline {$\Rightarrow$}~: si $\chi_A$ est mesurable, alors $A = \chi_A^{-1}(\{1\}) \in \mathcal A$.

\underline {$\Leftarrow$}~: si $A \in \mathcal A$, alors~:
\begin{itemize}
	\item $\chi_A^{-1}(\{0, 1\}) = X \in \mathcal A$ et $\chi_A^{-1}(\emptyset) = \emptyset \in \mathcal A$ puisque $\mathcal A$ est une $\sigma$-algèbre sur $X$~;
	\item $\chi_A^{-1}(\{1\}) = A \in \mathcal A$ par hypothèse~;
	\item et $\chi_A^{-1}(\{0\}) = A^\C \in \mathcal A$ par passage au complémentaire.
\end{itemize}
\end{proof}

\begin{ex} Soient $(X, \mathcal A)$ un espace mesurable et $f : X \to \R$ une fonction mesurable. Mq $f^+$ et $f^-$ sont mesurables.
\end{ex}

\begin{proof} Trivial~: $\min$ et $\max$ de fonctions mesurables sont mesurables et la fonction $\mathbf 0 : X \to \{0\} : x \mapsto 0$ est constante donc mesurable.
\end{proof}

\begin{ex} Mq $f : [0, 1) \to [0, 1) : x \mapsto \begin{cases}2x & \text{ si } x \in [0, 1/2) \\ 2x-1 & \text{ si } x \in [1/2, 1)\end{cases}$ est mesurable.

Mq pour tout $E \subseteq [0, 1)$ mesurable~: $\mathcal L(E) = \mathcal L(f^{-1}(E))$.
\end{ex}

\begin{proof} Soit $M \subseteq [0, 1)$ mesurable.
\[f^{-1}(M) = (f^{-1}(M) \cap [0, 1/2)) \sqcup (f^{-1}(M) \cap [1/2, 1)) = \{x \in [0, 1/2) \st f(x) \in M\} \sqcup \{x \in [1/2, 1) \st f(x) \in M\}.\]
Notons respectivement $M_1$ et $M_2$ ces deux ensembles. $M_1 = \{x \in [0, 1/2) \st 2x \in M\} = \{x/2\}_{x \in M} \in \mathcal M$ et
$M_2 = \{x \in [1/2, 1) \st 2x-1 \in M\} = \{(1+x)/2\}_{x \in M} \in \mathcal M$ car la transformation affine d'un ensemble $\mathcal L$-mesurable
est $\mathcal L$-mesurable. Donc $f^{-1}(M) = M_1 \sqcup M_2 \in \mathcal M$ car $\mathcal M$ est une $\sigma$-algèbre.

De plus, $\mathcal L(f^{-1}M) = \mathcal L(M_1 \sqcup M_2) = \mathcal L(M_1) + \mathcal L(M_2)$. Par invariance par translation de $\mathcal L$, on a
$\mathcal L(M_2) = \mathcal L(M_1) = \frac 12\mathcal L(M)$. Donc $\mathcal L(f^{-1}M) = 2\mathcal L(M_1) = \mathcal L(M)$.
\end{proof}

\begin{ex}[Vrai ou Faux] Justifier~:
\begin{enumerate}
	\item Soit $f : \R \to \R \st f \circ f$ est mesurable. Alors $f$ est mesurable.
	\item Soit $f : \R \to \R \st \abs f$ est mesurable. Alors $f$ est mesurable.
	\item Soient $f : \R \to \R$ mesurable et $g : \R \to \R$ continue. Alors $g \circ f$ est mesurable.
	\item Si $f : \R \to \R$ est continue presque partout, alors $f$ est mesurable.
\end{enumerate}
\end{ex}

\begin{proof}~
\begin{enumerate}
	\item Faux. Prenons $N \not \in \mathcal M \st N \cap \{0, 1\} = \emptyset$. Alors $\chi_N$ n'est pas mesurable, mais $\chi_N \circ \chi_N = \mathbf 0$
	est constante donc mesurable.
	\item aux. Prenons à nouveau $N \not \in \mathcal M$. On pose $f \coloneqq \chi_N - \chi_{N^\C} : \R \to \{-1, +1\}$ où donc $f(x) = 1$ si $x \in N$ et $f(x) = -1$ sinon.
	$f$ n'est pas mesurable, mais $\abs f = \mathbf 1 : \R \to \{1\} : x \mapsto 1$ est constante donc mesurable.
	\item Mq continuité implique Borel-mesurabilité. Soit $f \in \mathcal C^0(\R, \R)$. Pour $b \in \R$~: $f^{-1}((\minfty, b))$ est ouvert et est donc borélien.
	Donc ici $g$ et $f$ sont toutes deux Borel-mesurables et la composition d'applications Borel-mesurables est Borel-mesurable.
	\item On pose $N \coloneqq \{x \in \R \st f \text{ n'est pas continue en } x\}$. $N$ est négligeable et $f$ est continue sur $\R \setminus N$. On trouve pour $b \in \R$~:
	\[f^{-1}((\minfty, b)) = \left(f^{-1}((\minfty, b)) \cap N^\C\right) \sqcup \underbrace {\left(f^{-1}((\minfty, b)) \cap N\right)}_{\subseteq N \text{ donc } \in \mathcal M}.\]
\end{enumerate}
\end{proof}

\newpage
\section{Séance 5}
\begin{ex}
\begin{enumerate}
	\item Mq la relation $\sim$ définie sur $[0, 1]$ par $x \sim y \iff x-y \in \Q$ est une relation d'équivalence.
	\item On note $\hat x \coloneqq \R / \sim$. On pose $F \coloneqq \bigcup_{x \in \R}\rho(\hat x)$ où $\rho(\hat x)$ est un représentant de $\hat x$ ($\rho$ est bien
	définie par l'axiome du choix). Mq~:
	\[[0, 1] \subseteq \underbrace {\bigcup_{q \in \Q \cap [-1, 1]}(F+q)}_{\eqqcolon \tilde F} \subseteq [-1, 2].\]
	\item Mq si $q_1 \neq q_2 \in \Q \cap [-1, 1]$, alors $(F+q_1) \cap (F+q_2) = \emptyset$.
	\item Mq $F$ n'est pas $\mathcal L$-mesurable par l'absurde.
\end{enumerate}
\end{ex}

\begin{proof}~
\begin{enumerate}
	\item $\sim$ est trivialement une relation d'équivalence.
	\item $\underline {[0, 1] \subseteq \tilde F}$~: soit $x \in [0, 1]$. Par définition de $F$~: $\exists \tilde x \in \hat x \cap F$ et $x-\tilde x \in \Q$.
	On a alors $x \in (\Q \cap [-1, 1]) + \tilde x \subseteq (\Q \cap [-1, 1]) + F = \bigcup_{q \in \Q \cap [-1, 1]}(F+q)$.

	$\underline {\tilde F \subseteq [-1, 2]}$~: $F \subseteq [0, 1]$, et donc~:
	\[\tilde F = \bigcup_{q \in \Q \cap [-1, 1]}(F+q) \subseteq \bigcup_{y \in [-1, 1]}(F+y) \subseteq [-1, 2].\]
	\item Supposons par l'absurde $\exists \tilde x \in (F+q_1) \cap (F+q_2)$. On a alors $\exists y_1, y_2 \in F \st y_1+q_1 = \tilde x = y_2+q_2$. Donc
	$\tilde x - y_1 \in \Q \ni \tilde x - y_2$, ou encore $\tilde x \sim y_1$ et $\tilde x \sim y_2$, et donc par transitivité~: $y_1 \sim y_2$, i.e. $\hat y_1 = \hat y_2$,
	or $F$ contient exactement un seul représentant de chaque classe d'équivalence, ce qui est une contradiction. Donc $F+q_1$ et $F+q_2$ sont disjoints.
	\item Supposons que $F$ soit $\mathcal L$-mesurable. On a (par le point 2)~:
	\[1 \leq \mathcal L\left(\bigsqcup_{q \in \Q \cap [-1, 1]}(F+q)\right) \leq 3.\]

	Or $\forall q \in \Q : \mathcal L(F+q) = \mathcal L(F)$ car $\mathcal L$ est invariante par translations.
	\begin{itemize}
		\item Si $\mathcal L(F) = 0$, alors $\sum_{q \in \Q \cap [-1, 1]}\mathcal L(F+q) = 0 \lneqq 1$, ce qui est une contradiction.
		\item Donc $\mathcal L(F) \gneqq 0$, et donc $\sum_{q \in \Q \cap [-1, 1]}\mathcal L(F+q) = \sum_{q \in \Q \cap [-1, 1]}\mathcal L(F) = \pinfty \gneqq 3$, ce qui est également
		une contradiction.
	\end{itemize}

	On en déduit que $F$ n'est pas $\mathcal L$-mesurable.
\end{enumerate}
\end{proof}

\begin{ex}[Ensemble triadique de Cantor] Pour $k \geq 0$, on pose~:
\[A_k \coloneqq \bigcup_{\alpha \in \{0, 2\}^k}\left(\sum_{i=1}^k\frac {\alpha_i}{3^i}\right) + [0, 3^{-k}].\]
On définit l'\textit{ensemble triadique de Cantor} par $\mathscr C \coloneqq \bigcap_{k \geq 0}A_k$.
\begin{enumerate}
	\item Mq $\forall k \geq 0$~: $A_k$ est formé de $2^k$ intervalles fermés disjoints deux à deux et $\mathcal L(A_k) = (2/3)^k$.
	\item Mq $\mathscr C$ est un borélien non vide et de mesure de Lebesgue nulle.
	\item Mq le développement infini en base 3 est unique ssi les chiffres de la décomposition (notés $a_k : [0, 1] \to \{0, 1, 2\}$ pour le $k$ème chiffre)
	sont soit 0 soit 2. Mq $x \in \mathscr C \iff \forall k \geq 0 : a_k(x) \neq 1$.
	\item En déduire que $\mathscr C$ est en bijection avec $[0, 1]$.
\end{enumerate}
\end{ex}

\begin{proof}~
\begin{enumerate}
	\item $\abs {\{0, 2\}^k} = 2^k$ , donc $A_k$ est bien composé de $2^k$ intervalles fermés. Soient $\alpha \neq \beta \in \{0, 2\}^k$ et mq~:
	\[\left[\left(\sum_{i=1}^k\frac {\alpha_i}{3^i}\right) + [0, 3^{-k}]\right] \cap \left[\left(\sum_{i=1}^k\frac {\beta_i}{3^i}\right) + [0, 3^{-k}]\right] = \emptyset.\]

	Puisque $\alpha \neq \beta$, on sait que $\exists j \in \intint 1k \st \alpha_j \neq \beta_j$. On pose $\gamma_k : \{0, 2\}^k \to [0, 1] :
	\alpha \mapsto \sum_{i=1}^k\frac {\alpha_i}{3^i}$. Par écriture (\textbf{finie~!}) en base 3, on a~:
	\[\gamma(\alpha) = 0.\alpha_1|\alpha_2|\ldots|\alpha_k \qquad \text{ et } \qquad \gamma(\beta) = 0.\beta_1|\beta_2|\ldots|\beta_k\]
	où $|$ désigne la concaténation, ce qui implique que $\abs {\gamma(\alpha) - \gamma(\beta)} \gneqq 3^{-k}$. WLOG, on suppose $\gamma(\alpha) < \gamma(\beta)$.
	On en déduit finalement $\gamma(\alpha) + 3^{-k} \lneqq \gamma(\beta)$, or~:
	\[\forall x \in \gamma(\alpha) + [0, 3^{-k}] : \gamma(\alpha) \leq x \leq \gamma(\alpha) + 3^{-k}.\]
	Et~:
	\[\forall x \in \gamma(\beta) + [0, 3^{-k}] : \gamma(\beta) \leq x \leq \gamma(\beta) + 3^{-k}.\]

	On en déduit que $\left(\gamma(\alpha) + [0, 3^{-k}]\right) \cap \left(\gamma(\beta) + [0, 3^{-k}]\right) = \emptyset$.

	Et finalement, par additivité de $\mathcal L$, on a~:
	\[\mathcal L(A_k) = \sum_{\alpha \in \{0, 2\}^k}\mathcal L(\gamma(\alpha)+[0, 3^{-k}]) = \sum_{\alpha \in \{0, 2\}^k}\mathcal L([0, 3^{-k}]) = 2^k3^{-k} = (2/3)^k.\]

	\item $\forall k \geq 0 : \{0, 1\} \subset A_k$, donc $\mathscr C \neq \emptyset$. De plus, $\forall k \geq 0 : A_k$ est une union de boréliens et est donc borélien.
	$\mathscr C$ est intersection de boréliens et est donc également un borélien.

	De plus la suite $(A_k)_{k \geq 0}$ est décroissante, donc par continuité de la mesure~: $\mathcal L(\mathscr C) = \lim_{k \to \pinfty}\mathcal L(A_k) = 0$.
	Mq $(A_k)_{k \geq 0}$ est décroissante. Fixons $k \geq 0$. Soit $x \in A_{k+1}$, mq $x \in A_k$. On sait que $\exists! \alpha \in \{0, 2\}^{k+1} \st x \in \gamma(\alpha)+[0, 3^{-k}]$.
	On pose $\tilde \alpha \in \{0, 2\}^k \st \forall j \in \intint 1k : \tilde \alpha_j = \alpha_j$. On a donc~:
	\[\gamma_k(\tilde \alpha) \leq \gamma_{k+1}(\alpha).\]
	Mais on a également~:
	\[\gamma_k(\tilde \alpha) + 3^{-k} \geq \gamma_{k+1}(\alpha) + 3^{-k-1}\]
	car~:
	\[(\underbrace {\gamma_k(\tilde \alpha) - \gamma_{k+1}(\alpha)}_{\geq -2 \cdot 3^{-k-1}}) + (\underbrace {3^{-k} - 3^{-k-1}}_{=2\cdot 3^{-k-1}}) \geq 0.\]

	Donc les $A_k$ forment bien une suite décroissante.

	\item On remarque que pour $\alpha \in \{0, 2\}^k : x \in A_k \iff \{a_j(x)\}_{j=1}^k \subseteq \{0, 2\}$. Et puisque $\mathscr C = \bigcap_{k \geq 0}A_k$~:
	\[x \in \mathscr C \iff x \in \bigcap_{k \geq 0}A_k \iff \forall k \geq 0 : \{a_j(x)\}_{j=1}^k \subseteq \{0, 2\} \iff \{a_j(x)\}_{j \geq 0} \subseteq \{0, 2\}.\]

	\item On a alors $x \in \mathscr C \iff x = \sum_{k \geq 1}\frac {a_k(x)}{3^k}$ où les $a_k(x) \neq 1$, i.e. $a_k(x) \in \{0, 2\}$. On peut alors poser~:
	\[\theta : \mathscr \to [0, 1] : x = \sum_{k \geq 1}\frac {a_k(x)}{3^k} \mapsto \sum_{k \geq 1}\frac {a_k(x)}{2^{k+1}},\]
	qui est une bijection entre $\mathscr C$ et $[0, 1]$. On en déduit que $\abs {\mathscr C} = \abs {[0, 1]}$, et donc $\mathscr C$ est non-dénombrable.
\end{enumerate}
\end{proof}

\begin{ex} On définit la bijection $f = \theta^{-1}$ (inverse de la bijection ci-dessus).
\begin{enumerate}
	\item Mq $f$ est strictement croissante et est non-continue.
	\item Soit $E \subset [0, 1]$ un ensemble non-mesurable au sens de Lebesgue. Mq $f(E)$ est $\mathcal L$-mesurable mais non Borélien.
\end{enumerate}
\end{ex}

\begin{proof}~
\textcolor{red}{Attention~: résolution erronée. Il est montré que $[0, 1]$ et $\mathscr C$ ne sont pas homéomorphes, et pas que $f$ n'est pas continue.}
\begin{enumerate}
	\item $f$ est trivialement strictement croissante par construction, et $f$ n'est pas continue car $\mathscr C$ n'est pas connexe donc $\mathscr C$ n'est pas homéomorphe à $[0, 1]$.

	Mq $\mathscr C$ n'est pas connexe~: il faut trouver $U, V$ ouverts pour la topologie induite de $\mathscr C$ par la topologie usuelle de $\R$ tels que $U \cap V = \emptyset$ et
	$U \cup V = \mathscr C$. Pour cela on prend arbitrairement $x \in [0, 1] \setminus \mathscr C$ (par exemple $x = 1/2$) et on définit~:
	\begin{itemize}
		\item $U \coloneqq (-1, x) \cap \mathscr C$~;
		\item $V \coloneqq (x, 2) \cap \mathscr C$.
	\end{itemize}

	$U$ et $V$ sont bien des ouverts pour la topologie induite car $(-1, x)$ et $(x, 2)$ sont des ouverts de $\R$. De plus~:
	\begin{itemize}
		\item $U \cap V = (-1, x) \cap \mathscr C \cap (x, 2) \cap \mathscr C = \emptyset$~;
		\item $U \cup V = \left((-1, x) \cup (x, 2)\right) \cap \mathscr C = (-1, 2) \cap \mathscr C \setminus \{x\} = \mathscr C$.
	\end{itemize}

	{\small \color{white}{Notons qu'il était franchement pas facile celui-là~: il a fallu pas mal réfléchir...}}
	\item Soit $E \subseteq [0, 1]$ quelconque. $f(E) \subseteq \mathscr C$ donc $f(E)$ est $\mathcal L$-négligeable. Or $\mathcal L$ est complète donc $f(E)$ est mesurable.
\end{enumerate}
\end{proof}

\newpage
\section{Séance 6}

\begin{ex} Soient $X \neq \emptyset$, $a \in X$ et $\mathcal A = \mathcal P(X)$. Dans l'espace mesuré $(X, \mathcal A, \delta_a)$, montrons que pour $f : X \to \R^+$ mesurable~:
\[\int_X f\dif\delta_a = f(a).\]
\end{ex}

\begin{proof} Supposons d'abord $f$ positive. Soit $g \in \mathcal S^+(X, \mathcal A)$. Pour $(a_i)_{i=1}^k$ les valeurs prises par $g$, par définition on a~:
\[\int g\dif\delta_a = \sum_{i=1}^ka_i\mu(g^{-1}(\{a_i\})) = a_\ell\delta_a(g^{-1}(\{a_\ell\})) = a_\ell = g(a)\]
où $a_\ell = g(a)$. Dès lors, par définition de l'intégrale d'une fonction positive mesurable~:
\[\int f\dif\delta_a = \sup_{\begin{subarray}{l} g \in \mathcal S^+ \\ g \leq f \end{subarray}}\int g\dif\delta_a.\]

On sait que $f(a)$ majore tous les $g(a)$ pour $g \in \mathcal S^+ \st g \leq f$ donc $\int f\dif\delta_a \leq f(a)$. De plus pour $\tilde g = f(a)\chi_{\{a\}}$, on a
$\tilde g \in \mathcal S^+$ et $\tilde g \leq f$, or $\int \tilde g\dif\delta_a = \tilde g(a) = f(a)$. Donc $\int f\dif\delta_a \geq f(a)$. On a alors bien l'égalité.

Dans le cas général où $f$ n'est pas non-négative, soit $f(a) \geq 0$ et donc $f$ est égale à une fonction positive $\delta_a$-ae, soit $f(a) < 0$ et donc $\int f^+\dif\delta_a = 0$,
donc $\int f\dif\delta_a = -\int f^-\dif\delta_a = -f^-(a) = f(a)$.
\end{proof}

\begin{ex} Sur l'espace mesurable $(\R, \B)$, on définit la mesure de Lebesgue $\mathcal L$ et la mesure $\mu$ suivante~:
\[\mu : \B \to \overline \R^+ : B \mapsto \sum_{k \in B \cap \Z}\frac 1{1 + (k+1)^2}.\]

Déterminer si les fonctions suivantes sont intégrables~:
\begin{align*}
	f(x) &=
	\begin{cases}
		\pinfty &\text{ si } x=0 \\
		\ln\abs x & \text{ si } 0 < \abs x < 1 \\
		0 &\text{ sinon}
	\end{cases} \\
	g(x) &=
	\begin{cases}
		\frac 1{x^2-1} &\text{ si } \abs x < 1 \text{ et } x \in \Q \\
		\frac 1{\sqrt {\abs x}} &\text{ si } \abs x < 1 \text{ et } x \in \Q^\C\\
		\frac 1{x^2} & \text{ si }\abs x \geq 1
	\end{cases} \\
	h(x) &\equiv 1
\end{align*}
pour les mesures $\mathcal L$ et $\mu$ comme défini ci-dessus (pour $f$ et $h$).
\end{ex}

\begin{proof}~
\begin{itemize}
	\item Pour la fonction $f$~:
	\begin{itemize}
		\item[$\mathcal L$] Utilisons le théorème de convergence monotone, i.e. construisons une suite croissante $(f_n)_{n \geq 1}$ d'applications mesurables telle que
		$f_n \xrightarrow[n \to \pinfty]{} \tilde f$ (où $\tilde f = \restr f{(0, 1)} = f\chi_{(0, 1)}$). Pour $n \geq 1$, posons~:
		\[f_n : \R \to \R : x \mapsto -\chi_{(\frac 1n, 1)}(x)\ln x.\]

		$f_n$ est majorée par $x \mapsto \chi_{(1/n, 1)}(x)\sup_{y \in (1/n, 1)}-\ln(y) = \chi_{(1/n, 1)}\ln(n)$ qui est intégrable. On en déduit que $f_n$ est intégrable.\footnote{
		Ainsi, toute fonction bornée définie sur un ensemble de mesure finie est intégrable.} Par l'exercice~\ref{ex:7.1}, une fonction bornée Riemann-intégrable est Lebesgue-intégrable et
		les intégrales coïncident. On a donc~:
		\[\int f_n\dif\mathcal L = \int_{\frac 1n}^1-\ln(x)\dif x = 1 - \frac {\ln n+1}n.\]

		En appliquant le théorème de convergence monotone, on trouve~:
		\[\int f\dif\mathcal L = \lim_{n \to \pinfty}\int f_n\dif\mathcal L = \lim_{n \to \pinfty}\left(1 - \frac {1+\ln n}n\right) = 1.\]

		De manière similaire, on trouve que $\int f\chi_{(-1, 0)}\dif\mathcal L = 1$, et on en déduit~:
		\[\int f\dif\mathcal L = \int f\chi_{(-1, 0)}\dif\mathcal L + \int f\chi_{(0, 1)}\dif\mathcal L = 1+1 = 2.\]
		\item[$\mu$:]       $f = \chi_{(-1, 1) \setminus \{0\}}\ln\abs \cdot + (\pinfty)\chi_{\{0\}}$. Donc $f \equiv 0$ sur $\R \setminus (-1, 1)$. Or $(-1, 1) \cap \Z = \{0\}$.
		Donc $f = \pinfty\chi_{\{0\}}$ $\mu$-ae où $x \mapsto \pinfty\chi_{\{0\}}(x) \in \mathcal S^+$. On trouve donc~:
		\[\int f\dif\mu = \int \pinfty\chi_{\{0\}}\dif\mu = \pinfty\mu(\{0\}) = \pinfty \cdot \frac 1{1+1} = \pinfty.\]

		On peut également remarquer que pour $g \in \mathcal S^+ \st g \lneqq f$~:
		\[\int g\dif\mu = \int g\chi_{\{0\}}\dif\mu = g(0)\frac 12.\]
		Donc $\sup\{\int g\dif\mu \st g \in \mathcal S^+, g \lneqq f\}$ n'est pas borné.
	\end{itemize}
	\item Pour la fonction $g$~: $g = \abs x^{-1/2}\chi_{(-1, +1)} + x^{-2}\chi_{(\minfty, 1] \cup [1, \pinfty)}$ $\mathcal L$-ae et ces deux fonction sont Riemann-intégrables. Donc
	$g$ est $\mathcal L$-intégrable et~:
	\[\int g\dif \mathcal L = 2\left(\int_0^1x^{-1/2}\dif x + \int_1^\pinfty x^{-2}\dif x\right) = 2(2+1) = 6.\]
	\item Pour la fonction $h \in \mathcal S^+(\R, \B)$~:
	\begin{itemize}
		\item[$\mathcal L$:] $h$ n'est pas $\mathcal L$-intégrable car $\int h\dif\mathcal L = 1\mathcal L(\R) = \pinfty$.
		\item[$\mu$:]        $h$ est $\mu$-intégrable car~:
		\begin{align*}
			\int h\dif\mu &= \mu(\R) = \sum_{k \in \Z}\frac 1{1+(k+1)^2} = \sum_{k \in \N^*}\frac 1{1+(k-1)^2} + \sum_{k \in \N}\frac 1{1+(k+1)^2} \\
				&= \frac 12 + \sum_{k \in \N^*}\frac 1{1+k^2} + \sum_{k \in \N^*}\frac 1{1+k^2} = \frac 12 + 2\sum_{k \in \N^*}\frac 1{1+k^2} \leq \frac 12 + 2\sum_{k \in \N^*}k^{-2} \\
				&= \frac 12 + 2\zeta(2) = \frac 12 + \frac {\pi^2}3 < \pinfty.
		\end{align*}
	\end{itemize}
\end{itemize}
\end{proof}

\begin{ex} Soient $(X, \mathcal A, \mu)$ un espace mesuré et $(f_k)_{k \geq 1}$ une suite de fonctions mesurables telles que~:
\[f_1 \geq f_2 \geq f_3 \geq \ldots \geq 0.\]

On définit $f \coloneqq  \lim_{k \to \pinfty}f_k$ la limite point par point. Mq si $f_1 \in L^1(X, \mathcal A, \mu)$, alors~:
\[\lim_{k \to \pinfty}\int_Xf_k\dif\mu = \int_Xf\dif\mu.\]

Donner un contre-exemple avec $f_1 \not \in L^1(X, \mathcal A, \mu)$.
\end{ex}

\begin{proof} Pour $n \geq 2 : f_n \leq f_1$ donc $f_n$ est intégrable car majorée par une fonction intégrable. Par la convergence dominée, on a $f$ intégrable
également et $\int f_k\dif\mu \xrightarrow[k \to \pinfty]{} \int f\dif\mu$.

Pour un contre-exemple, prenons la suite constante $f_k : x \mapsto x^{-1}$. $\forall x \in X : f_k(x) \xrightarrow[k \to \pinfty]{} x^{-1}$ qui n'est pas intégrable.
\end{proof}

\begin{ex} Supposons $\mu(X) \lneqq \pinfty$. Soit $(f_k)_{k \geq 0}$ une suite de fonctions mesurables positives sur $X$ telles que
$f_k \xrightarrow[k \to \pinfty]{CVU \text{ sur }X} f$. Mq si $\forall k \geq 0 : f_k \in L^1(X, \mathcal A, \mu)$, alors~:
\[f \in L^1(X, \mathcal A, \mu) \qquad \text{ et } \qquad \lim_{k \to \pinfty}\int_Xf_k\dif\mu = \int_Xf\dif\mu.\]
\end{ex}

\begin{proof} À $n > 0$ fixé, on a~:
\[f = \abs f = \abs {f_n + f - f_n} \leq \abs {f_n} + \abs {f - f_n}.\]
Dès lors, par monotonie de l'intégrale~:
\[\int f\dif \mu \leq \int f_n\dif \mu + \int \abs {f-f_n}\dif\mu.\]

La convergence uniforme revient à dire~:
\[\forall \varepsilon > 0 : \exists N > 0 \st \forall n > N : \sup_{x \in X}\abs {f - f_n}(x) < \varepsilon.\]
En particulier, il existe $N > 0 \st \forall n > N : \sup_{x \in X}\abs {f - f_n}(x) < 1$. Dès lors pour $n > N$~:
\[\int f\dif\mu \leq \int f_n\dif \mu + \int \abs {f - f_n}\dif \mu \leq \int f_n \dif\mu + \mu(X) < \pinfty\]
car $f_n \in L^1$ et $\mu(X) < \pinfty$.

On en déduit $f \in L^1(X, \mathcal A, \mu)$. De plus~:
\[\abs {\int f_n\dif\mu - \int f\dif\mu} = \abs {\int (f_n-f)\dif\mu} \leq \int \abs {f_n - f}\dif\mu \xrightarrow[n \to \pinfty]{} 0,\]
i.e. $\int f_n\dif\mu \xrightarrow[n \to \pinfty]{} \int f\dif\mu$.
\end{proof}

\begin{ex} On définit pour $k \geq 0$~:
\[\alpha_k \coloneqq \int_0^k\left(1 - \frac xk\right)^k\exp(x/2)\dif x \qquad \text{ et } \qquad \beta_k \coloneqq \int_0^k\left(1 + \frac xk\right)^k\exp(-2x)\dif x.\]
Calculer $\alpha \coloneqq \lim_{k \to \pinfty}\alpha_k$ et $\beta \coloneqq \lim_{k \to \pinfty} \beta_k$.
\end{ex}

\begin{proof} On pose $f_k(x) \coloneqq \left(1-\frac xk\right)^ke^{x/2}\chi_{[0, k]}(x)$ et $g_k(x) = \left(1+\frac xk\right)^ke^{-2x}\chi_{[0, k]}(x)$.
Puisque $f_k(x) \nearrow e^{-x/2}\chi_{\R^+}(x)$ et $g_k(x) \nearrow e^{-x}\chi_{\R^+}(x)$ (deux limites $\in L^1$), par la convergence dominée, on a~:
\[\lim_{k \to \pinfty} \alpha_k = \lim_{k \to \pinfty}\int f_k\dif\mu = \int\lim_{k \to \pinfty}f_k\dif\mu = [2e^{x/2}]_{\minfty}^0 = 2,\]
et~:
\[\lim_{k \to \pinfty}\beta_k = \lim_{k \to \pinfty}\int g_k\dif\mu = \int\lim_{k \to \pinfty}g_k\dif\mu = [-e^{-x}]_0^\pinfty = 1.\]
\end{proof}

\begin{ex} Soit $f \in L^1(X, \mathcal A, \mu)$. Mq~:
\[\forall \varepsilon > 0 : \exists \delta > 0 \st \forall A \in \mathcal A : \mu(A) < \delta \Rightarrow \int_A\abs f\dif\mu < \varepsilon.\]
\end{ex}

\begin{proof} Par l'absurde, supposons qu'il existe $\varepsilon > 0 \st$~:
\[\forall n > 0 : \exists A_n \in \mathcal A \st \mu(A_n) < \frac 1n \text{ et } \int_{A_n}f\dif \mu \geq \varepsilon.\]

Posons $f_n \coloneqq f\chi_{A_n}$. $\forall n > 0 : \abs {f_n} \leq \abs f$ et $\abs {\lim_{n \to \pinfty}f_n} \leq \abs f$ et ces applications sont mesurables. Donc par
la convergence dominée, on a~:
\[\lim_{n \to \pinfty}\int f_n\dif\mu = \int\lim_{n \to \pinfty}f_n\dif \mu.\]

Mq $\lim_{n \to \pinfty}f_n = 0$ $\mu$-ae. Soit $N \coloneqq \{\lim_{n \to \pinfty}f_n \neq 0\}$. Sur $N^\C$, on a $\lim_{n \to \pinfty}f_n = f$. Dès lors~:
\[\exists K > 0 \st \forall x \in N : \forall n > K : x \in A_n,\]
et donc~: $\forall n > K : N \subset A_n$, ce qui implique $\forall n > K : \mu(N) \leq \mu(A_n) < \frac 1K$, i.e. $\mu(A) = 0$.

Finalement, on déduit~:
\[\varepsilon \geq \lim_{n \to \pinfty}\int f_n\dif \mu = \int\lim_{n \to \pinfty}f_n\dif \mu = 0,\]
ce qui est une contradiction.
\end{proof}

\newpage
\section{Séance 7}
\begin{ex}\label{ex:7.1}
Soit $f : [a, b] \to \R$ bornée. Mq~:
\begin{enumerate}
	\item si $f$ est Riemann-intégrable, alors $f$ est Lebesgue-intégrable et~:
	\[\int_a^bf(x)\dif x = \int_{[a, b]}f\dif\mathcal L.\]
	\item les fonctions $h$ et $H$ définies ci-dessous sont bien définies et $h \leq f \leq H$ sur $[a, b]$~:
	\begin{align*}
		h(x) &= \lim_{\delta \to 0}\inf_{\abs {y-x}<\delta}f(y), \\
		H(x) &= \lim_{\delta \to 0}\sup_{\abs {y-x}<\delta}f(y).
	\end{align*}
	\item $f$ est continue en $x$ ssi $H(x) = h(x)$.
	\item $H$ et $h$ sont Lebesgue-mesurables et~:
	\[\int_{[a, b]}H\dif\mathcal L = \inf_PU(f; P) \qquad \text{ et } \qquad \int_{[a, b]}h\dif\mathcal L = \sup_PL(f; P).\]
	\item En déduire que $f$ est Riemann-intégrable ssi l'ensemble des discontinuités de $f$ est négligeable.
\end{enumerate}
\end{ex}

\begin{proof}~
\begin{enumerate}
	\item \underline {\textbf {Notations}~:}
	Pour $P = (x_i)_{i=0}^{N_P}$ une partition de $[a, b]$, on pose pour $i \in \intint 0{N_P-1} : P(i] \coloneqq (x_i, x_{i+1}]$. On pose également les fonctions~:
	\begin{align*}
		f_P &\coloneqq \sum_{j=0}^{N_P - 1}m_j\chi_{P(j]} + f(a)\chi_{\{a\}} \\
		f^P &\coloneqq \sum_{j=0}^{N_P - 1}M_j\chi_{P(j]} + f(a)\chi_{\{a\}}
	\end{align*}

	Pour $P$ une partition de $[a, b]$ et $y \in (a, b) \st \forall i \in \intint 0{N_P}$, on pose $P \oplus y \coloneqq (x'_j)_{j=0}^{N_P+1}$ où
	$\exists j \in \intint 1{N_P} \st x'_j = y$ et $(x'_1, \ldots, x'_{j-1}, x'_{j+1}, \ldots, x'_{N_P+1}) = P$. Pour deux partitions
	$P_1 = (x_{1,j})_{j=0}^{N_{P_1}}$ et $P_2 = (x_{2,j})_{j=0}^{N_{P_2}}$, on pose également $P_1 \oplus P_2$ la plus petite partition $(x_j)_{j=0}^{N_P}$ où~:
	\[\forall j \in \intint 0{N_{P_1}} : \forall \ell \in \intint 0{N_{P_2}} : \exists i, k \in \intint 0{N_P}	\st x_{1,j} = x_i \text{ et } x_{2,\ell} = x_k.\]

	\underline {\textbf {Résolution}~:}

	Observons que pour une partition $P$ et $y \in (a, b)$, on a~:
	\begin{itemize}
		\item $f^{P \oplus y} \leq f^P$~;
		\item $f_{P \oplus y} \geq f_P$.
	\end{itemize}

	En effet, cela découle directement du fait que pour $P \oplus y = (x'_j)_{j=0}^{N_P+1}$ et pour $j \st y = x'_j$~:
	\[\inf_{t \in (x_j, x_{j+1}]}f(t) \geq \inf_{t \in (x_{j-1}, x_{j+1}]}f(t) \quad\text{ et }\quad \inf_{t \in (x_{j-1}, x_j]}f(t) \geq \inf_{t \in (x_{j-1}, x_{j+1}]}f(t).\]
	On déduit similairement pour le $\sup$~:
	\[\sup_{t \in (x_j, x_{j+1}]}f(t) \leq \sup_{t \in (x_{j-1}, x_{j+1}]}f(t) \quad\text{ et }\quad \sup_{t \in (x_{j-1}, x_j]}f(t) \leq \sup_{t \in (x_{j-1}, x_{j+1}]}f(t).\]

	Ainsi, on remarque que pour deux partitions $P_1$ et $P_2$, on a les inégalités suivantes par le même raisonnement~:
	\begin{itemize}
		\item $f^{P_1 \oplus P_2} \leq f^{P_1}$~;
		\item $f_{P_1 \oplus P_2} \geq f_{P_1}$.
	\end{itemize}

	Par définition du $\sup$, il existe une suite de partitions $(P'_n)_{n \geq 0} \st U(f; P'_n) \xrightarrow[n \to \pinfty]{} \inf_PU(f; P)$ et
	une suite de partitions $(\tilde P_n)_{n \geq 0} \st L(f; \tilde P_n) \xrightarrow[n \to \pinfty]{} \sup_PL(f; P)$. Du plus les suites $(U(f; P'_n))_{n \geq 0}$
	et $(L(f; \tilde P_n))_{n \geq 0}$ sont respectivement décroissante et croissante. Par la remarque ci-dessus, on remarque que si on pose $P_n \coloneqq P'_n \oplus \tilde P_n$,
	on obtient une suite de partitions $(P_n)_n$ telle que~:
	\[\lim_{n \to \pinfty}U(f; P_n) = \inf_PU(f; P) = \sup_PL(f; P) = \lim_{n \to \pinfty} L(f; P_n).\]
	En particulier~: $U(f; P_n) - L(f; P_n) \xrightarrow[n \to \pinfty]{} 0$.

	Remarquons ensuite que les $f_{P_n}$ sont des applications mesurables puisque des des combinaisons linéaires de fonctions caractéristiques sur des boréliens (en effet
	les intervalles ouverts à gauches sont des boréliens et le singleton $\{a\}$ en est un également). De plus~:
	\[\int_{[a, b]} f_{P_n}\dif\mathcal L = L(f; P_n) \quad\text{ et }\quad \int_{[a, b]} f^{P_n}\dif\mathcal L = U(f; P_n).\]

	Puisque $U(f; P_n) - L(f; P_n) \xrightarrow[n \to \pinfty]{} 0$, on a $\int_{[a, b]}(f^{P_n}-f_{P_n})\dif \mathcal L \xrightarrow[n \to \pinfty]{} 0$.
	Du coup $\lim_{n \to \pinfty}(f^{P_n} - f_{P_n}) = 0$ $\mathcal L$-ae, i.e. $f_{P_n} \xrightarrow[n \to \pinfty]{\mathcal L\text{-ae}} f$

	Dès lors, par le théorème de la convergence monotone, on a que $f$ est mesurable et que~:
	\[\int_{[a, b]} f\dif\mathcal L = \int_{[a, b]} \lim_{n \to \pinfty}f_{P_n}\dif\mathcal L = \lim_{n \to \pinfty}\int_{[a, b]}f_{P_n}\dif\mathcal L = \lim_{n \to \pinfty}L(f; P_n)
	= \sup_PL(f; P) = \int_a^bf(x)\dif x.\]

	\item Puisque $f$ est bornée, pour tout $x \in [a, b], \delta \gneqq 0$, on a bien~:
	\[\abs {\inf_{y \in B_\delta(x) \cap [a, b]}f(y)} \lneqq \pinfty \quad\text{ et }\quad \abs {\sup_{y \in B_\delta(x) \cap [a, b]}f(y)} \lneqq \pinfty.\]

	De plus, pour tout $x \in [a, b], \delta \gneqq 0$~:
	\[\inf_{y \in B_\delta(x) \cap [a, b]} f(y) \leq f(x) \leq \sup_{y \in B_\delta(x) \cap [a, b]}f(y).\]

	\item \underline {$\Leftarrow$~:} $H(x) = h(x)$ est équivalent à~:
	\[\forall \varepsilon > 0 : \exists \delta_\varepsilon > 0 \st \forall \gamma > 0 : \gamma \leq \delta_\varepsilon \Rightarrow
	\sup_{y \in B_\gamma(x)}f(y) - \inf_{y \in B_\gamma(x)}f(y) \leq \varepsilon.\]

	En particulier, à $\varepsilon > 0$ fixé, pour $y \in [a, b]$, si $\abs {x-y} \leq \delta_\varepsilon$, alors~:
	\[\abs {f(x)-f(y)} \leq \sup_{z \in B_{\delta_\varepsilon}(x)}f(z) - \inf_{z \in B_{\delta_\varepsilon}}f(z) \leq \varepsilon.\]
	On a donc bien la continuité de $f$ en $x$.

	\underline {$\Rightarrow$~:} à $\varepsilon > 0$ fixé, il existe $\delta > 0 \st \abs {x-y} < \delta \Rightarrow \abs {f(x)-f(y)} < \varepsilon$. Donc~:
	\begin{align*}
		\sup_{y \in B_\delta(x)}f(y) - \inf_{y \in B_\delta(x)}f(y) &= \left(\sup_{y \in B_\delta(x)}f(y) - f(x)\right) - \left(\inf_{y \in B_\delta(x)}f(y) - f(x)\right) \\
		&\leq \abs {\sup_{y \in B_\delta(x)}f(y)-f(x)} + \abs {f(x)-\inf_{y \in B_\delta(x)}f(y)} \leq 2\varepsilon.
	\end{align*}

	Or cette inégalité est vraie pour tout $\varepsilon > 0$. On en déduit que $H(x)-h(x) = 0$.

	\item Pour $\beta \in \R$, remarquons~:
	\[\overbracket {f^{-1}((\minfty, \beta])}^{\circ} = \{\xi \in [a, b] \st \exists r_\xi > 0 \st B_{r_\xi}(\xi) \subseteq f^{-1}((\minfty, \beta])\}.\]

	Dès lors~:
	\begin{align*}
	     & x \in H^{-1}((\minfty, \beta]) \\
	\iff & H(x) \leq \beta  \\
	\iff & \lim_{\delta \to 0}\sup_{y \in B_\delta(x)}f(y) \leq \beta \\
	\iff & \exists \delta > 0 \st \forall y \in B_\delta(x) : f(y) \leq \beta \\
	\iff & \exists \delta > 0 \st B_\delta(x) \subseteq f^{-1}((\minfty, \beta]) \\
	\iff & x \in \overbracket {f^{-1}((\minfty, \beta])}^{\circ},
	\end{align*}
	i.e. $H$ est mesurable car l'intérieur d'un ensemble est un ouvert.

	De manière similaire, on a $x \in h^{-1}([\alpha, \pinfty)) \iff x \in \overbracket {f^{-1}([\alpha, \pinfty)}^{\circ}$. Donc $h$ est également mesurable.

	Remarquons ensuite que pour toute partition $P$ : $H \leq f^P$ sur $[a, b] \setminus P$ et donc $H \leq f^P$ $\mathcal L$-ae. Par monotonie de l'intégrale~:
	\[\int H\dif\mathcal L \leq \int f^P\dif\mathcal L.\]
	En particulier~:
	\[\int H\dif \mathcal L \leq \inf_P\int f^P\dif\mathcal L.\]

	Reconsidérant maintenant la suite $(P_n)_{n \geq 0} \st U(f; P_n) \xrightarrow[n \to \pinfty]{} \inf_PU(f; P)$ et $L(f; P_n) \xrightarrow[n \to \pinfty]{} \sup_PL(f; P)$
	et supposons la croissante (ce qui n'est pas abusif~: il suffit de poser la suite de partitions $\hat P_0 \coloneqq P_0$ et $\hat P_n \coloneqq \hat P_{n-1} \oplus P_n$
	qui est croissante au sens de l'inclusion et qui satisfait les limites puisque $f^{\hat P_n} \leq f^{P_n}$ et $f_{\hat P_n} \geq f_{P_n}$).
	Pour toute partition $P$ et pour tout $x \in [a, b] \setminus P$, on note $j_{P,x} \in \intint 0{N_P-1} \st x \in P(j_{P,x}]$. Fixons alors $x \in [a, b] \setminus
	\bigcup_{n \geq 0}P_n$ (i.e. $\forall n \geq 0 : x \not \in P_n$).

	Séparons deux cas~:
	\begin{itemize}
		\item[{si $\mathcal L(P_n(j_{P_n,x}]) \xrightarrow[n \to \pinfty]{} 0$}~:] $\forall \delta > 0 : \exists N > 0 \st P_N(j_{P_N,x}] \subset B_\delta(x)$.
		Posons une suite $(\delta_n)_{n \geq 0}$ décroissante vers 0 et posons $H_n(x) \coloneqq \sup_{y \in B_{\delta_n}(x)}f(y)$.

		On remarque que $\forall n \geq 0 : \exists K > 0 \st \forall k \geq K : H_n(x) \geq f^{P_k}(x) \geq H(x)$. Or $H_n(x) \xrightarrow[n \to \pinfty]{} H(x)$, et donc
		$f^{P_n}(x) \xrightarrow[n \to \pinfty]{} H(x)$.

		\item[sinon~:] $\forall (\alpha, \beta] \subset \bigcap_{n \geq 0}P_n(j_{P_n, x}] : \exists K > 0 \st \forall k \geq K :$
		\[\sup_{y \in (\alpha, \beta]}f(y) = \sup_{y \in P_k(j_{P_k, x}]}f(y).\]

		En effet, s'il existe $(\alpha, \beta]$ qui ne satisfait pas la propriété, alors les $P_n$ pourraient être raffinés par la suite $(P'_n)_{n \geq 0}$ définie par
		$P'_n \coloneqq P_n \oplus \alpha \oplus \beta$. Dans ce cas~: $f^{P_n} \geq f^{P'_n}$ ce qui contredit~:
		\[U(f; P_n) \xrightarrow[n \to \pinfty]{} \inf_PU'f; P).\]

		On déduit donc~:
		\[\lim_{\delta \to 0}\sup_{y \in B_\delta(x)}f(y) = \sup_{y \in \bigcap_{n \geq 0}P_n(j_{P_n,x}]}f(y),\]
		i.e.~:
		\[H(x) = \lim_{\delta \to 0}\sup_{y \in B_\delta(x)}f(y) = \sup_{y \in \bigcap_{n \geq 0}P_n(j_{P_n,x}]}f(y) = \lim_{n \to \pinfty}f^{P_n}(x).\]
	\end{itemize}
	On déduit de ces deux cas $f^{P_n} \xrightarrow[n \to \pinfty]{\mathcal L\text{-ae}} H$ et donc~:
	\[\int H\dif\mathcal L = \lim_{n \to \pinfty}\int f^{P_n}\dif\mathcal L = \inf_PU(f; P).\]

	On déduit similairement le cas $\int h\dif\mathcal L = \sup_PL(f; P)$.

	\item $f$ est Riemann-intégrable $\iff \int H\dif\mathcal L = \int h\dif\mathcal L \iff \int (H-h)\dif\mathcal L = 0 \iff H = h$ $\mathcal L$-ae
	et donc $\iff \mathcal L(\{x \in [a, b] \st H(x) \neq h(x)\}) = 0$, i.e. $f$ est continue $\mathcal L$-ae.
\end{enumerate}
\end{proof}

\begin{ex}~
\begin{enumerate}
	\item Mq si $f : [a, \pinfty) \to \R$ est Riemann-intégrable et bornée sur $[a, b]$ pour tous $b > a$, alors~:
	\[\int_{[a, \pinfty)}f\dif\mathcal L = \int_a^\pinfty f(x)\dif x.\]
	\item Mq si $f : [a, b] \to \R$ est Riemann-intégrable et bornée sur $[c, b]$ pour tous $c \in (a, b)$, alors~:
	\[\int_{[a, b]}f\dif\mathcal L = \int_a^bf(x)\dif x.\]
\end{enumerate}
\end{ex}

\begin{proof}~
\begin{enumerate}
	\item Posons la suite croissante de fonctions $f_n \coloneqq f\chi_{[a, a+n]}$. Puisque les $f_n$ sont bornées et Riemann-intégrables (par hypothèse), par l'exercice précédent,
	on sait que les $f_n$ sont mesurables et que $\int_{[a, a+n]}f\dif\mathcal L = \int_a^{a+n}f(x)\dif x$. Puisque $f_n \xrightarrow[n \to \pinfty]{} f$, on
	a que $f$ est mesurable. De plus, par le théorème de la convergence monotone~:
	\[\int f\dif\mathcal L = \lim_{n \to \pinfty}\int f_n\dif\mathcal L = \lim_{n \to \pinfty}\int_a^{a+n}f(x)\dif x = \int_a^\pinfty f(x)\dif x.\]

	\item De manière similaire, on pose la suite croissante de fonctions $f_n \coloneqq f\chi_{[a+\frac 1n, b]}$ qui converge vers $f$ sur $(a, b]$ (donc $\mathcal L$-ae).
	Puisque les $f_n$ sont mesurables, on déduit à nouveau que $f$ est mesurable également et par la convergence monotone~:
	\[\int f\dif\mathcal L = \lim_{n \to \pinfty}\int f_n\dif\mathcal L = \lim_{n \to \pinfty}\int_{a+\frac 1n}^bf(x)\dif x = \int_a^bf(x)\dif x.\]
\end{enumerate}
\end{proof}

\begin{ex} $\Q$ est dénombrable donc $\Q \cap [0, 1]$ l'est aussi. Donc $\Q \cap [0, 1] = \{q_k\}_{k \geq 0}$. Pour $k \geq 0$, on définit~:
\[f_k : [0, 1] \to \R : x \mapsto \begin{cases}1 &\text{ si } x \in \{q_\ell\}_{\ell=0}^k \\0 &\text{ sinon}.\end{cases}\]

\begin{enumerate}
	\item Mq $\forall k \geq 0 : f_k$ est Riemann-intégrable et déterminer~:
	\[\int_0^1f_k(x)\dif x.\]
	\item Mq $f_k \xrightarrow[k \to \pinfty]{CVS \text{ sur } [0, 1]} f$. Que peut-on en déduire~?
	\item Mq $f$ est Lebesgue-intégrable et vérifier les hypothèses du théorème de la convergence dominée.
\end{enumerate}
\end{ex}

\begin{proof}~
\begin{enumerate}
	\item À $k$ fixé, $f_k$ est bornée sur son compact de définition et est continue sur $[0, 1] \setminus \{q_\ell\}_{\ell=0}^k$, donc $f_k$ est Riemann-intégrable. De plus~:
	\[\int_0^1f_k(x)\dif x = \sum_{\ell=1}^k\int_{q_{\ell-1}}^{q_\ell}f_k(x)\dif x = \sum_{\ell=1}^k0 = 0.\]

	De manière plus élégante, remarquons que $f_k$ est la fonction caractéristique d'un ensemble de mesure nulle, donc $f_k$ est mesurable et de plus~:
	\[\int f_k\dif\mathcal L = \mathcal L(\{q_\ell\}_{\ell=0}^k) = 0,\]
	donc $f_k$ est Lebesgue-intégrable. De plus $f_k$ est bornée sur son compact de définition (qui est forcément de mesure finie), donc $f_k$ doit être Riemann-intégrable.

	\item Soit $x \in [0, 1]$. Si $x \in \Q^\C$, alors $\forall k \geq 0 : f_k(x) = 0$. Si $x \in \Q$, alors $\exists K \geq 0 \st x = q_K$, ce qui implique
	$\forall k \geq K : f_k(x) = 1$. Donc $f_k \xrightarrow[k \to \pinfty]{} \chi_{\Q \cap [0, 1]}$. Notons que $f_k$ converge simplement mais pas uniformément. En effet~:
	\[\forall k \geq 0 : \sup_{x \in [0, 1]}\abs {f_k(x)-\chi_\Q(x)} = 1.\]

	Puisque $\chi_{\Q \cap [0, 1]}$ n'est pas Riemann-intégrable, on a l'existence d'une fonction $\mathcal L$-intégrable mais pas R-intégrable.

	\item Voir la remarque ci-dessus pour l'intégrabilité au sens de Lebesgue. Les $f_k$ sont bien mesurables et $f = \chi_{\Q \cap [0, 1]} \leq \chi_{[0, 1]}$ qui est intégrable.
	Donc par la convergence dominée, on a bien $\int f\dif\mathcal L = 0$.
\end{enumerate}
\end{proof}
\end{document}
