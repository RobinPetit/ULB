\documentclass{article}

\usepackage[T1]{fontenc}
\usepackage[utf8]{inputenc}
\usepackage{fullpage}
\usepackage[parfill]{parskip}
% math packages
\usepackage{amsmath, amsthm, amssymb}
\usepackage{mathtools}
\usepackage{mathdots}
\usepackage{commath}

\usepackage{hyperref}

\makeatletter
\def\thm@space@setup{%
	\thm@preskip=.4cm%
	\thm@postskip=\thm@preskip%
}
\makeatother

\newcommand{\N}{\mathbb N}
\renewcommand{\P}{\mathbb P}

% amsthm
\newtheorem{thm}{Théorème}[section]
\newtheorem{prp}[thm]{Proposition}
\renewcommand{\proofname}{\it{Démonstration}}
\theoremstyle{definition}
\newtheorem{déf}[thm]{Définition}
\theoremstyle{remark}
\newtheorem*{rmq}{Remarque}

\author{R. Petit}
\title{MATHF-105 : Probabilités \\ Résumé}
\date{Année académique 2015 - 2016}

\begin{document}

\pagenumbering{Roman}
\maketitle
\tableofcontents
\newpage
\pagenumbering{arabic}

\section{Espaces de probabilités}
	\subsection{Définition}
		\begin{déf} L'ensemble $\Omega$ est l'\textbf{espace des chances}, l'ensemble des résultats possibles d'un phénomène aléatoire. \end{déf}

		\begin{rmq}~
		\begin{itemize}
			\item $\Omega$ peut être fini (dénombrable) ou infini~;
			\item $\Omega = \left\{0, 1\right\}^\N$ est l'ensemble des suites à valeur dans $\{0, 1\}$~;
			\item $\Omega$ peut être un espace dit \textit{fonctionnel} quand le résultat d'une expérience est une fonction.
		\end{itemize}
		\end{rmq}

		\begin{déf} Un événement $E$ est un ensemble de réalisations possibles à une expérience tel que $E \subseteq \Omega$. \end{déf}

		\begin{rmq} L'ensemble $\mathcal P(\Omega)$ n'est pas toujours dénombrable. Et donc l'ensemble $\mathcal P(\Omega)$ est-il le bon ensemble pour décrire
		les événements~?
		
		\begin{itemize}
			\item Si $\abs \Omega \in \N$~: oui~;
			\item si $\abs \Omega \not \in \N$~: non.
		\end{itemize}
		\end{rmq}

		\begin{déf} $\mathcal F$ est la \textbf{classe des événements}. On mesure la \textit{probabilité d'occurrence} d'un événement $A \in \mathcal F$.
		On introduit une fonction d'ensemble $\P$ où~:
		\[\P : \mathcal F \to [0, 1] : A \mapsto \P(A).\]

		On impose~:
		\begin{itemize}
			\item[$(i)$] $\P(\emptyset) = 0$~;
			\item[$(ii)$] $\P(\Omega) = 1$~;
			\item[$(iii)$] $\forall A, B \in \mathcal F : A \cap B = \emptyset \Rightarrow \P(A \cup B) = \P(A) + \P(B)$.
		\end{itemize}
		\end{déf}

\end{document}
