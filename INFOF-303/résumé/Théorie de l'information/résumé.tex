\documentclass{article}

\usepackage{commath}
\usepackage[french]{babel}
\usepackage{palatino, eulervm}
\usepackage[utf8]{inputenc}
\usepackage[T1]{fontenc}
\usepackage{fullpage}
\usepackage{mathtools}
\usepackage{amsmath, amsthm, amsfonts, amssymb}
\usepackage[parfill]{parskip}
\usepackage[bottom]{footmisc}

\title{INFOF-303 : Réseaux, information et communication}
\author{R. Petit}
\date{année académique 2016 - 2017}

\DeclareMathOperator{\tq}{\text{ t.q. }}

\newcommand{\N}{\mathbb N}

% amsthm
\newtheorem{thm}{Théorème}[section]
\theoremstyle{definition}
\newtheorem{déf}[thm]{Définition}
\theoremstyle{remark}
\newtheorem*{rmq}{Remarque}

\begin{document}
\pagenumbering{Roman}
\maketitle
\tableofcontents
\newpage
\setcounter{page}{1}
\pagenumbering{arabic}

\section{Introduction et inégalités de Kraft/Mc Millan}
	\subsection{Définitions}
		\begin{déf} Soit $\Sigma$ un ensemble de cardinalité finie. Si on appelle les éléments $\sigma \in \Sigma$ des \textit{symboles},
		on dit que $\Sigma$ est un \textit{alphabet}. \end{déf}

		\begin{déf} Soit $\Sigma$ un alphabet. On pose $\ell \in \N$, un naturel. Toute séquence de $\ell$ symboles de $\Sigma$ concaténés est appelée
		\textit{mot} de l'alphabet $\Sigma$. Si $m$ est un mot, on peut écrire $m \in \Sigma^\ell$. \end{déf}

		\begin{déf} Soit $\Sigma$ un alphabet. On définit l'ensemble~:
		\[\Sigma^{*} \coloneqq \bigcup_{\ell \in \N}\Sigma^\ell.\]
		\end{déf}

		\begin{rmq} L'ensemble $\Sigma^{*}$ contient donc tous les mots de cardinalité naturelle qui peuvent être faits à l'aide de l'alphabet $\Sigma$.
		On peut également noter que l'ensemble $\Sigma^*$ est toujours de cardinalité infinie alors que, par définition, l'alphabet $\Sigma$ est de
		cardinalité finie. \end{rmq}

		\begin{déf} Soit $\Sigma$ un alphabet et $m \in \Sigma^*$ un mot sur $\Sigma$. On définit la fonction~:
		\[\ell_\Sigma : \Sigma^* \to \N : m \mapsto n \in \N \tq m \in \Sigma^n \subsetneqq \Sigma^*.\]
		On appelle cette fonction la fonction \textit{longueur} des mots sur $\Sigma$. \end{déf}

		\begin{rmq} Lorsque l'alphabet du mot n'est pas ambigu, on note simplement cette fonction $\ell$. \end{rmq}

		\begin{déf} Soient deux alphabets $S$ et $C$. Soit $K : S \to C^* : s \mapsto (c_i)_{i \leq n}$. On dit que la fonction $K$ est une fonction de codage
		si $K$ est injective. Dans ce contexte, le terme \textit{univoque} est préféré à \textit{injectif}. \end{déf}

		\begin{rmq}~
		\begin{itemize}
			\item Par une fonction de codage, chaque symbole de l'alphabet de départ $S$ est codé par une \textit{suite} de symboles de l'alphabet d'arrivée $C$~;
			\item l'ensemble $K(S) \neq C^*$ car $K$ est injective et donc $\abs {K(S)}  = \abs S$. Or $\abs S \in \N$ et $\abs {C^*} = +\infty$. La fonction
			      $K$ ne peut donc pas être bijective~;
			\item il est usuel de noter les cardinaux des ensembles $S$ et $C$ respectivement par $q$ et $r$.
		\end{itemize}
		\end{rmq}

		\begin{déf} On étend la fonction de codage $K : S \to C^*$ codant un unique symbole de $S$ en la fonction~:
		\[K : S^k \to C^* : (s_i)_{1 \leq i \leq k} \mapsto \left(K(s_i)\right)_{1 \leq i \leq k} =
			\left(\left(c_{ij}\right)_{1 \leq j \leq n}\right)_{1 \leq i \leq k} = \left(c_{ij}\right)_{\underset{1 \leq i \leq k}{1 \leq j \leq n}}.\]
		\end{déf}

		\begin{rmq} En théorie de l'information, la fonction $K$ de codage est totalement déterministe afin de permettre le décodage. \end{rmq}
	
	\subsection{Familles de codes}
		\subsubsection{Codes blocs}
			\begin{déf} Soient $S$ et $C$ deux alphabets et $K : S \to C^*$ une fonction de codage. S'il existe $n \in \N$ tel que $K(S) \subseteq C^n$, on
			dit que $K$ est une fonction de \textit{code bloc}. \end{déf}

			\begin{rmq} Une fonction de code bloc code donc tous les symboles de $S$ par une suite d'un nombre fixé de symboles de $C$. \end{rmq}

			\begin{rmq} La majorité des codes correcteurs d'erreurs (CCE) sont des codes blocs. En effet, si un canal de transmission est bruyant, il faut
			connaitre au préalable la longueur des blocs à lire afin de les décoder et de les corriger. \end{rmq}

		\subsubsection{Codes préfixes}
			\begin{déf} Soient $S = \{s_1, s_2, \ldots, s_n\}$ et $C$ deux alphabets et $K : S \to C^*$ une fonction de codage. On dit que $K$ est une
			fonction de \textit{code préfixe} si~:
			\[\forall 1 \leq i \leq n : \not \exists 1 \leq j \leq n \tq \left(i \neq j\right) \land
				\left(\forall 1 \leq k \leq \min\left\{\abs {K(s_i)}, \abs{K(s_j)}\right\} : \left(K(s_i)\right)_k = \left(K(s_j)\right)_k\right).\]
			\end{déf}

			\begin{rmq}~
			\begin{itemize}
				\item Autrement dit, une fonction de code est dite \textit{préfixe} lorsqu'aucun mot du code n'est préfixe d'un autre mot du code~;
				\item les codes préfixes présentent l'avantage d'être déchiffrables à la volée à l'aide d'un automate fini
				      (ou d'un arbre de décision $n$-aire où $n = \abs C$\footnote{Les codes blocs peuvent également être représentés par un arbre de
				      décision $n$-aire. On peut alors dire qu'un code est un code bloc si et seulement si l'arbre de décision associé est complet.})~;
				\item les codes blocs sont un cas particulier de code préfixe~: en effet, aucun mot du code n'est préfixe d'un autre étant donné qu'ils
				      ont tous la même longueur et que la fonction de code est injective.
			\end{itemize}
			\end{rmq}

	\subsection{Théorèmes de Kraft et Mc Millan}
		\begin{thm}[Inégalité de Kraft] Soient $S = \{s_1, \ldots, s_q\}$ et $C = \{c_1, \ldots, c_r\}$. On pose $\ell_i \coloneqq \ell(K(s_i))$.
		Alors, il existe un code préfixe $K : S \to C^*$ si et seulement si~:
		\[\sum_{i=1}^qr^{-\ell_i} \leq 1.\]
		\end{thm}

		\begin{proof} Réorganisons les $s_i$ de manière à ce que $\forall 1 \leq i \lneqq q : \ell_i \leq \ell_{i+1}$.

		Montrons d'abord que s'il existe un code préfixe, alors l'inégalité est vérifiée.

		Soit $\mathcal A$, l'arbre $r$-aire complet de hauteur $l_q$. Pour $1 \leq i \leq q$, on pose $a_i \coloneqq K(s_i)$. Notons $\mathcal A_i$ le
		sous-arbre $r$-aire ayant $a_i$ pour racine. Par définition de code préfixe, la famille $\{\mathcal A_i\}_{1 \leq i \leq q}$ est distincte deux à deux.

		On observe aisément que $\mathcal A_i$ est un arbre $r$-aire de hauteur $\ell_q - \ell_i$. On a donc~:
		\[\abs {\mathcal A_i} = r^{\ell_q-\ell_i}.\]

		On peut donc écrire~:
		\[r^{\ell_q} = \abs {\mathcal A} \geq \abs {\bigcup_{k=0}^q\mathcal A_k} = \sum_{k=0}^q \abs {\mathcal A_k} = \sum_{k=0}^qr^{\ell_q-\ell_k}.\]
		En divisant de part et d'autre par $r^{\ell_q}$, on obtient~:
		\[1 \geq \sum_{k=0}^qr^{-\ell_k}.\]

		Supposons maintenant que l'inégalité est vérifiée et montrons qu'il existe un code préfixe.

		Si $(\ell_i)_{1 \leq i \leq q}$ est un vecteur de naturels satisfaisant l'inégalité de Kraft et tels que~:
		\[\forall 1 \leq i \lneqq q : \ell_i < \ell_{i+1},\]
		alors on construit $\mathcal A$ un arbre $r$-aire de hauteur $\ell_q$. Pour tout $i \leq q$, on élague l'arbre $\mathcal A_i$ en supprimant un nœud
		de hauteur $\ell_i$. Cela supprime à l'itération $i$, $r^{\ell_q-\ell_i}$ nœuds de l'arbre. Donc à la $q$ème itération, sont supprimés au total~:
		\[\sum_{k=0}^qr^{\ell_q-\ell_k} = r^{\ell_q}\sum_{k=0}^qr^{-\ell_k} \leq r^{\ell_q}\]
		nœuds de l'arbre. Il est donc possible de placer les $q$ mots afin de former un code préfixe dans l'arbre car si ce n'était pas possible, l'arbre
		devrait contenir strictement moins de nœuds que $r^{\ell_q}$, ce qui n'est pas le cas.
		\end{proof}

		\begin{thm}[Théorème de Mc Millan] Tout code univoque satisfait l'inégalité de Kraft. \end{thm}

		\begin{proof} Soient $i_1, \ldots, i_n \in \N$, $C$ et $S$ deux alphabets. Soit $K : S \to C^*$ une fonction de code. On pose~:
		\[j \coloneqq \abs {K(s_{i_1}s_{i_2}\ldots s_{i_n})}.\footnote{Ici, $j$ est une fonction de $n$ paramètres $i_1$ jusque $i_n$,
			mais on peut considérer la valeur constante car les valeurs $i_k$ sont fixées.}\]
		Si $x_j$ est le nombre de mots de longueur $j$, on sait que $x_j \leq r^j$. Par définition de $j$, on peut écrire~:
		\[j = \sum_{k=1}^n\ell_{i_k}.\]

		On pose~:
		\[\alpha \coloneqq \sum_{k=1}^qr^{-\ell_k}.\footnote{Idem.}\]
		Dès lors, on a~:
		\[\alpha^n = \left(\sum_{k=1}^qr^{-\ell_k}\right)^n = \sum_{\gamma_1, \ldots, \gamma_n = 1}^nr^{-\sum_{k=1}^n\ell_{\gamma_k}}.\]

		On pose~:
		\[\mu \coloneqq \max_{\gamma_1, \ldots, \gamma_n}\left\{\sum_{i=1}^n\ell_{\gamma_i}\right\}.\]
		On peut alors exprimer
		\[\alpha^n \leq \sum_{j=1}^\mu x_jr^{-j} \leq \sum_{j=1}^\mu r^{-1}r^j = \sum_{j=1}^\mu 1 = \mu.\]

		Or, $\mu = n \cdot \max_i\{\ell_i\}$. Notons $L \coloneqq \max_i \{\ell_i\}$. On a alors~:
		\[\alpha^n \leq nL.\]

		En divisant par $n$ de part et d'autre, on obtient~:
		\[\frac {\alpha^n}n \leq L,\]
		où $L$ est une constante naturelle. La suite $\left(\frac {\alpha^n}n\right)_n$ est donc bornée par $L$. On peut alors déduire que la limite de cette suite
		existe également.\footnote{Il faut pour cela que la suite $\left(\abs {\frac {\alpha^n}n}\right)_n$ soit bornée, mais la suite
		$\left(\frac {\alpha^n}n\right)_n$ est définie positive, donc la suite valeur absolue est la même, et est donc bornée.}

		Dès lors, $\abs \alpha = \alpha \leq 1$.
		\end{proof}
\end{document}
