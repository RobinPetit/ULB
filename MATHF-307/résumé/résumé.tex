\documentclass{article}

\usepackage{times}  % times new roman police for document
\usepackage[cm]{fullpage}  % Use full page and remove big margins
\usepackage[T1]{fontenc}
\usepackage{amsmath}  % math package
\usepackage{amsfonts}  % same
\usepackage{commath}  % same
\usepackage{mathtools}  % sale
\usepackage[bottom]{footmisc}  % force footnotes to be on bottom of pages
\usepackage[utf8]{inputenc}  % UTF-8 file
\usepackage[parfill]{parskip}  % adjust lists
\usepackage{titlesec}  % reduce space between paragraphs

\newenvironment{lst}
	{\begin{minipage}[t]{.9\linewidth}\begin{itemize}}
	{\end{itemize}\end{minipage}}

\title{Mathématique discrètes}
\date{2015-16, premier quadrimèstre}
\author{Robin P.}

\titlespacing\paragraph{0pt}{1pt plus 1pt minus 1pt}{4pt plus 1pt minus 1pt}
\titlespacing\subparagraph{0pt}{1pt plus 0pt minus 0pt}{1pt plus 1pt minus 1pt}

\DeclareMathOperator{\tq}{\text{ tq }}
\DeclareMathOperator{\Z}{\mathbb Z}
\DeclareMathOperator{\Imappl}{Im}
\DeclareMathOperator{\Ker}{Ker}
\DeclareMathOperator{\modulo}{mod}
\newcommand{\cmod}[1]{\, (\modulo #1)}
\newcommand{\floor}[1]{{\left\lfloor#1\right\rfloor}}
\newcommand{\ceil}[1]{{\left\lceil#1\right\rceil}}

\begin{document}
	\pagenumbering{Roman}
	\maketitle
	\tableofcontents
	\pagebreak
	\clearpage
	\setcounter{page}{1}
	\pagenumbering{arabic}

	\section{Théorie des graphes}

		\subsection{Définitions}

			\paragraph{Intro} Le problème des sept ponts de Königsberg est une des origines de la théorie des graphes : Au XVIIIe siècle, Leonhard
			Euler fut réquisitionné pour résoudre le problème suivant. \textit{La ville de Königsberg est composée de sept ponts reliant quatre zones
			habitables d'une manière précise. Est-il possible de passer par tous les ponts et de revenir au point de départ en n'empruntant qu'une et
			une seule fois chaque pont ?} Euler a donc simplifié le problème en schématisant les zones habitables de manière ponctuelle et les ponts
			par des segments. C'est ainsi qu'il a trouvé que c'était impossible.

			\paragraph{Def}
				\begin{lst}
					\item Une graphe $\Gamma$ est un triplet $(V, E, \gamma)$ où $V$ est un ensemble fini dont les éléments sont appelés \textit{
						  sommets} du graphe, $E$ est un ensemble fini dont les éléments sont appelés \textit{arêtes} du graphe ; et $\gamma$ est
						  une fonction $\gamma : E \to \mathcal Paire(V)$ où $\mathcal Paire(V)$ est un ensemble non-ordonné. On note souvent
						  $\Gamma = (V, E)$ en omettant volontairement l'expression de $\gamma$.

					\item Soient $\gamma(e) = \{x, y\}, e \in E, x, y \in V$. On dit que $x$ et $y$ sont \textit{adjacents} et que $e$ est
					\textit{incident} à $x$ et $y$.
				\end{lst}

			\paragraph{Def}
				\begin{lst}
					\item Soit $\Gamma = (V, E)$, un graphe. $\gamma(e) = \{x, x\}, e \in E, x\in V$ est appelé un lacet.
					\item Si au moins deux arête sont incidentes à $x, y \in V$, on les appelle \textit{arêtes multiples}.
					\item Un graphe est simple s'il n'a ni lacet ni arête multiple. Dans ce cas, on omet $\gamma$ et on note $\Gamma = (V, E)$
						  avec $E \subseteq \mathcal Paire(V)$.
				\end{lst}

			\paragraph{Def} Soit $\Gamma = (V, E)$, un graphe. Le degré d'un sommet $v \in V$ est le nombre d'arêtes incidentes à $v$ tel que
			les lacets comptent pour deux arêtes. Ce degré se note $\deg(v) \in \mathbb N \; \forall v \in V$.

			\paragraph{Exemple} Dans le cas du graphe d'une molécule de chimie organique comme un alcane par exemple, on a :
			\[\deg(C) = 4, \deg(H) = 1.\]

			\paragraph{Théorème} Soit $\Gamma = (V, E)$, un graphe. Alors $\sum_{v \in V}\deg(v) = 2\#E$.

			\paragraph{Corollaire} La somme des degrés d'un graphe est pair.

			\paragraph{Def} Le graphe complet $\mathcal K_n$ est le graphe simple à $n$ sommets pour lequel chaque sommet est adjacent à tous
			les autres. Autrement dit, $\forall x, y \in V, x \neq y, \exists e \in E \tq \gamma(e) = \{x, y\}$.

			\paragraph{Def} Un graphe $\Gamma' = (U, F)$ est un sous-graphe de $\Gamma = (V, E)$ si $U \subseteq V$ \textbf{et} $F \subseteq E$.
			On note alors $\Gamma' \leq \Gamma$. (Et pas $\Gamma' \subseteq \Gamma$ !!)

		\subsection{Chemins dans les graphes}
			\paragraph{Def}
				\begin{lst}
					\item Soit $\Gamma = (V, E), v, w \in V$. Un chemin de $v$ à $w$ de longueur $n$ est une séquence alternée de $(n+1)$ sommets
						  $(v_0, \ldots, v_n)$ et $n$ arêtes $(e_1, \ldots, e_n)$ de la forme $(v = v_0, e_1, v_1, \ldots, e_{n-1}, v_{n-1}, e_n, v_n = w)$
						  tel que $e_i$ est incident à $v_{i-1}$ et $v_i \; \forall i \in \{1, \ldots, n\}$ et $e_i \neq e_j \; \forall i \neq j$.

					\item Un chemin ne possédant aucune répétition de sommet sauf peut-être $v_0 = v_n$ est dit \textit{simple}.
				\end{lst}

			\paragraph{Remarque} Dans les graphes simples, on ne note que les sommets et pas les arêtes.

			\paragraph{Def}
				\begin{lst}
					\item Un graphe $\Gamma = (V, E)$ est connexe si $\forall x, y \in V$, il existe un chemin de $x$ à $y$.
					\item Soit $\Gamma = (V, E)$, un graphe et $x \in V$, la \textit{composante connexe} de $\Gamma$ contenant $x$ est
						  $\Gamma' \leq \Gamma$ dont les sommets et les arêtes sont ceux contenus dans un chemin de $\Gamma$ démarrant en $x$.
				\end{lst}

			\paragraph{Def} Soit $\Gamma = (V, E)$ et $v \in V$. Un cycle est un chemin de $v$ à $v$. Un cycle simple est un cycle de $v$ à $v$ dans
			lequel seul $v$ est répété.

		\subsection{Arbres}
			\paragraph{Def} Un arbre est un graphe simple, connexe qui ne contient aucun cycle. Les sommets de degré $1$ sont appelés \textit{feuilles}.

			\paragraph{Prop} Si $T$ est un arbre avec $p \geq 2$ sommets, alors $T$ contient au moins 2 feuilles.

			\paragraph{Dém} Soit $T$ un arbre à $p$ sommets, alors considérons un chemin $(v_0, \ldots, v_k)$ de longueur maximale avec
			$v_i \in V \; \forall i 1 \leq i \leq r$. Alors $v_0$ et $v_k$ sont de degré $1$. Effectivement, si $\deg(v_0) \geq 2$, alors
			$\exists e \in E \tq \gamma(e) = \{x, v_0\}, x \neq v_1$. Par définition de $T$, il ne contient aucun cycle, donc
			$x \neq v_i \; \forall i \in \{2, \ldots, k\}$ et $(x, v_0, \ldots, v_k)$ est de longueur supérieure à $(v_0, \ldots, v_k)$. Or $(v_0, \ldots, v_k)$ est de
			longueur maximale. Pareil pour $v_k$.

			\paragraph{Théorème} Soit $T$, un graphe simple à $p$ sommets. Les assertions suivantes sont équivalentes :
				\begin{itemize}
					\item[$(i)$] $T$ est un arbre ;
					\item[$(ii$)] $T$ a $p-1$ arêtes et aucun cycle ;
					\item[$(iii$)] $T$ a $p-1$ arêtes et est connexe.
				\end{itemize}

			\paragraph{Dém} En montrant que $(i) \Rightarrow (ii)$, que $(ii) \Rightarrow (iii)$, et que $(iii) \Rightarrow (i)$, le théorème est démontré.
				\begin{itemize}
					\item Montrons que $(i) \Rightarrow (ii)$ en montrant qu'un arbre à $p$ sommets a $p-1$ arêtes par récurrence.

						  Pour $p = 1$, $T$ a $0$ arête. \\
						  Pour $p \geq 2$, $T$ a au moins $2$ feuilles. Enlevons-en une et l'arête incidente à cette feuille. On obtient $T' \leq T$, un sous-arbre
						  à $p-1$ sommets et $p-2$ arêtes en assumant que $T$ en avait $p-1$.

					\item Montrons que $(ii) \Rightarrow (iii)$ en montrant qu'aucun cycle $\Rightarrow$ connexe.

						  Supposons par l'absurdre que $T$ n'est pas connexe. Soient $T_i$, les $t$ composantes connexes de $T$ tel que $t \geq 2$. Chaque $T_i$
						  est un arbre (pas de cycle), donc chaque $T_i$ a $p_i - 1$ arêtes pour $p_i$, le nombre de sommets de $T_i$. Par définition,
						  $\sum_{i=1}^tp_i = p$, le nombre de sommets de $T$. Et $\sum_{i=1}^t(p_i-1) = (p-1)$, le nombre d'arêtes de $T$. Or
						  $\sum_{i=1}^t(p_i - 1) = \sum_{i=1}^tp_i - t = p - t$. Cela implique $t = 1$, or par construction, nous avions $t \geq 2$. Il
						  y a contradiction. $T$ est donc connexe.

					\item Montrons que $(iii) \Rightarrow (i)$ en montrant que si $T$ a $p-1$ arêtes et est connexe, alors $T$ est un arbre.

						  Supposons par l'absurde que $T$ n'est pas un arbre. Il existe donc un cycle car $T$ est simple par hypothèse et connexe par construction.
						  On considère le sous-graphe $T'$ de $T$ obtenu en retirant une arête du cycle. Si $T'$ contient encore un cycle, réitérer le procédé.
						  Si $T'$ ne contient plus de cycle, $T'$ est un arbre. Or $T'$ a toujours $p$ sommets (car aucun sommet de $T$ n'a été retiré) et un nombre
						  strictement inférieur à $p-1$ arêtes. Or dans $(i) \Rightarrow (ii)$, il a été montré qu'un arbre à $p$ sommets $\Rightarrow (p-1)$ arêtes.
						  Il y a donc contradiction et alors $T$ est un arbre.
				\end{itemize}

			\paragraph{Exemple} Les alcanes ($C_nH_{2(n+1)}$) sont des molécules représentables avec un graphe à $3n + 2$ sommets dont $n$ de degré 4 et $2n + 2$ de
			degré 1. Donc, comme $\sum_{v \in V}\deg(v) = 2\#E$, $4n + 1(2n + 2) = 6n + 2 = 2(3n+1)$, alors $\#E = 3n+1$, on a bien $\#V - \#E = 1$.

			\paragraph{Def}
				\begin{lst}
					\item Deux graphes $\Gamma_1 = (V_1, E_1, \gamma_1)$ et $\Gamma_2 = (V_2, E_2, \gamma_2)$ sont \textit{isomorphes} s'il existe une
					bijection $f : V_1 \to V_2$ et une bijection $g : E_1 \to E_2$ telles que $\forall e \in E, e$ est incident à
					$v, w \in V_1 \Leftrightarrow g(e) \in E_2 $ est incident à $f(v), f(w) \in V_2$.

					\item Le couple $(f, g)$ est appelé un \textit{isomorphisme de graphe}, et on note $\Gamma_1 \sim \Gamma_2$.
				\end{lst}

			\paragraph{Remarques}
				\begin{lst}
					\item lorsque $\Gamma_1$ et $\Gamma_2$ sont simples, $E_i \subseteq \mathcal Paire(V_i)$ avec $i \in \{1, 2\}$, alors la bijection
						  $g$ est induite par $f$, c'est à dire $e = \{v, w\} \in E_1 \Rightarrow g(e) := \{f(v), f(w)\} \in E_2$.

					\item Deux graphes isomorphes ont les mêmes propriétés.

					\item Le problème d'énumérer les alcanes non isomorphes fut résolu par Cayley.
				\end{lst}


		\subsection{Graphes hamiltoniens}
			\paragraph{Exemple} En prenant l'exemple du dessin du dodécaèdre en perspective, on peut se demander s'il existe un moyen de parcourir
			l'entièreté des sommets en un seul cycle. C'est la question que s'est posée Hamilton.

			\paragraph{Def}
				\begin{lst}
					\item Un cycle hamiltonien dans un graphe $\Gamma$ est un cycle simple contenant tous les sommets de $\Gamma$.
					\item Les graphes ayant un cycle hamiltonien sont dits \textit{graphes hamiltoniens}.
				\end{lst}

			\paragraph{Def} Un graphe $\Gamma = (V, E)$ est biparti si on peut écrire $V = B \cup W$ avec $B \cap W = \{\}$ ($B$ et $W$ sont des partitions)
			et si toute arête de $\Gamma$ est incidente à $v \in B$ et à $w \in W$.

			\paragraph{Remarque} Un graphe complet avec $p \geq 2$ sommets n'est pas biparti.

			\paragraph{Lemme} Si $\Gamma$ est biparti, tous ses cycles simples sont de longueur paire.

			\paragraph{Dém} Soit $\Gamma$, un graphe biparti. Par l'absurde, supposons qu'il existe un cycle simple $(v_0, e_1, \ldots, v_{2n}, e_{2n}, v_{2n+1})$
			de longueur impaire. Par définition, $v_i \in B \; \forall i$ pair et $v_i \in W \; \forall i$  impair (ou inversement). Or, par définition du cycle,
			on a $v_0 = v_{2n+1}$. Il y a donc contradiction, le cycle est alors de longueur paire.

			\paragraph{Théorème} Un graphe biparti avec un nombre impair de sommets n'est pas hamiltonien.

			\paragraph{Dém} Soit $\Gamma$, un graphe hamiltonien avec un nombre impair de sommets. Par définition, il contient un cycle simple passant
			par tous ses sommets, à savoir contenir un cycle de longueur impaire, ce qui est impossible par le lemme précédent.

			\paragraph{Théorème (de Dirac)} Soit $\Gamma = (V, E)$, un graphe simple avec $p \geq 3$ sommets et $\forall v \in V, \deg(v) \geq \frac p2$, alors
			$\Gamma$ est hamiltonien.

			\paragraph{Dém} Montrons d'abord que $\Gamma$ est connexe puis qu'il est cyclique. De là, il restera à prouver que ce cycle est hamiltonien.
				\begin{itemize}
					\item Montrons que $\Gamma$ est connexe. Supposons donc par l'absurde qu'il ne le soit pas. Alors la plus petite de ses composantes
						  connexes $\Gamma' \leq \Gamma$ a moins de $\frac p2$ sommets. Or, $\forall v \in V, \deg(v) \geq \frac p2$. Il y a
						  contradiction donc $\Gamma$ est connexe.
					\item Montrons maintenant que $\Gamma$ est cyclique.

						  Soit $C = (v_0, \ldots, v_k)$, un plus long chemin simple dans $\Gamma$ avec $v_0 \neq v_k$ et $k < p$. Par hypothèse,
						  $\deg(v_0) \geq \frac p2$ et $\deg(v_k) \geq \frac p2$. Tous les sommets adjacents à $v_0$ sont dans
						  $\{v_i \tq 1 \leq i \leq k\}$ et tous les sommets adjacents à $v_k$ sont dans $\{v_i \tq 0 \leq i \leq k-1\}$. Comme
						  $k < p$ (donc $\frac k2 < \frac p2$), il doit exister $i \in \{0, \ldots, k-1\} \tq \{v_i, v_k\}, \{v_{i+1}, v_0\} \in E$.
						  Il existe donc un cycle $\overline C = (v_0, v_1, \ldots v_i, v_k, v_{k-1}, \ldots, v_{i+1}, v_0)$.
					\item Reste à prouver que $\overline C$ est hamiltonien.

						  Supposons par l'absurde qu'il ne l'est pas. Il existe donc $y \not \in \overline C$. Comme $\Gamma$ est connexe, on peut supposer
						  qu'$\exists \{v_j, y\} \in E$ avec $ \in \{0, \ldots, k\}$. On peut donc construire un chemin $\widetilde C = (y, v_j,
						  \ldots v_0, v_{i+1}, v_k, v_i, \ldots, v_{j+1})$, or $C$ était un plus long chemin. Donc $\not \exists y$.
				\end{itemize}

			\paragraph{Illustration} Un code de Gray d'ordre $n$ est un arrangement cyclique de $2^n$ mots binaires de longueur $n$ tel que deux
			mots adjacents ne diffèrent que d'un seul élément.

			Un tel code peut être construit sur base d'un graphe hamiltonien. Pour construire un code de Gray d'ordre $(n+1)$ sur base d'un code
			de Gray d'ordre $n$, il faut écrire le code de Gray d'ordre $n$ en le suffixant de $0$ à chaque élément, puis le faire suivre du même
			code de Gray d'ordre $n$, à l'envers, suffixé de $1$ à chaque élément.

		\subsection{Graphes eulériens}
			\paragraph{Def}
				\begin{lst}
					\item Un cycle eulérien dans un graphe $\Gamma$ est un cycle simple contenant toutes les arêtes de $\Gamma$.
					\item Un graphe $\Gamma$ est eulérien s'il contient un cycle eulérien.
				\end{lst}

			\paragraph{Proposition} Si un graphe est eulérien, alors tous ses sommets sont de degré pair.

			\paragraph{Lemme} Soit $\Gamma$ un graphe $| \, \forall v \in V, \deg(v)$ est pair. Alors l'ensemble des arêtes se partitionne
			en une union (arête-)disjointe de cycles.

			\paragraph{Dém} Prouvons-le par récurrence sur $q$, le nombre d'arêtes.
				\begin{itemize}
					\item Le lemme est vrai pour $q = 2$.
					\item Supposons qu'il soit vrai pour $q \leq k$ et prouvons qu'il l'est pour $q = k+1$.

						  Soient $\Gamma = (V, E)$, un graphe de $k+1$ arêtes et $v_0 \in V$. On démarre un chemin en $v_0$ et on ajoute des sommets
						  jusqu'à répétition d'un des sommets. On le note $v_j$. Soit $\Gamma' = (V', E') \leq \Gamma = (V, E)$ dont $V' = V$ et
						  $E' = E \setminus C$ avec $C$ le chemin de $v_j$ à $v_k$. Donc $\#E' \leq k$. Par hypothèse de récurrence, les arêtes
						  de $E'$ se partitionnent en une union arête-disjointe de cycles $C_1, \ldots, C_n$. Comme $E' = E \setminus C$ ne contient
						  aucune arête de $C$ par définition, l'union $C, C_1, \ldots, C_n$ est une partition arête-disjointe de $E$.
				\end{itemize}

			\paragraph{Théorème (d'Euler-Hierholzer)} Soit $\Gamma$ un graphe connexe. $\Gamma$ est eulérien $\iff \forall v \in V, \deg(v)$ est pair.

			\paragraph{Dém}
				\begin{lst}
					\item Par la proposition, $\Gamma$ eulérien $\Rightarrow$ degrés pairs.
					\item Par le lemme, degrés pairs $\Rightarrow E$ se partitionne en une union arête-disjointe de cycles.
				\end{lst}

		\subsection{Application : le problème du voyageur de commerce (TSP) et arbres couvrants minimums (ACM)}
			\paragraph{Énoncé} Un vendeur doit visiter un certain nombre de villes avant de rentrer chez lui (d'où il est parti).
			Comment doit-il choisir sa route afin de minimiser les distances ?

			\paragraph{Objectif} Déterminer un cycle hamiltonien de poids minimum dans $\Gamma = (V, E, \gamma, w)$ un graphe valué tel que $V$ est
			l'ensemble des villes, $E$ l'ensemble des routes, $w : E \to \mathbb R$ une fonction associant à chaque arête un poids réel.

			\paragraph{Remarque} Un graphe complet $\mathcal K_n$ a $\frac 12(n-1)!$ cycles hamiltoniens différents. On ne connait pas d'algorithme
			efficace pour résoudre ce problème.

			\paragraph{Def} Un arbre couvrant dans un graphe $\Gamma = (V, E)$ est un arbre $T = (V', E') \leq \Gamma$ tel que $V = V'$.

			\paragraph{Algorithme de Kurska} Il existe un algorithme (celui de Kurska) qui permet de trouver des arbres couvrants de poids minimum
			dans un graphe valué. La procédure est la suivante :

			\begin{itemize}
				\item[$(i)$] Choisir une arête de plus petit poids pas encore dans $V'$ de manière à ne pas créer de cycle ;
				\item[$(ii)$] Réitérer tant que $\#V' \neq \#V$.
			\end{itemize}

			Comme dans un arbre, $\#E = \#V - 1$, cet algorithme s'exécute en $\#V - 1$ étapes. Il est donc en complexité $O(n)$ avec $n$ le nombre
			de sommets.

			\paragraph{Remarque} Si $C$ est un cycle hamiltonien dans $\Gamma$, alors $C \setminus \{e\}$ est un arbre couvrant. Une solution du TSP
			est toujours plus grande ou égale au poids d'un arbre couvrant minimum.

			De plus, soient $\Gamma = (V, E)$ et $v \in V$. Tout cycle hamiltonien $C_H \leq \Gamma$ contient deux arêtes incidentes à $v$. Le reste
			du chemin est un autre arbre couvrant de $\Gamma \setminus \{v\}$. Une solution du TSP est donc toujours plus grande ou égale à la somme
			des poids des 2 arêtes incidentes à $v$ + le poids d'un arbre couvrant de $\Gamma \setminus \{v\}$.

		\subsection{Relations et ordres partiels}
			\paragraph{Def} Soit $P$ un ensemble. Une ordre partiel sur $P$ est une relation sur $P$ (un ensemble de couples $(p_1, p_2) \in P^2$)
			notée $p_1 \leq p_2$ suivant les propriétés suivantes $\forall p, q, r \in P$ :

			\begin{itemize}
				\item $p \leq p$ (réflexivité) ;
				\item $(p \leq q) \land (q \leq r) \Rightarrow (p \leq r)$ (transitivité) ;
				\item $(p \leq q) \land (q \leq p) \Rightarrow (p = q)$ (antisymétrie).
			\end{itemize}

			\paragraph{Remarque} On note $(P, \leq)$ un ensemble partiellement ordonné.

			\paragraph{Def} $\mathcal P(E)$ est l'ensemble des parties de $E$, c'est-à-dire l'ensemble de tous les sous-ensembles de $E$.

			\paragraph{Remarque} Un ordre partiel $(P, \leq)$ peut se représenter à l'aide d'un graphe dirigé que l'on simplifie en enlevant les lacets
			(axiome de réflexivité) et en enlevant les arêtes que l'on peut obtenir par transitivité. De plus, par antisymétrie, il n'y a pas de
			cycle. Par convention, on enlève les flèches et le dessine de bas en haut.

			\paragraph{Def} Soit $(P, \leq)$, un ordre partiel. Son diagramme de Hasse est le graphe simple $\Gamma = (V, E)$ tel que :
				\begin{itemize}
					\item $e = \{x, y\} \in E \iff (x \leq y) \land (\not \exists z \in V \tq x \leq z \leq y)$.
					\item $x \leq y \Rightarrow$ $x$ plus bas que $y$ dans sa représentation.
				\end{itemize}

			\paragraph{Def} Soit $(P, \leq)$ un ordre partiel.
				\begin{itemize}
					\item Une chaine dans $P$ est un sous-ensemble $C \subseteq P$ tel que $\forall c_1, c_2 \in C, (c_1 \leq c_2) \lor (c_2 \leq c_1)$.
					\item Une antichaine dans $P$ est un sous-ensemble $A \subseteq P$ tel que $\forall a_1, a_2 \in A, \lnot ((a_1 \leq a_2) \lor (a_2 \leq a_1))$.
				\end{itemize}

			\paragraph{Théorème (de Dilworth)} Soit $(P, \leq)$ un ensemble fini partiellement ordonné. Il existe une antichaine $A$ et une partition
			de $P$ par des chaines tel que $\#Q = \#A$ avec $Q$ l'ensemble des partitions de $P$.

			\paragraph{Dém} Ce théorème se prouve par récurrence sur $\#P$ :
				\begin{itemize}
					\item si $\#P = 0$, alors $\#Q = \#A = \#P = 0$ car $Q = A = P = \emptyset$.
					\item si $\#P > 0$, notons $a = \max P$. Pour un certain $k$ naturel, $\exists C_1, \ldots, C_k$, une partition de
						  $P \setminus \{a\}$ et $A_0$ une antichaine de $P \setminus \{a\}$ telle que $\#A_0 = k$ par hypothèse de récurrence.

						  Prouvons qu'il y a toujours une antichaine dans $P$. $\forall 1 \leq i \leq k, \exists x_i$ le plus grand élément dans
						  $A_0 \cap C_i$. L'ensemble $A = \{x_i \tq 1 \leq i \leq k\}$ est une antichaine car
						  $\forall i \neq j, \exists y \in (A \cap C_j)$ tel que $y \leq x_j$ (par définition de $x_j$) et comme
						  $y \not \leq x_j$, $x_j \leq x_i$ (transitivité).

						  Prouvons maintenant qu'il y a une antichaine de même cardinal que $Q$ :
						  \begin{itemize}
							\item Si $a$ majore l'un des $x_i$, soit $K = \{a\} \cup \{z \in C_i \tq z \leq x_i\}$, alors $P \setminus K$ n'a pas
								  d'antichaine de cardinal $k$, donc $P \setminus K$ peut être, par hypothèse de récurrence, partitionné en $k-1$
								  chaines et $A \setminus \{x_i\}$ est une antichaine de cardinal $\#A - 1 = k-1$. $P$ est donc partitionnable en $k$
								  chaines (les $k-1$ de de $P \setminus K$ et $K$) et $\exists$ une antichaine $A$ de cardinal $k$.
							\item Sinon, $A \cup \{a\}$ est une antichaine de cardinal $k+1$ et $P$ est partitionnable en $k+1$ chaines :
								  $C_0 = \{a\}, C_1, \ldots, C_k$.
						  \end{itemize}
				\end{itemize}

			\paragraph{Remarque}
				\begin{lst}
					\item Soit $(P, \leq)$ un ordre partiel fini avec une partition $Q$ de $P$ et une antichaine $A \subseteq P$. Alors $\#A < \#Q$
						  car si $\#A > \#Q$, alors $\exists i, j, k \tq \exists C_i \ni \{a_j, a_k\}$. Ce qui veut dire que $a_j$ et $a_k$ sont
						  comparables car dans une chaine, ce qui est impossible par définition de l'antichaine.
					\item Soit $(P, <)$, une ordre total. Pour toute antichaine $A \subseteq P, \#A \in \{0, 1\}$.
				\end{lst}

			\paragraph{Def} Soit $\Gamma = (V, E)$ une graphe simple.
				\begin{itemize}
					\item Un couplage $M$ de $\Gamma$ est un sous-ensemble d'arêtes de $E$ deux à deux non adjacentes. Un sommet
						  $v \in V$ incident à une arête $e \in M$ est dit couplé.
					\item Un transversal $T$ de $\Gamma$ est un sous-ensemble de sommets de $T$ tel que $\forall e \in E, \exists v \in T \tq $
						  $e$ est incidente à $v$.
				\end{itemize}

			\paragraph{Def} Soit $\Gamma = (V = B \sqcup W, E)$, un graphe biparti et $M$ un couplage. Un chemin alterné dans $\Gamma$ est un chemin
			qui démarre en un sommet de $B$ non couplé et alterne une arête de $E \setminus M$ et une arête de $M$ et ainsi de suite.

			\paragraph{Théorème (de König)} Soit $\Gamma = (V = B \sqcup W, E)$, un graphe biparti. La cardinalité maximale d'un couplage de $\Gamma$
			est égale à la cardinalité minimale d'un transversal de $\Gamma$.

			\paragraph{Dém} Soient $\Gamma = (V = B \sqcup W, E)$, un graphe biparti et $M$ un couplage de cardinalité maximale dans $\Gamma$.
			$\forall m \in M$, choisissons un de ses sommets incidents comme suit :

			\begin{itemize}
				\item Le sommet $\in W$ s'il existe un chemin alterné arrivant à ce sommet,
				\item le sommet $\in B$ sinon.
			\end{itemize}

			Notons $U$ l'ensemble des sommets obtenus. Montrons que $U$ est un transversal de $\Gamma$ (avec $\#U = \#M$). Si $\#U < \#M$,
			alors $\exists e \in M \tq \not \exists v \in U | v \in \gamma(e)$.

			Soit $e = \{b, w\} \in E$. Il faut montrer que soit $b \in U$, soit $w \in U$. On peut supposer $e \not \in M$ car si $e \in M$, alors
			$(b \in U) \lor (w \in U)$ par définition. Comme $M$ est maximal, $\exists e' = \{b', w'\} \in M \tq (b = b') \lor (w = w')$.
			On peut supposer $b = b'$ car sinon $w = w'$, donc $\{b, w\} = \{b, w'\}$ ou encore $\{b, w\}$ est un chemin alterné, ce qui implique
			que $w \in U$ par définition de $U$. Toujours par définition de $U$, si $b = b' \in U$, alors $w' \in U$. $\exists$ donc $P$, un chemin
			alterné dans $\Gamma$ terminant en $w'$.

			\begin{itemize}
				\item Soit $P' := P \setminus \{\{w, b\}, \{w', b\}\}$. $P'$ est un chemin alterné arrivant en $w$, donc $w \in U$ par définition de
					  $U$.
				\item Soit $P' := P \cup \{w, b\} \cup \{w', b\}$. Alors il existe un couplage de cardinalité $\#M$, il y a donc contradiction.
			\end{itemize}

			\paragraph{Lemme} Soient $(P, \leq)$, un ordre partiel, $\Gamma = (V = (B = P \times \{1\}) \sqcup (W = P \times \{2\}), E)$ un graphe
			biparti et $T$ un transversal dans $\Gamma$. Le sous-ensemble $A \subseteq P$ défini comme suit :
			$A := \{p \in P \tq (p, 1), (p, 2) \not \in T\}$ est une antichaine.

			\paragraph{Dém} Supposons, par l'absurde, que $A$ n'est pas une antichaine (est une chaine). Alors $\exists a \neq b \in A$ tels que
			$a \leq b$ (par définition de la chaine dans $\Gamma$). $\exists$ donc $\{(a, 1), (b, 2)\} \in E$, donc $a \in T$ ou $b \in T$. De là,
			il faut déduire que soit $a$ soit $b$ $\not \in A$, il y a contradiction, $A$ est bien une antichaine.

			\paragraph{Proposition} Les théorèmes de Dilwoth et de König sont équivalents.

			\paragraph{Dém} Pour montrer que König $\iff$ Dilworth, montrons séparément les implications. Premièrement, montrons que König
			$\Rightarrow$ Dilworth, nous montrerons Dilworth $\Rightarrow$ König par la suite.

			Soit $(P, \leq)$, un ordre partiel. Construisons un graphe biparti $\Gamma = (V = B \sqcup W, E)$ avec $B = P \times \{1\}$ et avec
			$W = P \times \{2\}$ tel que pour $p, q \in P, \{(p, 1), (q, 2)\} \in E \iff (p \leq q) \land (p \neq q)$.

			Soient $M$ et $T$, respectivement un couplage et un transversal de cardinalité maximale et minimale dans $\Gamma$. Le théorème de König
			dit que $\#M = \#T$. Définissons $A \subseteq P := \{p \in P \tq (p, 1), (p, 2) \not \in T\}$, par le lemme précédent, nous savons
			que $A$ est une antichaine. De là découle $\#A = \#\{p \in P \tq (p, 1), (p, 2) \not \in T\} \leq \#P - \#T$.

			Soit $Q$, un ensemble de $n$ partitions de $P$ : $Q = \{C_1, \ldots, C_n\}$.

			\begin{itemize}
				\item Soit $C_i = \{p_0, \ldots, p_l\}$ avec $l \geq 1$ si $\{(p_N, 1), (p_{N+1}, 2)\} \in E$ et $(p_l, 1), (p_l, 2)$ pas incidents
					  à $M$.
				\item Soit $C_i = \{p\}$ si $(p, 1), (p, 2)$ ne sont pas incidents à $M$.
			\end{itemize}

			De là, $Q$ est une partition de $P$. Nous pouvons exprimer $\#P = \sum_i \#C_i = \#M + \#Q$. Autrement dit, $\#Q = \#P - \#M = \#P - \#T
			\leq \#A$. Or, comme $\#A \leq \#Q$ par la remarque ci-dessus, alors $\#Q = \#A$.

			Reste à prouver que Dilworth $\Rightarrow$ König.

			Soit $\Gamma = (V = B \sqcup W, E)$, un graphe biparti. On définit un ordre partiel $(P, \leq)$ tel que $P = V$ et $\forall (b, w) \in
			B \times W, b \leq w \iff \exists e \in E \tq \{b, w\} = \gamma(e)$. Par le théorème de Dilworth, il existe $A$ une antichaine et
			$Q$ un ensemble de partition en antichaines de $P$ tels que $\#A = \#Q$. Seules les arêtes $\{b, w\}$, les singletons $\{v\}$ peuvent
			être des chaines (en omettant les chaines triviales $\{\}$). Comme $\forall q \in Q, q$ est une partition de $P$, l'ensemble $M = E \cap Q$
			est un couplage car $\forall v \in V, \exists! q \in Q \tq v \in q$. De plus, comme $\Gamma$ est biparti, l'ensemble
			$T := \{b \in P \setminus A \tq a \in A, \{a, b\} \in E\} = P \setminus A$ est un transversal de $\Gamma$.
			Pour montrer König, il reste à prouver que $\#M = \#T$.

			$(\forall v \in T, \exists e \in M \tq v \in \gamma(e)) \Rightarrow (\#M \geq \#T)$. De plus, $(\forall e = \{b, w\} \in M,
			\exists v \in T \tq (v = b) \lor (v = w)) \Rightarrow (\#T \geq \#M)$. Donc $\#T = \#M$.

	\section{Arithmétique modulaire}

		\subsection{Les entiers et la division euclidienne}

			\paragraph{Rappel} L'ensemble des entiers se note $\Z$ tel que $\mathbb N \subset \Z$. Plus précisément,
			$\Z = \mathbb N_0 \cup -\mathbb N_0 \cup \{0\}$. Cet ensemble est défini par deux opérations :

			\begin{itemize}
				\item l'addition interne $+ : \Z \times \Z \to \Z : a, b \mapsto a+b$.

					  Cette opération respecte les propriétés suivantes :

					  \begin{itemize}
						\item[$(i)$] associativité ;
						\item[$(ii$)] existence du neutre ;
						\item[$(iii$)] existence de l'opposé ;
						\item[$(iv)$] commutativité.
					  \end{itemize}

					  $(\Z, +)$ est donc un groupe commutatif.
				\item la multiplication interne $\cdot : \Z \times \Z \to \Z : a, b \mapsto ab$.

					  Cette opération respecte les propriétés suivantes :

					  \begin{itemize}
						\item[$(i)$] associativité ;
						\item[$(ii)$] double distributivité (gauche et droite) sur l'addition ;
						\item[$(iii)$] commutativité ;
						\item[$(iv)$] inexistence de diviseur de zéro ;
						\item[$(v)$] existence du neutre.
					  \end{itemize}

					  $(\Z, +, \cdot)$ est donc un anneau ($(\Z, +)$ est un groupe et $\cdot$ respecte $(i)$ et $(ii)$) unital $(v)$
					  commutatif $(iii)$ intègre $(iv)$.
			\end{itemize}

			\paragraph{Axiome} Il existe sur $\Z$ une relation d'ordre $\leq$ telle que :

			\begin{itemize}
				\item[$(i)$] $\leq$ est un ordre total ;
				\item[$(ii)$] $\forall a, b, c \in \Z, a \leq b \iff a + c \leq b + c$ ;
				\item[$(iii)$] $\forall a, b, c \in \Z, (c \geq 0) \land (a \leq b) \Rightarrow ac \leq bc$.
			\end{itemize}

			\paragraph{Def} On définit la valeur absolue d'un entier $a \in \Z$ telle que :

			\[|a| = \left\{\begin{aligned}a &\text{ si $a \geq 0$} \\-a &\text{ sinon}\end{aligned}\right..\]

			La valeur absolue est conçue telle que $\forall a, b \in \Z, (|a| = 0) \iff (a = 0), |ab| = |a||b|$.

			\paragraph{Def} Soient $a, b \in \Z$. $a$ divise $b$ ($a|b$) $\iff \exists c \in \Z \tq ac = b$.

			\paragraph{Remarque} La divisibilité est une relation car elle suit les propriétés de réflexivité, transitivité et d'antisymétrie.

			\paragraph{Théorème (division euclidienne)} Soient $a, b \in \Z \tq b \neq 0$. $\exists! q, r \in \Z \tq a = bq + r$
			avec $0 \leq r < b$.

			\paragraph{Def} Un nombre $p \in \mathbb N$ est premier $\iff (p \not \in \{0, 1\}) \land (\not \exists q \in \mathbb N \setminus \{1, p\}
			\tq q|p)$.

			\paragraph{Def} Un entier $d$ est un GCD de $a$ et $b$ $\iff (d|a) \land (d|b) \land (c \in \Z, (c|a) \land (c|b) \Rightarrow c|d)$.
			On le note $d = GCD(a, b)$. De plus, $\forall a \in \Z, GCD(a, 0) := |a|$.

			\subsubsection{L'algorithme d'Euclide}

			\paragraph{Proposition} Soient $a, b \in \Z, b \neq 0, q, r \in \Z \tq a = bq + r$. Alors $GCD(a, b) = GCD(b, r)$.

			\paragraph{Dém} Soient $a, b, q, r \in \Z \tq a = bq + r$ et $c \in \Z$. Comme $a = bq +  r$, si $(c|b) \land (c|r)$,
			alors $c|a$. Et comme $r = a - bq$, si $(c|a) \land (c|b)$, alors $c|r$.

			\paragraph{Algorithme} Soient $a, b \in \Z, b \neq 0$. Pour déterminer $GCD(a, b)$, on suppose $a, b \geq 0$ car $GCD(a, b) =
			GCD(-a, b) = GCD(a, -b) = GCD(-a, -b)$. Par le théorème de la division euclidienne, $\exists! q_1, r_1 \in \Z \tq a = bq_1 + r_1$
			avec $0 \leq r_1 < |b|$. Si $r_1 = 0$, alors $GCD(a, b) = GCD(b, 0) = |b|$. Sinon, on itère en prenant $b$ et $r_1$. Par la
			DE\footnote{Division euclidienne.}, $\exists! q_2, r_2 \tq b = r_1q_2 + r_2$ avec $0 \leq r_2 < r_1$. Si $r_2 = 0$, alors
			$GCD(b, r_1) = GCD(r_1, 0) = r_1$. Sinon, on réitère sur la formule $r_i = r_{i+1}q_{i+2} + r_{i+2}$ avec $a, b = r_{-1}, r_0$. $\exists$
			donc $N \tq r_N = 0$. On a donc $GCD(a, b) = GCD(b, r_1) = GCD(r_i, r_{i+1}) = GCD(r_{N-1}, r_N) = r_{N-1}$.

			\paragraph{Proposition} Soient $a, b \in \Z, b \neq 0$. $\exists! s, t \in \Z \tq GCD(a, b) = sa + tb$.

			\paragraph{Dém} Soient $a, b \in \Z$. Supposons $a, b \geq 0$. Soient $r_i, 1 \leq i \leq n$, les restes successifs de la DE
			de l'algorithme d'Euclide avec $r_i = r_{i+1}q_{i+2} + r_{i+2}$, $0 \leq r_{i} < r_{i+1}$.

			On construit $s$ et $t$ comme suit :

			\[\begin{aligned}
				GCD(a, b) = r_n &= r_{n-2} - q_nr_{n-1} \\
								&= s_1t_{n-2} + t_1r_{n-1} \;\; \text{	où $s_1 = 1, t_1 = -q_n$} \\
								&= s_1r_{n-2} + t_1(r_{n-3} - q_{n-1}r_{n-2}) \\
								&= s_2r_{n-3} + t_2r_{n-2} \;\; \text{	où $s_2 = t_1, t_2 = (s_1 - t_1q_{n-1})$} \\
								&= ...
			\end{aligned}\]

			On construit, inductivement, $s_k = t_{k-1}, t_k = (s_{k-1} - t_{k-1}q_{n-(k-1)}) \tq GCD(a, b) = s_kr_{n-(k+1)} + t_kr_{n-k}$.

			De là, $s_nr_{n-1} + t_nr_0 = s_na + t_nb$.

			\subsubsection{Décomposition en nombres premiers}

			\paragraph{Def} Deux entiers $a, b \in \Z_0$ sont premiers entre eux $\iff GCD(a, b) = 1$.

			\paragraph{Proposition (lemme de Bézout)} Soient $a, b \in \Z$. $a$ et $b$ sont premiers entre eux
			$\iff \exists s, t \in \Z \tq sa + tb = 1$.

			\paragraph{Lemme (de Gauss)} Soient $a, b \in \Z \tq a$ et $b$ sont premiers entre eux, $c \in \Z \tq b|ac$. Alors $b|c$.

			\paragraph{Dém} Soient $a, b$ deux entiers premiers entre eux. Donc $GCD(a, b) = 1$, ou encore $\exists! s, t \in \Z \tq
			1 = sa + tb$. De là, $c = sac + tbc$. Or, on sait que $b|ac$ et $b|b$, donc $b|sac + tbc = c$.

			\paragraph{Corollaire} Soient $a, b, p \in \Z \tq p$ est premier. Si $p|ab$, alors $(p|a) \lor (p|b)$.

			\paragraph{Dém} Soient $a, b, p \in \Z \tq p$ est premier. Si $p|a$, le corollaire est bon. Sinon, si $p\not|a$, comme $p$
			est premier, $a$ et $p$ sont premiers entre eux. Donc par le lemme de Gauss, $p|b$.

			\paragraph{Théorème (décomposition en facteurs premiers)} $\forall z \in \Z, \exists n \in \mathbb N, p_1 \ldots p_n$, n nombres
			premiers différents deux à deux et $e_1 \ldots e_n \in \mathbb N_0 \tq z = (\pm 1)\prod_{i=1}^np_i^{e_i}$. Cette expression est unique
			(à l'ordre d'expression des $i^e$ termes).

		\subsection{Groupes, anneaux et entiers $\mod n$}

			\subsubsection{Définitions}

			\paragraph{Def} Un groupe $(G, *)$ est un ensemble non vide $G$ muni d'une loi de composition $* : G \times G \to G : g, h \mapsto g*h \tq$

			\begin{itemize}
				\item[$(i)$] $*$ est associative ;
				\item[$(ii)$] $\exists$ un neutre $e \in G \tq \forall g \in G, g*e = e*g = g$ ;
				\item[$(iii)$] $\forall g \in G, \exists$ un inverse $g^{-1} \in G \tq g^{-1} * g = g * g^{-1} = e$.
			\end{itemize}

			\paragraph{Def} Soient $(G, *)$ un groupe et un sous-ensemble $H \subseteq G$. Si $(H, *)$ est un groupe, on note $H \leq G$
			ou $(H, *) \leq (G, *)$ le sous-groupe $H$.

			\paragraph{Proposition} Soient $(G, *)$ un groupe et $H \subseteq G$. $H \leq G \iff$ :

			\begin{itemize}
				\item[$(1)$] $e \in H$ ;
				\item[$(2)$] $\forall g, h \in G, g*h^{-1} \in H$.
			\end{itemize}

			\paragraph{Dém} Pour montrer la double implication il faut montrer les implications séparément. Montrons d'abord que $H \leq G \Rightarrow
			(1) \land (2)$. Ensuite, montrons que $(1) \land (2) \Rightarrow H \leq G$.

			Montrons d'abord la première implication. Comme $H \leq G$, $H$ est un groupe, donc $(1)$ et $(2)$ sont vérifiés naturellement.
			Montrons ensuite la seconde implication. La propriété des groupes $(i)$ se montre par $(2)$ car $\forall g, h \in H, g*(h^{-1})^{-1}
			\in H$ (il faut que $h^{-1}$ soit dans $H$, ce qui est prouvé dans $(iii)$). La propriété $(ii)$ se montre par la proposition $(1)$.
			La propriété $(iii)$ se montre comme suit : soit $g = e \in H$. $\forall h \in H, e * h^{-1} = h^{-1} \in H$ par la proposition $(2)$.

			\paragraph{Def} Soit $(G, *)$ un groupe. Si la loi de composition $*$ est commutative, $G$ est un groupe abélien (commutatif) et est noté
			$(G, +)$\footnote{Le symbole $+$ précise que la composition est commutative.}.

			\paragraph{Proposition} Soit $S \subseteq \Z \tq \#S > 0, (S, +) \leq \mathbb (Z, +)$. $\exists k \in \Z \tq S = k \Z$.

			\paragraph{Dém}
			\begin{lst}
				\item Si $S = \{0\}$, alors $S = 0\Z$.
				\item Si $\#S > 1$, on prend $k = \min\{s \in S \tq s > 0\}$ (le plus petit entier positif $\in S$). Reste à prouver que $S = k\Z$.
					  Supposons par l'absurde qu'$\exists s \in S \tq s \not \in k\Z$. Par la division euclidienne, $\exists! q, r \in Z \tq s = kq + r$
					  avec $0 \leq r < k$. Si $r \neq 0$, sinon $s \in k\Z$. $r = kq - s \in S$ car $(s \in S) \land (k \in S \Rightarrow kq \in S)$.
					  Or $0 < r < k$ alors que $k$ est le plus petit entier positif de $S$. Donc il y a contradiction.
			\end{lst}

			\paragraph{Proposition} Soient $k, l \in \Z$. On définit $k\Z + l\Z := \{kz_1 + lz_2 \tq z_1, z_2 \in \Z\}$. $k\Z + l\Z = GCD(k, l)\Z$.

			\paragraph{Dém}
			\begin{lst}
				\item Montrons que $GCD(k, l)\Z \subseteq k\Z + l\Z$.

					  Par la proposition d'Euclide $\exists s, t \in \Z \tq sk + tl = GCD(k, l)$. Donc $\forall z \in \Z, GCD(k, l)z \in GCD(k, l)\Z =
					  (sk + tl)z = k(sz) + l(tz) \in k\Z + l\Z$.
				\item Montrons ensuite que $k\Z + l\Z \subseteq GCD(k, l)\Z$.

					  Par définition, $\forall z_1, z_2 \in \Z, \exists y_1, y_2 \tq GCD(k, l)y_1 = z_1$ et $GCD(k, l)y_2 = z_2$.
					  $\forall z_1, z_2 \in \Z, kz_1 + lz_2 \in k\Z + l\Z = GCD(k, l)(y_1k + ly_2) \in GCD(k, l)\Z$.
			\end{lst}

			\subsubsection{Groupes quotients}

			\paragraph{Remarque} Ici, $(G, +)$ représente un groupe abélien, l'inverse de $g \in G$ n'est donc plus noté $g^{-1}$ mais $-g$ par convention.

			\paragraph{Def} Une classe latérale d'un sous-groupe $H \leq G$ est un ensemble $g + H := \{g + h \tq h \in H\}$ pour $g \in G$ fixé.

			\paragraph{Proposition} Soient $(G, +)$ un groupe abélien, $H$ un sous-groupe $\tq H \leq G$ et $g, g' \in G$. $g + H = g' + H \iff
			\forall h \in H, \exists! h' \in H \tq g+h = g'+h'$.

			\paragraph{Dém} Pour montrer ceci, il faut montrer les implications séparément. Commençons par montrer l'implication $\Rightarrow$.
			$\forall h \in H, g + h \in g + H = g' + H \Rightarrow g' + h' = g + h$ avec $h' \in H$. Reste à prouver que $h'$ est unique.
			Supposons donc $h', \widetilde{h} \in H \tq g' + h' = g' + \widetilde{h}$. En réorganisant, on obtient $e+h' = e+\widetilde{h}$ ou encore
			$h' = \widetilde{h}$.

			Montrons maintenant l'implication $\Leftarrow$. Comme $\forall f = g + h \in g + H, \exists! f' = g' + h' \in g' + H \tq f = f'$, alors
			$g + H \subseteq g' + H$. De plus, comme $\forall h, h'$ est unique et que $h$ et $h'$ viennent du même ensemble, $\#(g + H) = \#(g' + H)$.
			Donc $g + H = g' + H$.

			\paragraph{Def} On note $G/H$ l'ensemble des classes latérales de $H$ avec $H \leq G$. $G/H := \{g + H \tq g \in G\}$. S'il n'y a pas
			d'ambigüité sur le sous-groupe, on note $\overline{g} := g + H$.

			\paragraph{Exemple} Soient $(\Z, +)$, un graphe abélien et $7 \in \Z$. $\Z/7\Z := \{7\Z, 1 + 7\Z, \ldots, 6 + 7\Z\} :=
			\{\overline 0, \overline 1, \ldots, \overline 6\}$.

			\paragraph{Proposition} Soient $(\Z, +)$ et $k \in \Z$. $\Z/k\Z$ est une partition de $\Z$.

			\paragraph{Dém} Par définition, $\Z/k\Z$ est une partition de $\Z \iff (\forall i, j \in \{0, \ldots, k-1\}, c_i \cap c_j \neq \emptyset \iff i = j)
			\land (\bigcup_{c \in \Z/k\Z} c = \Z)$. Pour montrer l'union totale, montrons que $\forall r_1, r_2 \in \Z, (r_1 + k\Z) \cap (r_2 + k\Z) \neq 0
			\iff r_1 = r_2 + kz'$ avec $z' \in \Z$. Soit $z \in (r_1 + k\Z) \cap (r_2 + k\Z)$. Alors $\exists q_1, q_2 \in \Z$, par la DE, $\tq
			r_1 + kq_1 = z = r_2 + kq_2$. Donc $r_1 = r_2 + k(q_2 - q_1)$. Pour démontrer l'intersection vide, également par la DE, $\forall z \in \Z
			\exists q, r \in \Z \tq z = kq + r \in r \in k\Z$.

			\paragraph{Théorème} Soient $(G, +)$ un groupe abélien et $H \leq G$. Alors $G/H$ est muni d'une loi $\overline + \tq (G/H, \overline +)$ est un
			groupe abélien. Précisément, on définit $\forall g, g' \in G, \overline g \overline + \overline {g'} := \overline {g + g'}$.

			\paragraph{Remarque} Une opération d'addition a été définie sur base d'une autre opération. Il faut cependant vérifier si elle est
			\textit{bien définie}\footnote{Le terme \textit{bien défini} veut dire \textit{défini sans ambiguïté} sur une autre opération, à savoir,
			si on définit $*'$ sur un ensemble E' sur base de $*$ défini sur un ensemble E, alors il faut que $a*b = a'*b' \Rightarrow f(a)*'f(b) = f(a')*f(b')$
			où $f$ est une application définie par $f : E \to E'$.}, à savoir que si $\overline{\widetilde g} = \overline{g}$ et
			$\overline{\widetilde{g}'} = \overline{g'}$, alors il faut que $\overline{\widetilde{g}} \overline{+} \overline{\widetilde{g}'} =
			\overline{g} \overline{+} \overline{g'}$.

			\paragraph{Dém} Soient $\widetilde{g}, g, \widetilde{g}', g' \in G$ tels que $\overline{g} = \overline{\widetilde{g}}, \overline{g'} = \overline{\widetilde{g}'}$.
			Comme $\overline{g} = \overline{\widetilde{g}}$, on a $g - \widetilde{g} = h \in H$ et comme $\overline{\widetilde{g}'} = \overline{g'}$, on a
			$\widetilde{g}' - g = h' \in H$. De plus, $\overline{\widetilde{g}} \overline{+} \overline{\widetilde{g}'} := \overline{\widetilde{g} + \widetilde{g}'}
			= \overline{(g-h) + (g'-h')} = \overline{(g+g') - (h+h')} = \overline{g+g'} = \overline{g} \overline{+} \overline{g'}$.

			Maintenant que nous savons que l'addition est bien définie, il faut prouver qu'elle confère à $G/H$ une structure de sous-groupe commutatif :

			\begin{itemize}
				\item[$(1)$] $e \in G$, alors $\overline{e}$ est le neutre pour $G/H$.
				\item[$(2)$] Soit $g \in G$. $-g = -(g+H) := (-g) + H = \overline{-g}$.
				\item[$(3)$] $\overline{+}$ est commutatif car cet opérateur est défini selon l'opérateur $+$ de $G$ qui, lui, est commutatif.
			\end{itemize}

			\paragraph{Def} Pour $n \in \mathbb N_0$, $n\Z \leq \Z$, on définit le groupe des entiers modulo $n$ comme le groupe quotient $(\Z/n\Z, \overline{+})$
			où $\overline{a} \overline{+} \overline{b} = \overline{a+b}$.

			\subsubsection{Isomorphismes de groupes}

			\paragraph{Def} Soient $(G, *), (G', *')$, deux groupes. Un morphisme de groupes est une application $f : G \to G'$ telle que
			$\forall g, h \in G, f(g*h) = f(g) *' f(h)$.

			\paragraph{Def} Un morphisme de groupes $f : G \to G'$ est dit :

			\begin{itemize}
				\item injectif si $\forall g, h \in G, f(g) = f(h) \Rightarrow g = h$ ;
				\item surjectif si $\forall g' \in G', \exists g \in G \tq f(g) = g'$ ;
				\item bijectif s'il est injectif et surjectif.
			\end{itemize}

			\paragraph{Def} Soient $(G, *), (G, *')$ deux groupes et $f : G \to G'$ un morphisme de groupes.

				\[\begin{aligned}
					\Imappl(f) &:= \{f(g) \tq g \in G\} = f(G) \subseteq G', \\
					\Ker(f)	&:= \{g \in G \tq f(g) = e' \in G'\} = f^{-1}(e') \subseteq G.
				\end{aligned}\]

			\paragraph{Proposition} $\Ker(f)$ est un sous-groupe de $G$ et $\Imappl(f)$ est un sous-groupe de $G'$.

			\paragraph{Lemme} Soit $(G, *)$ un groupe. Si $x \in G = x * x$, alors $x = e$ où $e$ est le neutre de $G$.

			\paragraph{Dém} Soient $(G, *), x \in G \tq x = x * x$. Par définition de $G$, $\exists x^{-1} \in G \tq x * x^{-1} = e$.
			En ajoutant $x^{-1}$ de part et d'autre de l'équation, on obtient $x * x^{-1} = x * x * x^{-1} \iff e = x * e = x$.

			\paragraph{Lemme} Soient $(G, *), (G', *')$ et $f : G \to G'$ un morphisme de groupes. $\forall x \in G, f(x)^{-1} = f(x^{-1})$.

			\paragraph{Dém} Soient $f : G \to G'$ un morphisme de groupes. $e' \in G' = f(e) = f(x^{-1} * x) = f(x^{-1}) *' f(x) \iff
			f(x)^{-1} = f(x^{-1})$.

			\paragraph{Proposition} Soient $(G, *), (G', *')$, deux groupes et $f : G \to G'$, un morphisme de groupes. Alors :

			\begin{itemize}
				\item $f$ est injective $\iff$ $\Ker(f) = \{e\}$
				\item $f$ est surjective $\iff$ $\Imappl(f) = G'$
			\end{itemize}

			\paragraph{Dém} Montrons d'abord la première équivalence (l'injectivité).

			\begin{itemize}
				\item Prouvons d'abord $f$ injective $\Rightarrow \Ker(f) = \{e\}$. Pour ce faire, montrons que $e \in \Ker(f)$ puis montrons qu'
				$\not \exists x \neq e \in \Ker(f)$.
				\begin{itemize}
					\item $f(e) = f(e *e) = f(e) *' f(e)$. Or, par le lemme ci-dessus, $f(e) = e' \in G'$, donc $e \in \Ker(f)$.
					\item Par définition d'une fonction injective, $\forall g, h \in G, f(g) = f(h) \Rightarrow g = h$. Donc $\forall g \in G \tq
					f(g) = f(e) \Rightarrow g = e$. $e$ est donc le seul élément de $\Ker(f)$.
				\end{itemize}

				\item Prouvons ensuite $\Ker(f) = \{e\} \Rightarrow$ $f$ injective. Soient $g_1, g_2 \in G$. $f(g_1) = f(g_2) \Leftrightarrow
				f(g_1) *' (f(g_2))^{-1} = f(g_2) *' (f(g_2 ))^{-1} = e'$. Comme par le lemme précédent, $f(x)^{-1} = f(x^{-1})$, alors
				$f(g_1 * g_2^{-1}) = e'$. Donc $g_1 * g_2^{-1} \in \Ker(f)$. Cependant, par hypothèse, on peut dire que si $x \in \Ker(f)$, alors
				$x = e$. Donc $g_1 * g_2^{-1} = e$, ou encore $g_1 = g_2$.
			\end{itemize}

			Il faut encore maintenant montrer la seconde équivalence (la surjectivité).

			Cette preuve est cependant immédiate :
			$\Imappl(f) = G' \iff \{f(g) \tq g \in G\} = G' \iff \forall g' \in G', \exists g \in G \tq f(g) = g' \iff$ $f$ est surjective.

			\paragraph{Def} Soient $(G, *), (G', *')$ deux groupes.

			\begin{itemize}
				\item Un isomorphisme de groupe est un morphisme bijectif $f : G \to G'$.
				\item $(G, *)$ et $(G', *')$ sont dits isomorphes s'$\exists f : G \to G'$ un isomorphisme. On note $G \sim G'$.
			\end{itemize}

			\paragraph{Exemple} $\exp : \mathbb R \to \mathbb R_0^+$ est un isomorphisme entre $(\mathbb R, +)$ et $(\mathbb R_0^+, \cdot)$

			\subsubsection{Les anneaux}

			\paragraph{Def} Un anneau $(A, +, \cdot)$ est un ensemble $A$ non vide muni d'une opération de deux opérations ($+ : A \times A \to A$
			et $\cdot : A \times A \to A$) respectant les propriétés suivantes :

			\begin{itemize}
				\item $(A, +)$ est un groupe commutatif ;
				\item $\cdot$ est une opération associative ;
				\item $\cdot$ est une opération distributive sur $+$.
			\end{itemize}

			\paragraph{Def}
			\begin{lst}
				\item $(A, +, \cdot)$ est un anneau commutatif si l'opération multiplicative $\cdot$ est associative ;
				\item $(A, +, \cdot)$ est un anneau unital s'$\exists 1 \in A \tq \forall a \in A, 1 \cdot a = a = a \cdot 1$.
			\end{lst}

			\paragraph{Exemples}
			\begin{lst}
				\item Mq $0 \cdot a = 0$.Par définition, $0 = 0 + 0$. Donc $\forall a \in A, 0 \cdot a = (0 + 0) \cdot a \iff 0a = 0a + 0a$.
				Par le lemme précédent, $x = x + x \Rightarrow x = e$ où $e$ est le neutre. Ici, le neutre est $0$. Donc $\forall a \in A, 0a = 0$.
				\item Mq dans un anneau unital $(A, +, \cdot)$, $\forall a \in A, (-1) \cdot a = -a$. Comme vu plus haut, $\forall a \in A, 0a = 0$.
				Donc $\forall a \in A, (1 + (-1)) \cdot a = 0 \iff 1 \cdot a + (-1) \cdot a = 0 \iff (-1) \cdot a = -a$.
			\end{lst}

			\paragraph{Proposition} Soit $k \in \Z_0 \setminus \{1\}$. $(\Z/k\Z, \overline{+}, \overline{\cdot})$ où $\overline{.}$ est défini par
			$\overline{.} : \Z/k\Z \times \Z/k\Z \to \Z/k\Z \tq \overline{l} \overline{\cdot} \overline{l'} := \overline{l \cdot l'}$ est un anneau commutatif.

			\paragraph{Dém} À nouveau, il faut montrer que $\overline{\cdot}$ est bien défini.
			Soient $l, l', \widetilde{l}, \widetilde{l}' \in \Z \tq \overline{l} = \overline{\widetilde{l}}, \overline{l'} = \overline{\widetilde{l'}}$.
			Alors, $l - \widetilde{l} = kz_1 \in k\Z$ et $l' - \widetilde{l}' = kz_2 \in k\Z$ avec $z_1, z_2 \in \Z$. Donc, par définition :

			\[\begin{aligned}
				\overline{\widetilde{l}} \overline{\cdot} \overline{\widetilde{l}'} := \overline{\widetilde{l} \cdot \widetilde{l}'}
				= \overline{(l - kz_1) \cdot (l' - kz_2)} &= \overline{l \cdot  l' - k \cdot (z_1 \cdot l' + z_2 \cdot l) + k \cdot k \cdot z_1 \cdot z_2} \\
				&= \overline{(l \cdot l') + k \cdot ((k \cdot z_1 \cdot z_2 - z_1 \cdot l - z_2 \cdot l') \in \Z)} = \overline{l \cdot l'}.
			\end{aligned}\]

			Les propriétés de commutativité découlent directement des propriétés des entiers $\in \Z$.

		\subsection{Interprétation des GCD, nombres premiers, nombres premiers entre eux}

			\paragraph{Def} Soit $(A, +, \cdot)$, un anneau unital.
			\begin{itemize}
				\item $a \in A$ est inversible s'$\exists b \in A \tq a \cdot b = 1 = b \cdot a$ ;
				\item $a \in A \neq 0$ est un diviseur de 0 s'$\exists b \in A \neq 0 \tq a \cdot b = 0$.
			\end{itemize}

			\paragraph{Exemple} Soient $0 < a \leq b < k \in \mathbb N_0 \tq a \cdot b = k$. Alors $\overline a$ et $\overline b$ sont des diviseurs
			de 0 dans $\Z/k\Z$.

			\paragraph{Proposition} Si $a \in A$ est un diviseur de 0, alors $a$ n'est pas inversible.

			\paragraph{Dém} Soit $a \in A$. Par définition de $a$, $\exists b \in A \neq 0 \tq a \cdot b = 0$.
			Supposons, par l'absurde, $a$ inversible. Donc $\exists c \in A \tq ac = 1 = ca$. De là, $b(ac) = b \iff (ba)c = b \iff 0c = b \iff b = 0$.
			Or $b$ doit être non nul, il y a contradiction. $a$ n'est pas inversible.

			\paragraph{Proposition} Soient $k \in \mathbb N_0 \setminus \{1\}$, $k \in \Z$. $\overline z$ est inversible dans $\Z/k\Z \iff GCD(k, z) = 1$.

			\paragraph{Dém} Montrons que si $k$ et $z$ sont premiers entre eux, alors $\overline z$ est inversible.

			$\overline{t} \overline{\cdot} \overline{z} = \overline{tz} = \overline{1-sk} \in\Z / k\Z = \overline 1$ car $sk$ est un multiple de $k$, avec $t, s \in \Z \tq
			sk + tl = 1$ (par Euclide). On a donc $\overline t$, l'inverse de $\overline z$.

			Montrons ensuite que si $\overline z$ est inversible, alors $k$ et $z$ sont premiers entre eux.

			$\overline z$ est inversible $\Rightarrow \exists t \in \Z \tq \overline t \overline \cdot \overline z = \overline 1 = \overline{tl}$. Donc $\exists t, s \in \Z
			\tq tl - 1 = sk$ avec $s \in \Z$. Ou encore $tl + (-s)k = 1$. Par le lemme de Bézout, $k$ et $z$ sont premiers entre eux.

			\paragraph{Remarque} Nous avons vu l'algorithme qui permet de déterminer $s, t \in\Z \tq sk + tl = GCD(k, l)$. Si $GCD(k, l) = 1$, alors $\overline t$ est
			l'inverse de $\overline z$ dans $\Z / k\Z$.

			\paragraph{Def} $(\mathbb K, +, \cdot)$ est un champ si $(\mathbb K, +,\cdot)$ est un anneau unital commutatif tel que $\forall k \neq 0 \in \mathbb K,
			\exists k^{-1}\tq kk^{-1} = 1$.

			\paragraph{Exemple} $\Z/4\Z$ possède des diviseurs de $0$ donc ce n'est pas un champ.

			\paragraph{Proposition} Soit $k \in \mathbb N \setminus \{1\}$. Alors $\Z/k\Z$ est un champ $\iff$ $k$ est un nombre premier.

			\paragraph{Dém} Corollaire de la proposition précédente.

			\paragraph{Remarque} $\forall p \in \Z \tq p$ est premier, $\Z/p\Z$ est un champ à $p$ éléments.

			\subsubsection{Relations de congruence}

			\paragraph{Def} Soient $a, b, k\in \Z \tq |k| > 1$. On dit que $a$ est \textit{congru à $b$ modulo $k$} et on note $a \equiv b \cmod k$ $\iff a-b \in k\Z \iff
			\overline a = \overline b$ dans $\Z/k\Z$.

			\paragraph{Propriétés}
				\begin{enumerate}
					\item La congruence modulo $k$ est une relation d'équivalence (réflexive, transitive et symétrique) ;
					\item $\forall a_1, b_1, a_2, b_2 \in \Z, |k| > 1$, si $a_1 \equiv a_2 \cmod k$ et $b_1 \equiv b_2 \cmod k$, alors :
						\begin{itemize}
							\item $a_1 + b_1 \equiv a_2 + b_2 \cmod k$ ;
							\item $a_1b_1 \equiv a_2b_2 \cmod k$.
						\end{itemize}
					En conséquence, $\forall c \in \Z, a_1c \equiv a_2c \cmod k$.
				\end{enumerate}

		\subsection{La cryptologie : le système RSA (Rivest, Shamir, Adlerman)}

			\paragraph{Lemme} $\forall n \in \mathbb N, (n+1)^p \equiv n^p + 1 \cmod p$ si $p$ est premier.

			\paragraph{Dém} Par le binôme de Newton,$(n+1)^p = \sum_{i=0}^p n^i\binom pi = n^p + 1 + \sum_{i=1}^{p-1}n^i\binom pi$.

			Montrons maintenant par récurrence sur $i$ que $p \, | \, \binom pi \forall 0 < i < p$.

			Quand $i = 1$, $\binom pi = p$, or $p \, | \, p$, donc ok.

			Supposons maintenant que $p \, | \, \binom pi$ et démontrons que donc $p \, | \, \binom p{i+1}$. $\binom p{i} = \frac {p!}{i!(p-i)!} \Rightarrow \binom p{i+1} =
			\frac {p!}{(i+1)!(p-i-1)!} = \frac {p!(p-i)}{i!(i+1)(p-i-1)!(p-i)} = \binom pi \frac {p-i}{i+1}$. Or $p \, | \, \binom pi$, donc $\exists  \in \mathbb N \tq
			\binom pi = pb$. De là, on a $\binom p{i+1} = pb\frac {p-i}{i+1}$. Et $\binom p{i+1} \in \mathbb N$ par définition. Donc comme $p$ est premier et $i < p$,
			on sait $(i+1) \, \not | \, p$ et donc qu'il faut que $(i+1) \, | \, b(p-i)$. Donc $\exists t \in \mathbb N \tq b(p-i) = t(i+1)$, ou encore $\binom p{i+1} = pt$.
			Et comme $p \, | \, pt$, on sait que $p \, | \, \binom p{i+1}$.

			Comme $p \, | \, \binom p{i+1} \Leftrightarrow \binom p{i+1} \equiv 0 \cmod p$. Donc $n^p + 1 + \sum_{i=1}^{p-1}\binom pi n^i \equiv n^p + 1 \cmod p$.

			\paragraph{Théorème (Petit théorème de Fermat)} Soit $p \in \mathbb N$ un nombre premier. Et soit $a \in \mathbb N \tq p \, \not | \, a$.
			Alors $a^p \equiv 1 \cmod p$.

			\paragraph{Dém} Montrons par récurrence sur $a$ que $a^p \equiv a \cmod p$.

			Pour $a = 1$, on a $a^p = 1^p \equiv 1 \cmod p$, donc ok.

			Supposons maintenant que $a^p \equiv a \cmod p$ et montrons que $(a+1)^p \equiv a+1 \cmod p$. Par le lemme précédent, nous avons
			$(a+1)^p \equiv a^p + 1 \cmod p$.
			Et par hypothèse de récurrence, nous avons $a^p \equiv a \cmod p$. Donc, en combinant les deux, nous obtenons $(a+1)^p \equiv (a+1) \cmod p$.

			Cela veut dire que dans $\Z/p\Z$, $\overline{a^p} = \overline a$. Mais comme $p \, \not | \, a$ et que $\Z/p\Z$ set un champ, $\exists \overline b$, un inverse
			de $\overline a$. Donc en multipliant par $\overline b$ de part et d'autre, on obtient $\overline b \overline{a^p} = \overline b \overline{a}^p
			= \overline b \overline a \overline a^{p-1} = \overline a^{p-1} = \overline b \overline a = \overline 1$. Autrement dit, $a^{p-1} \equiv 1 \cmod p$.

			\paragraph{Fonctionnement de RSA} deux personnes Alice et Bob veulent communiquer de manière sûre. Alice choisit deux nombres premiers $p, q \in \mathbb N$.
			Ce couple est appelé la \textit{clef privée}. Ensuite Alice calcule $N := pq$ et $\phi(N) := (p-1)(q-1) = N - (p-q+1)$. Alice choisit également $e \in \Z \tq
			GCD(\phi(N), e) = 1$. $e$ est appelé \textit{l'exposant de chiffrement}. Par le choix de $e$, $\exists 0 \leq s \leq \phi(N) \tq t\phi(N) + se = 1$.
			Attention, $s$ doit rester secret pour la sécurité du fonctionnement ! Ensuite, Alice publie le couple $(N, e)$.

			Si Bob souhaite envoyer un message sûr à Alice, il doit transformer son message en un entier $M \in \Z \tq 0 < M < N$. Il utilise ensuite la clef publique de
			pour publier à son tour $\widetilde M \equiv M^e \cmod N$.

			Pour déchiffrer ce message, Alice utilise $s$ (tel que $\overline s$ est l'inverse de $\overline e$ dans $\Z/\phi(N)\Z$) pour trouver
			$(\widetilde M)^e \equiv M^{es} \cmod N \equiv M \cmod N$, par le théorème suivant.

			\paragraph{Théorème} $\forall 0 < M < N = pq \in \mathbb N \tq p, q$ sont premiers, soit $u \equiv 1 \cmod{\phi(N)}$. Alors $M^u \equiv M \cmod N$.

			\paragraph{Dém} $(0 < M < N = pq) \Rightarrow (p \not | \, M) \lor (q \not | \, M)$. Soit $u \tq u \equiv 1 \cmod{\phi(N)}$, donc
			$u = 1 + t\phi(n) = 1 + t(p-1)(q-1)$.

				\subparagraph{cas 1 : $(p \not | \, M) \land (q \not | \, M)$} $M^u = M^{1 + t(p-1)(q-1)} = MM^{(p-1)(q-1)}$.
				Comme $p \not | \, M$, par le PTF\footnote{Petit Théorème de Fermat.}, $(M^{t(q-1)})^{(p-1)} \equiv 1 \cmod p$, et comme $q \not | \, M$, par le PTF,
				$(M^{t(p-1)})^{(q-1)} \equiv 1 \cmod q$.
				Donc $(M^u \equiv M \cmod p) \land (M^u \equiv M \cmod q) \Rightarrow (p \, | \, (M^u - M)) \land (q \, | \, (M^u - M)) \Rightarrow (pq \, | \, (M^u - M))$.
				Donc $\exists m \in \Z \tq M^u - M = mpq \iff M^u = M + mpq \equiv M \cmod{(pq=N)}$.

				\subparagraph{cas 2 : $(p \not | \, M) \land (q \, | \, M)$} $(q \not | \, M) \Rightarrow (M^{t(p-1)})^{(q-1)} \equiv 1 \cmod q \Rightarrow
				M^{t(p-1)(q-1)} = 1 + lq$, pour $l \in \Z$. Donc $M^u = M(1 + lq) = M + Mlq$, et comme $p \, | M$, on a $M^u = M + pclq$ pour $c \in \Z$. En réorganisant,
				on obtient $M^u = M + lcN$, ou encore $M^u \equiv M \cmod N$.

				\subparagraph{cas 3 : $(p \, | \, M) \land (q \not | \, M)$} Se prouve par symétrie du cas 2.

	\section{Combinatoire énumérative}

		\subsection{Comptage élémentaire}

			\subsubsection{Principes de base}

			\paragraph{Def} Une fonction $f : A \to B$ est dite :
			\begin{itemize}
				\item injective si $\forall a, a' \in A, f(a) = f(a') \iff a = a'$ ;
				\item surjective si $\forall b \in B, \exists a \in A \tq f(a) = b$ ;
				\item bijective si elle est injective et surjective.
			\end{itemize}

			\paragraph{Remarque}
			\begin{lst}
				\item si $\#B < \#A, \not \exists f : A \to B$ injective ;
				\item si $\#B > \#A, \not \exists f : A \to B$ surjective ;
				\item si $\#B \neq \#A, \not \exists f : A \to B$ bijective.
			\end{lst}

			\paragraph{Théorème} Soit $f : A \to B$. Alors $f$ est bijective $\iff \exists g : B \to A \tq ((g \circ f) = Id_A) \land ((f \circ g) = Id_B)$.

			\paragraph{Dém}
			\begin{lst}

				\item $(\Rightarrow)$ Si $f$ est bijective, elle admet un inverse $f^{-1} \tq (f \circ f^{-1}) = Id_B$ et $(f^{-1} \circ f) = Id_A$.

				\item $(\Leftarrow)$ S'$\exists $ une telle fonction $g$, prouvons que $f$ est bijective :

				\begin{itemize}
					\item $\forall b \in B, \forall a, a' \in A, f(a) = f(a') \Rightarrow g(f(a)) = g(f(a')) \Rightarrow (g \circ f)(a) = (g \circ f)(a') \Rightarrow a = a'$
					\item Posons $a = g(b)$. Donc $f(a) = f(g(b)) = (f \circ g)(b) = b$.
				\end{itemize}
			\end{lst}

			La fonction $f$ est alors bien injective et surjective (donc bijective).

			\paragraph{Def} Soient deux ensembles $A$ et $B$. On note $B^A$ l'ensemble des fonctions allant de $A$ dans $B$. $B^A := \{f : A \to B\}$.

			\paragraph{Remarque} Le nombre d'éléments de cet ensemble (donc le nombre de fonctions allant de $A$ dans $B$ est $\#(B^A) = (\#B)^{\#A}$.

			\subsubsection{Cardinalité}

			\paragraph{Def} Pour $n \in \mathbb N_0$, on définit $[n] := \{k \in \mathbb N_0 \tq k \leq n\}$. On pose $[0] = \emptyset$.

			\paragraph{Def} Deux ensembles $A$ et $B$ sont de même cardinalité s'$\exists f : A \to B$ bijective. On note alors $\#A = \#B$ ou $|A| = |B|$.

			\paragraph{Def} Un ensemble $E$ est fini s'il est de même cardinalité que $[n]$ pour $n \in \mathbb N$. On note alors $\#E = n$.

			\paragraph{Théorème} Soient $A, B$ deux ensembles finis de même cardinalité, $f : A \to B$. Les conditions suivantes sont équivalentes :

			\begin{itemize}
				\item[$(i)$]   $f$ est injective ;
				\item[$(ii)$]  $f$ est surjective ;
				\item[$(iii)$] $f$ est bijective.
			\end{itemize}

			\paragraph{Dém} Soient $A$ et $B$ deux ensembles de cardinalité $n$, et soit $f : A \to B$. Montrons d'abord que $(i) \Rightarrow (ii)$. Montrons ensuite que
			$(ii) \Rightarrow (i)$. Après avoir prouvé $(i) \iff (ii)$, il faut prouver que $(i) \land (ii) \iff (iii)$.

			\begin{itemize}
				\item Par définition, $f$ injective $\Rightarrow \forall a, a' \in A, f(a) = f(a') \iff a = a'$. Donc $\forall a' \neq a \in A, f(a) \neq f(a')$.
				Ce qui veut dire que le nombre d'éléments $b \in B$ n'admettant pas de préimage dans $A$ est $\#B - \#A$. Or $\#B = \#A$, donc
				$\forall b \in B, \exists a \in A \tq f(a) = b$, ce qui est la définition d'une fonction surjective.
				\item Par définition, $f$ surjective $\Rightarrow \forall b \in B, \exists a \in A \tq f(a) = b$. Donc le nombre d'éléments $b \in B$ admettant
				strictement plus d'une préimage $a \in A$ est $\#A - \#B$. Or $\#B = \#A$, donc $\not \exists b \in B \tq \exists a \neq a' \in A \tq f(a) = f(a') = b$.
				Autrement dit, $\forall a, a' \in A, f(a) = f(a') \iff a = a'$, ce qui est la définition d'une fonction injective.
				\item Par définition, $f$ est bijective $\iff$ $f$ est injective $\land$ $f$ est surjective.
			\end{itemize}

			\paragraph{Théorème (principe d'addition)} Soient $A_i, 0 < i \leq k$, des ensembles finis deux à deux disjoints. Alors $\#\left(\bigcup_{i=1}^kA_i\right) =
			\sum_{i=1}^k\#A_i$.

			\paragraph{Dém} Démontrons cela par récurrence sur $k$, le nombre d'ensembles.

			Pour $i = 1$, $\#A_1 = \#A_1$, donc ok.

			Supposons que $\#\cup_{i=1}^kA_i = \sum_{i=1}^k\#A_i$ et prouvons que $\#\cup_{i=1}^{k+1} = \sum_{i=1}^{k+1}$.

			$\cup_{i=1}^{k+1} = \cup_{i=1}^kA_i \cup A_k$. Donc $\#\cup_{i=1}^{k+1} = \#(\cup_{i=1}^kA_i) + \#A_k - \#((\cup_{i=1}^kA_i) \cap A_{k+1})$.
			Or, par définition, les $A_i$ sont disjoints deux à deux. Donc $((\cup_{i=1}^kA_i) \cap A_{k+1}) = \emptyset$. On a donc $\#(\cup_{i=1}^kA_i) + \#A_{k+1} - 0$.
			Et par hypothèse de récurrence, $\#(\cup_{i=1}^kA_i) = \sum_{i=1}^k\#A_i$. Donc $\#(\cup_{i=1}^kA_i) + \#A_{k+1} = \sum_{i=1}^k\#A_i + \#A_{k+1} =
			\sum_{i=1}^{k+1}\#A_i$.

			\paragraph{Def} Soient $A_i, 0 < i \leq k$, des ensembles. On définit leur produit cartésien par :

			\[\prod_{i=1}^kA_i := A_1 \times A_2 \times \ldots \times A_k = \{(a_i)_{i \in [k]} \tq a_i \in A_i\}.\]

			\paragraph{Théorème (principe de multiplication)} Soient $A_i, 0 < i \leq k$ des ensembles finis, pour $k \in \mathbb N_0$. Alors :
			$\#\left(\prod_{i=1}^kA_i\right) = \prod_{i=1}^k\#A_i$.

			\subsubsection{Factorielle}

			\paragraph{Def} La factorielle de $n \in \mathbb N_0$ est le nombre $n! := \prod_{i=0}^ni$. On pose $0! := 1$.

			\paragraph{Théorème} Soient $A$ et $B$ deux ensembles tels que $\#A = \#B = n$. Il existe $n!$ bijections $f : A \to B$.

			\paragraph{Dém} Soit $A = \{a_1, \ldots, a_n\}$. Pour construire une bijection $f : A \to B$, on choisit les images des éléments de $A$ par $f$. Pour $a_1$,
			on a $n$ choix ($\#B$). Pour $a_2$, on a $(n-1)$ choix ($\#(B \setminus \{a_1\})$). Pour $a_i$, le nombre de choix s'élève donc à $(n-i+1)$.
			Le nombre de bijections est donc $\prod_{i=1}^ni = n!$.

			\paragraph{Proposition} Soient $A$ et $B$ deux ensembles finis de cardinalité respective $k$ et $n$ avec $k < n$. Alors le nombre de fonctions $f : A \to B$
			injectives est $n(n-1)(n-2)\ldots(n-k+1) = \frac {n!}{(n-k)!}$.

			\paragraph{Dém} Soient $A = \{a_1, \ldots a_k\}$ et $B = \{b_1, \ldots a_n\}$ avec $k < n$. Pour construire une fonction $f : A \to B$, il faut
			choisir $\forall a \in A$ une image $b \in B$. Pour que la fonction soit injective, lors du choix de $a_1$, on a $n$ choix, mais lors du choix de $a_2$,
			on n'a plus que $(n-1)$ choix (pour garantir la propriété d'injection). Donc pour $a_k$, il restera $(n-k+1)$ choix. Le nombre total de fonctions injectives
			est donc $n(n-1)(n-2)\ldots(n-k+1) = \frac {n!}{(n-k)!}$.

			\paragraph{Remarque} Une injection $f : [k] \to [n]$ revient à choisir $k$ éléments parmi $n$ (en tenant compte de l'ordre).

			\paragraph{Exemple} Soit $E$ un ensemble. Le nombre d'ordres totaux sur $E$ est $(\#E)!$.

			\subsubsection{Croissance de $n!$}

			\paragraph{Remarque} En regardant la fonction $n \mapsto (n!)^2 = n(n-1)\ldots 1n(n-1)\ldots 1 = \prod_{k=1}^nk(n-k+1)$.
			Cette fonction est une fonction du second degré en $k$, donc une parabole. On sait donc déterminer son maximum (en $\frac {n+1}2$). De là, il est possible
			de borner le carré de la factorielle de la sorte :

			\[\begin{aligned}
				\forall k \in [n], n &\leq (n-k+1) &&\leq \left(\frac {n+1}2\right)^2 \\
								   n^n &\leq \prod_{k=1}^nk(n-k+1) &&\leq \left(\frac {n+1}2\right)^{2n} \\
								   n^n &\leq (n!)^2 &&\leq \left(\frac {n+1}2\right)^{2n} \\
								   n^{\frac n2} &\leq n! &&\leq \left(\frac {n+1}2\right)^n.
			\end{aligned}\]

			\subsubsection{Coefficients binomiaux}

			\paragraph{Def} Pour $n, k \in \mathbb N \tq k \leq n$, le coefficient binomial $\binom kn$ (se prononce $n$ choose $k$) est défini par
			$\binom kn := \frac {n!}{k!(n-k)!}$ et représente le nombre de sous-ensembles à $k$ éléments dans $[n]$ (l'ordre n'a pas d'importance).

			\paragraph{Propriété (symétrie)} $\forall k \leq n \in \mathbb N, \binom kn = \binom n{n-k}$.

			\paragraph{Dém} Par définition, $\binom kn$ est le nombre de sous-ensembles de $[n]$ à $k$ éléments.

			Soit $S \subseteq [n] \tq \#S = k$. Alors $T := [n] \setminus S \subseteq [n] \tq \#T = n-k$. Soit $A := \{S \subseteq [n] \tq \#S = k\},
			B := \{T \subseteq [n] \tq \#T = n-k\}$. On construit $f : A \to B : S \mapsto f(S)$, où $f(S) := [n] \setminus S$. En vérifiant que $f$ est bien une bijection,
			on a $\#A = \#B$, donc, par définition du coefficient binomial, $\binom nk = \binom n{n-k}$.

			\paragraph{Propriété (induction)} Pour $n, k \in \mathbb N_0, k < n, \binom kn = \binom {n-1}{k-1} + \binom{n-1}{k}$.

			\paragraph{Dém} Soient $S \subseteq [n]$, $e \in [n]$. Il y a donc deux cas possibles : soit $e \in S$, soit $e \not \in S$.

			On a :

			\[\begin{aligned}
				\{S \subseteq [n]\ \tq \#S = k\} &= \{S \subseteq [n] \tq (\#S = k) \land (e \in S)\} \cup \{S \subseteq [n] \tq (\#S = k) \land (e \not \in S)\} \\
												 &= \{T \subseteq [n] \setminus \{e\} \tq \#T = k-1\} \cup \{S \subseteq [n] \setminus \{e\} \tq \#S = k\}.
			\end{aligned}\]

			Si ces ensembles sont égaux, ils sont de même cardinalité. On a donc :

			\[\#\{S \subseteq [n]\ \tq \#S = k\} = \#(\{T \subseteq [n] \setminus \{e\} \tq \#T = k-1\} \cup \{S \subseteq [n] \setminus \{e\} \tq \#S = k\}).\]

			Comme les deux ensembles à droite de l'égalité sont d'intersection vide, on a :

			\[\begin{aligned}
				\#\{S \subseteq [n]\ \tq \#S = k\} &= \#\{T \subseteq [n] \setminus \{e\} \tq \#T = k-1\} &&+ \#\{S \subseteq [n] \setminus \{e\} \tq \#S = k\} \\
				\binom nk						  &= \binom {n-1}{k-1}								   &&+ \binom {n-1}k.
			\end{aligned}\]

			\paragraph{Théorème (absorption)} Soient $k, n \in \mathbb N_0, k \leq n$.

			\[k\binom nk = n\binom {n-1}{k-1}.\]

			\paragraph{Dém} Le membre de gauche est la cardinalité d'un ensemble $A := \{(S \subseteq [n], e) \tq \#S = k, e \in S\}$ et le membre de droite est
			la cardinalité d'un ensemble $B := \{(e, T) \tq e \in [n], T \subseteq [n] \setminus \{e\}, \#T = k-1\}$. Construisons $f : A \to B$, une bijection afin de prouver
			l'égalité des cardinaux. Soit $f : A \to B : (S, e) \mapsto (e, S \setminus \{e\})$. Montrons maintenant que $f$ est bijective :

			\begin{itemize}
				\item injectivité : montrons que $\forall S, S' \subseteq [n], e, e' \in [n], f(S, e) = f(S', e') \Rightarrow (S, e) = (S', e')$.
				Par définition de $f$, on sait $f(S, e) = (e, T := S \setminus \{e\})$ et $f(S', e') = (e', T' := S' \setminus \{e'\})$. Si $(e, T) = (e', T')$,
				alors il faut $e = e'$ (égalité membre à membre). Dans ce cas, on a $T = S \setminus \{e\} = S \setminus \{e'\} = T' = S' \setminus \{e'\}$.
				On a dès lors $S = S'$ en ajoutant $e$ dans les deux ensembles.

				\item surjectivité : montrons que $\forall (e, T) \in B, \exists (S, e') \in A \tq f(S, e') = (e, T)$.
				Soient $e \in [n]$ et $S \subseteq [n]$. Prenons $e' = e$ et $S = T \cup \{e\}$. On a dès lors $f(S, e') = (e', S \setminus \{e'\})
				= (e, S \setminus \{e\}) = (e, T)$.
			\end{itemize}

			Nous avons montré que $f$ est bijective, donc $\#A = \#B$.

			\paragraph{Théorème (somme parallèle)} Soient $m, k \in \mathbb N, k \leq n$.

			\[\sum_{n=k}^m\binom nk = \binom {m+1}{k+1}.\]

			\paragraph{Dém} Montrons ceci par récurrence sur $m$.

			\begin{itemize}
				\item $m = k$ : $\binom kk = 1 = \binom {k+1}{k+1}$, donc ok.
				\item Supposons la propriété vraie pour $m$ et montrons la pour $m+1$ :

				\[\sum_{n=k}^{m+1}\binom nk = \left(\sum_{n=k}^m\binom nk\right) + \binom {m+1}k.\]

				Par hypothèse de récurrence, on a :

				\[\sum_{n=k}^{m+1}\binom nk = \binom {m+1}{k+1} + \binom {m+1}{k} = \binom {m+2}{k+1}.\]
			\end{itemize}

			\paragraph{Théorème (binôme de Newton)} Soient $x, y \in \mathbb R, n \in \mathbb N$.

			\[(x+y)^n = \sum_{i=0}^n\binom nix^iy^{n-i}.\]

			\paragraph{Dém} $(x+y)^n = (x+y)(x+y)\ldots(x+y)$. Chaque terme $x^iy^{n-i}$ est donc un produit de $n$ éléments tel qu'il faut choisir $i$ fois l'élément $x$.
			Ce facteur $x^iy^{n-i}$ est donc choisi $\binom ni$ fois.

			\paragraph{Remarque} La preuve est également possible de manière plus algébrique par récurrence sur $n$.

			\paragraph{Applications/interprétations du binôme de Newton}

			\begin{enumerate}
				\item Le nombre de mots de $n$ bits contenant $k$ symboles 1 et $n-k$ symboles 0 est $\binom nk$ ;
				\item soit un réseau routier quadrillé de routes parallèles ou perpendiculaires deux à deux. Comme un chemin de $(0, 0)$ à $(a, b) \in \mathbb N^2$
				peut être vu comme un mot binaire où le symbole 1 veut dire \textit{haut} et le symbole 0 veut dire \textit{droite}, on peut facilement se ramener au
				cas précédent et dire que le nombre de plus courts chemins est $\binom {a+b}a = \binom {a+b}b$ ;
				\item le nombre de solutions $(x_1, \ldots, x_d) \in \mathbb N^d, d \geq 1$ de l'équation suivante :

				\[\sum_{i=1}^dx_i = s, s \in \mathbb N\]

				est $\binom {s+d-1}s$ car on peut à nouveau se ramener au premier cas en remplaçant les symboles + par le symbole 1 et il reste $s$ symboles 0
				à répartir dans les $d$ valeurs $x_i$. On cherche donc à faire un mot binaire avec $s$ symboles 0 et $d-1$ symboles 1.
			\end{enumerate}

			\subsubsection{Coefficients multinomiaux}

			\paragraph{Def} Pour $n, t \in \mathbb N, (k_i)_{i \in [t]} \tq \sum_{i=1}^tk_i=n$, on définit le le coefficient multinomial comme suit :

			\[\binom {n}{k_1, k_2, \ldots k_t} = \binom n{(k_i)_{i\in[t]}}.\]

			Ce coefficient est le nombre de partitions ordonnées de l'ensemble $[n]$ en $t$ sous-ensembles $S_1, S_2, \ldots, S_t \subseteq [n]$ tels que
			$\forall i \in [t], \#S_i = k_i$.

			\paragraph{Remarque} la notion d'ordre des partitions veut dire que les sous-ensembles sont nommés et qu'échanger deux noms en laissant le contenu
			intact est une autre partition que la partition initiale. Par exemple pour $[2]$, les partitions suivantes : $\{S_1 = \{1\}, S_2 = \{2\}\}$ et
			$\{S_1 = \{2\}, S_2 = \{1\}\}$ ne sont pas équivalentes.

			\paragraph{Remarque (bis)} Le coefficient multinomial $\binom n{n, n-k}$ correspond au coefficient binomial $\binom nk = \binom n{n-k}$.

			\paragraph{Interprétations du coefficient multinomial}

			\begin{itemize}
				\item Le nombre de répartitions possibles de $n$ objets dans $t$ sacs distinguables en mettant $k_i$ objets dans le sac $S_i \forall i \in [t]$ est donné
				par le coefficient multinomial $\binom n{(k_i)_{i\in[t]}}$ pour $n, t \in \mathbb N, k_1, \ldots, k_t \in \mathbb N$ ;
				\item le nombre de fonctions $f : [n] \to [t]$ telles que $\#f^{-1}(\{i\}) = k_i \forall i \in [t]$ est donné par le même coefficient multinomial ;
				\item le nombre de mots de longueur $n$ faits dans un alphabet $\Sigma = \{\sigma^{(1)}, \sigma^{(2)}, \ldots, \sigma^{(t)}\}$ tels que
				le nombre d'occurrences du symbole $\sigma^{(i)}$ est $k_i \forall i \in [t]$.
			\end{itemize}

			\paragraph{Def} Pour $n, t \in \mathbb N, (k_i)_{i\in[t]} \tq \sum_{i=1}^tk_i=n$, on définit le coefficients multinomial comme suit :

			\[\binom {n}{k_1, \ldots, k_t} = \binom {n}{(k_i)_{i\in[t]}} = \frac {n!}{\prod_{j=1}^t(k_j)!}.\]

			\paragraph{Propriétés} $\forall n, t \in \mathbb N, (k_i)_{i\in[t]} \tq \sum_{i=1}^tk_i=n$ :

			\begin{itemize}
				\item Soit $\pi : [t] \to [t]$ bijective (une permutation de $[t]$).

				\[\binom {n}{(k_i){i\in[t]}} = \binom {n}{(k_{\pi(i)})_{i\in[t]}}.\]

				\item

				\[\forall i \in [t], k_i \neq 0 \Rightarrow \binom {n}{(k_i)_{i\in[t]}} = \frac k{k_i} \binom {n-1}{k_1, k_2, \ldots, k_{i-1}, k_i-1, k_{i+1}, \ldots, k_t}.\]

				\item Si $\forall j \in [t], k_j \neq 0$, alors

				\[\binom n{k_1, \ldots, k_t} = \sum_{j=1}^t\binom{n-1}{(k_i)_{i\in[t] \setminus \{j\}}, k_j-1}.\]
			\end{itemize}

			\paragraph{Théorème} Soient $n, t \in \mathbb N, (x_i)_{i\in[t]} \in \mathbb R^t$. Alors

			\[\left(\sum_{i=1}^tx_i\right)^n
			= \sum_{\underset {\tq \sum_ik_i=n}{(k_1, \ldots, k_t) \in \mathbb N^t}}\left[\binom {n}{(k_i)_{i \in [t]}} \prod_{j=1}^tx_j^{k_j}\right].\]

		\subsection{Preuves bijectives}

			\paragraph{Intuition} Lorsque l'on a un ensemble fini $A$ d'objets mathématiques, et que l'on veut connaitre son cardinal, on peut trouver un autre
			ensemble que l'on sait dénombrer, puis créer une bijection $f : A \to B$ afin de conclure $\#A = \#B$.

			\subsubsection{Arbres étiquetés}

			\paragraph{Def} Un arbre étiqueté à $n$ sommets est un arbre à $n$ sommets dont les sommets sont numérotés de $1$ à $n$.

			\paragraph{Def} Un graphe orienté $\vec {\Gamma}= (V, E)$ est un ensemble de sommets $v \in V$ et d'\textit{arcs} $e \in E \subseteq V \times V$
			où un arc est une arête orientée (pour $a, b \in V$, $\lnot((a, b) \in E \Rightarrow (b, a) \in E)$).

			\paragraph{Théorème (de Cayley)} Le nombre d'arbres étiquetés à $n$ sommets est $n^{n-2}$.

			\paragraph{Dém} Soient $a_n$, le nombre d'arbres étiquetés à $n$ sommets, $\mathcal A_n := \{$arbres étiquetés à $n$ sommets avec deux sommets spéciaux notés
			$\bigcirc$ et $\square$ (possiblement les mêmes)$\} = \{\Gamma = ((V, E), \bigcirc, \square)\}$. Les sommets $\bigcirc$ et $\square$ sont appelés respectivement
			extrémité gauche et droite.

			On a de manière évidente $\#\mathcal A_N \geq a_n$ voire même $\#\mathcal A_n = n^2a_n$. En effet, chaque arbre étiqueté à $n$ sommet a $n^2$ possibilités pour
			placer $\bigcirc$ et $\square$. Maintenant montrons qu'il existe une bijection $f : [n]^{[n]} \to \mathcal A_n$. Ainsi, on aura $\#\mathcal A_n = n^n$ ou encore
			$a_n = n^{n-2}$. Il faut construire un arbre étiqueté à $n$ sommets dont deux spéciaux notés $\bigcirc$ et $\square$ $\in \mathcal A_n$ à partir d'une fonction
			$F \in [n]^{[n]}$.

			Construisons l'arbre en deux étapes  : d'abord construisons un graphe dirigé à partir de la fonction puis supprimons les cycles afin d'en faire un arbre.

			\subparagraph{Étape 1 :} On pose $\vec{\Gamma_F} = ([n], \{(i, F(i)) \tq i \in [n]\})$, un graphe ayant pour sommets les naturels $\in [n]$
			et pour arêtes les couples de la fonction $F$. Un tel graphe a les propriétés suivantes :

				\begin{lst}
					\item[$(i)$] $\vec {\Gamma_F}$ a exactement $n$ sommets ;
					\item[$(ii)$] $\forall v \in V, \exists! e \in E \tq e = (v, x)$ pour $x \in V$ ;
					\item[$(iii)$]  $\forall v \in V, \#\{(x, v) \in E \text{ pour $x \in V$}\} \leq n$ ;
					\item[$(iv)$] Chaque composante a exactement un cycle dirigé. En effet, chaque sommet a exactement un arc sortant donc il n'y a pas de cycle non dirigé.
								  De plus, $\vec {\Gamma_F}$ a $n$ sommets et $n$ arêtes, il ne peut donc être un arbre ou acyclique.
					\item[$(v)$] Si$\widetilde C$ est un cycle, alors $f_{\sVert[0]_{\{\text{sommets de $\widetilde C$}\}}}$ est une bijection.
				\end{lst}

			\subparagraph{Étape 2 : }On définit $\mathcal M := \{i_1, i_2, \ldots, i_t\}$ tel que $\forall j \in [m], i_j$ est un cycle dirigé dans $\vec {\Gamma_F}$ avec
			$i_1 < i_2 < \ldots < i_m$. La restriction $F_{\sVert[0]_{\mathcal M}}: \mathcal M \to \mathcal M$ est une bijection définie par :

			\[F_{\sVert[0]_{\mathcal M}}= \begin{pmatrix}i_1 & i_2 & \ldots & i_m \\f(i_1) & f(i_2) & \ldots & f(i_m)\end{pmatrix}.\]

			De là, on construit un graphe dirigé $\vec {\Gamma_F}$ :

				\begin{itemize}
					\item on enlève les arcs $\mathcal M \times F(\mathcal M)$ ;
					\item on construit un chemin d'arcs $\{(F(i_j), F(i_{j+1})) \tq i_j \in [\#\mathcal M - 1]\}$ (cela permet de réduire de 1 le nombre
						  d'arêtes et de connecter le graphe) ;
					\item on marque $F(i_1)$ par $\bigcirc$ et $F(i_m)$ par $\square$ ;
						\item on supprime l'orientation des arcs.
				\end{itemize}

			$\vec {\Gamma_F}$ a donc $n-1$ arêtes, $n$ sommets et est simple (un arbre). Il est étiqueté (par la fonction $F$) et a 2 sommets particuliers
			($\bigcirc$ et $\square$).

			Les deux étapes qui viennent d'être décrites donnent donc une fonction $f : [n]^{[n]} \to \mathcal A_n$. Il reste à montrer que cette fonction est bijective.
			Pour cela, construisons $g : \mathcal A_n \to [n]^{[n]}$ telle que $(f \circ g) = Id_{\mathcal A_n}$ et $(g \circ f) = Id_{[n]^{[n]}}$.

			\underline{construction de la fonction $g$ : } On prend $C$, l'unique chemin entre $\square$ et $\bigcirc$.On en construit la matrice $\Delta$ de dimension
			$2 \times m$ où la première ligne est la permutation ordonnée de $C$ et la seconde est $C$ dans l'ordre de $\bigcirc$ à $\square$. On enlève tous les arcs de $C$.
			On crée des arcs allant de $\Delta_{1j}$ à $\Delta_{2j} \forall j$. Il reste à orienter les arêtes restantes en direction des cycles. On a alors $\vec {\Gamma_F}$.
			Définissons donc maintenant $G : [n] \to [n] : \Delta_{1j} \mapsto \Delta_{2j} \forall j$.

			On a donc bien construit $g : \mathcal A_n \to [n]^{[n]}$ respectant les bonnes propriétés d'identité. On en déduit $\#\mathcal A_n = \#[n]^{[n]} = n^n$,
			ou encore $a_n = n^{n-2}$.

			\subsubsection{Arbres binaires enracinés} %TODO

			\paragraph{Def} Un arbre binaire enraciné est un arbre dont tous les sommets sont de degré 1, 2 ou 3 tel qu'un et un seul sommet soit de degré 2, chaque arête a un
			label $l \in \{G, D\}$ et on peut représenter un tel arbre dans le plan en rangeant les sommets par étage :

			\begin{itemize}
				\item étage 1 : le sommet de degré 2 appelé racine ;
				\item étage $k > 1$ :
					\begin{lst}
						\item un sommet n'est adjacent qu'à un et un seul sommet de l'étage $(k-1)$ ;
						\item un sommet est adjacent à 0 ou 2 sommets de l'étage $(k+1)$ ;
						\item il n'y a pas d'autre adjacence.
					\end{lst}
			\end{itemize}

			\paragraph{Def} Les sommets de degré 1 sont appelés feuilles et les sommets de degré 3 sont appelés sommets internes.

			\paragraph{Def} Une triangulation d'un polygone à $(n+1)$ côtés est une découpe du polygone en triangles dont les sommets sont les sommets du polygone.

			\paragraph{Théorème} Le nombre d'arbres binaires enracinés à $n$ sommets est égal au nombre de triangulations d'un polygone à $(n+1)$ côtés.

			\paragraph{Dém} Soient $\mathcal A_n := \{$ arbres binaires enracinés à $n$ feuilles $\}$, $\mathcal T_n := \{$ triangulation d'un polygone à $(n+1)$ côtés $\}$.
			Construisons $f : \mathcal T_n \to \mathcal A_n$ bijective.

			\begin{itemize}
				\item Nommons les côtés de $\tau \in \mathcal T_n$ par $c_0$ (sur le côté supérieur), $c_1, \ldots, c_n$ dans le sens anti-horlogique ;
				\item plaçons la racine de $a \in \mathcal A_n$ dans le triangle adjacent à $c_0$ ;
				\item plaçons un sommet interne ($\deg v = 3$) dans chaque autre triangle ;
				\item plaçons les feuilles sur les côtés $c_1, \ldots, c_n$ ($(n+1)$ faces et $n$ feuilles) ;
				\item les arêtes vont relier les sommets placés dans des triangles adjacents. Et les arêtes vont relier les sommets internes aux feuilles dans un même triangle.
			\end{itemize}

			Reste à montrer que $f$ est bijective.  % TODO

			\paragraph{Remarque} $\mathcal T_n = \frac 1n\binom {2(n-1)}{n-1}$ (démonstration plus tard).

		\subsection{Relations de récurrence}

			\subsubsection{Le tri fusion}

			\paragraph{Intro} Le tri fusion applique la notion de \textit{Divide \& conquer}, à savoir diviser le set de données à traiter afin de traiter
			séparément des cas plus petits et unifier la solution à la fin.

			\paragraph{Rappel} Le principe du tri fusion (\textit{merge sort}) est de diviser le vecteur à trier en deux, trier les deux sous-vecteur et
			unifier les deux sous-veceturs de manière à avoir l'union des deux triées. Pour trier chaque sous-vecteur, le principe est récursif jusqu'à obtenir
			un sous-vecteur déjà trié (de taille unitaire).

			\paragraph{Proposition} Le nombre de copies exécutées lors d'un merge sort sur un vecteur de taille $N = 2^n$ avec $m \in \mathbb N$ est $N\log_2N$.

			\paragraph{Dém} Comptons $C_N$, le nombre de copies effectuées. On sait que si le vecteur est de taille unitaire, il est déjà trié, il n'y a donc aucune copie.
			De plus, lors de la récursion, les sous-vecteurs sont de taille respective $\left \lceil \frac N2 \right \rceil$ et $\left \lfloor \frac N2 \right \rfloor$.
			Cependant comme $N = 2^n$, on sait que $2 \, | \, N$. Dès lors, $\left \lceil \frac N2 \right \rceil = \frac N2 = \left \lfloor \frac N2 \right \rfloor$.
			On trouve donc :

			\[
			  \left\{
				\begin{aligned}
				  C_N &= 2C_{\frac N2} + N \;\;\; \forall n \geq 2 \\
				  C_1 &= 0
				\end{aligned}
			  \right.
			\]

			Si on pose $a_n := \frac {C_{2^n}}{2^n}$, on obtient :

			\[
			  \left\{
				\begin{aligned}
				  a_n &= a_{n-1} + 1 \\
				  a_0 &= 0
				\end{aligned}
			  \right.
			\]

			Cette relation de récurrence peut être simplifiée en $a_n = n$, ou encore $C_{2^n} = 2^nn$. Et comme $N = 2^n$, on a $C_N = N\log_2N$.

			\paragraph{Prop} Le nombre de copies effectuées par le tri fusion pour trier un vecteur $V$ de taille $2^n$ est $C_N = N\log_2N$.

			\paragraph{Def} Un ensemble de droites du plan est en position générale si toute paire de droites s'intersecte en exactement un point et tout triplet de droites
			a une intersection vide.

			\paragraph{Def} $\Phi_2(n) : \mathbb N \to \mathbb N$ est le nombre de régions du plan délimitées par un ensemble de $n$ droites en position générale.

			\paragraph{Exemple}


				\begin{center}
					\begin{tabular}{c | c}
						$n$ & $\Phi_2(n)$ \\ \hline
						$0$ & $1$ \\
						$1$ & $2$ \\
						$2$ & $4$ \\
						$3$ & $7$ \\
						$4$ & $11$
					\end{tabular}
				\end{center}

			On "devine" $\Phi_2(n) = \Phi(n-1) + n$.

			\paragraph{Remarque} Montrons que $\Phi_2(n)$ ne dépend pas de la position des $n$ droites.

			\paragraph{Théorème} $\forall n \in \mathbb N$, le nombre de régions du plan délimitées par $n$ droites en position générale ne dépend pas du choix des droites.
			$\Phi_2(n)$ est donc bien défini. De plus $\forall n \in \mathbb N$ :

			\[\left\{\begin{aligned}
				\Phi_2(n) &= \Phi_2(n-1)+n \;\;\;\;\text{ si $n \geq 1$}, \\
				\Phi_2(n) &= 1.
			\end{aligned}\right.\]

			\paragraph{Dém} Montrons que $\Phi_2(n)$ est bien définie :

			\begin{itemize}
				\item $\Phi_2(0) = 1$ est bien défini car sans droite, le plan n'est pas séparé ;
				\item $\Phi_2(1) = 2$ est bien défini car une droite coupe le plan en deux ;
				\item Supposons $n \geq 2$ et que $\Phi_2(n-1)$ est bien défini et montrons que $\Phi_2(n)$ est bien définie.

					  Soient $D_1, D_2, \ldots, D_{n-1}$ $(n-1)$ droites dans le plan en position générale. Le plan est donc subdivisé en $\Phi_2(n-1)$ régions (par hypothèse).
					  Soit $D_n$ une droite du plan telle que $D_1, \ldots, D_n$ sont en position générale. La droite $D_n$ admet donc $(n-1)$ points d'intersection
					  et est donc séparée en $n$ intervalles. Ces intervalles "coupent" chacun une région en 2 parties. Les $n$ droites $D_1, \ldots, D_n$ séparent
					  donc le plan en $\Phi_2(n-1) + n$ régions.
			\end{itemize}

			\paragraph{Remarque} $\Phi_2(n) = \Phi_2(n-1) + n = \Phi_2(n-2) + (n-1) + n = \Phi_2(0) + 1 + 2 + \ldots + n = 1 + \sum_{k=1}^nk = 1 + \frac {n(n+1)}2 =
			1 +\binom {n+1}2 = 1 + \binom n1 + \binom n2 = \sum_{k=0}^2\binom nk$.

		\subsection{Récurrences linéaires}
			\subsubsection{Récurrence linéaire de premier ordre}

			\paragraph{Def}
			\begin{lst}
				\item Une récurrence linéaire de premier ordre est une récurrence de la forme :

				\[\left\{\begin{aligned}
					x_n &= c_n x_{n-1} + d_n \;\;\;\text{ si $x \geq 1$} \\
					x_0 &= 0
				\end{aligned}\right.\;\;\;\;(*).\]

				\item Une solution d'une telle récurrence linéaire est une suite $(x_n)_{n \in \mathbb N}$.
			\end{lst}

			\paragraph{Théorème} La récurrence $(*)$ a pour solution explicite la suite $(x_n)_{n \in \mathbb N}$ où $x_n$ est définie ainsi :

			\[x_n = \sum_{i=1}^nd_i\prod_{j=i+1}^nc_j = d_1c_2c_3 \ldots c_n + d_2c_3c_4 \ldots c_n + \ldots + d_{n-1}c_n + d_n.\]

			\paragraph{Dém} Prouvons-le par récurrence sur $n$.

			\begin{itemize}
				\item $n  =  1$  : $x_1 = c_n x_0 + d_1$, ok.
				\item $n \geq 2$ : supposons que ce soit vrai pour $x_{n-1}$ et démontrons pour $x_n$ :

					  \[\begin{aligned}
						x_n &= c_nx_{n-1} + d_n \\
							&= c_n \left[\sum_{i=1}^{n-1}d_i\prod_{j=i+1}^{n-1}c_j\right] + d_n \\
							&= \left[\sum_{i=1}^{n-1}d_ic_n\prod_{j=i+1}^{n-1}c_j\right] + d_n \\
							&= \left[\sum_{i=1}^{n-1}d_i\prod_{j=i+1}^nc_j\right] + d_n \\
							&= \sum_{i=1}^nd_i\prod_{j=i+1}^nc_j.\;\;\;\;\;\;\square
					  \end{aligned}\]
			\end{itemize}

			\subsubsection{Récurrences linéaires à coefficients constants}

			\paragraph{Exemple} De combien de manières différentes peut-on paver un rectangle de dimensions $2 \times n$ avec des dominos (non-numérotés) ?

			\paragraph{Def} Notons $P_n := $ le nombre de tels pavages d'un rectangle $2 \times n$. Dès lors :

			\[\left\{\begin{aligned}
				P_n &= P_{n-1} + P_{n-2} \;\;\;\;\; \text{ si $n \geq 3$}, \\
				P_1 &= 1, P_2 = 2.
			\end{aligned}\right.\]

			On trouve :

			\begin{center}\begin{tabular}{c|c}
				$n$ & $P_n$ \\
				\hline \\
				$0$ & non défini \\
				$1$ & $1$ \\
				$2$ & $2$ \\
				$3$ & $3$ \\
				$4$ & $5$ \\
				$5$ & $8$
			\end{tabular}\end{center}

			\paragraph{Def} La suite de Fibonacci est l'unique suite $(F_n)_{n \in \mathbb N}$ telle que :

			\[\left\{\begin{aligned}
				F_n &= F_{n-1} + F_{n-2} \;\;\;\;\text{ si $n \geq 3$}, \\
				F_0 &= 0, F_1 = 1.
			\end{aligned}\right.\]

			\paragraph{Def}

			\begin{lst}
				\item Une récurrence linéaire homogène à coefficient constants est un système d'équations de la forme :

					  \begin{equation}\label{eq:RLCCH}
					  	x_n = c_{d-1}x_{n-1} + c_{d-2}x_{d-2} + \ldots + c_0x_{n-d}
					  \end{equation}

					  où $c_i \in \mathbb C\;\;\forall i \in \{0, 1, \ldots, d-1\}$ ;
				\item si $c_0 \neq 0$, l'ordre de récurrence est $d \in \mathbb N$ ;
				\item une solution est une suite $(x_n) \in \mathbb C^{\mathbb N}$ qui satisfait l'équation $(1)$.
			\end{lst}

			\paragraph{Rappel} $\mathbb C^{\mathbb N}$ peut être muni d'une structure d'espace vectoriel :
			\begin{itemize}
				\item addition : $\forall (x_n), (y_n) \in \mathbb C^{\mathbb N}, (x_n) + (y_n) = (x_n + y_n)$ ;
				\item multiplication : $\forall (x_n) \in \mathbb C^{\mathbb N}, \lambda \in \mathbb C, \lambda(x_n) = (\lambda x_n)$.
			\end{itemize}

			\paragraph{Remarque} $\dim \mathbb C^{\mathbb N} = +\infty$.

			\paragraph{Def} Le polynôme caractéristique de la récurrence linéaire homogène à coefficients constants $(1)$ est un polynôme $P(t)$ de degré $d$ est défini par :

			\[P(t) = -t^d + c_{d-1}t^{d-1} = c_{d-2}t^{d-2} + \ldots + c_0 = \sum_{i=0}^dc_it^i \text{ avec $c_d = -1$}.\]

			\paragraph{Rappel} On cherche, pour Fibonacci, des solutions de la forme $F_n = \beta$ pour $\beta \in \mathbb C$. L'équation
			$F_n = F_{n-1} + F_{n-2}$ devient donc $\beta^n = \beta^{n-1} + \beta^{n-2} \iff \beta^2 = \beta + 1 \iff \beta^2 - \beta - 1 = 0$.
			Cette équation du second degré est de discriminant $\Delta = 5$ et a donc pour solutions $\beta_1 = \frac {1 + \sqrt 5}2$ et $\beta_2 = \frac {1 - \sqrt 5}2$.
			Par construction, on sait donc que les suites $(\varphi^n)$ et $(\bar\varphi^n)$ sont des solutions de la RLCC des nombres de Fibonacci où $\varphi = \frac {1 + \sqrt 5}2$ est
			le nombre d'or et où $\bar\varphi = \frac {1-\sqrt 5}2$ est son conjugué.

			\paragraph{Remarque} Si $(\varphi)^n$ et $(\bar\varphi^n)$ sont des solutions de la RLCC de Fibonacci, alors $\forall \lambda, \mu \in \mathbb C, (\lambda\varphi^n + \mu\bar\varphi^n)$ est
			également une solution.

			\paragraph{Théorème} Notons $W$ l'ensemble des solutions de la RLCC (avec $c_0 \neq 0$). Alors :

			\begin{itemize}
				\item[$(i)$] $W$ est un sous-espace vectoriel de $\mathbb C^{\mathbb N}$ ;
				\item[$(ii)$] $\dim W = d$.
			\end{itemize}

			\paragraph{Dém} Pour montrer le point $(i)$, il faut montrer que la somme et le produit par un scalaire sont stables et que $(0) \in W$.
			\begin{itemize}
				\item $(0) \in W$ est trivial (en insérant $0$ pour tous les $c_i$ dans la définition de la RLCC, on voit que $(0) \in W$) ;
				\item $\lambda(x_n) + \mu(y_n) \coloneqq (\lambda x_n + \mu y_n)$. Donc soit $n \geq d$. On a :

				\[\lambda x_n + \mu y_n = \lambda (c_{d-1}x_{n-1} + \ldots + c_0x_{n-d}) + \mu (c_{n-1}y_{n-1} + \ldots + c_0y_{n-d}) = \sum_{i=1}^dc_{d-i}(\lambda x_{n-i} + \mu y_{n-i}).\]

				Or $(\lambda x_n + \mu y_n)$ est une racine du polynôme caractéristique et est onc une solution de $(1)$. Dès lors, $(\lambda x_n + \mu y_n) \in W$
			\end{itemize}

			Pour montrer le point $(ii)$, il faut trouver une bijection linéaire $f : \mathbb C^d \to W$ pour montrer que $W$ et $\mathbb C^d$ sont isomorphes.

			Soit $f : W \to \mathbb C^d : (x_n)_n \mapsto(x_i)_{i\in[d-1]}$. Montrons que $f$ est un isomorphisme :

			\begin{itemize}
				\item montrons que $f$ est linéaire :

				\[f(\lambda (x_n) + \mu (y_n)) = f((\lambda x_n + \mu y_n)_n) = (\lambda x_i + \mu y_i)_{i\in[d-1]} = \lambda (x_i)_{i\in[d-1]} + \mu (y_i)_{i\in[d-1]} = \lambda f((x_n)) + \mu f((y_n)).\]

				\item montrons que $f$ est bijective :

				\begin{itemize}
					\item surjectif : $\forall (x_0, \ldots, x_{d-1}) \in \mathbb C^d : \exists (\tilde x_n) \in W \tq f((x_n)) = (x_0, \ldots, x_{d-1})$.
						  Construisons $(\tilde x_n)$ comme suit :

						  \[\tilde x_i = \left\{\begin{aligned}&x_i &\text{ si $i < d$}\\&\sum_{i=j}^dc_{n-j}x_{i-j} &\text{ si $i \geq d$}\end{aligned}\right.\]

					\item injectif : $\forall (x_n), (y_n) \in W : f((x_n)) = f((y_n)) \Rightarrow (x_n) = (y_n)$. On sait que deux suites $(x_n), (y_n)$ sont égales si et seulement si
						  $x_i = y_i \forall i$. Ici, $f((x_n)) = f((y_n)) \iff x_i = y_i \forall i \in [d-1]$ car les termes suivants sont construits sur base des précédents. Dès lors :

						  \begin{align*}
						  	x_d &= c_{d-1}x_{d-1} + c_{d-2}x_{d-2} + \ldots + c_0x_0 = c_{d-1}y_{d-1} + c_{d-2}y_{d-2} + \ldots + c_0y_0 = y_d \\
							x_t &= c_{d-1}x_{t-1} + c_{d-2}x_{t-2} + \ldots + c_0x_{t-d} = c_{d-1}y_{t-1} + c_{d-2}y_{t-2} + \ldots + c_0y_{t-d} = y_t.
						  \end{align*}
				\end{itemize}
			\end{itemize}

			\begin{flushright}$\square$\end{flushright}

			\paragraph{Def} Soient $P(t)$ un polynôme à coefficients complexes, $a \in \mathbb C$ une racine de $P$ est de multiplicité $m(a)$ si $P(t)$ est divisible par
			$(t-a)^{m(a)}$ mais pas par $(t-a)^{m(a)+1}$.

			\paragraph{Théorème} Construisons la RLCC homogène d'ordre $d > 1$ :

			\[-x_n + c_{d-1}x_{n-1} + \ldots + c_0x_{n-d} = 0 \;\;\;\;\;\forall n \geq d\]

			où $c_i \in \mathbb C \; \forall i \in [d-1]$ et $c_0 \neq 0$. Notons $P(t)$ son polynôme caractéristique. Toute solution de cette RLCC est une combinaison linéaire
			des $d$ suites de la forme $(n^j\beta^n)$ où $\beta$ est une racine de $P(t)$ et $j \in [m(\beta)-1]$. Soient $\beta_1, \beta_2, \ldots, \beta_k \in \mathbb C$ les racines
			de $P(t)$ deux à deux distinctes de multiplicité respective $m(\beta_1), m(\beta_2), \ldots, m(\beta_k)$. C'est-à-dire :

			\[P(t) = -\prod_{i=1}^k(t-\beta_i)^{m(\beta_k)}.\]

			Donc toute solution $(y_n)$ s'écrit :

			\[y_n = \sum_{\delta=1}^k\sum_{j_\delta=0}^{m(\beta_\delta)-1}\lambda_{j_\delta}n^{j_\delta}\beta^n.\]

			où les $\lambda_{j_i}$ sont des constantes ne dépendant pas de $n$.

			\paragraph{Exemple}

			\begin{enumerate}
				\item $F_n = F_{n-1} + F_{n-2} \; \forall n \geq 2$. Donc $P(t) = -t^2 + t + 1$. Les racines sont $\varphi$ et $\bar\varphi$. Dès lors, les solutions de la RLCC sont
					  $(\varphi^n)$ et $(\bar\varphi^n)$. L'ensemble des solutions est donc $\{(\lambda \varphi^n + \mu\bar\varphi^n) : \lambda, \mu \in \mathbb C\}$.
					  Pour retomber sur les conditions initiales ($F_0 = 0, F_1 = 1$), on a :

					  \[\left\{\begin{aligned}\lambda+\mu &= 0\\\lambda \varphi + \mu\bar\varphi &= 1\end{aligned}\right.\]

					  La solution est donc $(\lambda, \mu) = (\frac {\sqrt 5}5, -\frac {\sqrt 5}5)$. Dès lors, on peut déterminer :

					  \[F_n = \frac {\sqrt 5}5(\varphi^n-\bar\varphi^n).\]
				\item Soit la relation suivante : $x_n = -(x_{n-1} + x_{n-2})$ où $x_0 = 0$ et $x_1 = 1$. Le polynôme caractéristique est $P(t) = t^2 + t + 1 $ et admet pour racines :
					  $\frac {-1 \pm \sqrt 3i}2$. Dès lors les solutions sont sous la forme :

					  \[\left(\lambda \left(\frac {-1+\sqrt 3i}2\right)^n + \mu \left(\frac {-1-\sqrt 3i}2\right)^n\right).\]

					  Afin de retomber sur les conditions initiales, il faut $(\lambda, \mu) = \left(\frac {-i}{\sqrt 3}, \frac i{\sqrt 3}\right)$. La solution à cette RLCC est donc :

					  \[\left(\frac i{\sqrt 3}\left(\left(\frac {-1-\sqrt 3i}2\right)^n - \left(\frac {-1+\sqrt 3i}2\right)^n\right)\right)_n.\]

				\item $x_n = 2x_{n-1} - x_{n-2} \; \forall n \geq 2$. Le polynôme caractéristique est donc $P(t) = t^2 - 2t + 1 = (t-1)^2)$. Dès lors les solutions sont sous la forme
					  $(x_n)$ où $x_n = \lambda 1^n + \mu(n \cdot 1^n)$ avec $\lambda, \mu \in \mathbb C$.
			\end{enumerate}

			\paragraph{Résolution générale d'une RLCC} Une RLCC non-homogène st un système d'équation sous la forme :

			\begin{equation}\label{eq:RLCCNH}
				-x_n + c_{d-1}x_{n-1} + c_{d-2}x_{n-2} + \ldots + c_1x_{n-d+1} + c_0x_{n-d} = a_n
			\end{equation}

			où $(a_n) \in \mathbb C^{\mathbb N}$. Et se résout de la manière suivante :
			\begin{enumerate}
				\item déterminer $S^{EHA}$, l'ensemble des solutions de l'équation homogène associée \eqref{RLCCH};
				\item trouver une solution particulière $(x^{SP}_n) \in \mathbb C^{\mathbb N}$ de la RLCC non-homogène \eqref{eq:RLCCNH} ;
				\item exprimer l'ensemble des solutions de la RLCC non-homogène :

					  \[S \coloneqq S^{EHA} + ((x^{SP}_n)).\]
			\end{enumerate}

			Donc toute suite $(z_n)$ est une solution de la RLCC non homogène si elle s'écrit demanière unique comme :

			\[(z_n) = (y_n) + (x^{SP}_n) \;\text{ où } (y_n) \in S^{EHA}.\]

			\subsubsection{Récurrences \textit{Divide and Conquer}}

			\paragraph{Exemple} Le tri fusion : $C_N = C_\floor{\frac N2} + C_\ceil{\frac N2} + N$.

			\underline{Recherche binaire}

			\paragraph{Objectif} Localiser $x$ dans un vecteur trié de taille $N$.

			\paragraph{Algorithme} Comparer $x$ au $\ceil{\frac N2}$ élément du vecteur (appelé \textit{pivot}). Si $x = $ pivot, ok. Si $x < $ pivot, l'algo s'appelle
			récursivement sur le sous-vecteur des éléments 1 à $\ceil {\frac N2}-1$. Si $x > $ pivot, pareil mais sur $\ceil{\frac N2}+1$ à $N$.

			\paragraph{Théorème} Le nombre de comparaisons , dans le pire des cas, effectuées par par la recherche binaire dans un vecteur de taille $N$ est $B_N$, solution de :

			\[\left\{\begin{aligned}B_N &= B_\floor{\frac N2} + 1\\B_1 &= 1\end{aligned}\right.\]

			On a $B_N = \floor{\log_2 N} + 1$, qui représente le nombre de bits nécessaires à l'encodage binaire du nombre $N$.

			\paragraph{Dém} Soit $\widetilde B_N$, le nombre de bits dans la représentation binaire de $N$. Montrons que $\widetilde B_N = \widetilde B_\floor{\frac N2} + 1$. On sait :

			\[N = \sum_{i=0}^{\widetilde B_N-1}a_i2^i \; \text{ où } a_i \in \{0, 1\}, a_{\widetilde B_N} = 1.\]

			De plus, on sait que :

			\[\floor{\frac N2} = \floor{\frac 12\sum_{i=1}^{\widetilde B_N-1}a_i2^i} = \floor{\sum_{i=1}^{\widetilde B_N-1}a_i2^{i-1} + \frac {a_0}2} = \sum_{i=0}^{\widetilde B_N-2}a_{i+1}2^i.\]

			Donc $\floor{\frac N2}$ nécessite autant de bits que $\sum_{i=0}^{\widetilde B_N-2}a_{i+1}2^i$. Le nombre de bits dans le premier est $\widetilde B_\floor {\frac N2}$
			et le nombre de bits du second est $B_N - 1$. Comme le nombre 1 s'écrit de la même manière en toute base, on sait $\widetilde B_1 = 1$.
			Le nombre de bits nécessaires à l'encodage binaire du nombre $N$ est bien $\widetilde B_N$, une solution à la relation de récurrence précédente.

			Montrons maintenant que $\widetilde B_N = \floor{\log_2 N} + 1$. Soit $N = 2^b + a_12^{b-1} + \ldots + a_b2^0$. On voit que $\widetilde B_N = b + 1$. De plus,
			on sait que $\frac N{2^b} = 1 + \frac {a_1}2 + \ldots + \frac {a_b}{2^b}$. Si $1 < \frac N{2^b} < 2$, alors $0 < \log_2 N - b < 1$, ou encore $b < \log_2 N < b+1$.
			De là, $\floor{\log_2 N} = b$. Dès lors, $\widetilde B_N = b + 1 = \floor{\log_2 N} + 1$. $\;\;\;\square$

			\underline{Tri fusion}

			\paragraph{Def} Soit un vecteur de taille $N$. On définit $C_N$ par le nombre de copies effectuée par le tri fusion sur ce vecteur. $C_N$ est une solution de la relation suivante :

			\[\left\{\begin{aligned}C_1 &= 0\\C_N &= C_\floor{\frac N2} + C_\ceil{\frac N2} + N\end{aligned}\right.\]

			\paragraph{Proposition} On peut déterminer $C_N = (N-1) + \sum_{i=1}^{N-1}\widetilde B_i$.

			\paragraph{Dém} Soient $C_N, C_{N+1}$. On pose $D_N \coloneqq C_{N+1} - C_N$. On a donc :

			\begin{align*}
				D_N &= C_{N+1} - C_N = \left(C_\ceil{\frac {N+1}2} + C_\floor{\frac {N+1}2} + (N+1)\right) - \left(C_\ceil{\frac N2} + C_\floor{\frac N2} + N\right) \\
					&= C_\ceil{\frac {N+1}2} - C_\ceil{\frac N2} + C_\floor{\frac {N+1}2} - C_\floor{\frac N2} + 1 = C_\ceil{\frac {N+1}2} - C_\floor{\frac N2} + 1 \\
					&= C_{\floor{\frac N2}+1} - C_\floor{\frac N2} + 1 = D_\floor{\frac N2} + 1.
			\end{align*}

			Avec $D_1 = C_2 - C_1 = 2 - 0 = 2$. En posant $n \coloneqq \floor{\log_2 N}$, on peut définir $\alpha_n \coloneqq D_N$. De là, on sait que $\alpha_n = \alpha_{n-1} + 1$ qui peut
			se simplifier en $\alpha_n = n + k$ où $k = 2$ selon les conditions initiales. On en déduit $D_N = \alpha_n = n + 2 = \floor{\log_2 N} + 2$. Maintenant, on sait :

			\begin{align*}
				C_N &= C_N - 0 = C_N - C_1 = C_N + \sum_{i=2}^{N-1}(-C_i+C_i) - C_1 = \sum_{i=2}^{N}(C_i-C_{i-1}) = \sum_{i=2}^{N}D_{i-1} = \sum_{i=1}^{N-1}D_i \\
					&= \sum_{i=1}^{N-1}(\floor{\log_2 N} + 2) = \sum_{i=1}^{N-1}(\floor{\log_2 N + 1}) + \sum_{i=1}^{N-1}1 = (N-1) + \sum_{i=1}^{N-1}\widetilde B_i. \;\; \square
			\end{align*}

			\paragraph{Théorème} Le nombre de copies effectuées par le tri fusion pour un vecteur de taille $N$ est exactement :

			\[C_N = N\floor{\log_2 N} + 2N - 2^{\floor{\log_2 N} + 1}.\]

			\paragraph{Remarque} Ce nombre est également un majorant du nombre de comparaisons.

			\paragraph{Dém} Premièrement, montrons que $\sum_{i=1}^N\widetilde B_i = \sum_{i=0}^{\floor{\log_2 N}}(N-2^i)$. En effet, dans l'ensemble des nombres de 1 à $N-1$,
			ils contiennent tous un bit sur le bit de poids le plus faible ($= (N-1) = (N-2^0)$), tous sauf le premier contiennent le second bit ($= (N-1) = (N-2^1)$), tous sauf le premier
			et les deux suivants contiennent le troisième bit ($= (N-4) = (N-2^2)$), etc. De manière plus générale, les $2^i-1$ premiers nombres ne contiennent pas le $i^e$ bit.
			Le nombre de nombres contenant le $i^e$ bit est donc $((N-1)-(2^i-1)) = (N-2^i)$. Et le nombre de bits est $\widetilde B_i = \floor {\log_2 N} + 1$. On a donc
			$\sum_{i=1}^{N-1}(\floor{\log_2 N} + 1) = \sum_{i=0}^{\floor{\log_2 N}}(N-2^i) = N(\floor{\log_2 N} + 1) - \sum_{i=0}^{\floor{\log_2 N}}2^i$.

			Dès lors :

			\begin{align*}
				C_N &= (N-1) + \sum_{i=1}^{N-1}(N-2^i) = (N-1) + N\floor{log_2 N} + N - (2^{\floor{\log_2 N} + 1} - 1) \\
					&= 2N - 1 + N\floor{\log_2 N} - 2^{\floor{\log_2 N}+1} + 1 = 2N + N\floor{\log_2 N} + 2^{\floor{\log_2 N}+1}. \;\;\;\square
			\end{align*}

			\underline{Comportements asymptotiques}

			Considérons deux fonctions $f, g : \mathbb N \to \mathbb R^+$ s'annulant en un nombre fini de valeurs.

			\paragraph{Def}

			\begin{enumerate}
				\item Si $\lim_{n\to\infty}\frac {f(n)}{g(n)} = 1$, on dit que $f$ et $g$ sont \textit{asymptotiquement équivalents}, ce qui se note $f \sim g$ ;
				\item si $\exists C > 0, n_0 \in \mathbb N \tq f(n) \leq Cg(n) \;\;\forall n \geq n_0$, on dit que $f$ est un grand O de $g$, ce qui se note $f = O(g)$ ;
				\item si $\lim_{n\to\infty}\frac {f(n)}{g(n)} = 0$, on dit que $f$ est un petit o de $g$, ce qui se note $f = o(g)$ ;
				\item si $f = O(g)$, alors $g = \Omega(f)$ ;
				\item si $f = o(g)$, alors $g = \omega(f)$ ;
				\item si $f = O(g)$ et $g = O(f)$, on dit que $f$ et $g$ ont un même comportement asymptotique, ce qui se note $f = \Theta(g)$.
			\end{enumerate}

			\paragraph{Remarque} Si $f = o(g)$, alors $f = O(g)$. De même, si $g = \omega(f)$, alors $g = \Omega(f)$.

			\paragraph{Exemple} Chaque fonction ci-dessous est un $o(\cdot)$ de la précédente :

			\[n^n, 2^n, n^2, n, \sqrt n, \log(n)^2, \log(n), \log(\log(n)).\]

			\subsubsection{Récurrences \textit{Divide and Conquer} générales}

			\paragraph{Objectif} Obtenir le comportement asymptotique de coût en temps (ou en espace) d'un algorithme qui résout un problème de taille $N$ en :

			\begin{itemize}
				\item produisant un certain nombre $\alpha$ de sous-problèmes de taille $\floor{\frac N\beta}$ ou $\ceil {\frac N\beta}$ ;
				\item s'appliquant récursivement sur chaque problème ;
				\item recombinant les solutions des $\alpha$ sous-problèmes pour trouver une solution du problème original.
			\end{itemize}

			Ce qui se note $a_N = \alpha a_{\frac N\beta} + f(N)\;\;\forall N \in \mathbb N$ avec $f : \mathbb N \to \mathbb R^+$et où $\frac N\beta$ est interprété
			tantôt comme $\floor{\frac N\beta}$, tantôt comme $\ceil {\frac N\beta}$.

			Commençons par étudier l'équation fonctionnelle suivante :

			\begin{equation}\label{eq:EqFonctionnelle}
				\left\{
					\begin{aligned}
						a(x) &= \alpha a\left(\frac N\beta\right) + x &\text{ si } x > 1\\
						a(x) &= 0 &\text{ si } x \leq 1
					\end{aligned}
				\right.
			\end{equation}

			avec $\alpha, \beta \in \mathbb R$ tels que $\alpha > 0$ et $\beta > 1$.

			\paragraph{Théorème} Si la fonction $a : \mathbb R^+ \to \mathbb R^+$ est une solution de de \eqref{eq:EqFonctionnelle}, alors :

			\begin{enumerate}
				\item si $\alpha > \beta$ :

					  \[a(x) = \Theta\left(x^{\log_\beta\alpha}\right) ;\]

				\item si $\alpha = \beta$ :

					  \[a(x) \sim x\log_\beta x = \Theta(x\log_2 x) ;\]

				\item si $\alpha < \beta$ :

					  \[a(x) \sim \frac \beta{\beta-\alpha}x = \Theta(x).\]
			\end{enumerate}

			\paragraph{Exemple} Soit un tri fusion sur un vecteur de taille $N = 2^n$, où $\alpha = \beta = 2$, alors $C_N = N\log_2 N$.

			\paragraph{Dém} Soit une fonction $a : \mathbb R^+ \to \mathbb R^+$, solution de \eqref{eq:EqFonctionnelle}. Dès lors :

			\begin{align*}
				a(x) &= x + \alpha a\left(\frac x\beta\right) = x + \alpha\left[\frac x\beta + \alpha a\left(\frac x{\beta^2}\right)\right] = x + \frac \alpha\beta x + \alpha^2 a\left(\frac x{\beta^2}\right) = x + \frac \alpha\beta x + \frac {\alpha^2}{\beta^2}x + \alpha^3 a\left(\frac x{\beta^3}\right) \\
					 &= \ldots = \left[1 + \frac \alpha\beta + \frac {\alpha^2}{\beta^2} + \ldots + \frac {\alpha^{t-1}}{\beta^{t-1}}\right]x = \left[\sum_{i=0}^{t-1}\frac {\alpha^i}{\beta^i}\right]x = \left[\frac {1 - \frac {a^t}{\beta^t}}{1 - \frac \alpha\beta}\right]x.
			\end{align*}

			Où $t \coloneqq \ceil{\log_\beta x}$ tel que $t \stackrel{x\to+\infty}{\longrightarrow}+\infty$.

			\underline{cas 3 : } $\alpha < \beta$, on a :

			\[a(x) = \left[\frac {1 - \frac {\alpha^t}{\beta^t}}{1 - \frac \alpha\beta}\right]x \stackrel{x\to+\infty}\longrightarrow \left[\frac {1 - 0}{1 - \frac \alpha\beta}\right]x = \frac \beta{\beta - \alpha}x.\]

			À l'aide de cette convergence, on peut exprimer $a(x) = \Theta\left(\frac \beta{\beta-\alpha}x\right) = \Theta(x)$.

			\underline{cas 2 : } $\alpha= \beta$, on a :

			\[a(x) = x\sum_{i=0}^{t-1}\frac {\alpha^i}{\beta^i} = x\sum_{i=0}^{t-1}1 = xt =x\ceil{\log_\beta x} = \Theta(x\log_2 x).\]

			\underline {cas 1 : } $\alpha > \beta$, on a :

			\begin{align*}
				a(x) &= \left[\frac {1 - \frac {\alpha^t}{\beta^t}}{1 - \frac \alpha\beta}\right]x = \frac {\alpha^t}{\beta^t}\left[\frac {\frac {\beta^t}{\alpha^t} - 1}{1 - \frac \alpha\beta}\right]x
				= \left(\frac \alpha\beta\right)^\ceil{\log_\beta x}\left[\frac {\left(\frac \beta\alpha\right)^t - 1}{1 - \frac \alpha\beta}\right]x
				= \left(\frac \alpha\beta\right)^{\ceil{\log_\beta x} + \log_\beta x - \log_\beta x}\left[\frac {\left(\frac \beta\alpha\right) - 1}{1 - \frac \alpha\beta}\right]x \\
					 &= \left(\frac \alpha\beta\right)^{\ceil{\log_\beta x} - \log_\beta x}\frac {\alpha^{\log_\beta x}}{\beta^{\log_\beta x}}\left[\frac {\left(\frac \alpha\beta\right)^t - 1}{1 - \frac \alpha\beta}\right]x = \left(\frac \alpha\beta\right)^{\ceil{\log_\beta x} - \log_\beta x}\frac {\alpha^{\log_\beta x}}x\left[\frac {\left(\frac \beta\alpha\right)^t - 1}{1-\frac \alpha\beta}\right]x.
			\end{align*}

			Comme $\left(\frac \beta\alpha\right)^t \to 0$ quand $t \to +\infty$ et que la formule de changement de base d'une puissance est $\alpha^{\psi(x)} = \left(\beta^{\psi(x)}\right)^{\log_\beta \alpha}$,
			l'expression devient :

			\begin{align*}
				a(x) &\sim \left(\frac \alpha\beta\right)^{\ceil{\log_\beta x} - \log_\beta x}\left(\beta^{\log_\beta x}\right)^{\log_\beta \alpha}\left[\frac {-1}{1-\frac \alpha\beta}\right]
				= \left(\frac \alpha\beta\right)^{\ceil{\log_\beta x} - \log_\beta x}x^{\log_\beta \alpha}\left[\frac \beta{\alpha-\beta}\right]
			\end{align*}

			De plus, $\left(\frac \alpha\beta\right)^{\ceil{\log_\beta x} - \log_\beta x}$ est toujours borné par $\frac \alpha\beta$. Dès lors, on peut dire que :

			\[a(x) = \Theta\left(\frac \alpha\beta x^{\log_\beta \alpha}\frac \beta{\alpha-\beta}\right) = \Theta\left(x^{\log_\beta \alpha}\right).\]

			\begin{flushright}$\square$\end{flushright}

			\paragraph{Théorème (Master Thorem)} Soient $\alpha \geq 1, \beta > 1$ deux réels, $f : \mathbb N \to \mathbb R^+$ et $(a_N)_{N\in\mathbb N}$, une solution de la récurrence
			\textit{Divide and Conquer} suivante :

			\[a_N = \alpha a_{\frac N\beta} + f(N).\]

			\underline{cas 1 : } si $f(N) = O\left(N^{\log_\beta(\alpha)-\epsilon}\right)$ pour $\epsilon > 0$, alors :

			\[a_N = O\left(N^{\log_\beta\alpha}\right).\]

			\underline{cas 2 : } si $f(N) = \Theta\left(N^{\log_\beta\alpha}\right)$, alors :

			\[a_N = O\left(N^{\log_\beta\alpha}\log_2 N\right).\]

			\underline{cas 3 : } si $f(N) = \Omega\left(N^{\log_\beta(\alpha) + \epsilon}\right)$ pour $\epsilon > 0$, et si $\alpha f\left(\frac N\beta\right) \leq C f(N)$ pour $C < 1$ et si
			$N$ est \textit{suffisamment} grand, alors :

			\[a_N = \Theta(f(N)).\]

			\paragraph{Exemple} La recherche binaire (dichotomique) : $B_N = _{\floor{\frac N2}} + 1$. On a : $\alpha = 1, \beta = 2$ et $f(N) = 1 = \Theta\left(N^{\log_\beta\alpha}\right)$
			(car $\log_2 1 = 0$). Dès lors, on sait par le M.T. que $O\left(N^{\log_\beta\alpha}\log_2 N\right) = O(\log_2 N)$. En effet, $B_N = \floor{\log_2 N} + 1$.

			\paragraph{Application (algorithme de Strassen)}

			\paragraph{Ojectif} Calculer le produit de deux matrices avec un comportement asymptotique meilleur que $O(n^3)$. Soient deux matrices carrées $A, B \in M_{n \times n}(\mathbb R)$.
			Soit $C$ leur produit. Le nombre d'opérations pour déterminer les valeurs de $C$ est $2n^3 - n^2$.\\
			En effet, vu que :

			\[C_{ij} = \sum_{k=1}^nA_{ik}B_{kj},\]

			pour chacun des $n^2$ coefficients, il y a $n$ produits et les $n$ termes doivent être additionnés ($(n-1)$ sommes). Le nombre d'opérations est donc
			$n^2(n-1) + n^2 n = n^3-n^2+n^3 = 2n^3-n^2$.

			\paragraph{L'algorithme de Strassen}

			\subparagraph{Idée pour $n = 2$} l'algo \textit{usuel} de multiplication ferait donc $2 \cdot 2^3 - 2^2 = 12$ opérations (8 multiplications et 4 additions).
			L'algo de Strassen calcule préalablement 7 valeurs :

			\begin{align*}
				S_1 &= (A_{11} - A_{22})(B_{21} + B_{22}) \\
				S_2 &= (A_{11} + A_{22})(B_{11} + B_{22}) \\
				S_3 &= (A_{11} - A_{21})(B_{11} + B_{12}) \\
				S_4 &= (A_{11} + A_{12})B_{22} \\
				S_5 &= A_{11}(B_{12}-B_{22}) \\
				S_6 &= A_{22}(B_{21} - B_{11}) \\
				S_7 &= (A_{21} + A_{22})B_{11}
			\end{align*}

			Ensuite, on définit :

			\begin{align*}
				C_{11} &= S_1 + S_2 - S_4 + S_6 \\
				C_{12} &= S_4 + S_5 \\
				C_{21} &= S_6 + S_7 \\
				C_{22} &= -S_3 + S_4 + S_5 - S_7
			\end{align*}

			\subparagraph{Principe général de l'algo} Soit un corps $\mathbb K$. Soit deux matrices $\in M_{n \times n}(\mathbb K)$ :

			\[A = \begin{pmatrix}A_{11} & A_{12} \\ A_{21} & A_{22}\end{pmatrix}, B = \begin{pmatrix}B_{11} & B_{12} \\ B_{21} & B_{22}\end{pmatrix}.\]

			On définit leur produit par une nouvelle matrice $C \in M_{n \times n}(\mathbb K)$ :

			\[C = \begin{pmatrix}C_{11} & C_{12} \\ C_{21} & _{22}\end{pmatrix}.\]

			Ici, $A_{ij}, B_{ij}$ et $C_{ij}$ sont les sous-matrices $\in M_{\frac n2 \times \frac n2}(\mathbb K)$. L'algo de Strassen s'applique exactement de la même manière que pour e cas où $n=2$
			sauf que les additions et les produits ne se font plus sur de scalaires $\in \mathbb K$ mais bien des sous-matrices.

			Dans le cas où $n=2^k$, on a la relation de récurrence suivante sur $T(n) \coloneqq $ le nombre total d'opérations arithmétiques pour multiplier deux matrices $\in M_{n \times n}(\mathbb K)$
			à l'aide de l'algorithme de Strassen :

			\[T(n) = 7T\left(\frac n2\right) + 18n^2.\]

			Selon le M.T., on sait $f(n) = 18n^2 = \Theta(n^2), \alpha = 7, \beta = 2$. Soit $\epsilon = \log_\beta\alpha-2$. On a donc $f(n) = O(n^2) = O\left(n^{\log_\beta(\alpha) - \epsilon}\right)$.
			On sait dès ors que $T(n) = \Theta\left(n^{\log_\beta\alpha}\right) \simeq \Theta(n^{2.81})$.

			\paragraph{Remarque} Si $n \neq 2^k$, on rajoute des 0 de padding.

			\paragraph{Remarque} L'algorithme de Strassen est \textbf{asymptotiquement} meilleur que l'algorithme usuel pour calculer le produit de deux matrices. Cependant, pour des petites valeurs
			de $n$, il est bien pire. Il existe d'autres algorithmes (le plus \textit{intéressant asymptotiquement} actuellement est l'algorithme de Coppersmith-Winograd qui est en
			$O(n^{2.376})$, mais ces algorithmes ne sont efficaces que sur des matrices de plus en plus grande au point qu'ils ne puissent être utilisés en pratique).

			\paragraph{Conjecture} Il existe des algorithmes pour la multiplication de deux matrices $n \times n$ qui sont en $O(n^{2 +\epsilon})$ pour tout $\epsilon > 0$. En effet, l'exposant doit
			être au moins égal à deux car tous les éléments de la matrices doivent se voir assigner une valeur et il y en a $n^2$.

		\subsection{Fonctions génératrices}

			\paragraph{Idée} Transformer une suite $(a_n)$ en une fonction définie par une série.

			\paragraph{Objectif} Trouver des formules pour les termes de la suite $(a_n)$.

			\subsubsection{Exemple introductif}

			La suite de Fibonacci est représentée par la RLCC suivante :

			\[\left\{\begin{aligned}F_n &= F_{n-1} + F_{n-2} \;\;\forall n \geq 2\\F_0 &= 0, F_1 = 1\end{aligned}\right.\]

			On associe une fonction à la suite de Fibonacci $(F_n)_n$ :

			\[f(x) = \sum_{n=0}^{+\infty}F_nx^n.\]

			Cette fonction est une \textit{série formelle}, à savoir une série dont on ne se soucie pas de la convergence.

			On peut étendre cette fonction :

			\[f(x) = \sum_{n=0}^{+\infty}F_nx^n = 0 + x + \sum_{n=2}^{+\infty}F_nx^n = x + \sum_{n=2}^{+\infty}(F_{n-1} + F_{n-2})x^n = x + x\sum_{n=0}^{+\infty}F_nx^n + x^2\sum_{n=0}^{+\infty}F_nx^n = x + xf(x) + x^2f(x).\]

			Dès lors, on peut exprimer, en isolant $f(x)$ :

			\[f(x)  - f(x)(x+x^2) = x \iff f(x) = \frac x{1-x-x^2}.\]

			\paragraph{Remarque} Soit $\mathbb R[X]$ l'anneau des polynômes réels, soit $\mathbb R[X)$ l'anneau des séries formelles. Il existe des polynômes ($1-\lambda x$ par exemple) n'admettant
			pas d'inverse multiplicatif dans $\mathbb R[X]$ mais qui admettent un inverse multiplicatif dans $\mathbb R[X)$. En effet, $(1-x)(1 + \lambda x + \lambda^2x^2 + \ldots) =
			1 + \lambda x - \lambda x + \lambda^2 x^2 - \lambda^2 x^2 + \ldots = 1$.

			Dès lors, on note $\frac 1{1-\lambda x} = 1+\lambda x+\lambda^2 x^2+\ldots$.

			Trouvons maintenant un autre moyen d'exprimer $f(x)$ :

			\[f(x) = \frac x{1-x-x^2} = \frac x{(1-\varphi x)(1 - \bar\varphi x)} = \frac A{1-\varphi x} + \frac B{1-\bar\varphi x}.\]

			On trouve $A = \frac 1{\sqrt 5}$ et $B = \frac {-1}{\sqrt 5}$. Dès lors :

			\[f(x) = \frac 1{\sqrt 5} \left(\frac 1{1-\varphi x} - \frac 1{1-\bar\varphi x}\right).\]

			Or on sait que $\frac 1{1-\varphi x} = \sum_{n=0}^\infty (\varphi x)^n$ et $\frac 1{1-\bar\varphi x} = \sum_{n=0}^\infty(\bar\varphi x)^n$. Donc :

			\[f(x) = \frac 1{\sqrt 5}\left(\sum_{n=0}^\infty(\varphi x)^n - \sum_{n=0}^\infty(\bar\varphi x)^n\right) = \frac 1{\sqrt 5}\sum_{n=0}^\infty(\varphi^n - \bar\varphi^n)x^n.\]

			Or on sait que $f(x) = \sum_{n=0}^\infty F_Nx^n$. Dès lors, on sait que $F_n = (\varphi^n - \bar\varphi^n)$.

			\subsubsection{Fonctions génératrices ordinaires}

			\paragraph{Def} La fonction génératrice ordinaire (FGO) de la suite $(a_n)_n$ est définie par $A(x) = \sum_{n=0}^\infty a_nx^n$. On note $[x^n]A(x)$ le coefficient de $A(x)$ devant $x^n$.

			\paragraph{Remarque}
			\begin{lst}
				\item Dans les séries formelles, on ignore les question sur la convergence ;
				\item on manipule les séries formellement (comme des polynômes) ;
				\item certaines suites vont converger pour certaines valeurs de $x$.
			\end{lst}

			\paragraph{Théorème} Soient $A(x), B(x)$, les FGOs des suites $(a_n), (b_n)$. Alors :

			\begin{itemize}
				\item[$(i)$]    $A(x) + B(x)$ est la FGO de la suite $(a_n + b_n)$ ;
				\item[$(ii)$]   $xA(x)$ est la FGO de la suite $(a_{n-1})_{n\in\mathbb N^*}$ ;
				\item[$(iii)$]  $\int_0^xA(t)\dif t$ est la FGO de la suite $\left(\frac {a_{n-1}}n\right)_{n\in\mathbb N^*}$ ;
				\item[$(iv)$]   $\frac {A(x) - a_0}x$ est la FGO de la suite $(a_{n+1})_n$ ;
				\item[$(v)$]    $\pd xA(x)$ est la FGO de $(na_n)_{n\in\mathbb N^*}$ ;
				\item[$(vi)$]   $A(x)B(x)$ est la FGO de la suite $\left(\sum_{k=0}^na_kb_{n-k}\right)_n$ ;
				\item[$(vii)$]  $(1-x)A(x)$ est la FGO de la suite $\left(a_n - a_{n-1}\right)_{n\in\mathbb N^*}$ ;
				\item[$(viii)$] $\frac {A(x)}{1-x}$ est la FGO de $\left(\sum_{k=0}^na_k\right)_n$.
			\end{itemize}

			\paragraph{Démonstration (partielle) du théorème}

			\begin{itemize}
				\item[$(iii)$]  Étant donné que les séries sont traitées comme des séries formelles, les hypothèses de convergences ne sont pas considérées. Dès lors :

				                \[\int_0^xA(t)\dif t = \int_0^x\sum_{n=0}^\infty a_nt^n\dif t = \sum_{n=0}^\infty\int_0^xa_nt^n\dif t = \sum_{n=0}^\infty\frac {a_n}{n+1}x^{n+1}.\]

								La suite est donc $\left(0, a_0, \frac {a_1}2, \frac {a_2}3, \ldots\right) = \left(\frac {a_n}{n+1}\right)_{n\in\mathbb N^*}$.
				\item[$(v)$]    $\pd xA(x) = \sum_{n=0}^\infty a_n\pd xx^n = \sum_{i=1}^\infty na_nx^{n-1}$. La suite est donc $(a_1, 2a_2, \ldots) = ((n+1)a_{n+1})_n$.
				\item[$(vi)$]   \[A(x)B(x) = \left(\sum_{n=0}^\infty a_nx^n\right)\left(\sum_{n=0}^\infty b_nx^n\right) = \sum_{n=0}^\infty \left(\sum_{k=0}^na_kb_{n-k}\right)x^n.\]

				                La suite est donc $\left(\sum_{k=0}^n a_kb_{n-k}\right)_n$.
				\item[$(viii)$] En partant du fait que $\frac 1{1-x} = \sum_{n=0}^\infty x^n$ et que $A(x) = \sum_{n=0}^\infty a_n x^n$, par le point $(vi)$, on a la suite :

				                \[\left(\sum_{k=0}^na_k\beta_{n-k}\right)_n = \left(\sum_{k=0}^na_k\right)_n.\]
			\end{itemize}

			\begin{flushright}$\square$\end{flushright}

			\paragraph{Exemple} Déterminer la FGO de la suite $(n)_n$. On sait que $\frac 1{1-x}$ est la FGO de $(1)_n$. Par le point $(viii)$ du théorème, on sait que
			$\left(\frac 1{1-x}\right)^2$ est la FGO de $\left(\sum_{k=0}^n1\right)_n = (n)_n$.

			\paragraph{Proposition} Soit $k \in \mathbb N$ fixé. La FGO de la suite $\left(\binom nk\right)_n$ où $\binom nk \coloneqq 0$ si $n < k$ est :

			\[\frac {x^k}{(1-x)^{k+1}}.\]

			\paragraph{Dém} Prouvons-le par récurrence sur $k$.

			\underline{cas de base : } $k = 0$. Alors $\binom nk = 1 \; \forall n$. La FGO est donc $\frac 1{1-x} = \frac {x^0}{(1-x)^{0+1}}$.

			\underline{Pas de récurrence : } supposons la propriété vraie pour $k$ et prouvons-la pour $(k+1)$.

			\[\frac {x^{k+1}}{(1-x)^{k+2}} = \frac x{1-x}\frac {x^k}{(1-x)^{k+1}}.\]

			Soient $A(x) = \frac x{1-x}$ et $B(x) = \frac {x^k}{(1-x)^{k+1}}$. Par hypothèse, $B(x) = \sum_{n=0}^\infty \binom nk x^n$. Et par le théorème précédent, on sait que
			$A(x) = \sum_{n=1}^\infty x^n$. Dès lors, $C(x) = A(x)B(x)$ est donné par :

			\[C(x) = \sum_{n=0}^\infty\left(\sum_{i=0}^n a_kb_{b-i}\right)x^n = \sum_{n=0}^\infty\left(\sum_{i=0}^n a_kb_{n-i}\right)x^n = \sum_{n=0}^\infty\left(\sum_{i=1}^n b_{n-i}\right)x^n
			  = \sum_{n=0}^\infty\left(\sum_{i=0}^{n-1}\binom ik\right)x^n = \sum_{n=0}^\infty \binom n{k+1}x^n\]

			La fonction $C(x) = \frac {x^{k+1}}{(1-x)^{k+2}}$ est donc bien la FGO de la suite $\left(\binom n{n+1}\right)_n$.

			\begin{flushright}$\square$\end{flushright}

			\underline{Les nombres harmoniques}

			\paragraph{Def} Pour $n \in \mathbb N$, on définit :

			\[H_n \coloneqq \left\{\begin{aligned}&0 &\text{ si } n=0\\&\sum_{k=1}^\infty\frac 1{x^k}&\text{ si } n>0\end{aligned}\right.\]

			\paragraph{Proposition} La FGO de la suite $(H_n)_n$ est :

			\[H(x) = \left(\frac 1{1-x}\right)\ln\left(\frac 1{1-x}\right).\]

			\paragraph{Dém} Partons de la fonction $x \mapsto \frac 1{1-x}$, FGO de $(1)_n$. Par le point $(iii)$, on sait que :

			\[\int_0^x\frac 1{1-t}\dif t = -\ln(1-x) = \ln\left(\frac 1{1-x}\right)\]

			est la FGO de la suite $\left(0, 1, \frac 12, \frac 13, \ldots\right)$. On a tous les termes de la suite, il faut maintenant les sommer. Par le point $(viii)$, on sait que :

			\[\frac 1{1-x}\ln\left(\frac 1{1-x}\right)\]

			est la FGO de $\left(\sum_{k=1}^n \frac 1{x^k}\right)$, qui est ce que l'on voulait.

			\begin{flushright}$\square$\end{flushright}

			\underline{Les nombres de Catalan}

			\paragraph{Def} Le $n^e$ nombre de Catalan, noté $C_n$ est le nombre de parenthésages possibles pour un produit de $n$ facteurs $x_1, x_2, \ldots, x_n$ (avec $C_0 = 0$).

			\paragraph{Lien avec les arbres enracinés à $n$ feuilles} On construit l'arbre et on associe un facteur à chaque feuille. Chaque nœud interne correspond à un produit parenthésé.

			\paragraph{Proposition} $\left(C_n\right)_n$ est la solution de la récurrence :

			\[C_n = C_1C_{n-1} + C_2C_{n-2} + \ldots + C_{n-1}C_1 = \sum_{k=1}^{n-1}C_kC_{n-k}.\]

			De plus, $C_n = \frac 1n\binom {2(n-1)}{n-1}$ pour tout $n \geq 1$.

			\paragraph{Dém} Montrons que $C_n = \sum_{k=1}^{n-1}C_kC_{n-k}$.

			Soit $C(x) \coloneqq \sum_{n=0}^\infty C_n x^n$, la FGO de la suite des nombres de Catalan. On a :

			\[C(x) = C_1x + C_2x^2 + \ldots = x + (C_1C_1)x^2 + (C_1C_2 + C_2C_1)x^3 + \ldots = x + C(x)C(x).\]

			On a donc $C(x) = x + C(x)^2$, ou encore $(C(x))^2 - C(x) + x = 0$. Cette équation du second degré a deux solutions possibles :
			$C_1(x) = \frac {1+\sqrt {1-4x}}2$ et $C_2(x) = \frac {1-\sqrt {1-4x}}2$. Or il faut $C(0) = 0$, donc la solution est $C(x) = C_2(x)$.
			Par la formule de Taylor de $C(x)$ autours du $x=0$, on a (en prenant $f(x) = \sqrt {1-4x}$) :

			\[C(x) = \frac 12 - \frac 12\sum_{n=0}^\infty f^{(n)}(0)\frac {x^n}{n!} = \sum_{n=0}^\infty \frac 12(-1)^{n+1}\binom {\frac 12}n(-4x)^n.\]

			Où $\binom {\frac 12}n$ est défini par :

			\[\binom {\frac 12}n \coloneqq \left\{\begin{aligned}&1 &\text{ si } n=0\\&\frac 1{n!}\prod_{k=1}^{n-1}\left(\frac 12-k\right) &\text{ si } n > 0\end{aligned}\right.\]

			Or $C(x) = \sum_{n=0}^\infty C_nx^n$. Dès lors, on sait que :

			\[C_n = (-1)^{n+1}\binom {\frac 12}n\frac 124^n = (-1)^{n+1} 2^{2n}\frac 1{2n!}\prod_{k=0}^{n-1}\left(\frac 12 - k\right) = 2^{2n}(-1)^{n+1}\frac 1{4n!}\prod_{k=1}^{n-1}\left(\frac 12-k\right).\]

			Or tous les termes du produit (il y en a $(n-1)$) sont négatifs. Dès lors, en rentrant le $(-1)^{n+1} = (-1)^{n-1}$ dans le produit, et en mettant $\frac 12$
			en évidence à chaque terme, on obtient :

			\[C(x) = \frac {2^{2n}}{4n!}\left(\frac 12\right)^{n-1}\prod_{k=1}^{n-1}(2k-1) = \frac {2^n}{2n!}\prod_{k=1}^{n-1}(2k-1).\]

			En utilisant $\frac {n!}{n!}$ pour artifice de calcul, on obtient $n!2^n = 2\cdot4\cdot6\cdot\ldots\cdot2n$ au numérateur. Dès lors, on a :

			\[\frac {2n}{2n!n!}\prod_{k=1}^{2n-2}k = \frac 1{n}\frac {(2n-2)!}{(n-1)!(n-1)!} = \frac 1n\binom {2(n-1)}{n-1}.\]

\end{document}
