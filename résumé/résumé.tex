\documentclass{article}

\usepackage{commath}
\usepackage{amsmath}
\usepackage{amsfonts}
\usepackage[utf8]{inputenc}
\usepackage[T1]{fontenc}
\usepackage{times}
\usepackage[cm]{fullpage}
\usepackage{cases}
\usepackage[parfill]{parskip}
\usepackage{mathtools}

\title{Résumé de Stats \& Probas (MATHH-2002)}
\date{Premier quadirmestre 2015-2016}
\author{}

\DeclareMathOperator{\tq}{\text{ t.q. }}
\DeclareMathOperator{\Nms}{\mathcal N(\mu, \sigma)}
\DeclareMathOperator{\Nzo}{\mathcal N(0, 1)}
\renewcommand{\pd}[1]{\frac {\partial}{\partial #1}}

\begin{document}
\pagenumbering{Roman}
\maketitle
\tableofcontents
\newpage
\pagenumbering{arabic}

\section{Statistique descriptive}
	\subsection{Introduction}
		Il existe différents types de variables qui sont manipulées en statistiques (et en probabilités) : soit une variable est \textit{quantitative} (numérique)
		(en quel cas elle est soit discrète soit continue) soit elle est \textit{qualitative}. Si elle est qualitative, elle ne peut être représentée par un nombre car il
		serait alors sous-entendu que toutes les opérations applicables aux nombres le seraient également sur ces variables particulières, ce qui insérerait un \textbf{biais}.

		Les variables qualitatives sont regroupées en quatre sous-parties : \textit{nominales}, \textit{ordinales}, \textit{intervalles}, \textit{ratios}. Ce qui différencie
		ces sous-parties est l'ensemble des opérations que l'on peut appliquer. Deux variables nominales ne peuvent qu'être comparées $=,\neq$, deux variables ordinales peuvent
		\textbf{en plus} être ordonnées, deux variables intervalles acceptent \textbf{en plus} les opérations additives et les variables rations peuvent servir d'opérandes
		pour des opérations multiplicatives.
	
	\subsection{Notations et définitions}
		Il est important de commencer par déterminer les notations et les notions qui vont suivre afin de ne pas les confondre :

		\begin{itemize}
			\item une variables aléatoire est notée par une lettre \textbf{latine majuscule} : $X$ ;
			\item les valeurs possibles pour une telle variable aléatoire (discrète) sont notées par la même lettre latine mais \textbf{minuscule} avec un indice tel que
				  $\forall i \in \{1, \ldots, p\} : x_i \leq x_{i+1}$ ;
			\item l'\textit{effectif} est le nombre d'observations réalisées et est habituellement noté $n$ ;
			\item la \textit{fréquence absolue} d'une valeur $x_i$ est le nombre d'observations de cette valeur et est notée $n_i$ ;
			\item la \textit{fréquence relative} d'une valeur $x_i$ est la proportion d'observations correspondant à cette valeur et est notée $f_i = \frac {n_i}n$ ;
			\item le mode est la valeur ayant la plus grande fréquence (donc la plus observée) ;
			\item la fréquence cumulée (absolue ou relative) d'une valeur $x_i$ est la somme des fréquences $f_i + f_{i-1} + \ldots + f_1$ (dans le cas continu, la somme
				  devient une intégrale, à voir plus tard) ;
			\item la distribution de fréquences est l'ensemble des couples $(x_i, f_i)$ (donc une fonction $f_X(x_i)$ dans le cas d'une variable continue).
		\end{itemize} 

	\subsection{Représentations graphiques}
		Une distribution de fréquences peut être totalement aléatoire mais il est assez fréquent de voir quatre différentes classes de graphiques :

		\begin{itemize}
			\item les graphiques en cloche (moyenne d'une distribution aléatoire de taille, de nombre, etc.) ;
			\item les graphiques en \textit{U} (interventions/jour sur la vie d'un objet) ;
			\item les graphiques en \textit{i} décroissants de gauche à droite (nombre de gagnants du Lotto en fonction du montant) ;
			\item les graphiques en \textit{j} croissant de gauche à droite (risque de panne en fonction de l'âge d'une machine).
		\end{itemize}
	
	\subsection{Valeurs typiques d'une variable aléatoire}
		Une variable aléatoire peut être représentée et manipulée au travers de certaines valeurs particulières. Ces valeurs sont réparties en \textit{paramètres de position}
		et \textit{paramètres de dispersion}.

		\subsubsection{Paramètres de position}
			La moyenne (arithmétique) est définie par :
			
			\[\overline{x} = \frac 1n\sum_{i=1}^px_in_i = \sum_{i=1}^px_if_i.\]

			La médiane et définie par :

			\[\widetilde{x} \tq = F^*(\widetilde{x}) = \frac 12.\]

			où $F^*$ est la fréquence cumulée relative. Le mode est défini comme les maxima locaux de la distribution.

			Ces valeurs ne sont cependant pas suffisantes pour bien comprendre l'\textbf{entièreté} de la distribution. Par exemple, la moyenne est très sensible aux valeurs
			aberrantes alors que la médiane l'est beaucoup moins. De plus, une moyenne n'est pas unique : il existe plusieurs possibilités pour calculer une moyenne
			(représentant une valeur différente) : arithmétique, géométrique, harmonique, etc.

		\subsubsection{Paramètres de dispersion}
			L'écart moyen est défini par :

			\[\frac 1n\sum_{i=1}^pn_i(x_i-\overline{x}).\]

			Mais cette valeur n'est pas intéressante car la moyenne est définie comme l'\textit{exact} centre de toutes les valeurs. Dès lors, les valeurs $< \overline x$
			sont compensées par les valeurs $> \overline x$ de manière à ce que l'écart moyen soit toujours nul :

			\[\frac 1n\sum_{i=1}^pn_i(x_i-\overline x) = \frac 1n\sum_{i=1}^pn_ix_i - \frac 1n\sum_{i=1}^pn_i\overline x =\overline x - \overline{x}\frac 1n\sum_{i=1}^pn_i = 0.\]

			Il est alors possible de parler d'écart moyen absolu :

			\[\frac 1n\sum_{i=1}^p|n_i(x_i-\overline x)|.\]

			Mais la notion de variance est beaucoup plus agréable à manipuler et est construite de manière à pénaliser les gros écarts à la moyenne plus fort que les petits
			écarts :

			\[s^2 = \frac 1n\sum_{i=1}^p(x_i-\overline x)^2 \geq 0.\]
			
			Sur base de cette notion de variance, l'écart-type $s$ est défini par $s = \sqrt{s^2}$.

			L'amplitude est la différence entre la plus petite et la plus grande valeur. Cependant, telle la moyenne, l'amplitude est très sensible aux valeurs dégénérées.

			Les quartiles quant à eux représentent une division des valeurs en 4 grâce à trois séparateurs :

			\[q_1 \tq F^*(q_1) = \frac 14, \;\;q_2 = \widetilde x,\;\; q_3 \tq F^*(q_3) = \frac 34.\]
			
			Les quartiles $q_1$ et $q_3$ servent à définir l'intervalle \textit{itnterquartile} $IQ = q_3-q_1$. Cet intervalle donne une manière de repérer les valeurs
			aberrantes : si $x < q_1 - 2IQ$ ou si $x > q_3 + 2IQ$, alors on considère $x$ comme dégénéré.
			
	\subsection{Statistique descriptive à deux dimensions}
		Soient $X(x_1, \ldots, x_p)$ et $Y(y_1, \ldots, y_q)$ avec un effectif de $n$ observations. On définit $n_{ij} \coloneqq$ la fréquence absolue du couple $(x_i, y_j)$.
		On définit également les fréquences absolues restreintes $n_{i \cdot} = \sum_{j=1}^qn_{ij}$ et $n_{\cdot j} = \sum_{i=1}^pn_{ij}$.
	   
		Les valeurs typiques que l'on utilise pour caractériser une telle situation sont :
		
		\begin{itemize}
			\item \textit{les moyennes marginales} définies par :

			\[\overline x = \frac 1n\sum_{i=1}^pn_{i\cdot}x_i, \overline y = \frac 1n \sum_{j=1}^qn_{\cdot j}y_j ;\]

			\item les variances marginales définies par :

			\[s_1^2 = \frac 1n\sum_{i=1}^pn_{i\cdot}(x_i-\overline x)^2, s_2^2 = \frac 1n\sum_{j=1}^qn_{\cdot j}(y_j-\overline y)^2 ;\]

			\item les moyennes conditionnelles définies par :

			\[\overline{x_k} = \frac 1{n_{\cdot k}}\sum_{i=1}^pn_{ik}x_i, \overline{y_k} = \frac 1{n_{k\cdot}} \sum_{j=1}^qn_{kj}y_j.\]
		\end{itemize}
	   
		Afin de déterminer s'il existe une relation linéaire entre ces deux variables aléatoires, on utilise la \textbf{covariance} :

		\[m_{11} = \frac 1n\sum_{i=1}^p\sum_{j=1}^qn_{ij}(x_i-\overline x)(y_j-\overline y).\]

		En effet, si ce paramètre est positif, c'est qu'il y a une majorité d'observations qui se situent dans les cadrans inférieur gauche et supérieur droit\footnote{En
		considération une découpe du plan en 4 : en traçant un axe $y = \overline y$ et un autre $x = \overline x$.} et donc que la droite relative à la relation linéaire est
		croissante. De manière similaire, si $m_{11}$ est négatif, c'est qu'il y a majorité d'observations dans les cadrans supérieur gauche et inférieur droit.
		Attention cependant à ne pas trouver une relation linaire là où il n'y en a pas.

		\subsubsection{Bornes de la covariance}
			Il existe une propriété disant que $|m_{11} \leq s_1s_2|$. De manière plus précise, si $|m_{11}| = s_1s_2$, c'est que tous les points se situent sur un axe non
			parallèle aux axes $Ox$ et $Oy$.

			\underline{Démonstration :}

			\[\begin{aligned}
				\forall u \in \mathbb R : &\frac 1n\sum_{i=1}^p\sum_{j=1}^qn_{ij}(u(x_i-\overline x) + (y_j-\overline y))^2 &&\geq 0 \\
										  &\frac {u^2}n\sum_{i=1}^p\sum_{j=1}^qn_{ij}(x_i-\overline x)^2 + \frac {2u}n\sum_{i=1}^p\sum_{j=1}^qn_{ij}(x_i-\overline x)(y_j-\overline y) + \frac 1n\sum_{i=1}^p\sum_{j=1}^qn_{ij}(y_j-\overline y)^2 &&\geq 0 \\
										  & \frac {u^2}n\sum_{i=1}^pn_{i\cdot}(x_i-\overline x)^2 + 2u m_{11} + \sum_{j=1}^qn_{\cdot j}(y_j-\overline y)^2 &&\geq 0 \\
										  &u^2s_1^2 + 2um_{11} + s_2^2 &&\geq 0
			\end{aligned}\]

			On a donc une équation du second degré en $u$ de discriminant $\Delta = (2m_{11})^2 - 4s_1^2s_2^2$. Or du fait que cette équation du second degré soit $\geq 0$, il
			y a soit 1 soit 0 racine. Il faut donc $\Delta \leq 0$ ou encore $4m_{11}^2 - 4s_1^2s_2^2\leq 0$ dès lors, on a $m_{11}^2 \leq s_1^2s_2^2$. En prenant de part et
			d'autre la racine carrée, on a $|m_{11}| \leq s_1s_2$.
			
			Plus précisément, si l'équation vaut 0, c'est que le discriminant est nul et qu'il y a une seule racine. Il faut donc avoir $\Delta = 0$, ou encore $m_{11} = s_1s_2$.
			$\square$

		\subsubsection{Coefficient de corrélation}
			Vu que $m_{11}$ et $s_1s_2$ sont liés, on définit :

			\[r \coloneqq \frac {m_{11}}{s_1s_2},\]

			le coefficient de corrélation de $X$ et $Y$. Ce coefficient montre à quel point les deux variables aléatoires $X$et $Y$ sont en relation linéaire.
			En effet, si tous les points sont placés sur une droite, alors on a $|r| = 1$ et dans les autres cas, on a $|r| < 1$.

		\subsubsection{Régression linéaire, erreur et variance résiduelle}
			Lorsque deux variables $X$ et $Y$ sont en relation linéaire, on détermine l'équation de la droite $d : ax + b$. $a, b$ sont ici les valeurs qui caractérisent
			notre droite. Cependant, la droite ne convient pas \textbf{exactement} à \textbf{tous} les points. Il y a donc une erreur que l'on peut calculer :

			\[\epsilon(a, b) = \sum_{i=1}^p\sum_{j=1}^qn_{ij}(y_j - (ax_i+b))^2.\]

			Cette fonction doit être minimisée afin de trouver les paramètres optimaux. Pour minimiser la fonction, il faut déterminer son gradient et l'annuler :

			\[\begin{aligned}
				\pd{a}\epsilon(a, b) &= \pd{a}\sum_{i=1}^p\sum_{j=1}^qn_{ij}\left[y_j - (ax_i + b)\right]^2 = \sum_{i=1}^p\sum_{j=1}^qn_{ij}\pd{a}[y_j-(ax_i+b)]^2 \\
									 &= \sum_{i=1}^p\sum_{j=1}^qn_{ij}2[y_j-(ax_i + b)](-x_i) \\
									 &= 2\sum_{i=1}^p\sum_{j=1}^qn_{ij}ax_i^2 + 2\sum_{i=1}^p\sum_{j=1}^qn_{ij}bx_i - 2\sum_{i=1}^p\sum_{j=1}^qn_{ij}y_jx_i.
			\end{aligned}\]

			Or cette dérivée partielle doit être nulle pour trouver le minimum :

			\[\begin{aligned}
				a\sum_{i=1}^p\sum_{j=1}^qn_{ij}x_i^2 + b\sum_{i=1}^p\sum_{j=1}^qn_{ij}x_i &= \sum_{i=1}^p\sum_{j=1}^qn_{ij}x_iy_j \\
				a\sum_{i=1}^px_i^2\sum_{j=1}^qn_{ij} + b\sum_{i=1}^px_i\sum_{j=1}^qn_{ij} &= \sum_{i=1}^p\sum_{j=1}^qn_{ij}x_iy_j \\
				a\sum_{i=1}^pn_{i\cdot}x_i^2 + b\sum_{i=1}^pn_{i\cdot}x_i &= \sum_{i=1}^p\sum_{j=1}^qn_{ij}x_iy_j.
			\end{aligned}\]

			De même pour $b$ :

			\[\begin{aligned}
				\pd b\epsilon(a, b) &= \sum_{i=1}^p\sum_{j=1}^qn_{ij}\pd b[y_j-(ax_i+b)]^2 \\
									&= \sum_{i=1}^p\sum_{j=1}^qn_{ij}2[y_j-(ax_i+b)](-1) \\
									&= 2\sum_{i=1}^p\sum_{j=1}^qn_{ij}[(ax_i+b)-y_j].
			\end{aligned}\]

			Or à nouveau, cette dérivée partielle doit s'annuler pour trouver le minimum :

			\[\begin{aligned}
				2\sum_{i=1}^p\sum_{j=1}^qn_{ij}[(ax_i+b)-y_j] &= 0 \\
				\sum_{i=1}^p\sum_{j=1}^qn_{ij}ax_i + \sum_{i=1}^p\sum_{j=1}^qn_{ij}b &= \sum_{i=1}^p\sum_{j=1}^qn_{ij}y_j \\
				a\sum_{i=1}^px_i\sum_{j=1}^qn_{ij} + b\sum_{i=1}^p\sum_{j=1}^qn_{ij} &= \sum_{j=1}^qy_j\sum_{i=1}^pn_{ij} \\
				a\sum_{i=1}^pn_{i\cdot}x_i + bn &= \sum_{j=1}^qn_{\cdot j}y_j \\
				an\overline x + bn &= n\overline y.
			\end{aligned}\]

			On a donc deux équations à deux inconnues :

			\begin{numcases}{}
					a\sum_{i=1}^pn_{i\cdot}x_i^2 + bn\overline x = \sum_{i=1}^p\sum_{j=1}^qn_{ij}x_iy_j \\
					an\overline x + bn = n\overline y
			\end{numcases}

			En développant $(1) - \overline x(2)$, on obtient ceci :

			\[\begin{aligned}
				\left[a\sum_{i=1}^pn_{i\cdot}x_i^2 + bn\overline x\right]-\left[an\overline x^2 + bn\overline x\right] &= \sum_{i=1}^p\sum_{j=1}^qn_{ij}x_iy_j - n\overline x\overline y \\
				a\left[\sum_{i=1}^pn_{i\cdot}x_i^2 - n\overline x^2\right] &= \sum_{i=1}^p\sum_{j=1}^qn_{ij}x_iy_j - n\overline x\overline y \\
				a\left[\sum_{i=1}^pn_{i\cdot}x_i^2 + n\overline x^2 - 2n\overline x^2\right] &= \sum_{i=1}^p\sum_{j=1}^qn_{ij}x_iy_j + n\overline x\overline y - 2n\overline x\overline y \\
				a\left[\sum_{i=1}^pn_{i\cdot}(x_i^2 + \overline x^2 - 2\overline xx_i)\right] &= \sum_{i=1}^p\sum_{j=1}^qn_{ij}x_iy_j - \overline x\sum_{j=1}^qy_j\sum_{i=1}^pn_{ij} - \overline y\sum_{i=1}^px_i\sum_{j=1}^qn_{ij} + n\overline x\overline y \\
				a\sum_{i=1}^pn_{i\cdot}(x_i-\overline x)^2 &= \sum_{i=1}^p\sum_{j=1}^qn_{ij}(x_iy_j - \overline xy_j - x_i\overline y + \overline x\overline y) \\
				ans_1^2 &= \sum_{i=1}^p\sum_{j=1}^qn_{ij}(x_i-\overline x)(y_j-\overline y) \\
				ans_1^2 &= nm_{11} \\
				a &= \frac {m_{11}}{s_1^2}.
			\end{aligned}\]

			Et comme l'équation $(2)$ nous dit que $b = \overline y - a\overline x$, on a $b = \overline y - \frac {m_{11}}{s_1^2}\overline x$. Dès lors, l'équation de la droite est :

			\[D : y = \frac {m_{11}}{s_1^2}(x-\overline x) + \overline y.\]

			Il est également possible de faire une régression linéaire pour trouver $x$ en fonction de $y$. Dans ce cas, l'équation de droite devient :

			\[D : x = \frac {m_{11}}{s_1^2}(y-\overline y) + \overline x.\]

			La \textit{variance résiduelle} de $y$ en $x$ est donnée par :

			\[\begin{aligned}
				s_{yx}^2 &= \frac 1n\sum_{i=1}^p\sum_{j=1}^qn_{ij}[y_j-ax_i-b]^2 \\
						 &= \frac 1n\sum_{i=1}^p\sum_{j=1}^qn_{ij}[y_j-\frac {m_{11}}{s_1^2}x_i - \overline y + \frac {m_{11}}{s_1^2}]^2 \\
						 &= \frac 1n\sum_{i=1}^p\sum_{j=1}^qn_{ij}(y_j - \overline y)^2 + \frac 1n\sum_{i=1}^p\sum_{j=1}^qn_{ij}\left(\frac {m_{11}}{s_1^2}\right)^2(x_i-\overline x)^2 - \frac 2n\sum_{i=1}^p\sum_{j=1}^qn_{ij}(y_j-\overline y)\frac {m_{11}}{s_1^2}(x_i-\overline x) \\
						 &= s_2^2 + \frac {m_{11}^2}{s_1^4}s_1^2 - 2\frac {m_{11}}{s_1^2}m_{11} = s_2^2 - \frac {m_{11}^2}{s_1^2} = s_2^2 - \frac {m_{11}^2s_2^2}{s_1^2s_2^2}
							 = s_2^2\left(1 - \frac {m_{11}^2}{s_1^2s_2^2}\right) = s_2^2(1-r^2).
			\end{aligned}\]

			On peut comprendre cette valeur comme étant la proportion de la variance n'étant pas expliquée par le modèle linéaire.

		% Fini la section sur les stats, reste les probas

\section{Les probabilités}  % p.49
	\subsection{Introduction}
		Suite à des questions telles que \textit{qu'est-ce que l'aléatoire ?} ou \textit{comment le quantifier ?}, on arrive vite à la notion de probabilité. Le domaine des
		probabilités est très fortement lié au domaine des statistiques comme ce sera vu plus loin.

	\subsection{Notations et notions}
		Tout comme pour les statistiques, il est important de commencer par définir \textbf{précisément} les termes et conceptions qui vont suivre :

		\begin{itemize}
			\item une \textit{expérience aléatoire} est une expérience dont l'issue ne peut être définie précisément (ou avec certitude) ;
			\item l'univers $\Omega$ (ou $E$) est l'ensemble des issues possibles d'une expérience aléatoire ;
			\item un résultat est un élément $e \in E$ qui représente l'issue d'une variable aléatoire ;
			\item un événement $A \subseteq E$ est un sous-ensemble de possibilités sur lequel la probabilité d'occurrence peut être évaluée ;
			\item l'événement certain est l'événement $A = E$ de probabilité $1$ (donc dont l'occurrence est sûre) ;
			\item l'événement impossible $\emptyset \subset E$ est l'événement de probabilité $0$ (donc dont la non-occurrence est sûre).
		\end{itemize}

	\subsection{Lien avec les statistiques}
		Il existe plusieurs manières de lier les statistiques et les probabilités. La première est l'\textit{approche fréquentiste}. Cette théorie dit que quand $n \to \infty$,
		la fréquence relative $\frac {n_i}n$ converge en une valeur $\in [0, 1]$. C'est cette valeur que représente la probabilité.

		A partir de là, on peut définir la probabilité d'un événement $A \subseteq E$ comme suit\footnote{Quant le comptage n'est plus trivial, il faut passer par l'analyse
		combinatoire.} :

		\[P(A) = \frac {\#A}{\#E}.\]

	\subsection{Axiomes de Kolmogorov}
		Comme toute théorie mathématique, la théorie des probabilités est basées sur certains axiomes. Ces axiomes sont appelés \textit{axiomes de Kolmogorov}. Ils sont au nombre
		de trois et peuvent être exprimés comme suit :

		\begin{enumerate}
			\item toute probabilité est évaluée entre $0$ et $1$ compris : $\forall A \subseteq E : P(A) \in [0, 1]$ ;
			\item La probabilité de l'événement certain est de $1$ : $P(E) = 1$ ;
			\item Si $A \subseteq E$ et $B \subseteq E$ sont disjoints (incompatibles), la probabilité de leur union est la somme des probabilités :
				
				  \[\forall A, B \subseteq E : A \cap B = \emptyset \Rightarrow P(A \cup B) = P(A) + P(B).\]
		\end{enumerate}

		De là, une définition plus précise de la fonction de probabilité peut être exprimée :

		\[P : 2^E \to [0, 1] : A \mapsto P(A).\]

		\underline{Remarque : } Attention, il existe des événements possibles mais de \textbf{stochastiquement} impossibles. Cela veut dire qu'il existe des événements qui
		peuvent théoriquement apparaitre mais dont la fréquence relative tend vers $0$ quand $n$ tend vers $+\infty$. Par exemple, lors du lancer d'une pièce de monnaie,
		l'événement relatif au fait que la pièce ne tombe ni sur \textbf{pile}, ni sur \textbf{face} mais \textit{pile sur la tranche} est possible mais de probabilité nulle.

	\subsection{Evenements stochastiquement indépendants}
		Soient $A, B \subseteq E$, deux événements. Notons $P(A \, | \, B)$ la probabilité de l'événement $A$ en sachant que l'événement $B$ s'est produit.
		Si $P(A \, | \, B) = P(A)$, on dit que $A$ et $B$ sont (stochastiquement) indépendants (ce que l'on note $A \sqcup B$).

	\subsection{Probabilité conditionnelle}
		Dans le cas précédent, la notation $P(A \, | \, B) = \frac {P(A \cap B)}{P(B)}$ représente la probabilité que l'événement $A$ arrive \textbf{en sachant que}
		l'événement $B$ est arrivé. Pour pouvoir appeler cette quantité \textit{probabilité}, il faut montrer qu'elle vérifie les axiomes de Kolmogorov :

		\begin{enumerate}
			\item $P(A \, | \, B) = \frac {P(A \cap B)}{P(B)} \geq 0$ car $P(A \cap B), P(B) \geq 0$ ;
			\item $P(E \, | B) = 1$ car $P(E \cap B) = P(B)$, donc $\frac {P(B)}{P(B)} = 1$ ;
			\item $A \cap C = \emptyset \Rightarrow P(A \cup C \, | \, B) = P(A \, | \, B) + P(C \, | \, B)$ car
			$P(A \cup C \, | \, B) = \frac {P((A \cap B) \cup (C \cap B))}{P(B)} = \frac {P(A \cap B)}{P(B)} + \frac {P(C \cap B)}{P(B)} = P(A\, | \, B) + P(C \, | \, B)$.
		\end{enumerate}

		Les axiomes de Kolmogorov ayant été démontrés, il est acceptable d'utiliser cette notion afin de représenter une probabilité.

		\underline{Remarque : } Il y a une \textbf{très forte} corrélation entre \textit{événements incompatibles} et \textit{événements indépendants}. En effet, deux
		événements incompatibles ne peuvent pas apparaitre en même temps. Dès lors la probabilité d'un événement en sachant que l'autre est apparu est facilement calculable et
		vaut $0$. On peut donc dire que deux événements incompatibles sont très fort dépendants.

		De plus, si deux événements sont indépendants, alors $P(A \, | \, B) = P(A)= \frac {P(A \cap B)}{P(B)}$ ou encore $P(A)P(B) = P(A \cap B)$.

	\subsection{Théorème de Bayes}
		Le théorème de Bayes est un moyen d'«\textit{inverser}» les probabilités conditionnelles. Il sert en effet à exprimer $P(A \, | \, B)$ en connaissant $P(B  \, | \, A)$.
		Soient $m$ événements deux à deux incompatibles :

		\[\left\{\begin{aligned}
			B \subseteq \bigcup_{i=1}^mA_i \\
			\forall i \neq j : A_i \cap A_j = \emptyset \\
			P(B) \neq 0
		\end{aligned}\right.\]

		En disposant des informations $\forall i \in \{1, \ldots, m\} : P(A_i), P(B \, | \, A_i)$, on peut retrouver les informations $P(A_k \, | \, B)$ :

		\[P(A_k \, | \, B) = \frac {P(B \, | \, A_k)P(A_k)}{\sum_{i=1}^mP(B \, | \, A_i)P(A_i)}.\]

		Par exemple, en connaissant les données suivantes :
		
		\begin{itemize}
			\item la probabilité qu'une personne soit un homme ou une femme (respectivement $P(H)$ et $P(F)$) ;
			\item la probabilité qu'un gêne s'exprime chez une femme ($P(G \, | \, F)$) ;
			\item la probabilité que ce gêne s'exprime chez un homme ($P(G \, | \, H)$),
		\end{itemize}

		on peut déterminer la probabilité qu'une personne quelconque ayant le gêne exprimé soit un homme ou une femme. On sait, par le théorème de Bayes, exprimer les
		égalités suivantes :

		\begin{align}
			P(H \, | \, G) &= \frac {P(G \, | \, H)P(H)}{P(G \, | \, H)P(H) + P(G \, | \, F)P(F)} \\
			P(F \, | \, G) &= \frac {P(G \, | \, F)P(F)}{P(G \, | \, H)P(H) + P(G \, | \, F)P(F)}
		\end{align}

		\underline{Démonstration du théorème de Bayes : } En partant de l'égalité suivante (corollaire direct de la définition de probabilité conditionnelle) :

		\[P(B)P(A_k \, | \, B) = P(B \, | \, A_k)P(A_k),\]

		on obtient :

		\begin{align}
			P(A_k \, | \, B) = \frac {P(B \, | \, A_k)P(A_k)}{P(B)}.
		\end{align}

		Or on sait que $A_i \cap A_j = \emptyset \forall i \neq j$. Dès lors $(B \cap A_j) \cap (B \cap A_j) = \emptyset$ également. De plus, $\{A_i\}_i$ forme une partition de
		$E$. On a donc :

		\[P(B) = P(B \cap E) = P\left(B \cap \left[\bigcup_{i=1}^mA_i\right]\right) = P\left(\bigcup_{i=1}^m[B \cap A_i]\right).\]

		Et comme les couples $((B \cap A_i), (B \cap A_j))$ sont disjoints, on peut utiliser le troisième axiome de Kolmogorov :

		\[P(B) = \sum_{i=1}^mP(B \cap A_i).\]

		Par définition de la probabilité conditionnelle, on obtient :

		\begin{align}
			P(B) = \sum_{i=1}^mP(B \, | \, A_i)P(A_i).
		\end{align}

		En remettant $(5)$ et $(6)$ ensemble, on obtient le théorème de Bayes :

		\[P(A_k \, | \, B) = \frac {P(B \, | \, A_k)P(A_k)}{\sum_{i=1}^mP(B \, | \, A_i)P(A_i)}. \square\]

\section{Variables aléatoires}
	\subsection{Variables aléatoires discrètes}
		Une variable aléatoire (notée par une lettre latine majuscule par convention) est \textit{discrète} si le nombre de résultats possibles est dénombrable. Les valeurs
		qu'elle peut prendre sont notées $x_1, \ldots, x_n$ où $\forall i : x_i \leq x_{i+1}$. Ce sont ces valeur qui définissent la variable aléatoire. On note également
		$P_i$ la probabilité que la variable aléatoire $X$ prenne la valeur $x_i$. Dès lors, $P_i \coloneqq P(X = x_i)$.

		\underline{Remarque : } Il faut que ces valeurs $P_i$ respectent les propriétés des probabilités. Il faut donc $P_i \geq 0$ et $P(E) = 1$. Or $E = \cup_{i=1}^nx_i$,
		donc $1 = P(E) = P(\cup_{i=1}^nX=x_i) = \sum_{i=1}^nP_i$.

		\subsubsection{Fonction de répartition}
			Afin de déterminer le comportement d'une variable aléatoire, on utilise la notion de \textit{fonction de réaprtition}. On la définit comme suit :

			\[F_X : \mathbb R \to [0, 1] : x \mapsto P(\{e \in E \tq X(e) \leq x\}).\]

			Dans le cas d'une variable discrète, on a donc :

			\[F_X(x) = \sum_{x_i \leq x}P(X=x_i) \leq 1.\]

		\subsubsection{Lien avec les statistiques descriptives}
			En statistique descriptive, on avait plusieurs variables types pour désigner une situation. Par exemple, la moyenne était définie comme suit :

			\[\overline x = \frac 1n\sum_{i=1}^pn_ix_i = \sum_{i=1}^p\frac {n_i}nx_i.\]

			Or dans le cas des statistiques, on sait que la valeur $\frac {n_i}n$ converge vers $p_i$. La notion de moyenne est donc appelée $\mu$ (en stats on utilisait des
			lettres latines, en probabilités, on utilise les équivalents grecs) et définie comme suit :

			\[\mu_X = E(X) \coloneqq \sum_{i=1}^pp_ix_i.\]

			De même pour l'écart-type défini en statistique descriptive a un équivalent en probabilités :

			\[\sigma_X^2 = D^2(X) \coloneqq \sum_{i=1}^pp_i(x_i-\mu_X)^2.\]

	\subsection{Variables aléatoires continues}
		Abrégées VAC, les variables aléatoires continues sont des variables aléatoires dont l'ensemble des valeurs possibles n'est pas dénombrable (tel $\mathbb R$).
		Comme le nombre de possibilités pour cette variable est infini, on ne peut pas définir la probabilité comme la convergence $\frac {n_i}n$ pour $n \to \infty$ vu que
		ces valeurs convergeraient toutes vers $0$.

		Dès lors, on définit une autre notion : la \textit{densité de probabilité}. Cette fonction est définie sur base de la fonction de répartition $F_X$. Plus précisément,

		\begin{align}
			f_X(x) = \pd{x}F_X(x).
		\end{align}

		Le fait de prendre la dérivée permet de définir la fonction de densité de probabilité comme étant la pente de la tangente de la fonction de répartition. Cette définition
		est pertinente car au plus $F_X$ est croissante pour une valeur $x$, au plus il est probable que la valeur de la variable aléatoire lors d'une expérience aléatoire soit
		proche de $x$.

		\subsubsection{Vérification des axiomes}
			A nouveau, le fait de définir une nouvelle notation de probabilité implique de vérifier que l'appellation \textit{probabilité} est cohérente et acceptable.
			Il faut donc vérifier les axiomes de Kolmogorov. Le premier axiome implique $P(A) \geq 0$ ou encore $P(a \leq X \leq b) \geq 0$.
			Cela implique :

			\[\int_a^bf_X(x)\dif x \geq 0.\]

			La seule manière de garantir cela est d'imposer $f_X(x) \geq 0 \forall x$. Le deuxième axiome impose $P(E) = 1$. Donc :

			\[\int_{-\infty}^\infty f_X(x)\dif x = 1.\]

			Par définition, on a :

			\begin{align}
				F_X(x) = \int_{-\infty}^xf_X(t)\dif t.
			\end{align}

			On peut donc en déduire que $P(a \leq X \leq b) = P(X \leq b) - P(X \leq a)$. Ce qui donne lieu à une remarque importante : il a été mentionné dans l'introduction
			que la probabilité que la variable aléatoire prenne une valeur exacte est nulle car la convergence du quotient $\frac {n_i}n$ vaut $0$. Donc la valeur
			$P(X = a) = P(a \leq X \leq a) = F_X(a) - F_X(a) = 0$. On a donc bien en réalité une intégrale sur un point donc un résultat nul. Un corollaire de ceci est le fait
			que $\forall x : P(X \leq x) = P((X < x) \cup (X = x)) = P(X < x)$. Le fait de prendre - ou pas - la valeur $x$ dans l'évaluation de la probabilité n'intervient
			donc pas dans le résultat final. Ce corollaire sera souvent réutilisé sans être explicité.

			De plus, on peut justifier l'apparition d'une dérivée par le fait que l'aire sous la courbe est donnée par :

			\[A \coloneqq \lim_{\Delta t \to 0}\frac {F_X(t + \Delta t) - F_X(t)}{\Delta t} = \pd{t}F_X(t).\]

		\subsubsection{Lien avec les statistiques descriptives}
			Tout comme pour les variables aléatoires discrètes, les variables aléatoires continues peuvent être reliées aux valeurs particulières vues en statistiques
			descriptives. La notion de moyenne et celle d'écart-type sont données par :

			\begin{align}
				\mu_X &= E(X) \coloneqq \int_{-\infty}^\infty x f_X(x)\dif x \\
				\sigma_X^2 &= D^2(X) \coloneqq \int_{-\infty}^\infty(x-\mu_X)^2f_X(x)\dif x.
			\end{align}
	\subsection{Opérations sur des variables aléatoires}
		Lorsque l'on a deux variables aléatoires $X$ et $Y$ définies par leur fonction de densité respective ($f_X$ et $f_Y$), il serait intéressant de pouvoir définir une
		nouvelle variable $Z$ définie par une opération sur les deux autres (par exemple $Z = X+Y$).
		
		\subsubsection{Variables aléatoires à deux dimensions}
			Lorsque l'on passe à deux dimensions, on définit la fonction de répartition comme suit :

			\begin{align}
				F_{(V, W)}(x, y).
			\end{align}

			Avec une seule variable aléatoire, l'indépendance est définie par :

			\[A \sqcup B \iff P(A \cap B) = P(A)P(B).\]

			Cette définition doit être adaptée pour le cas de dimension 2. Il y a donc un théorème pour caractériser l'indépendance sur base de la fonction de répartition à deux
			dimensions :

			\underline{Théorème :} $\forall x_1, x_2, y_1, y_2 : V \sqcup W$

			\begin{enumerate}
				\item $\iff P(V \in [x_1, x_2] \land W \in [y_1, y_2]) = P(V \in [x_1, x_2])P(W \in [y_1, y_2])$
				\item $\iff F_{(V, W)}(x_2, y_2) = F_V(x_2)F_W(y_2)$
				\item %\vspace{1cm}
				
					  \begin{enumerate}
						\item[disc.] $\iff P_{ij} = P_iP_j \forall i, j$
						\item[cont.] $\iff f_{(V, W)}(x_2, y_2) = f_V(x_2)f_W(y_2)$
					  \end{enumerate}
			\end{enumerate}

			\underline{Remarque :} le cas discret et continu ont bien été différenciés pour le point $3.$ car il est question de la fonction de densité ce qui n'est pas
			pertinent dans le cas d'une variable discrète.

			\underline{Démonstration : } Pour démontrer toutes les équivalences, il faut démontrer $(i) \rightarrow (ii)$, $(ii) \rightarrow (iii)$ et $(iii) \rightarrow (i)$.

			\begin{itemize}
				\item[$(i) \rightarrow (ii)$] Si la propriété $(i)$ est vraie pour tous $x_1, x_2, y_1, y_2$, alors plus précisément avec $x_1, x_2 = -\infty$ :

					  \[P(V \in ]-\infty, x_2] \land W \in ]-\infty, y_2]) = P(V \in ]-\infty, x_2])P(W \in ]-\infty, y_2]) = P(V \leq x_2)P(W \leq y_2) = F_V(x_2)F_W(y_2).\]
				
				\item[$(ii) \rightarrow (iii)$] En utilisant la propriété $(ii)$, on a :

					  \[\begin{aligned}
						\frac {\partial^2}{\partial x \partial y}F_{(V, W)}(x, y) &= \pd{x}\left[\pd{y}F_{(V, W)}(x, y)\right] = \pd{x}\left[\pd{y}\left(F_V(x)F_W(y)\right)\right] \\
						&= \pd{x}\left[F_V(x)\pd{y}F_W(y)\right] = \pd{x}[F_V(x)f_W(y)] = f_W(y)\pd{x}F_V(x) = f_V(x)f_W(y).
					  \end{aligned}\]

				\item[$(iii) \rightarrow (i)$] En intégrant doublement l'égalité $(iii)$ de part et d'autre, on obtient :
				
				\[\begin{aligned}
					\int_{x_1}^{x_2}\int_{y_1}^{y_2}f_V(x)f_W(y)\dif x\dif y &= \int_{x_1}^{x_2}f_V(x)\left(\int_{y_1}^{y_2}f_W(y) \dif y\right)\dif x
						= \int_{x_1}^{x_2}f_V(x)P(W \in [y_1, y_2])\dif x \\
					&= P(W \in [y_1, y_2])\int_{x_1}^{x_2}f_V(x)\dif x = P(V \in [x_1, x_2])P(W \in [y_1, y_2]) \\
					&= \int_{x_1}^{x_2}\int_{y_1}^{y_2}f_{(V, W)}(x, y)\dif x\dif y = P( \in [x_1, x_2] \land W \in [y_1, y_2]). \square
				\end{aligned}\]
			\end{itemize}

			En statistique descriptive à deux dimensions, il y avait le paramètre $m_{11}$ qui permettait d'évaluer la corrélation (linéaire) entre deux événements.
			En probabilités, il existe la même notion définie comme suit :

			\begin{align}
				\mu_{11} &\coloneqq \sum_{i=1}^p\sum_{j=1}^qp_{ij}(x_i-\mu_X)(y_j-\mu_Y) \;\;\;\;&&\text{ si VAD}\\
				\mu_{11} &\coloneqq \int_{-\infty}^\infty\int_{-\infty}^\infty(x-\mu_X)(y-\mu_Y)f_{(X, Y)}(x, y)\dif x\dif y &&\text{ si VAC}.
			\end{align}

			On peut donc définir $X \sqcup Y \Rightarrow \mu_{11} = 0$. En effet :

			\[\begin{aligned}
				\int_{-\infty}^\infty\int_{-\infty}^\infty(x-\mu_X)(y-\mu_Y)f_{(X, Y)}(x, y)\dif x\dif y &\stackrel{X \sqcup Y}{=}
					\int_{-\infty}^\infty\int_{-\infty}^\infty(x-\mu_X)(y-\mu_Y)f_X(x)f_Y(y)\dif x\dif y \\
				&= \int_{-\infty}^\infty(x-\mu_X)f_X(x)\left(\int_{-\infty}^\infty(y-\mu_Y)f_Y(y)\dif y\right)\dif x \\
				&= \int_{-\infty}^\infty(x-\mu_X)f_X(x)\dif x\int_{-\infty}^\infty(y-\mu_Y)f_Y(y)\dif y \\
				&= \left(\int_{-\infty}^\infty y f_Y(y)\dif y - \int_{-\infty}^\infty\mu_Yf_Y(y)\dif y\right)\left(\int_{-\infty}^\infty x f_X(x)\dif x - \int_{-\infty}^\infty\mu_Xf_X(x)\dif x\right) \\
				&= (E(Y) - \mu_Y1)(E(X) - \mu_X1) = (\mu_Y - \mu_Y)(\mu_X - \mu_X) = 0. \square
			\end{aligned}\]
		
		\subsubsection{Transformée d'une variable aléatoire}
			Soit $X$, une VAC avec une fonction de densité $f_X(x)$. Soit une nouvelle variable $Z \coloneqq G(X)$, une transformée croissante monotone de $X$. Comment peut-on
			déterminer la fonction de densité $f_Z(z)$ ?

			Commençons par déterminer ce que vaut $F_Z(z)$ :

			\[F_Z(x) \coloneqq P(Z \leq x) = P(G(X) \leq x) = P(X \leq G^{-1}(x)) = F_X(G^{-1}(x)) = (F_X \circ G^{-1})(x).\]

			Or on sait que $f_Z(x) = \pd xF_Z(x)$. Dès lors :

			\[f_Z(x) = \pd x(F_X \circ G^{-1})(x) = f_X(G^{-1}(x))\pd xG^{-1}(x)\]

			Posons $y(x) \coloneqq G^{-1}(x)$, donc $x = G(y)$, ou encore $\pd xx = \pd xG(y) \iff 1 = G'(y)y'$. Ce qui veut dire que $y' = \frac 1{G'(y)}$. Et par la définition
			de $y$, on a : $y' = \frac 1{G'(G^{-1}(x))}$.
			
			Dès lors :

			\[f_Z(x) = f_X(G^{-1}(x))y'(x) = f_X(G^{-1}(x))\frac 1{G'(G^{-1}(x))}.\]

			Attention, ce résultat n'est valable que si $G$ est une transformée croissante. Si $G$ est décroissante, le raisonnement est le même à l'exception près que lors
			de la probabilité suivante :

			\[P(G(X) \leq x),\]

			la prise de la fonction inverse $G^{-1}(x)$ impose de changer le sens de l'inégalité (car la fonction est décroissante). On a donc :

			\[F_Z(x) = P(X \geq G^{-1}(x)) = 1 - P(X \leq G^{-1}(x)) = 1 - F_X(G^{-1}(x)).\]

			Dès lors, la dérivée donne :

			\[f_Z(x) = -f_X(G^{-1}(x))\frac 1{G'(G(^{-1}(x))}.\]

			Le résultat est donc presque similaire si ce n'est un signe $-$ supplémentaire.

		\subsubsection{Transformée affine}
			Soit $X$ une VAC définie par $f_X$. Si on définit $Z \coloneqq aX + b$ avec $(a, b) \in (\mathbb R_0^+, \mathbb R)$, on peut trouver $f_Z$ de deux manières : soit on passe
			par la formule démontrée ci-dessus, soit on repart de la définition et on évite des calculs fastidieux :

			\[f_Z(x) = \pd xF_Z(x) = \pd xP(Z \leq x) = \pd xP(aX + b \leq x) = \pd xP\left(X \leq \frac {x-b}{a}\right) = \pd xF_X\left(\frac {x-b}a\right) = f_X\left(\frac {x-b}a\right)\frac 1a.\]

		\subsubsection{Opérations arithmétiques}
			Soient $X, Y$ deux VAC définies par $f_X$ et $f_Y$. La variable $Z \coloneqq X + Y$ est également définie par une fonction $f_Z$. Or cette fonction ne peut être
			$f_Z(x) = f_X(x) + f_Y(x)$ car il faut que l'intégrale sur $\overline {\mathbb R}$ soit égale à 1. Or :

			\[\int_{-\infty}^\infty f_Z(x)\dif x = \int_{-\infty}^\infty(f_X(x) + f_Y(x))\dif x = 1 + 1 = 2\]

			ne convient pas du tout à cet axiome ! Cette définition de $f_Z$ n'est donc logiquement pas exacte. Il faut à nouveau partir de la définition de $F_Z$ :

			\[F_Z(t) \coloneqq P(Z \leq t) = P(X+Y \leq t) = \iint_{X+Y \leq t}f_{(X, Y)}(x, y)\dif x\dif y\]

			Pour avancer, il nous faut changer de variables :

			\[\left\{\begin{aligned}
				x &= \beta\\
				y  &= \alpha - \beta
			\end{aligned}\right.\]

			On a donc :

			\[F_Z(t) = \int_{-\infty}^t\left(\int_{-\infty}^\infty f_{(X, Y)}(\beta, \alpha-\beta)\abs{\begin{vmatrix}\pd \alpha x & \pd \beta x \\ \pd \alpha y & \pd \beta y\end{vmatrix}}\dif \beta\right)\dif\alpha\]

			Où

			\[\begin{vmatrix}\pd \alpha x & \pd \beta x \\ \pd \alpha y & \pd \beta y\end{vmatrix} = \begin{vmatrix}0 & 1 \\1 & -1\end{vmatrix}\]

			est le déterminant du jacobien (à prendre en valeur absolue).

			Donc on a au final :

			\[F_Z(t) = \int_{-\infty}^t\left(\int_{-\infty}^\infty f_{(X, Y)}(\beta, \alpha - \beta)\dif \beta\right)\dif \alpha.\]

			Et comme $F_Z(t) = \int_{\infty}^t f_Z(x)\dif x$, on a au final :

			\[f_Z(\alpha) = \int_{-\infty}^\infty f_{(X, Y)}(\beta, \alpha-\beta)\dif \beta\]

			Et si au lieu de faire une somme des deux variables $X$ et $Y$, nous avions voulu les multiplier, le procédé aurait été le même :

			\[F_Z(t) = P(XY \leq t),\]

			et les variables à poser auraient été :

			\[\left\{\begin{aligned}
				x &= \beta \\
				y &= \frac \alpha\beta
			\end{aligned}\right.\]

			et donc le déterminant du jacobien aurait donné :

			\[\abs{\begin{vmatrix}\pd \alpha x & \pd \beta x \\ \pd \alpha y & \pd \beta y\end{vmatrix}} =
			\abs{\begin{vmatrix}0 & 1 \\ \frac 1\beta & -\frac \alpha{\beta^2}\end{vmatrix}} =
			\frac 1{\abs\beta}\]

			de manière à obtenir :

			\[F_Z(t) = \int_{-\infty}^t\left(\int_{-\infty}^\infty f_{(X, Y)}\left(\beta, \frac \alpha\beta\right)\frac {\dif \beta}{\abs\beta}\right)\dif \alpha.\]

			Et donc on peut déterminer $f_Z$ par le même principe :

			\[f_Z(\alpha) = \int_{-\infty}^\infty f_{(X, Y)}\left(\beta, \frac \alpha\beta\right)\frac {\dif\beta}{\abs\beta}.\]

	\subsection{Manipulation de transformées}
		Nous avons trouvé comment déterminer les fonctions de densité pour de nouvelles variables créées par opérations sur d'autres. Maintenant tentons de les manipuler
		pour déterminer l'espérance ou l'écart-type :

		\begin{align}E(G(V)) = \int_{-\infty}^\infty G(v)f_V(v)\dif v\end{align}

		\underline{Démonstration : } On sait par définition :

		\[E(W=G(V)) = \int_{-\infty}^\infty wf_W(w)\dif w,\]

		qui peut être partitionné en $\overline G$ et $\underline G$, les sous-ensemble où $G(v)$ est croissante (respectivement décroissante). On a dès lors :

		\[E(W) = \int_{\overline G}wf_W(w)\dif w + \int_{\underline G}wf_W(w)\dif w.\]

		De là, on applique la formule de la transformée (exprimer $f_W$ en fonction de $f_V$) :

		\[E(W) = \int_{\overline G}wf_V(G^{-1}(w))\frac {\dif w}{G'(G^{-1}(w))} - \int_{\underline G}wf_V(G^{-1})\frac {\dif w}{G'(G^{-1}(w))}.\]

		\underline {Remarque : } Effectivement, le $+$ devient $-$ lors du changement de variable pour la raison vue précédemment (fonction décroissante).
		En réitérant un changement de variable ($w = G(y)$), on rechange le signe $-$ en $+$ et on a donc une somme de deux intégrales de la même fonction sur un partitionnement
		de $\overline {\mathbb R}$ :

		\[E(W) = \int_{-\infty}^\infty G(y)f_V(y)\dif y. \;\;\;\square\]

		On peut également montrer la linéarité de l'opérateur $E(V)$ :

		\[E(aV + b) = \int_{-\infty}^\infty(av+b)f_V(v)\dif v = a\int_{-\infty}^\infty vf_V(v)\dif v + b\int_{-\infty}^\infty f_V(v)\dif v = aE(V) + b. \;\;\;\square\]

		\subsubsection{Espérance d'une somme}
			Montrons que l'espérance d'une somme de variables aléatoires correspond à la somme des espérances.
			En effet, soient $X$ et $Y$ deux VAC. Définissons $Z \coloneqq X+Y$. On peut déterminer l'espérance de cette variable comme suit (par définition) :

			\[E(Z) = \int_{-\infty}^\infty zf_Z(z)\dif z = \int_{-\infty}^\infty zf_{X+Y}(z)\dif z =
				\int_{-\infty}^\infty z\left(\int_{-\infty}^\infty f_{(X, Y)}(u, z-u)\dif u\right)\dif z.\]

			Posons donc $z = \alpha + \beta$ et $u = \alpha$ (le déterminant du jacobien vaut $1$ en valeur absolue).

			\[\begin{aligned}
				E(Z) &= \int_{-\infty}^\infty\int_{-\infty}^\infty(\alpha + \beta)f_{(X, Y)}(\alpha, \beta) \dif \alpha \dif \beta =
				\int_{-\infty}^\infty\int_{-\infty}^\infty\alpha f_{(X, Y)}(\alpha, \beta)\dif\alpha\dif\beta + \int_{-\infty}^\infty\int_{-\infty}^\infty\beta f_{(X, Y)}(\alpha, \beta)\dif\alpha\dif\beta \\
					 &= \int_{-\infty}^\infty\alpha\left(\int_{-\infty}^\infty f_{(X, Y)}(\alpha, \beta)\dif\beta\right)\dif\alpha + \int_{-\infty}^\infty\beta\left(\int_{-\infty}^\infty f_{(X, Y)}(\alpha, \beta)\dif\alpha\right)\dif\beta \\
					 &= \int_{-\infty}^\infty\alpha f_X(\alpha)\dif\alpha + \int_{-\infty}^\infty\beta f_Y(\beta)\dif\beta = E(X) + E(Y).\;\;\;\;\square
			\end{aligned}\]

		\subsubsection{Espérance d'un produit}
			Tout comme pour la somme, montrons que l'espérance d'un produit correspond au produit des équivalences. Soient $X$, $Y$ et $Z \coloneqq XY$.

			\[E(Z) = \int_{-\infty}^\infty zf_Z(z)\dif z = \int_{-\infty}^\infty z\left(\int_{-\infty}^\infty f_{(X, Y)}\left(z, \frac zu\right)\frac {\dif u}{\abs u}\right) \dif z\]

			Posons donc à nouveau $u = \alpha$ et $z = \alpha\beta$ (le déterminant du jacobien vaut $\alpha$ en valeur absolue). On a donc :

			\[E(Z) = \int_{-\infty}^\infty\int_{-\infty}^\infty(\alpha\beta)f_{(X, Y)}(\alpha, \beta)\dif\alpha\dif\beta.\]

			Prenons pour hypothèse $X \sqcup Y$. Dès lors,

			\[E(Z) = \int_{-\infty}^\infty\alpha f_X(\alpha)\dif\alpha\int_{-\infty}^\infty\beta f_Y(\beta)\dif\beta = E(X)E(Y).\;\;\;\;\square\]

		\subsubsection{Calcul de l'écart-type}
			Par la définition de l'écart-type, on a :

			\[D^2(X) = \int_{-\infty}^\infty (x-E(X))^2f_X(x)\dif x.\]

			Donc en définissant une fonction quelconque $G : x \mapsto (x-E(X))^2$, on a alors :

			\[D^2(X) = \int_{-\infty}^\infty G(x)f_X(x)\dif x = E\left[G(X)\right] = E\left[(x-E(X))^2\right].\]

			On a donc pu, en réalité, définir le calcul de l'écart-type sur base du calcul d'espérance. L'avantage est le fait que toute les propriétés de l'espérance sont alors
			utilisables pour l'écart-type. Par exemple, pour déterminer l'écart-type d'une transformée linéaire, on a :

			\[\begin{aligned}
				D^2(aX+b) &= E(((ax+b) - (E(aX+b)))^2) = E((ax+b-(aE(X)+b))^2) = E((ax-aE(X))^2) \\
						  &= E(a^2(x-E(X))^2) = a^2E((x-E(X))^2) = a^2D^2(X).\;\;\;\square
			\end{aligned}\]

			Concernant une somme, on a :

			\[\begin{aligned}
				D^2(X+Y) &= E((x+y-E(X)-E(Y))^2) = E(((x-E(X)) + (y-E(Y)))^2) \\
						 &= E((x-E(X))^2) + E((y-E(Y))^2) + 2E((x-E(X))(y-E(Y))) \\
						 &= D^2(X) + D^2(Y) + 2\int_{-\infty}^\infty\int_{\-\infty}^\infty(x-E(X))(y-E(Y))f_{(X, Y)}(x, y)\dif x\dif y = D^2(X) + D^2(Y) + 2\mu_{11}.
			\end{aligned}\]

			Ce résultat est intéressant car $\mu_{11}$ peut être négatif. Dans le cas où effectivement le coefficient de corrélation linéaire est négatif, on a
			$D^2(X+Y) < D^2(X)+D^2(Y)$. Cela a des applications entre autres en économie : dans ce cas, investir dans deux actions $X$ et $Y$ permet de
			\textit{limiter la variance} donc avoir plus d'uniformité dans les gains. Ce peut être intuitif dans le sens qu'il est possible d'investir uniquement dans une
			action ayant des gains entre janvier et juillet compris ou dans une action ayant des gains uniquement entre août et décembre. Dans ce cas, investir
			\textbf{simultanément} dans les deux actions permet d'avoir des gains pendant toute l'année (donc limiter la variance).

	\subsection{Fonction génératrice des moments}
		On définit une fonction $\Psi_X(t) : t \mapsto E(\exp(tX))$ que l'on appelle \textit{fonction génératrice des moments}. On la nomme ainsi car elle permet de déterminer
		les moments d'ordre $k$ d'une variable aléatoire :

		\[\frac {\partial^k}{\partial t^k}\Psi_X(t)\sVert[2] _{t=0} = \frac {\partial^k}{\partial t^k}\int_{-\infty}^\infty\exp(tx)f_X(x)\dif x = \int_{-\infty}^\infty x^kf_X(x)\dif x = E(X^k).\]

		Cette fonction permet donc effectivement de déterminer les moments d'ordre $k$ en dérivant et en évaluant en $t = 0$. \\
		Ce résultat peut se montrer par le développement en série de l'exponentielle :

		\[\exp(v) = \sum_{i=0}^\infty\frac {v^k}{k!}.\]

		Dès lors, en réinjectant cela dans la définition de $\Psi_X$, on obtient :

		\[\Psi_X(t) = E\left[\sum_{i=0}^\infty\frac {(tX)^k}{k!}\right] = \sum_{i=0}^\infty\frac {t^k}{k!}E(X^k).\]

		Donc en dérivant $k$ fois, les $k$ premiers termes s'annulent (par dérivation du $t$). On a donc :

		\[\frac {\partial^k}{\partial t^k}\Psi_X(t) = \sum_{i=k}^\infty E(X^k)\frac {t^{i-k}}{(i-k)!}.\]

		Dès lors, en évaluant la $k$ème dérivée en $t = 0$, tous les termes sauf le premier s'annulent afin qu'il ne reste que le moment d'ordre $k$ :

		\begin{align}
			\frac {\partial^k}{\partial t^k}\Psi_X(t)\sVert[2]_{t=0} = E(X^k).\;\;\;\;\square
		\end{align}

		\subsubsection{Fonction génératrice d'une somme}
			Il faut savoir qu'une variable aléatoire peut être définie soit par sa fonction de répartition, soit par sa fonction de densité, soit par sa fonction génératrice
			des moments (il est possible de passer de l'un à l'autre de manière univoque). De là, il est aussi important de regarder ce que donnent les opération classiques
			sur la fonction génératrice que sur les autres.

			Dans le cas d'une somme, on a $\Psi_Z(t) = \Psi_{X+Y}(t) = E[\exp(t(X+Y))] = E[\exp(tY)\exp(tX)]$. En prenant pour hypothèse l'indépendance de $X$ et $Y$
			($X \sqcup Y$), on a $\Psi_Z(t) = E[\exp(tX)]E[\exp(tY)] = \Psi_X(t)\Psi_Y(t)$.

\section{Variables aléatoires particulières}
	\subsection{Binomiales et variables de Bernoulli}
		Dans le cas où une expérience aléatoire a pour seuls résultats possibles \textit{la réussite} et \textit{l'échec} avec une probabilité respective de $p$ et $(1-p)$,
		on appelle \textbf{variable de Bernoulli} une variable représentant le résultat de cette expérience. La probabilité d'une telle variable est donc exprimée par :

		\[P(X = x) = \left\{\begin{aligned}&p &\text{ si $x = 1$}\\&q=(1-p)\;\; &\text{ si $x = 0$}\\&0 &\text{ sinon}\end{aligned}\right.\]

		Comme une telle situation est réductrice, on définit une distribution \textbf{normale} (ou \textbf{binomiale}) lorsqu'une variable $X$ est définie comme
		$n$ répétitions successives \textbf{et indépendantes} d'une expérience de Bernoulli. On note cela $X \sim \mathcal B(n, p)$, ce qui veut dire : « $X$ suit une
		distribution binomiale représentant une répétition de $n$ expériences de Bernoulli ayant pour probabilité de réussite $p$ ». $X$ peut donc prendre comme différentes
		valeurs le nombre d'expériences se soldant en réussite. Dès lors, on a :

		\[P(X = x) = \binom nkp^k(1-p)^{n-k}.\]

		\subsubsection{Les axiomes de Kolmogorov}
			À nouveau, pour s'assurer de l'appellation \textit{probabilité}, il faut vérifier que les axiomes de Kolmogorov soient vérifiés :

			$P(X=k) \geq 0 \; \forall k$. En effet, $P(X=k)$ est donné par un produit de valeurs toutes positives. Dès lors, la probabilité en elle-même est positive.

			$P(E) = 1.$ En effet :
			
			\[P(E) = P\left(\bigcup_{k=0}^nX=k\right) = \sum_{k=0}^kP(X=k) = \sum_{k=0}^n\binom nkp^k(1-p)^{n-k} = \left[p + (1-p)\right]^n = 1.\]

		\subsubsection{Valeurs particulières}
			L'intérêt de définir des variables \textit{spéciales} comme les binomiales ou d'autres (plus loin) est de pouvoir trouver des règles correspondantes
			à toute variable de ce type. En effet, une variable binomiale est paramétrée par les valeurs $n$ et $p$, respectivement le nombre de répétitions de l'expérience
			et la probabilité de réussite. Cependant, on peut tout de même exprimer $E(X)$ et $D^2(X)$ pour une variable de ce type :

			\[\begin{aligned}
				E(X) &= \sum_{i=0}^niP(X=i) = \sum_{i=1}^niP(X=i) = \sum_{i=1}^ni\binom nip^i(1-p)^{n-i} = \sum_{i=1}^n\frac {in!}{i!(n-i)!}p^i(1-p)^{n-i} \\
				&= \sum_{i=1}^n\frac {n!}{(i-1)!(n-i)!}p^i(1-p)^{n-i} = np\sum_{i=1}^n\frac {(n-1)!}{(i-1)!(n-i)!}p^{i-1}(1-p)^{n-i} = np1^{n-1} = np.
			\end{aligned}\]

			De même manière, on peut déterminer la variance. Mais commençons par faire une remarque applicable à \textbf{toute} VAC :

			\[D^2(X) = E((X-E(X))^2) = E(X^2) + E(E(X)^2) - 2E(XE(X)) = E(X^2) - 2E(X)E(E(X)) + E(X^2) = E(X^2) - E(X)^2.\]

			Appliquons donc cette formule pour déterminer la variance d'une variable binomiale :

			\[\begin{aligned}
				D^2(X) &= E(X^2) - E(X)^2 = \sum_{i=0}^ni^2P(X=i) - (np)^2 = \sum_{i=1}^ni^2P(X=i) - (np)^2 = \sum_{i=1}^n(i(i-1) + i)P(X=i) - (np)^2 \\
				&= \sum_{i=2}^n\frac {i(i-1)n!}{i!(n-i)!}p^i(1-p)^{n-i} + \sum_{i=1}^niP(X=i) - (np)^2 = \sum_{i=2}^n\frac {n!}{(i-2)!(n-i)!}p^i(1-p)^{n-i} + np - (np)^2\\
				&= n(n-1)p^2\sum_{i=2}^n\frac {(n-2)!}{(i-2)!(n-i)!}p^{i-2}(1-p)^{n-i} + np - (np)^2 = n(n-1)1^{n-2} + np - (np)^2 \\
				&= n^2p^2 - np^2 + np - (np)^2 = np(1-p).
			\end{aligned}\]

			On pouvait également utiliser la fonction génératrice des moments pour déterminer $E(X^2)$. Commençons par expliciter cette fonction dans le cas d'une binomiale :

			\[\begin{aligned}
				\Psi(t) &= E(\exp(tX)) = \sum_{i=0}^n\exp(ti)P(X=i) = \sum_{i=0}^n\exp(ti)\binom nip^i(1-p)^{n-i} = \sum_{i=0}^n\binom ni(\exp(t)p)^i(1-p)^{n-i} \\
				&= (\exp(t)p + (1-p))^n.
			\end{aligned}\]

			Dès lors, pour trouver $E(X)$, le moment d'ordre 1, on a :

			\[\pd t\Psi(t)\sVert[2]_{t=0} = \pd t\left[(1-p+\exp(t)p)^n\right]\sVert[3]_{t=0} = \left[n(1-p+\exp(t)p)^{n-1}p\exp(t)\right]\sVert[3]_{t=0} = n(1)^{n-1}p = np.\]

			Ensuite, pour $E(X^2)$, le moment d'ordre 2, on a :

			\[\begin{aligned}
				\frac {\partial^2}{\partial t^2}\Psi(t)\sVert[2]_{t=0} &= \pd t\left[n(1-p+\exp(t)p)^{n-1}p\exp(t)\right]\sVert[3]_{t=0} \\
				&= \left[n(n-1)(1-p+\exp(t)p)^{n-2}p\exp(t)p\exp(t) + n(1-p+\exp(t))^{n-1}p\exp(t)\right]\sVert[3]_{t=0} \\
				&= \left[n(n-1)(1-p+\exp(t)p)^{n-2}p^2\exp(2t) + n(1-p+\exp(t)p)^{n-1}p\exp(t)\right]\sVert[3]_{t=0} \\
				&= n(n-1)(1)^{n-2}p^2 + n(1)^{n-1}p = n^2p^2 - np^2 + np.
			\end{aligned}\]

			Et donc $D^2(X) = E(X^2) - E(X)^2 = -np^2 + np = np(1-p)$.

		\subsubsection{stabilité des binomiales}
			On a vu comment appliquer des opérations sur des variables aléatoires en les additionnant ou multipliant. Il est cependant intéressant de voir si des opérations sur
			deux variables de même type donnent une troisième variable de ce même type. Une bonne lanière de procéder à cela est de regarder la fonction génératrice des moments
			et de voir si on ré-obtient une fonction similaire. Soient $X \sim \mathcal B(n_1, p), Y \sim \mathcal B(n_2, p)$ tels que $X \sqcup Y$. Soit $Z = X + Y$.

			\[\Psi_Z(t) = E(\exp(tx))E(\exp(ty)) = (1-p+\exp(t)p)^{n_1}(1-p+\exp(t)p)^{n_2} = (1-p+\exp(t))^{n_1+n_2}.\]

			Donc Effectivement, on a bien $Z \sim \mathcal B(n_1+n_2, p)$. Cependant, si on définit $X \sim \mathcal B(n, p_1)$ et $Y \sim \mathcal B(n, p_2)$, alors on obtient :

			\[\Psi_Z(t) = \Psi_X(t)\Psi_Y(t) = (1-p_1+\exp(t)p_1)^n(1-p_2+\exp(t)p_2)^n = ((1-p_1+\exp(t)p_1)(1-p_2+\exp(t)p_2))^n.\]

			Et dans ce cas là, on ne peut pas faire ressembler cela à une fonction génératrice des moments d'une binomiale.

			Donc pour résumer, en sommant deux binomiales de même probabilité $p$ mais avec chacune leur nombre de répétitions $n_1$ et $n_2$, on obtient une troisième binomiale
			ayant cette même probabilité $p$ et un nombre de répétition $n_1+n_2$. Ce qui peut sembler intuitif, les répétitions étant indépendantes, les faire dans
			une première variable ou dans la seconde ne change rien au final et il y a bien $n_1+n_2$ répétitions d'une expérience aléatoire (\textit{a priori} identique ou
			assimilable) alors que si les binomiales n'ont pas la même probabilité, les sommer n'a pas de sens car elles représentent des choses trop différentes.

	\subsection{Variables de Poisson}
		Une variable de Poisson est une particularisation d'une binomiale soumise à des conditions particulières. Elles servent typiquement à représenter le nombre d'occurrences
		d'un événement dans un intervalle de temps $T$ donné. On peut visualiser cela comme une binomiale

		\[\mathcal B(\frac T{\Delta t}, a\Delta t).\]

		Or, pour réellement parler de temps, il faut approcher le continu par le discret. On fait donc tendre $\Delta t \to 0$. Dès lors, on a :

		\begin{itemize}
			\item $n \to \infty$ ;
			\item $p \to 0$ ;
			\item $np \to \frac T{\Delta t}a\Delta t = aT = \lambda$.
		\end{itemize}

		Ici, la valeur $\lambda$ est le paramètre de la distribution et est défini par $\lambda \coloneqq np$.

		\subsubsection{Distribution}
			Pour exprimer la fonction de distribution d'une variable de Poisson, partons de la distribution binomiale :

			\[P(X=k) = \binom nkp^k(1-p)^{n-k} = \frac {n!}{i!(n-i)!}p^i(1-p)^{n-k}.\]

			Et le quotient binomial $\binom nk$ peut se réécrire sous la forme :

			\[\frac {n!}{k!(n-k)!} = \frac 1{k!}\prod_{i=n-k+1}^ni = \frac 1{k!}n(n-1)\ldots(n-k+2)(n-k+1) = \frac 1{k!}n^k1(1-\frac 1n)(1-\frac 2n)\ldots(n-\frac {k-1}n).\]

			De plus, la quantité $(1-p)^{n-k}$ peut se réécrire sous la forme :

			\[(1-p)^{n-k} = \frac {(1-p)^n}{(1-p)^k} = \frac {(1-\frac {np}n)^n}{(1-p)^k} = \frac {(1-\frac \lambda n)^n}{(1-p)^k}.\]

			Dès lors, pour déterminer la distribution d'une variable de Poisson, on soumet la formule de la binomiale aux conditions décidées :

			\[\begin{aligned}
				P_\lambda(X=k) &= \lim_{p \to 0}\lim_{n\to\infty}P_{\mathcal B(n, p)}(X=k) = \lim_{p \to 0}\lim_{n \to \infty}\binom nkp^k(1-p)^{n-k} \\
				&= \lim_{p \to 0}\lim_{n \to \infty}\frac {n^k(1-\frac 1n)\ldots(1-\frac {k+1}n)}{k!}p^k\frac {(1-\frac \lambda n)^n}{(1-p)^k} \\
				&= \exp(-\lambda)\lim_{p \to 0}\lim_{n \to \infty}p^kn^k\frac 1{k!} = \frac {\lambda^k}{k!}\exp(-\lambda).
			\end{aligned}\]

		\subsubsection{Vérification des axiomes}
			Une fois de plus, une nouvelle distribution implique un nouvelle vérification de l'application des axiomes de Kolmogorov. De plus, il faut vérifier que la
			fonction de distribution est bien définie sur tout son domaine. Le domaine de $P_\lambda$ est $\mathbb N$, donc $k!$ est définie partout. De plus, $\exp$ a pour
			ensemble d'arrivée $\mathbb R_0^+$. Donc $P_\lambda(X=k)$ est un produit de valeurs positives, ce qui satisfait l'axiome 1. Pour le second, il faut :

			\[P(E) = P\left(\bigcup_{k=0}^\infty X=k\right) = \sum_{k=0}^\infty P(X=k) = \sum_{k=0}^\infty \frac {\lambda^k}{k!}\exp(-\lambda) = \exp(-\lambda)\sum_{k=0}^\infty \frac {\lambda^k}{k!} = \exp(-\lambda)\exp(\lambda) = 1.\]

		\subsubsection{Valeurs particulières}
			Nous avons trouvé pour une binomiale que $E(\mathcal B(n, p)) = np$. Faisons de même pour une variable de Poisson. Soit $X \sim P_\lambda$.

			\[E(X) = \sum_{k=0}^\infty kP(X=k) = \sum_{k=1}^\infty k\frac {\lambda^k}{k!}\exp(-\lambda) = \exp(-\lambda)\lambda\sum_{k=1}^\infty \frac {\lambda^{k-1}}{(k-1)!} = \lambda\exp(-\lambda)\exp(\lambda) = \lambda.\]

			Ce résultat est cohérent : dans une binomiale, $E(X) = np$ or une Poisson est un cas particulier de binomiale. Dès lors, il est cohérent d'obtenir $E(X) = \lambda$.

			Pour la variance, utilisons $D^2(X) = E(X^2)-E(X)^2$. Déterminons $E(X)$ ) l'aide de la fonction génératrice des moments :

			\[\Psi_\lambda(t) = E(\exp(tX)) = \sum_{k=0}^\infty \exp(tk)\frac {\lambda^k}{k!}\exp(-\lambda) = \sum_{k=0}^\infty \frac {(\exp(t)\lambda)^k}{k!}\exp(-\lambda)
			= \exp(-\lambda)\exp(\exp(t)\lambda) = \exp(\lambda(\exp(t)-1)).\]

			On peut vérifier l'espérance en dérivant une fois et en évaluant en $t=0$, maintenant regardons $E(X^2)$ :
			
			\[\begin{aligned}
				\frac {\partial^2}{\partial t^2}\Psi_X(t)\sVert[3]_{t=0} &= \pd t\left[\exp(\lambda(\exp(t)-1))\lambda\exp(t)\right]\sVert[3]_{t=0} 
					= \lambda\pd t\left[\exp(\lambda(\exp(t)-1) + t)\right]\sVert[3]_{t=0} \\
				&= \lambda\left[\exp(\lambda(\exp(t) - 1) + t)(\lambda\exp(t)+1)\right]\sVert[3]_{t=0} = \lambda(\lambda+1).
			\end{aligned}\]

			Ce que l'on peut conclure en :

			\[D^2(X) = E(X^2)-E(X)^2 = \lambda(\lambda+1) - \lambda^2 = \lambda^2 + \lambda - \lambda^2 = \lambda.\]

			À nouveau, ce résultat est cohérent par rapport à l'origine de la variable de Poisson qui vient de la binomiale. En effet, la variance d'une binomiale est donnée par
			$np(1-p)$ et dans une variable de Poisson, $np$ tend vers $\lambda$ alors que $p$ tend vers $0$. C'est donc logique que $np(1-p)$ tende vers $\lambda(1-0) = \lambda$.

		\subsubsection{Stabilité des Poissons}
			Tout comme pour les binomiales regardons si la famille des variables de Poisson est stable. Commençons par l'addition. Soient $X \sim P_{\lambda_1}$ et
			$Y \sim P_{\lambda_2}$ tels que $X \sqcup Y$. Soit $Z = X+Y$.

			\[\begin{aligned}
				\Psi_Z(t) &= \Psi_X(t)\Psi_Y(t) = \exp(\lambda_1(\exp(t)-1))\exp(\lambda_2(\exp(t)-1) \\
				&= \exp(\lambda_1(\exp(t)-1) + \lambda_2(\exp(t)-1))) = \exp((\exp(t)-1)(\lambda_1 + \lambda_2)).
			\end{aligned}\]

			$Z$ est donc défini comme étant également une Poisson de paramètre $\lambda = \lambda_1 + \lambda_2$. Les Poissons sont donc stables pour l'addition.

	\subsection{Les exponentielles négatives}
		En considérant un événement pouvant se produire pendant un intervalle de temps $\Delta t$ avec une probabilité de $a\Delta t$ tel que
		$P(X \leq T + t \, | \, X > T) = P(X \geq t)$, on a :

		\[F(t+\Delta t) - F(t) = (1-F(t))a\Delta t.\]

		En faisant tendre $\Delta t \to 0$ et en divisant de chaque côté par $\Delta t$, on obtient une équation différentielle :

		\[F'(t) = (1-F(t))a.\]

		Qui se résout en :

		\[F(t) = 1-C\exp(-at).\]

		En sachant qu'il y a pour condition initiale $F(0) = 0$, on trouve $C = 1$, ou encore $F(t) = 1 - \exp(-at)$ pour $t \geq 0$. Une telle variable aléatoire est appelée
		variable exponentielle négative. Si on a :

		\[F(t) = \left\{\begin{aligned}
			&1-\exp(-at)\,\, &\text{ si $t \geq 0$}, \\
			& 0 &\text{ sinon}.
		\end{aligned}\right.\]

		Et par définition de la fonction de densité, on a également :

		\[f(t) = \left\{\begin{aligned}a\exp(-at)\,\,&\text{ si $t \geq 0$},\\&0 &\text{ sinon}.\end{aligned}\right.\]

		\subsubsection{Axiomes}
			La vérification des axiomes se fait à nouveau : le premier axiome est vérifié car $a > 0$ et $\exp$ a $\mathbb R_0^+$ pour domaine d'arrivée. Pour le second axiome,
			il faut :

			\[\int_{-\infty}^\infty f(t)\dif t = \int_0^\infty a\exp(-at)\dif t = [-\exp(-at)]_0^\infty = 0 - (-1) = 1.\]

		\subsubsection{Valeurs particulières}
			Les valeurs particulières d'une exponentielle négative (notée $T \sim \text{Exp}_a$) est donnée par :

			\[E(T) = \int_{-\infty}^\infty tf_T(t)\dif t = \int_0^\infty tf_T(t)\dif t = \int_0^\infty t(-a\exp(-at))\dif t.\]

			Posons donc $f(t) = t$ et $g'(t) = -a\exp(-at)$, d'où $f'(t) = 1$ et $g(t) = \exp(-at)$. Dès lors :

			\[E(T) = [t\exp(-at)]_0^\infty - \int_0^\infty \exp(-at)\dif t = 0 - [-\frac 1a\exp(-at)]_0^\infty = -\left(-\frac 1a\right) = \frac 1a.\]

			Pareillement pour la variance :

			\[D^2(T) = E(T^2)-E(T)^2.\]

			Passons donc par la fonction génératrice des moments (supposons $a > t$) :

			\[\Psi_X(t) = E(\exp(tX)) = \int_{-\infty}^\infty \exp(tx)(-a\exp(-ax))\dif x = -a\int_0^\infty \exp((t-a)x).\]

			Posons $\omega = (t-a)x$ donc $\dif\omega = (t-a)\dif x$. Dès lors :

			\[-\frac a{t-a}\int_{-\infty}^0\exp(\omega)\dif\omega = \frac a{a-t}[1-0] = \frac a{a-t}.\]

			On peut donc calculer les moments d'ordre 1 et 2 :

			\[E(X) = \pd t\Psi(t)\sVert[3]_{t=0} = \pd t\left[\frac a{a-t}\right]\sVert[3]_{t=0} = \left[\frac a{(a-t)^2}\right]\sVert[3]_{t=0} = \frac a{a^2} = \frac 1a.\]

			Pour le moment d'ordre 2 :

			\[E(X^2) = \pd t\left[\frac a{(a-t)^2}\right]\sVert[3]_{t=0} = \left[\frac {2a}{(a-t)^3}\right]\sVert[3]_{t=0} = \frac 2{a^2}.\]

			Et comme on a $D^2(X) = E(X^2) - E(X)^2$, on sait : $D^2(X) = \frac 2{a^2} - \left(\frac 1a\right)^2 = \frac 1{a^2}$.

		\subsubsection{Stabilité}
			Comme nous allons le voir, contrairement aux familles vues précédemment, les exponentielles négatives ne sont pas stables même pour l'addition. Soient $X, Y$ deux
			variables indépendantes telles que $X \sim $ Exp$_a$, $Y \sim $ Exp$_b$. Soit $Z = X + Y$. On sait $\Psi_Z(t) = \Psi_Xt)\Psi_Y(t) = \frac a{a-t}\frac b{b-t}$.
			Il n'existe pas de $c \in \mathbb R^+$ tel que $\Psi_Z(t) = \frac c{c-t}$. Ce qui montre que les exponentielles négatives ne sont pas stables pour l'addition.

		\subsubsection{Propriété d'oubli}
			Comme mentionné plus haut, pour une telle variable, on a :

			\[P(T \leq t + \Delta t \, | \, ) = P(T \geq \Delta t).\]

			Pour le \textit{démontrer}, il faut utiliser la définition de probabilité conditionnelle.

			\[\begin{aligned}
				P(T \leq t + \Delta t \, | \, T \geq t) &= \frac {P((T \leq t + \Delta t) \cap (T \geq t))}{P(T \geq t} = \frac {P(t \leq T \leq t+\Delta t)}{1-P(T \leq t)} \\
				&= \frac {F(t+\Delta t) - F(t)}{1-F(t)} = \frac {1-\exp(-a(t+\Delta t) - (1-\exp(-at)))}{1-(1-\exp(-at))} = \frac {-\exp(-a(t+\Delta t)) + \exp(-at)}{\exp(-at)} \\
				&= \frac {\exp(-at) - \exp(-a(t+\Delta t))}{\exp(-at)} = 1 - \frac {\exp(-a(t+\Delta t))}{\exp(-at)} = 1 - \exp(-a\Delta t) = F(\Delta t).
			\end{aligned}\]

			Ce résultat veut dire que peu importe quand tombe $t$, la probabilité que l'événement apparaisse dans l'intervalle $\Delta t$ suivant est constante.

	\subsection{Variable normale}
		Une variable aléatoire est soit continue soit discrète. SI elle est continue, elle peut être une exponentielle négative comme vu ci-dessus, mais elle peut également être
		une \textit{variable normale}. Ce type de variable représente une répartition totalement aléatoire et se note $\Nms$ où $\mu$ et $\sigma$ sont respectivement
		la moyenne et l'écart type. Une normale suit la distribution suivante :

		\[f_{\Nms)}(x) = \frac 1{\sqrt {2\pi\sigma^2}}\exp\left(-\frac {(x-\mu)^2}{2\sigma^2}\right).\]

		\subsubsection{Axiomes de Kolmogorov}
			Une fois de plus, vérifions les axiomes de Kolmogorov afin de vérifier l'appellation \textit{fonction de densité} :

			\[\int_{-\infty}^{+\infty}f_{\Nms}(x)\dif x = \frac 1{\sqrt {2\pi\sigma^2}}\int_{-\infty}^{+\infty}\exp\left(-\frac {(x-\mu)^2}{2\sigma^2}\right)\dif x.\]

			Posons $y \coloneqq\frac {x-\mu}{\sqrt 2\sigma}$, dès lors $\dif y = \frac {\dif x}{\sqrt 2 \sigma}$.

			\[\int_{-\infty}^{+\infty}f_{\Nms}(x)\dif x = \frac 1{\sqrt \pi}\int_{-\infty}^{+\infty}\exp(-y^2)\dif y.\]

			Or l'intégrale sur $\mathbb R$ de $\exp(-x^2)$ est une intégrale bien connue et vaut $\sqrt \pi$. Dès lors, l'intégrale sur $\mathbb R$ de la fonction de distribution
			d'une normale est bien égale à 1.

			De plus, $f_{\Nms)}$ est un produit de valeurs positives pour tout $x$ et est donc positive partout.

		\subsubsection{Intérêts de la normale $\Nms$}
			En plus de pouvoir modéliser une répartition aléatoire, les variables aléatoires ont pour avantage d'être très simples à calculer : un ordinateur n'a aucun mal à évalue $\exp(x)$
			pour n'importe quel $x$, contrairement à une factorielle $k!$ qui devient difficile à calculer au dessus de valeurs de $k$ valant $\simeq$ 40. On va donc utiliser les variables
			normales afin d'approximer d'autres distributions en les soumettant à des caractéristiques particulières.

			De plus, les normales sont centrales en probabilités dû à un théorème (le théorème central limite qui sera vu plus loin).

		\subsubsection{Valeurs particulières}
			La définition d'une variables $\Nms$ fait intervenir deux paramètres : $\mu$ et $\sigma$. Prouvons que ces deux variables correspondent bien à la moyenne et à l'écart type
			de la distribution. Soit $X \sim \Nms$.

			\[E(X) = \int_{-\infty}^{+\infty}xf_{\Nms}(x)\dif x = \frac 1{\sqrt {2\pi\sigma^2}}\int_{-\infty}^{+\infty}x\exp\left(-\frac {(x-\mu)^2}{2\sigma^2}\right)\dif x.\]

			Posons $y \coloneqq\frac {x-\mu}{2\sigma}$. Dès lors $\dif y = \frac {\dif x}{2\sigma}$. On a donc :

			\[\begin{aligned}
				&= \frac 1{\sqrt {2\pi\sigma^2}}\int_{-\infty}^{+\infty}(\sqrt {2\sigma^2}y + \mu)\exp(-y^2)\sqrt {2\sigma^2}\dif y \\
				&= \frac 1{\sqrt \pi}\left[\int_{-\infty}^{+\infty}(\sqrt {2\sigma^2}y + \mu)\exp(-y^2)\dif y\right] \\
				&= \frac 1{\sqrt \pi}\left[\int_{-\infty}^{+\infty}\sqrt {2\sigma^2}y\exp(-y^2)\dif y + \int_{-\infty}^{+\infty}\mu\exp(-y^2)\dif y\right] \\
				&= \frac 1{\sqrt{\pi}}\left(0 + \mu\int_{-\infty}^{+\infty}\exp(-y^2)\dif y\right) = \frac 1{\sqrt \pi}\mu\sqrt {\pi} = \mu.
			\end{aligned}\]

			En effet, vu que la première intégrale est une fonction impaire sur des bornes opposées, elle vaut 0. On a donc bien $E(\Nms) = \mu$. Il reste à prouver que $D^2(\Nms) = \sigma^2$.
			Pour cela, utilisons la propriété disant $D^2(X) = E(X^2) - E(X)^2$. Donc afin de trouver $E(X^2)$, utilisons la fonction génératrice des moments donnée par $\Psi_X(t) = E(\exp(tx))$ :

			\begin{align*}
				\Psi_X(t) &= \int_{-\infty}^{+\infty}\exp(tx)\frac 1{\sqrt {2\pi\sigma^2}}\exp\left(-\frac {(x-\mu)^2}{2\sigma^2}\right)\dif x \\
			              &= \frac 1{\sqrt {2\pi\sigma^2}}\int_{-\infty}^{+\infty}\exp\left(-\left[\frac {(x-\mu)^2}{2\sigma^2} - tx\right]\right)\dif x \\
						  &= \frac 1{\sqrt {2\pi\sigma^2}}\int_{-\infty}^{+\infty}\exp\left(-\left[\frac {(x-\mu)^2}{2\sigma^2} - tx + \mu t + \frac {(\sigma t)^2}2 - \mu t - \frac {(\sigma t)^2}2\right]\right)\dif x \\
						  &= \frac {\exp(\mu t + \frac {(\sigma t)^2}2)}{\sqrt {2\pi\sigma^2}}\int_{-\infty}^{+\infty}\exp\left(\left(\frac {x-\mu}{\sqrt 2\sigma} - \frac {\sigma t}{\sqrt 2}\right)^2\right)\dif x.
			\end{align*}

			Posons $y \coloneqq \frac {x-\mu}{\sqrt 2\sigma} - \frac {t\sigma}{\sqrt 2}$, donc $\dif y = \frac {\dif x}{\sqrt 2}$. Dès lors :

			\begin{align*}
				\Psi_X(t) &= \frac {\exp(\mu t + \frac {(\sigma t)^2}2)}{\sqrt {2\pi\sigma^2}}\int_{-\infty}^{+\infty}\exp(-y^2)\sqrt 2 \sigma \dif y \\
				          &= \frac 1{\sqrt {2\pi\sigma^2}}\exp(\mu t + \frac {t^2\sigma^2}2)\sqrt 2\sqrt \pi \sigma \\
						  &= \exp(\mu t + \frac {t^2\sigma^2}2).
			\end{align*}

			Dérivons la maintenant deux fois et instancions-la en $t = 0$ afin de déterminer $E(X^2)$ :

			\begin{align*}
				\pd x\Psi_X(t) = \Psi_X(t)(\mu  + t\sigma^2).
			\end{align*}

			Et donc :

			\begin{align*}
				\pd x\pd x\Psi_X(t) = \Psi_X(t)(\mu + t\sigma^2)^2 + \sigma^2\Psi_X(t).
			\end{align*}

			Qui, instancié en 0 donne : $1(\mu)^2 + \sigma^21$. Dès lors, on sait trouver $D^2(X) = E(X^2) - E(X)^2 = \mu^2 + \sigma^2 - \mu^2 = \sigma^2$.

			Nous avons bien trouvé, ici, que $\mu$ et $\sigma$ représentent respectivement la moyenne et l'écart type de la distribution normale.

		\subsubsection{Stabilité des normales}
			Avec la fonction génératrice des moments, il nous est possible de déterminer si la famille des normales est une famille stable ou non. Soient $X \sim \mathcal N(\mu_1, \sigma_1)$
			et $Y \sim \mathcal N(\mu_2, \sigma_2)$, $Z = X + Y$. On a :

			\begin{align*}
				\Psi_Z(t) &= \Psi_X(t)\Psi_Y(t)  \exp\left(\mu_1 t + \frac {(t\sigma_1)^2}{2}\right) \exp\left(\mu_2 t + \frac {(t\sigma_2)^2}{2}\right) \\
				          &= \exp\left((\mu_1+\mu_2) t + t^2\frac {\sigma_1^2 + \sigma_2^2}{2}\right).
			\end{align*}

			Ce qui montre donc que la somme de deux normales est à nouveau une normale, telle que $Z \sim \mathcal N(\mu_1+\mu_2, \sqrt {\sigma_1^2 + \sigma_2^2})$.

		\subsubsection{La normale $\Nzo$}
			Soient $X \sim \Nms$, $Z = \frac {X-\mu}\sigma$. On sait exprimer :

			\begin{align*}
				\Psi_Z(t) &= E\left(\exp\left(t\frac {X-\mu}\sigma\right)\right) = E\left(\exp\left(\frac {tx}\sigma\right)\exp\left(\frac {-t\mu}\sigma\right)\right) =
				\exp\left(\frac {-t\mu}\sigma\right)E\left(\exp\left(\frac {tx}\sigma\right)\right) \\
				          &= \exp\left(\frac {-t\mu}\sigma\right)\Psi_X\left(\frac t\sigma\right) =
				\exp\left(\frac {-t\mu}\sigma\right)\exp\left(\mu \frac t\sigma + \frac {\frac {t^2}{\sigma^2}\sigma^2}2\right) =
			    \exp\left(\frac {-\mu t}\sigma + \frac {\mu t}\sigma + \frac {t^2}2 \right) = \exp\left(\frac {t^2}2\right) = \Psi_{\Nzo}(t).
			\end{align*}

			La fonction génératrice des moments de $Z$ est la fonction génératrice des moments d'une normale de moyenne 0 et d'écart-type 1. Cette propriété veut dire
			que toute distribution normale peut être ramenée à une normale $\Nzo$. Ce qui est extrêmement intéressant : pour calculer $P(\Nms \leq \alpha)$ pour toute
			distribution normale, il faudrait calculer une infinité de tables de valeurs. Maintenant, on sait que $P(\Nms \leq \alpha) = P(\Nzo \leq \frac {\alpha - \mu}\sigma)$.
			De plus, comme une variable $\Nzo$ est symétrique, $\forall \delta \geq 0, P(\Nzo \leq -\delta) = 1 - P(\Nzo \leq \delta)$. Les variales normales sont
			donc finalement très simples à calculer : il existe énormément de tables avec des précisions différentes.

\section{Théorèmes importants}
	\subsection{Théorème central limite}
		Soient $V_1, V_2, \ldots, V_n$, des variables aléatoires quelconques\footnote{Possiblement de distributions totalement différentes.}, $V = \sum_{i=1}^nV_i$ telles que
		$V_i \sqcup V_j \; \forall i \neq j$. Alors :

		\[\frac {V - \mu_V}{\sigma_V} \stackrel{n \to +\infty}{\longrightarrow} \Nzo.\]

		\underline{Démonstration :} Soient $V_1, \ldots, V_n$ des variables aléatoires iid\footnote{Indépendantes et Identiquement Distribuées.}. Soient $V = \sum_{i=1}^nV_i$,
		$Z = \frac {V - E(X)}{D(X)}$. Supposons que $\forall i : E(V_i) = \bar \mu, D(V_i) = \bar \sigma$. On sait que :

		\[\Psi_Z(t) = E\left[\exp\left(t\frac {V-E(V)}{D(V)}\right)\right].\]

		On sait également que :
	
		\[E(V) = E\left(\sum_{i=1}^nV_i\right) = \sum_{i=1}^nE(V_i) = \sum_{i=1}^n\bar \mu = n\bar \mu.\]

		Posons $W_i \coloneqq V_i  \bar \mu$. Dès lors, on sait que $E(W_i) = \bar \mu - \bar \mu = 0$. Donc :

		\begin{align*}
			\Psi_Z(t) &= E\left[ \exp\left(\frac t\sigma(V-E(V))\right) \right] = E\left[ \exp\left(\frac t\sigma \left(\sum_{i=1}^nV_i - \sum_{i=1}^n\bar\mu\right)\right) \right] \\
	                  &= E\left[ \exp\left(\frac t\sigma \sum_{i=1}^n(V_i-\bar\mu)\right) \right] = E\left[ \exp\left(\frac t\sigma\sum_{i=1}^nW_i\right) \right].
		\end{align*}
	
		On sait également que :

		\[\sigma^2 = D^(V) = D\left(\sum_{i=1}^nV_i\right) = \sum_{i=1}^nD^2(V_i) = \sum_{i=1}^n\bar\sigma = n\bar\sigma.\]

		Or en reprenant l'égalité précédente, on a :

		\begin{align*}
			\Psi_Z(t) &= E\left[ \exp\left(\frac t\sigma \sum_{i=1}^nW_i\right) \right] = E\left[ \prod_{i=1}^n\exp\left(\frac t\sigma W_i\right) \right] =
			\prod_{i=1}^nE\left[ \exp\left(\frac t\sigma W_i\right) \right] = \prod_{i=1}^n\Psi_{W_i}\left(\frac t\sigma\right) \\
		              &\stackrel{iid}= \left(\Psi_{W_j}\left(\frac t\sigma\right)\right)^n \;\;\;\;\;\text{ où $j \in \{1, 2, \ldots, n\}$ est choisi sans importance}.
		\end{align*}

		En exprimant $\Psi_{W_j}(t)$ sous la forme suivante :

		\[\Psi_{W_j}(t) = E[\exp(tW_j)] = E\left[\sum_{k=0}^{+\infty}\frac {(tW_j)^k}{k!}\right] = \sum_{k=0}^{+\infty}E(W_j^k) = 1 + E(W_j)t + E\left(\frac {W_j^2}2\right)t^2 + R(t^3),\]

		où $R(t^3)$ est un reste en $O(t^3)$ et $E(W_j) = 0$, on peut déterminer $D^2(W_j) = E(W_j^2) - E(W_j)^2 = E(W_j^2)$ et $D^2(W_j) = D^2(V_j - \bar\mu) = D^2(V_j) = \bar{\sigma}^2$.
		Dès lors, on a :

		\[\Psi_Z(t) = \left(\Psi_{W_j}\left(\frac t\sigma\right)\right)^n = \left(\Psi_{W_j}\left(\frac t{\sqrt n\bar\sigma}\right)\right)^n = \left( 1 + \frac {\bar{\sigma}^2}2\frac {t^2}{n\bar{\sigma}^2} + R\left(\left(\frac t{\sqrt n\bar\sigma}\right)^3\right)\right)^n.\]

		Le reste en $\frac 1{\sqrt {n^3}}$ devient négligeable quand $n \to +\infty$. Dès lors, on a :

		\[\Psi_Z(t) = \left(1 + \frac {t^2}{2n}\right)^n \stackrel{n\to+\infty}=\exp\left(\frac {t^2}2\right) = \Psi_{\Nzo}(t).\]

		Cela prouve bien qu'une somme infinie de variables aléatoires iid\footnote{hypothèse pour cette démonstration-ci qui peut être supprimée à l'aide d'une autre démonstration.} est une
		variable normale. Les variables normales sont donc effectivement totalement \textbf{centrales} dans le domaine des probabilités.i

	\subsection{Approximation par une normale}
		\subsubsection{Approximer une binomiale}
			Une variable binomiale peut vite devenir difficile à évaluer à cause des factorielles qui deviennent vite élevées et qu'il faut sommer bon nombre de fois. Il est donc appréciable de
			trouver un moyen d'approximer la valeur d'une binomiale. Comme on sait que la binomiale est stable pour l'addition, on peut écrire :

			\[B(n, p) = \sum_{i=1}^nB(1, p).\]

			Par le théorème central limite, on sait que plus $n$ est grand, plus l'approximation de $\frac {B(n, p)-np}{\sqrt {np(1-p)}}$ par une $\Nzo$ est précise. Dès lors,
			pour des grandes valeurs de $n$, on peut dire que :

			\[P(X \geq \beta) = P\left(\frac {X-np}{\sqrt {np(1-p)}} \geq \frac {\beta - np}{\sqrt {np(1-p)}}\right).\]

			En posant $\alpha \coloneqq \frac {\beta - np}{\sqrt {np(1-p)}}$, on a $P(X \geq \beta) = 1 - F_{\Nzo}(\alpha)$.

			On a donc deux moyens différents d'approximer une binomiale. Le premier étant une poisson :

			\[B(n, p) \stackrel{\begin{aligned}n&\to+\infty\\p&\to0\\np&\to\lambda\end{aligned}}{\longrightarrow}P_\lambda.\]

			Le second étant une normale :

			\[B(n, p) \stackrel{n\to+\infty}{\longrightarrow}\mathcal(np, \sqrt {np(1-p)}).\]

			Pour déterminer quelle est la meilleure approximation à choisir, il faut déterminer celle qui correspond le mieux aux hypothèses posées\footnote{Sur $n$, $p$ et $np$.}.

		\subsubsection{Approximation d'une Poisson}
			Tout comme une binomiale, une Poisson peut être approximée par une normale selon une certaine hypothèse. En effet, si $\lambda \to +\infty$, alors
			$P_\lambda \to \mathcal N(\lambda, \sqrt \lambda)$.

			Pour le montrer, il faut à nouveau passer par la stabilité des Poissons : $P_\lambda = \lambda P_1 + P_{\lambda-\lfloor\lambda\rfloor}$. Par le théorème central limite,
			on a la même propriété, à savoir : une normale $\mathcal N(\lambda, \sqrt \lambda)$ devient une bonne approximation d'une Poisson $P_\lambda$ quand $n \to +\infty$.

	\subsection{Théorème de Binaymé-Tchebycheff}
		Soient $X$ une variable aléatoire et $k \in \mathbb R_0^+$. Alors :

		\[P\left(\abs{X-E(X)} \geq kD(X)\right) \leq k^{-2}.\]

		\underline{Démonstration : } Soient $X$ une variable aléatoire quelconque et $k$ un réel positif. On sait que :
		
		\begin{align*}
			\sigma_X^2 &= D^2(X) = \int_{-\infty}^{+\infty}(x-\mu_X)^2f_X(x)\dif x \\
			           &= \int_{-\infty}^{\mu_X-k\sigma_X} (x-\mu_X)^2f_X(fx)\dif x + \int_{\mu_X - k\sigma_X}^{\mu_X + k\sigma_X}(x-\mu_X)^2f_X(x)\dif x + \int_{\mu_X + k\sigma_X}^{+\infty}(x-\mu_X)^2f_X(x)\dif x \\
					   &\geq \int_{-\infty}^{\mu_X-k\sigma_X}(x-\mu_X)f_X(fx)\dif x + \int_{\mu_X + k\sigma_X}^{+\infty}(x-\mu_X)^2f_X(x)\dif x.
		\end{align*}

		Soient $P_1 = ]-\infty, \mu_X-k\sigma_X], P_2 = [\mu_X+k\sigma_X, +\infty[$. On a :

		\begin{align*}
			\forall x \in P_1 : \exists \epsilon \geq 0 \tq x &= \mu_X - (k\sigma_X + \epsilon) \\
			                                      (x-\mu_X)^2 &= (\mu_X - (k\sigma_X+\epsilon) - \mu_X)^2 \geq k^2\sigma_X^2, \\
			\forall x \in P_2 : \exists \epsilon \geq 0 \tq x &= \mu_X + (k\sigma_X + \epsilon) \\
			                                      (x-\mu_X)^2 &= (\mu_X + (k\sigma_X+\epsilon) - \mu_X)^2 \geq k^2\sigma_X^2.
		\end{align*}

		Dès lors :

		\begin{align*}
			\sigma_X^2 &\geq k^2\sigma_X^2\int_{-\infty}^{\mu_X-k\sigma_X}f_X(x)\dif x + k^2\sigma_X^2\int_{\mu_X+k\sigma_X}^{+\infty}f_X(x)\dif x \\
			           &= k^2\sigma_X^2\left[ \int_{-\infty}^{+\infty}f_X(x)\dif x - \int_{\mu_X-k\sigma_X}^{\mu_X+k\sigma_X}f_X(x)\dif x \right] \\
					   &= k^2\sigma_X^2\left(1-P\left(\abs{X-\mu_X} \leq k\sigma_X\right)\right) \\
					   &= k^2\sigma_X^2\left(P\left(\abs{X-\mu_X} \geq k\sigma_X\right)\right).
		\end{align*}

		En divisant de part et d'autre par $k^2\sigma_X^2$, on obtient :

		\[\frac 1{k^2} \geq P\left(\abs{X-\mu_X} \geq k\sigma_X\right).\;\;\;\square\]

	\subsection{Théorème de Bernoulli}
		Si $F \sim B(n, p)$, alors :
		
		\[\forall \epsilon \geq 0 : P\left(\abs{\frac Fn - p} \geq \epsilon\right) \stackrel{n\to+\infty}{\longrightarrow}0.\]

		\underline{Démonstration : } Soient $F \sim B(n, p)$, $V = \frac Fn, \mu_V = E(V) = \frac 1nE(F) = \frac 1nnp = p$. On sait que :

		\[P\left( \abs{\frac Fn-p} \geq \epsilon \right) = P\left( \abs{V-\mu_V} \geq \epsilon\right).\]

		Or $D^2(V) = D^2\left(\frac Fn\right) = \frac 1{n^2}D^2(F) = \frac 1{n^2}np(1-) = \frac {p(1-p)}n$. Posons $k \coloneqq \frac \epsilon{\sqrt {p(1-p)}n}$. Dès lors :

		\[\lim_{n\to+\infty}P\left(\abs{\frac Fn - p} \geq \epsilon\right) = \lim_{n\to+\infty}P\left(\abs{V-\mu_V} \geq k\sigma_V\right) \leq \lim_{n\to+\infty}\frac 1{\frac {\epsilon^2}{\sqrt{\frac {p(1-p)}n}^2}} = \frac {p(1-p)}{n\epsilon^2}\lim_{n\to+\infty}\frac 1n = 0.\]
	
\section{Inférence statistique}
	\textbf{La population} est l'ensemble des objets étudiés et n'est pas observable (soit pas en entier, soit pas directement). \textbf{Un échantillon} est un sous-ensemble (aléatoire et simple)
	réellement observé de la population. Le sens de l'inférence statistique est de voir en quelle mesure il est possible inférer des conclusions venant de l'échantillon sur la population.

	Si l'échantillon n'est pas aléatoire simple, il risque d'y avoir un biais qui peut être important.

	\subsection{Biais des valeurs particulières}
		Soit une population $P$. Soit $A$ l'expérience correspondant au prélèvement d'un échantillon d'effectif $n$ dans $P$. Chaque échantillon est sous la forme $x_1, x_2 \ldots, x_n$.
		Soient $X_1, X_2, \ldots, X_n$ des expériences aléatoires telles que $X_i$ est l'expérience relative à la valeur de $x_i$. Du fait que l'échantillon est aléatoire et simple, on
		sait que la distribution de probabilité de $X_i$ est la même que la distribution de probabilité de la population pour tout $i$. De plus, chaque échantillon a une moyenne
		$\overline x = \frac 1n\sum_{i=1}^nx_i$. Notons $\overline X$ l'expérience aléatoire associant chaque échantillon à sa moyenne.

		En supposant que la population admet pour moyenne $\mu$, on peut déterminer l'espérance de la variable aléatoire représentant la moyenne de l'échantillon :

		\[E(\overline X) = E\left(\frac 1n\sum_{i=1}^nX_i\right) = \frac 1n\sum_{i=1}^nE(X_i) = n\frac \mu n = \mu.\]

		Dès lors, on peut affirmer que \textit{la moyenne de la moyenne est la moyenne}, ce qui veut dire que la valeur moyenne de la moyenne de l'échantillon est égale à la moyenne de la population.
		Ou encore : en moyenne, l'évaluation de la moyenne de l'échantillon donnera la moyenne de la population. Cela veut dire que la prise d'échantillon est un \textbf{estimateur non biaisé}.

		En supposant $X_i \sqcup X_j \; \forall i \neq j$ et en supposant que la population est d'écart type $\sigma$, on peut déterminer l'écart type de l'échantillon (dispersion par rapport
		à la moyenne) :

		\[D^2(\overline X) = D^2\left(\frac 1n\sum_{i=1}^nX_i\right) = \frac 1{n^2}\sum_{i=1}^nD^2(X_i).\]

		Comme tous les $X_i$ sont de même distribution que la population, $D^2(X_i) = \sigma^2 \;\forall i$. Dès lors, $D^2(\overline X) = \frac 1{n^2}n\sigma^2 = \frac 1n\sigma^2$. Quand $n$ est grand,
		$D^2(\overline X)$ est petit et donc $\overline X$ représente bien la moyenne de la population. De plus, si $n$ est \textit{très grand}, alors le théorème central-limite dit que
		$\overline X \sim \mathcal N\left(\mu, \frac \sigma{\sqrt n}\right)$. Soit $S$ l'événement aléatoire relatif à la variance de l'échantillon. Dès lors :

		\[E(S^2) = E\left(\frac 1n\sum_{i=1}^nX_i^2-\overline X^2\right) = \frac 1n\sum_{i=1}^nE(X_i^2) - E(\overline X^2).\]

		Or $D^2(X_i) = E(X_i^2) - E(X_i)^2$, donc $E(X_i^2) = D^2(X_i) + E(X_i)^2$. Et de manière similaire, $E(\overline X^2) = D^2(\overline X) + E(\overline X)^2$. Donc :

		\[E(S^2) = \frac 1n\sum_{i=1}^n(D^2(X_i) + E(X_i)^2) - (D^2(\overline X) + E(\overline X)^2) = \frac 1nn(\sigma^2 + \mu^2) - (\frac {\sigma^2}n + \mu^2) = \frac {n-1}n\sigma^2.\]

		Contrairement à $\overline X$, $S$ est un opérateur biaisé : en moyenne la variance de l'échantillon correspond à $\frac {n-1}n$ fois la variance de la population. Soit
		$\widetilde S^2 \coloneqq \frac 1{n-1} \sum_{i=1}^n(X_i-\overline X)^2$. On a donc :

		\[E(\widetilde S) = \frac 1{n-1}E\left(\sum_{i=1}^n(X_i-\overline X)^2\right) = \frac n{n-1}E\left(\frac 1n\sum_{i=1}^n(X_i-\overline X)^2\right) = \frac n{n-1}E(S^2) = \sigma^2.\]

		$\widetilde S$ est donc un meilleur estimateur que $S^2$ pour l'écart-type et la variance.

	\subsection{Estimer un paramètre}
		Quand il faut pouvoir estimer le paramètre d'une distribution de probabilité, il faut soit partir de sa définition (ex : le paramètre $\lambda$ d'une Poisson est l'espérance de la distribution)
		et la retrouver comme cela ($\lambda = \frac 1n\sum_{i=1}^nx_i$ pour la Poisson), soit utiliser le maximum de vraissemblance. \\
		Le maximum de vraissemblance (maximum likelihood en anglais) d'un paramètre $\theta$ est la valeur de $\theta$ qui maximise la probabilité d'observation de l'échantillon.

		On note $L_\theta(x_1, \ldots, x_n) \coloneqq P(X_1=x_1, \ldots, X_n=x_n)$ la probabilité que l'échantillon observé soit $x_1, x_2, \ldots, x_n$ en ayant le paramètre $\theta$
		(la fonction de vraissemblance). On a donc :

		\[L_\theta(x_1, \ldots, x_n) = \prod_{i=1}^nP(X_i=x_i ; \theta).\]

		Afin de maximiser la probabilité selon le paramètre $\theta$, il faut trouver pour quelle valeur la dérivée voit 0 :

		\[\pd \theta L_\theta(x_1, \ldots, x_n) = 0.\]

		Cependant, afin de ne pas devoir dériver un produit, on passe par la vraissemblance logarithmique (log-likelihood en anglais) :

		\[\pd \theta \log L_\theta(x_1, \ldots, x_n) = \pd \theta \log \prod_{i=1}^n P(X_i=x_i ; \theta) = \sum_{i=1}^n \pd \theta \log P(X_i=x_i).\]

		Par exemple, dans le cas d'une variable de Poisson, trouver le paramètre $\lambda$ sur base de l'échantillon $x_1, \ldots, x_n$ se fait de la manière suivante :

		\begin{align*}
			\sum_{i=1}^n\pd \lambda \log P_\lambda(X_i=x_i) &= \sum_{i=1}^n \pd \lambda \log\left(\frac {\lambda^{x_i}}{x_i!}\exp(-\lambda)\right)
			= \sum_{i=1}^n\pd \lambda \left(\log\lambda^{x_i} - \log x_i! + \log\exp-\lambda\right) \\
			&= \sum_{i=1}^n\pd\lambda\left(x_i\log\lambda - \log x_i! - \lambda\right) = \sum_{i=1}^n\left(x_i\frac 1\lambda - 1\right) = \frac 1\lambda\sum_{i=1}^nx_i - n.
		\end{align*}

		Or cette valeur vaut 0, donc :

		\[\frac 1\lambda\sum_{i=1}^nx_i - n = 0 \iff \lambda = \frac 1n\sum_{i=1}^nx_i.\]

		Ce qui dit que pour estimer le paramètre $\lambda$ d'une poisson il faut faire la moyenne des éléments. On retombe bien sur la définition.

	\subsection{Intervalle de confiance}
		L'idée derrière les intervalles de confiance est de trouver un intervalle de valeurs où le paramètre inconnu apparait avec une haute probabilité.

		Soit une population de moyenne $\mu$ inconnue et d'écart-type $\sigma$ connu. $\overline X \sim \mathcal N(\mu, \frac \sigma{\sqrt n})$ car $n$ est suffisamment grand.
		Soit $\epsilon \in [0, 1[$. On cherche un intervalle $[-z_\epsilon, z_\epsilon]$ tel que :

		\[P\left(-z_\epsilon \leq \frac {\overline X-\mu}{\frac \sigma{\sqrt n}} \leq z_\epsilon\right) = 1-\epsilon.\]

		Ou encore :

		\[P\left(\mu \in \left[\overline X \pm z_\epsilon\frac \sigma{\sqrt n}\right]\right) = 1-\epsilon\]

		L'intervalle dépend de $z_\epsilon$ qui dépend d'$\epsilon$. Donc plus $\epsilon$ est grand, plus la probabilité attendue est basse donc plus l'intervalle est réduit (plus $\abs{z_\epsilon}$
		est petit). La longueur de l'intervalle est donc $2z_\epsilon\frac \sigma{\sqrt n}$. En imposant une limite $d$ telle que $d \geq 2z_\epsilon\frac \sigma{\sqrt n}$, on sait
		déterminer le nombre le nombre minimum d'effectifs à avoir dans l'intervalle pour obtenir une telle précision : $n \geq \left(\frac {2z_\epsilon\sigma}d\right)^2$.

	\subsection{Test d'hypothèse}
		Si $\overline X$ est la variable aléatoire correspondant à la moyenne de l'échantillon, on sait que $\overline X \sim \mathcal N\left(\mu, \frac \sigma{\sqrt n}\right)$.
		Posons $T \coloneqq \frac {\overline X - \mu}{\frac \sigma{\sqrt n}}$. Soient deux hypothèses $H0$ et $H1$ telles que $H0 \equiv \lnot H1$.

		Dès lors, en reprenant la notion d'intervalle de confiance, il faut déterminer laquelle des deux hypothèses $H0$ ou $H1$ est admissible. L'intervalle de confiance permet
		de délimiter \textit{jusqu'à quel point l'hypothèse peut s'écarter de la réalité pour qu'elle soit admissible}. Par exemple, si $\overline X$ correspond à la moyenne des notes à un
		examen (la valeur observée est 5), si l'échantillon est d'effectif 9, que l'écart-type est 1, prenons pour hypothèses $H0 \equiv \mu = 12$  donc $H1 \equiv \mu \neq 12$. Pour $H0$, on a :

		\[T = \frac {\overline X - \mu}{\frac \sigma{\sqrt n}} = \frac {5 - 12}{\frac 1{\sqrt 9}} = -21.\]

		Or en prenant $\epsilon = 0.05$, on veut trouver un intervalle contenant $\mu$ avec une probabilité de $0.95$. La valeur de $z_\epsilon$ pour $\epsilon = 0.05$ est $1.96$. Dès lors,
		on voit bien que $-21 \not\in [-1.96, 1.96]$. L'hypothèse $H0$ est donc à rejeter. En faisant le même calcul si $\overline X$ avait pour valeur $12.5$, on aurait eu :

		\[T = \frac {\overline X - \mu}{\frac \sigma{\sqrt n}} = \frac {12.5 - 12}{\frac 1{\sqrt 9}} = \frac 32.\]

		En prenant le même $\epsilon$, on peut admettre l'hypothèse $H0$ car $\frac 32 \in [-1.96, 1.96]$.

		Le lien \textit{formel} entre test d'hypothèse en intervalle de confiance est que l'hypothèse $H0$ est rejetée si et seulement si l'hypothèse n'entre pas dans l'intervalle de confiance :

		\[RHO \iff \mu \not \in \left[\overline X \pm z_\epsilon \frac \sigma{\sqrt n}\right].\]

		Le lien peut également être formulé comme suit : \textit{la probabilité que l'hypothèse soit rejetée en faisant un faux négatif est $\epsilon$}. En effet, un test d'hypothèse
		est concluant si l'hypothèse $H0$ est rejetée et qu'elle n'est pas valable ou si l'hypothèse $H1$ est rejetée et qu'elle n'est pas valable. Dans le cas où $H0$ est rejetée
		alors qu'elle est valable (idem pour $H1$) est appelé faux négatif. On a donc :

		\[P(RH0 \, | \, H0) = \epsilon.\]

		On y voit que en prenant $\epsilon$ de plus en plus petit, on s'attend à une probabilité de plus en plus grande de trouver $H0$ et donc pour $\epsilon \to 0$, $H0$ ne sera jamais rejetée.
		Dès lors, $P(RH0 \, | \, H0) \to 0 = \epsilon$. À l'inverse, quand $\epsilon \to 1$, l'intervalle est réduit jusqu'à devenir le singleton $\{\mu\}$ et donc $H0$ sera toujours rejeté.
		Donc $P(RH0 \, | \, H0) \to 1 = \epsilon$.
	
	\subsection{Student et $\chi^2$}
		Soient $X_1, X_2, \ldots, X_n \sim \Nzo$ des variables aléatoires indépendantes. Alors :

		\[X \coloneqq \sum_{i=1}^nX_i^2 \sim \chi^2_{(n)}.\]

		Soient $V_0, V_1, V_2, \ldots, V_2 \sim \mathcal N(0, \sigma)$ des variables aléatoires indépendantes. Alors :

		\[V \coloneqq \frac {V_0}{\sqrt {\frac 1n\sum_{i=1}^nV_i}} \sim t_{(n)}.\]

		$X$ est appelée $\chi^2$ (\textit{chi carrée}) et $V$ est appelée student. Toutes deux sont de degré de liberté $n$.

		Si $\overline X \sim \mathcal N\left(\mu, \frac \sigma{\sqrt n}\right)$ est la variable correspondant à la moyenne de l'échantillon, $n$ est l'effectif de l'échantillon et $S$ son écart type, alors :

		\[\frac {nS^2}{\sigma^2} \sim \chi^2_{(n-1)}.\]

		De plus, pour une student d'écart type $\sigma = 1$, on a :

		\[t_{(n)} = \frac \Nzo{\sqrt {\frac 1n\sum_{i=1}^n\Nzo^2}} = \frac \Nzo{\sqrt {\frac 1n\chi^2_{(n)}}}.\]

		Dès lors :

		\[t_{(n-1)} = \frac \Nzo{\sqrt {\frac 1{n-1}\chi^2_{(n-1)}}} = \frac {\frac {\overline X - \mu}{\frac \sigma{\sqrt n}}}{\sqrt {\frac 1{(n-1)}\frac {nS^2}{\sigma^2}}}
		= \frac {(\overline X - \mu)\sqrt n\sqrt {n-1}\sigma}{\sigma\sqrt nS} = \frac {(\overline X-\mu)\sqrt {n-1}}S\]

		Or, si $\frac {(\overline X - \mu)\sqrt {n-1}}S \sim t_{(n-1)}$, alors :

		\[P\left(-t_{(n-1), \epsilon} \leq \frac {(\overline X - \mu)\sqrt {n-1}}S \leq t_{(n-1), \epsilon}\right) = 1-\epsilon.\]

		Où $\pm t_{(n-1), \epsilon}$ sont les quantiles d'une student à $(n-1)$ degrés de liberté. De là :

		\[P\left(\mu \in \left[\overline X \pm t_{(n-1), \epsilon}\frac S{\sqrt {n-1}}\right]\right) = 1-\epsilon.\]

		Ce résultat permet de se débarrasser de l'hypothèse disant que l'on connait $\sigma$, l'écart type de la population. En effet, il est possible de faire un intervalle de
		confiance et des tests d'hypothèses sans avoir à connaitre $\sigma$.
	
	\subsection{Analyse de deux populations distinctes}
		Soient deux populations $P_1, P_2$ suivant des distributions normales $\mathcal N(\mu_1, \sigma_1)$ et $\mathcal N(\mu_2, \sigma_2)$. Soient deux échantillons $E_1 \subset P_1, E_2 \subset P_2$
		d'effectif respectif $n_1$ et $n_2$. Soient $\overline X_1$ et $\overline X_2$ les variables aléatoires des moyennes de $E_1$ et $E_2$. On sait que
		$\overline X_1 \sim \mathcal N\left(\mu_1, \frac {\sigma_1}{\sqrt n_1}\right)$ et $\overline X_2 \sim \mathcal N\left(\mu_2, \frac {\sigma_2}{\sqrt n_2}\right)$

		Pour évaluer si les deux populations ont la même moyenne , on procède à nouveau à un test d'hypothèse. Posons :

		\[T \coloneqq \frac {(\overline X_1 - \overline X_2) - (\mu_1 - \mu_2)}{\sqrt {\frac {\sigma_1^2}{n_1} + \frac {\sigma_2^2}{n_2}}}.\]

		Si les moyennes sont égales ($\mu_1 = \mu_2$), alors on procède au test d'hypothèse suivant :

		\[\mu_1 = \mu_2 \iff 0 \in \left[\overline X_1 - \overline X_2 \pm z_\epsilon\sqrt {\frac {\sigma_1^2}{n_1} + \frac {\sigma_2^2}{n_2}}\right] \iff \frac {\overline X_1 - \overline X_2}{\sqrt {\frac {\sigma_1^2}{n_1} + \frac {\sigma_2^2}{n_2}}} < z_\epsilon.\]

\end{document}
