\documentclass{report}

\usepackage{hyperref}
\usepackage[french]{babel}
\usepackage{commath}
\usepackage{palatino, eulervm}
\usepackage[T1]{fontenc}
\usepackage[utf8]{inputenc}
\usepackage{fullpage}
\usepackage{amsmath, amsthm, amssymb, amsfonts}
\usepackage{mathtools}
\usepackage{stmaryrd}
\usepackage[bottom]{footmisc}
\usepackage[parfill]{parskip}

\title{Calcul différentiel et intégral II}
\author{R. Petit}
\date{année académique 2016 - 2017}

% amsthm
\newtheorem{thm}{Théorème}[chapter]
\newtheorem{prp}[thm]{Proposition}
\newtheorem{cor}[thm]{Corollaire}
\renewcommand{\proofname}{\it{Démonstration}}
\theoremstyle{definition}
\newtheorem{déf}[thm]{Définition}
\theoremstyle{remark}
\newtheorem*{rmq}{Remarque}
\newtheorem{ex}{Exemple}[chapter]

\DeclareMathOperator{\K}{\mathbb K}
\DeclareMathOperator{\C}{\mathbb C}
\DeclareMathOperator{\R}{\mathbb R}
\DeclareMathOperator{\Rp}{\R^{+}}
\DeclareMathOperator{\Rm}{\R^{-}}
\DeclareMathOperator{\N}{\mathbb N}
\DeclareMathOperator{\Ns}{\N^{*}}
\DeclareMathOperator{\tq}{\text{t.q.}}

% sequence; parameters: sequence letter, sequence index, index set
\newcommand{\seq}[3]{\left(#1_{#2}\right)_{#2 \in #3}} 
% sequence of instanciated function; parameters: sequence function letter, sequence index, function variable name, index set
\newcommand{\seqf}[4]{\left(#1_{#2}\left(#3\right)\right)_{#2 \in #4}}
% metric set convergence; parameters: variable name, variable limit, distance function
\newcommand{\mconv}[3]{\xrightarrow[#1 \to #2]{#3}}  
% Convergence
\newcommand{\CONV}[5]{\xrightarrow[#2 \to #3]{#4 \text{ #5 } #1}}
% Simple convergence
\newcommand{\CVS}[3]{\CONV{#1}{#2}{#3}{CVS}{sur}}
% Simple convergence on compacts
\newcommand{\CVSc}[3]{\CONV{#1}{#2}{#3}{CVS}{sur tout cpct de}}
% uniform convergence
\newcommand{\CVU}[3]{\CONV{#1}{#2}{#3}{CVU}{sur}}
% uniform convergence on compacts
\newcommand{\CVUc}[3]{\CONV{#1}{#2}{#3}{CVU}{sur tout cpct de}}

\newcommand{\restr}[2]{\left.#1\vphantom{\big|}\right|_{#2}}
\newcommand{\intint}[2]{\left\llbracket#1, #2\right\rrbracket}
\newcommand{\scpr}[2]{\left\langle #1, #2\right\rangle}

\newcommand{\pinfty}{+\infty}
\newcommand{\grantedproof}{\begin{proof} \underline{Admis.} \end{proof}}
\newcommand{\evn}{espace vectoriel normé}
\newcommand{\evnc}{{\evn} complet}

\setcounter{secnumdepth}{5}
\setcounter{tocdepth}{5}

\begin{document}

\pagenumbering{Roman}
\maketitle
\tableofcontents
\newpage
\setcounter{page}{1}
\pagenumbering{arabic}

\chapter{Suites et séries de fonctions}
	\section{Rappels}
		\subsection{Topologie métrique}
			\subsubsection{Espaces métriques}
				\begin{déf} Soit $X$ un ensemble. Une \textit{distance} sur $X$ est une application $d : X \times X \to \Rp$ telle que~:

				\begin{enumerate}
					\item $\forall x, y \in X : d(x, y) = d(y, x)$ (symétrie)~;
					\item $\forall x, y, z \in X : d(x, z) \leq d(x, y) + d(y, z)$ (inégalité triangulaire)~;
					\item $\forall x, y \in X : \left(d(x, y) = 0 \iff x = y\right)$ (séparation\footnote{Également appelé
					      \textit{principe d'identité des indiscernables.}}).
				\end{enumerate}
				\end{déf}

				\begin{déf}  On appelle \textit{espace métrique} $(X, d)$ un espace $X$ muni d'une distance $d$ sur $X$. \end{déf}

				\begin{déf} Soient $(X, d)$ un espace métrique, $\seq xn\N$ et $x \in X$ La suite $(x_n)$ converge vers $x$ dans $(X, d)$ lorsque~:
				\[\forall \varepsilon > 0 : \exists N \in \N \tq \forall n \geq \N : d(x_n, x) < \varepsilon.\]

				Cela se note~:
				\[x_n \mconv n\pinfty d x.\]
				\end{déf}

				\begin{prp} Soit $\seq xn\N$ une suite dans $(X, d)$, un espace métrique. Soient $x, y \in X$. Si~:
				\[x_n \mconv n\pinfty d x \qquad\qquad \text{et} \qquad\qquad x_n \mconv n\pinfty d y,\]
				alors $x = y$. \end{prp}

				\begin{proof} Soit $\varepsilon > 0$. Puisque $x_n \to x$ et $x_n \to y$, on sait qu'il existe $N_1, N_2 \in \N$ tels que~:
				\[\forall n \geq N_1 : d(x_n, x) < \frac \varepsilon2 \qquad\qquad \text{et} \qquad\qquad \forall n \geq N_2 : d(x_n, y) < \frac \varepsilon2.\]

				Dès lors, soit $N \coloneqq \max\{N_1, N_2\}$. On peut dire~:
				\[\forall n \geq N : d(x, y) \leq d(x, x_n) + d(x_n, y) < \frac \varepsilon2 + \frac \varepsilon2 = \varepsilon.\]

				On en déduit $d(x, y) = 0$ et donc $x = y$ par séparation. \end{proof}

			\subsubsection{Espaces vectoriels}
				\begin{déf} Soit $\K$, un sous-corps de $\C$. On appelle \textit{norme} sur le $\K$-e.v. $E$ toute application $n : E \to \Rp$ telle que~:

				\begin{enumerate}
					\item $\forall x \in E : \left(n(x) = 02 \iff x = 0\right)$~;
					\item $\forall x \in E : \forall \lambda \in \K : n(\lambda x) = \abs \lambda n(x)$~;
					\item $\forall x, y \in E : n(x + y) \leq n(x) + n(y)$.
				\end{enumerate}
				\end{déf}

				\begin{prp} Soit $(E, n)$ un $\K$-\evn. L'application $d$ suivante est une distance sur $E$ (on l'appelle la
				\textit{distance associée à la norme $n$})~:
				\[d : E \times E \to \Rp : (x, y) \mapsto n(y-x).\]
				\end{prp}

				\begin{proof} EXERCICE.
				\end{proof}

				\begin{rmq} Si $(E, n)$ est un \evn, $\seq xn\N$ est une suite de $E$, et si $x \in E$, alors on dit~:
				\[x_n \mconv n\pinfty n x\]
				lorsque~:
				\[x_n \mconv n\pinfty{} x\]
				au sens de la distance associée à la norme $n$.
				\end{rmq}

				\begin{ex} $\R$ est un $\R$-e.v. normé avec pour norme $n : x \mapsto \abs x$. \end{ex}

				\begin{ex} Soient $d \in \Ns$, $p \in [1, \pinfty)$. Pour $x = (x_i)_{1 \leq i \leq d} \in \C^d$, on définit~:
				\[n(x) = \norm x_p \coloneqq \left(\sum_{k=0}^d\abs {x_i}^p\right)^{\frac 1p}.\]
				On a alors $(\C^d, n)$ est un $\C$-\evn. Également $(\C^d, n)$ et $(\R^d, n)$  sont des $\R$-espaces vectoriels normés.
				\end{ex}

				\begin{déf} Soit $x \in \C^d$. On définit la \textit{norme infinie} de $x$ dans $\C^d$ par~:
				\[\norm x_\infty \coloneqq \max_{1 \leq i \leq d}\abs {x_i}.\]
				\end{déf}

				\begin{ex} Soit $d \in \Ns$. $(\C^d, \norm \cdot_\infty)$ est un $\C$-\evn. Également, $(\R^d, \norm \cdot_\infty)$
				et $(\C^d, \norm \cdot_\infty)$ sont des $\R$-espaces vectoriels normés. \end{ex}

				\begin{proof} EXERCICE.
				\end{proof}

				\begin{déf} Soit $\seq xn\N$ une suite. On dit que la suite $(x_n)$ est \textit{presque nulle} s'il existe $N \in \N$ tel que
				$\forall n \geq N : x_n = 0$. \end{déf}

				\begin{ex} Soient $P \in \C[x]$ et $\seq ak\N$ la suite presque nulle des coefficients de $P$. On pose~:
				\[\norm P_\infty \coloneqq \sup_{k \in \N}\abs {a_k} = \max_{k \in \N}\abs {a_k}.\]

				Alors $\norm \cdot_\infty$ est une norme sur $\C[x]$.
				\end{ex}

				\begin{proof} EXERCICE.
				\end{proof}

			\subsubsection{Ouverts, fermés, compacts}
				\begin{déf} Soit $(X, d)$ un espace métrique. On appelle \textit{boule ouverte} de centre $x \in X$ et de rayon $r \gneqq 0$ l'ensemble~:
				\[B(x, r[ \coloneqq \left\{y \in X \tq d(x, y) \lneqq r\right\}.\]

				On définit également la \textit{boule fermée} de centre $x$ et de rayon $r$ l'ensemble~:
				\[B(x, r] \coloneqq \{y \in X \tq d(x, y) \leq r\}.\]
				\end{déf}

				\begin{déf} Soit $(X, d)$ un espace métrique et soit $O \subset X$. On dit que $O$ est une partie \textit{ouvert} dans $X$ lorsque~:
				\[\forall x \in O : \exists r \gneqq 0 \tq B(x, r)  \subset O.\]
				\end{déf}

				\begin{rmq} Pour tout $X$, les ensembles $\emptyset$ et $X$ sont tous deux des ouverts de $X$. \end{rmq}

				\begin{déf} Soit $(X, d)$ un espace métrique. Une partie $F \subset X$  de $X$ est dite \textit{fermée} dans $X$ lorsque $X \setminus F$
				est ouvert. \end{déf}

				\begin{prp} Dans un espace métrique $(X, d)$, soit $\seq OiI$ une famille d'ouverts de $X$ indicés par un ensemble $I \neq \emptyset$.
				Alors $\left(\bigcup_{i \in I}O_i\right)$ est un ouvert de $X$. Si de plus $I$ est fini, alors $\left(\bigcap_{i \in I}\right)$ est
				un ouvert de $X$. \end{prp}

				\begin{ex} Prenons $X = \R$ et $O_i = (-1-\frac 1i, 1 + \frac 1i)$. Alors $\left(\bigcap_{i \in \Ns}O_i\right) = [-1, 1]$ qui n'est pas
				un ouvert de $X$. \end{ex}

				\begin{proof} EXERCICE.
				\end{proof}

				\begin{déf}[Compacts par Borel-Lebesgue] Soit $(X, d)$ un espace métrique. Une partie $K \subset X$ est dite \textit{compacte} si
				$K \neq \emptyset$ et si, de tout recouvrement de $K$ par des ouverts de $X$, on peut extraire un sous-recouvrement fini.

				C'est-à-dire lorsque~:
				\begin{enumerate}
					\item $K \neq \emptyset$~;
					\item $\forall I \neq \emptyset : \forall \seq OiI$ ouverts de $X \tq K \subset \left(\bigcup_{i \in I}O_i\right) : \exists J \subset I$
					      fini $\tq K \subset \left(\bigcup_{j \in J}O_j\right)$.
				\end{enumerate}
				\end{déf}

				\begin{prp}[Compacts par Bolzano-Weierstrass] Soit $(X, d)$ un espace métrique. Une partie $K$ de $X$ est compacte si et seulement si~:

				\begin{enumerate}
					\item $K \neq \emptyset$~;
					\item de toute suite de points de $K$, on peut extraire une sous-suite convergente dans $K$.
				\end{enumerate}
				\end{prp}

				\grantedproof

				\begin{ex} L'ensemble $[0, 1]$ est un compact de $\R$. \end{ex}

				\begin{prp} Soit $(X, d)$, un espace métrique et $K \subset X$, une partie compacte. Alors $K$ est fermé et borné. \end{prp}

				\begin{proof} EXERCICE. (Absurde)
				\end{proof}

				\begin{prp} Soit $(E, n)$ un $\K$-e.v. normé de dimension finie. Alors les parties compactes de $E$ sont les parties fermées
				bornées non nulles. \end{prp}

				\grantedproof

			\subsubsection{Suites de Cauchy}
				\begin{déf} Soit $(X, d)$, un espace métrique. On dit que $\seq xn\N$ est \textit{de Cauchy} dans $X$ lorsque~:
				\[\forall \varepsilon > 0 : \exists N \in \N \tq \forall m, n \geq N : d(x_n, x_m) < \varepsilon.\]
				\end{déf}

				\begin{prp} Si $\seq xn\N$ est convergente dans l'espace métrique $(X, d)$, alors elle est de Cauchy. \end{prp}

				\begin{proof} Si $x$ est la limite de la suite $(x_n)$, on pose $\varepsilon > 0$. Il existe $N \in \N$ tel que~:
				\[\forall n \geq \N : d(x, x_n) < \frac \varepsilon2.\]

				Donc $\forall m, n \geq N : d(x_m, x_n) \leq d(x_m, x) + d(x, x_n) < \varepsilon$.
				\end{proof}

				\begin{déf} Un espace métrique $(M, d)$ est dit \textit{complet} quand toute suite de Cauchy de points de $X$ converge dans $X$. \end{déf}

				\begin{déf} Un espace vectoriel $E$ est dit \textit{de Banach} lorsque toute suite de Cauchy de vecteurs de $E$ converge dans $E$. \end{déf}

				\begin{rmq} On remarque que dans un espace métrique complet, une suite converge si et seulement si elle est de Cauchy
				(ce qui est entre autres le cas de $\R$).

				De plus, les suites de Cauchy permettent, dans des espaces complets, de montrer que des suites convergent sans connaitre leur limite. \end{rmq}

				\begin{ex} Les espaces métriques $(\R, \abs \cdot)$ et $(\C, \abs \cdot)$ sont des espaces de Banach. Et pour tout $p \in [1, \pinfty)$ et
				$q \in \N$, les espaces métriques $(\R^q, \norm \cdot_p)$ et $(\C^q, \norm \cdot_p)$ sont des espaces de Banach. \end{ex}

			\subsubsection{Continuité}
				\begin{déf} Soient $(X, d_X)$ et $(Y, d_Y)$ deux espaces métriques. Une application $f : X \to Y$ est dite continue en $x_0 \in X$ lorsque~:
				\[\forall \varepsilon > 0 : \exists \delta \gneqq 0 \tq \forall x \in X :
					\left(d_X(x, x_0) < \delta \Rightarrow d_X(f(x), f(x_0)) < \varepsilon\right).\]

				On dit que $f$ est continue sur $A \subset X$ lorsque $f$ est continue en tout $a \in A$.
				\end{déf}

				\begin{prp} Une fonction $f : (X, d) \to (Y, d)$ est continue sur $X$ lorsque l'image réciproque par $f$ de $(Y, d)$ est un ouvert de $(X, d)$.
				\end{prp}

				\grantedproof

				\begin{prp} Une fonction $f : (X, d) \to (Y, d)$ est continue en $x_0 \in X$ si et seulement si l'image par $f$ de toute suite de points de
				$X$ convergente en $x_0$ est une suite convergente en $f(x_0)$. \end{prp}

				\grantedproof

				\begin{déf} Soit $f : (X, d) \to (Y, d)$. $f$ est dite \textit{lipschitzienne} de constante $K \geq 0$ lorsque~
				\[\forall (x, y) \in X^2 : d(f(x), f(y)) \leq d(x, y).\]
				\end{déf}

				\begin{prp} Si $f : (X, d) \to (Y, d)$ est lipschitzienne, alors elle est continue sur $X$. \end{prp}

				\begin{proof} EXERCICE.
				\end{proof}

				\begin{déf} Soit $\seq ak\N$, une suite dans un espace métrique $(X, d)$. On dit que $(a_k)$ est \textit{presque nulle} lorsqu'il existe
				$N \in \N$ tel que $\forall n \geq N : a_n = 0$. \end{déf}

				\begin{ex}~
				\begin{itemize}
					\item Pour tout $i \in \N$, l'application $c_i : \C[x] \to \C : P = \sum_{k=0}^\pinfty a_kx^k \mapsto a_i$ est continue de
					      $(\C[x], \norm \cdot_\infty)$ dans $(\C, \abs \cdot)$. En effet, pour $i \in \N$, $P = \sum_{k=0}^\pinfty a_kx^k$, et
					      $Q = \sum_{k=0}^\pinfty b_kx^k$, on a~:
						  \[\abs{c_i(P) - c_i(Q)} = \abs{a_i - b_i} \leq \norm{P-Q}_\infty = \max_{k \in \N}\abs{a_k - b_k}.\]

					      On en déduit que $c_i$ est lipschitzienne sur $\C[x]$ et donc continue sur $\C[x]$.
					\item Soit $n \in \N$. Posons~:
					      \[P_n = \sum_{k=0}^n\frac 1{k!}x^k \in \C[x].\]

					      On observe que $\seq Pn\N$ est de Cauchy dans $(\C[x], \norm \cdot_\infty)$ car~:
					      \[\norm{P_n - P_m}_\infty = \norm{\sum_{k=0}^n\frac 1{k!}x^k - \sum_{k=0}^m\frac 1{k!}x^k}_\infty.\]

					      On a alors~:
					      \[\norm{P_n - P_m}_\infty = \norm {\sum_{k=\min\{m, n\}+1}^{\max\{m, n\}}\frac 1{k!}x^k}_\infty =
					      	\max_{\min\{m, n\}+1 \leq k \leq \max\{m, n\}}\frac 1{k!} = \frac 1{(\min\{m, n\}+1)!}.\]

					      Montrons que $\seq Pn\N$ est de Cauchy. Supposons (par l'absurde) que $\seq Pn\N$ converge vers $P \in (\C[x], \norm \cdot_\infty)$.
					      Notons $(a_k) \subset \C$, la suite presque nulle des coefficients de $P$. Pour $i \in \N$, on a $c_i(P) = \frac 1{i!}$ quand
					      $n \geq i$. Or par la propriété de Lipschitz, on sait que $c_i(P_n) \mconv n\pinfty{} c_i(P) = a_i$. Or $(a_k)$ est presque nulle et
					      $a_i = \frac 1{i!}$. Il y a donc contradiction. Donc $(P_n)$ ne converge pas dans $(\C[x], \norm \cdot_\infty)$. Dès lors,
					      $(\C[x], \norm \cdot_\infty)$ n'est pas complet.
				\end{itemize}
				\end{ex}

	\section{Convergence de suites de fonctions}
		\subsection[Convergence simple]{Convergence simple\protect\footnote{La convergence simple est la notion de convergence «~minimale~» que l'on va exiger.
		Il existe des convergences encore plus élémentaires (voir théorie de l'intégration de Lebesgue), mais qui se trouvent en dehors des objectifs du cours.}}
			\begin{déf} Soit $X$ un ensemble et $(Y, d)$ un espace métrique. On dit que la suite $\seqf fnx\N$ où $f_n : X \to (Y, d)$
			\textit{converge simplement} sur $X$ lorsque~:
			\[\forall x \in X : \seqf fnx\N \text{converge dans } (Y, d).\]
			\end{déf}

			\begin{déf} Dans ce cas, la suite a pour limite simple la fonction~:
			\[f : X \to (Y, d) : x \mapsto \lim_{n \to \pinfty}f_n(x)\]
			et est bien définie. Cela se note~:
			\[f_n \xrightarrow[n \underset{X}{\to} \pinfty]{CVS} f \qquad\qquad \text{ou} \qquad\qquad f_n \CVS Xn\pinfty f.\]
			\end{déf}

			\begin{ex} Soient $X = [0, 1]$ et $Y = \R$. On pose $f_n(x) = x^n$ pour tout $n \in \N$.
			\begin{itemize}
				\item Si $x \in [0, 1)$, alors la suite $\seqf fnx\N$ est une suite géométrique de raison $x$ avec $\abs x < 1$ donc la suite converge vers 0~;
				\item si $x = 1$,a lors $f_n(x) = 1$ pour tout $n \in \N$. Donc la suite $\seqf fnx\N$ converge simplement sur $[0, 1]$ vers la fonction~:
				      \[f : [0, 1] \to \R : x \mapsto \begin{cases}0 &\text{ si } x  < 1 \\ 1 &\text{ si } x = 1\end{cases}.\]
			\end{itemize}
			\end{ex}

			\begin{rmq}~
			\begin{itemize}
				\item On a «~perdu~» la continuité des fonctions $f_n$ par passage à la limite~;
				\item ici, la convergence simple peut s'écrire ainsi, à l'aide de quantificateurs~:
				      \[\forall \varepsilon > 0 : \forall x \in X : \exists N \in \N \tq \forall n \geq N : d(f_n(x), f(x)) < \varepsilon.\]

				      On remarque donc que $N$ dépend de $x$ (ordre des quantificateurs).
			\end{itemize}
			\end{rmq}

		\subsection{Convergence uniforme}
			\begin{déf} Soient $X$ un ensemble, $(Y, d)$ un espace métrique, et $f_n : X \to (Y, d)$. On dit que $(f_n)$ \textit{converge uniformément} sur $X$
			vers $f : X \to (Y, d)$ lorsque~:
			\[\forall \varepsilon > 0 : \exists N \in \N \tq \forall n \geq N : \forall x \in X : d(f_n(x), f(x)) < \varepsilon.\]

			Cela se note~:
			\[f_n \CVU Xn\pinfty f.\]
			\end{déf}

			\begin{rmq} La définition est très proche de la convergence simple. La différence étant que pour une convergence uniforme, il faut que
			$N \in \N$ ne dépende pas de la valeur de $x$. \end{rmq}

			\begin{prp} Soient $X$ un ensemble, $(Y, d)$ un espace métrique, $\seqf fnx\N$ une suite de fonctions de $X$ dans $(Y, d)$ et $f : X \to (Y, d)$.
			Si $(f_n)$ converge uniformément sur $X$ vers $f$, alors $(f_n)$ converge simplement sur $X$ vers $f$. \end{prp}

			\begin{proof} EXERCICE.
			\end{proof}

			\begin{ex} Prenons $X = \R = Y$ et pour tout $n \geq 1$, définissons $f_n(x) = \sqrt{x^2 + \frac 1n}$. Fixons $x \in \R$. On trouve alors~:
			\[\seqf fnx\N = \seq {\sqrt{x^2 + \frac 1n}}n\N \to \sqrt {x^2} = \abs x.\]

			Donc~:
			\[f_n \CVS Xn\pinfty \abs \cdot.\]
			\end{ex}

			\begin{thm} Soient $(X, d)$, $(Y, d)$ deux espaces métriques. Soient $f_n : X \to Y$, $a \in X$. On suppose~:
			\begin{itemize}
				\item $\exists f \tq f_n \CVU Xn\pinfty f$~;
				\item $\forall n \in \N : f_n$ est continue en $a$.
			\end{itemize}

			Alors $f$ est continue en $a$.
			\end{thm}

			\begin{proof}  Soit $\varepsilon > 0$. Par convergence uniforme des $f_n$, on sait~:
			\[\exists N \in \N \tq \forall n \geq N : \forall x \in X : d(f_n(x), f(x)) < \frac \varepsilon3.\]

			De plus, la fonction $f_N$ est continue en $a$ par hypothèse. Dès lors, on sait qu'il existe $\delta$ tel que~:
			\[\forall x \in X : d(x, a) < \delta \Rightarrow d(f_N(x), f_N(a)) < \frac \varepsilon3.\]

			Ainsi, prenons $x \in X$ tel que $d(x, a) < \delta$. On a alors~:
			\[d(f(x), f(a)) \leq d(f(x), f_N(x)) + d(f_N(x), f(a)) \leq d(f(x), f_N(x)) + d(f_N(x), f_N(a)) + d(f_N(a), f(a)) \leq 3\frac \varepsilon3 = \varepsilon.\]
			\end{proof}

			\begin{cor} Si $f_n \in C^0(X, Y)$ et $f_n \CVU Xn\pinfty$, alors $f \in C^0(X, Y)$. \end{cor}

			\begin{proof} Les fonctions $f_n$ sont continues en tout point et $f_n \CVU Xn\pinfty$ par hypothèse. Dès lors, pour tout point $a \in X$, par le
			théorème précédent, on peut dire $f$ continue en $a$. Dès lors $f \in C^0(X, Y)$.
			\end{proof}

		\subsection{L'espace $B(X, E)$}
			\begin{déf} Soient $X \neq \emptyset$ et $(E, \norm \cdot_E)$ un \evn. On note~:
			\[B(X, E) \coloneqq \left\{f : X \to E \tq f \text{ est bornée sur } X\right\}.\]

			Pour $f \in B(X, E)$, on définit~:
			\[\norm f_\infty \coloneqq \sup_{x \in X}\norm {f(x)}_E.\]
			\end{déf}

			\begin{prp} $\left(B(X, E), \norm \cdot_\infty\right)$ est un \evn. \end{prp}

			\begin{proof} EXERCICE.
			\end{proof}

			\begin{thm}\label{thm:BXEcpltssiEcplt} $\left(B(X, E), \norm \cdot_\infty\right)$ est complet si et seulement si $(E, \norm \cdot_E)$ est complet.
			\end{thm}

			\begin{proof} Supposons d'abord $\left(B(X, E), \norm \cdot_\infty\right)$ complet et montrons que $(E, \norm \cdot_E)$ est complet.

			Soit $(x_n)_n$ une suite de Cauchy d'éléments de $E$. Soit $(f_n)$ une suite de fonctions de $B(X, E)$ telle que~:
			\[\forall n \in \N : \forall x \in X : f_n(x) = x_n.\]

			Puisque $(x_n)$ est de Cauchy, on sait que~:
			\[\forall \varepsilon > 0 : \exists N \in \N \tq \forall m, n \geq N : d(x_m, x_n) < \varepsilon.\]
			Or, avec $a \in X$ fixé, on peut alors dire $\forall m, n \geq N : d(f_m(a), f_n(a)) < \varepsilon$, et ce peu importe le $a$ choisi (car les
			$f_n$ sont constantes). On a donc $(f_n)$ une suite de Cauchy dans $B(X, E)$ car $d(f_m(a), f_n(a)) = \norm {f_m - f_n}_\infty$. Or, par
			complétude de $B(X, E)$, on sait qu'il existe $f \in B(X, E)$ telle que $f_n \to f$. La fonction $f$ est également constante. Posons $L$ la seule
			image de $f$. Soit $\varepsilon > 0$. On sait qu'il existe $n \in \N$ tel que $\forall n \geq N : \norm {f_n - f}_\infty < \varepsilon.$

			Or~:
			\[\varepsilon > \norm {f_n - f}_\infty = \sup_{x \in X} \norm {f_n(x) - f(x)}_E = \norm {f_n(a) - f(a)}_E = \norm {x_n - L}.\]
			Dès lors, on sait que $(x_n)$ converge dans $E$.

			Montrons maintenant que si $(E, \norm \cdot_E)$ est complet, alors $(B(X, E), \norm \cdot_\infty)$ est complet également.

			Soit $(f_n)_n$ une suite de Cauchy de fonctions de $(B(X, E), \norm \cdot_\infty)$. Fixons $\varepsilon > 0$. Il existe alors $N \in \N$ tel que~:
			\[\forall m, n \geq N : \norm {f_m - f_n}_\infty < \varepsilon.\]

			Soit $x \in X$. On observe que~:
			\[\forall m, n \geq N : \norm {f_n(x) - f_m(x)}_E \leq \norm {f_n - f_m}_\infty < \varepsilon.\]
			La suite $(f_n(x))_n$ est donc une suite de Cauchy dans $(E, \norm \cdot_E)$. Par complétude de $E$, on sait qu'il existe $f(x) \in E$ tel que
			$f_n(x) \to f(x)$. Montrons maintenant que $f \in B(X, E)$.

			La suite $(f_n)_n$ est de Cauchy et donc bornée. Soit $M \gneqq 0$ tel que $\forall n \in \N : \norm {f_n}_\infty < M$. Passons à la limite dans
			$(B(X, E)$. On a alors~:
			\[\forall n \in \N : \forall x \in X : \norm {f(x)}_E < M.\]

			Ainsi, $f \in B(X, E)$ par définition.

			Soit alors $\varepsilon > 0$. Pour tout $m, n \in \N$ et pour tout $x \in X$, on a~:
			\[\norm {f_n(x) - f_m(x)}_E  \leq \norm {f_n - f_m}_\infty \leq \varepsilon.\]

			Passons alors à la limite e $m$, ce qui donne~:
			\[\norm {f_n(x) - f(x)}_E \leq \norm {f_n - f}_\infty \leq \varepsilon.\]
			Dès lors~:
			\[\forall n \geq N : \norm {f_n - f}_\infty \leq \varepsilon.\]
			\end{proof}

			\begin{rmq} Quand $X \neq \emptyset$ et $Y = E$ est un \evn, on a~:
			\[f_n \CVU Xn\pinfty f \iff
				\begin{cases}&\exists N \in \N \tq \forall n \geq N : f_n - f \in B(X, E) \\ &f_n - f \xrightarrow[n \to \pinfty]{\norm \cdot_\infty} 0\end{cases}.\]
			\end{rmq}

		\subsection{Convergence uniforme sur tout compact}
			\begin{déf} Soit $X$, une partie non-vide d'un \evn de dimension finie $(E, \norm \cdot_E)$. Soit $(Y, d)$ un espace métrique.
			Une suite $f_n : X \to Y$ converge uniformément vers $f : X \to Y$ sur tout compact lorsque~:
			\[\forall \text{ compact } K \subset X : \restr {f_n}K \CVU Kn\pinfty \restr fK.\]

			Cela se note~:
			\[f_n \CVUc Xn\pinfty f.\]
			\end{déf}

			\begin{prp}\label{prp:cvuccontinues} Si la suite $f_n$ converge uniformément sur tout compact de $X$ et si toutes les fonctions $f_n$ sont continues
			en $a \in X$, alors $f$ est continue en $a$.
			\end{prp}

			\begin{proof} EXERCICE.
			\end{proof}

			\begin{ex} Prenons $X = Y = \R$. On définit $f_n(x) = \sum_{k=0}^n\frac {x^k}{k!}$. On a alors $f_n \CVS Xn\pinfty \exp$.

			De plus~:
			\[\norm {f_n - \exp}_\infty = \sup_{x \in \R} \abs {\sum_{k=0}^n\frac {x^k}{k!} - \exp(x)} = \pinfty.\]
			Donc $f_n$ ne converge pas uniformément vers $\exp$. Montrons maintenant que $f_n$ converge uniformément vers $\exp$ sur tout compact de $\R$.
			Soit $K \subset \R$ un compact. On sait qu'il existe $a, b \in \R, a < b$ tels que $K \subset [a, b]$. Pour $x \in [a, b]$, par Lagrange, on a~:
			\[\exp(x) - \sum_{k=0}^n\frac {x^k}{k!} = \frac {x^{n+1}}{(n+1)!}\exp(c_x),\]
			avec $c_x \in [a, b]$.

			Ainsi~:
			\[\abs {\exp(x) - \sum_{k=0}^n\frac {x^k}{k!}} \leq \frac {(b-a)^{n+1}}{(n+1)!}\sup_{x \in [a, b]}\exp(x) \xrightarrow[n \to \pinfty]{} 0.\]

			D'où $f_n \xrightarrow[n \to \pinfty]{[a, b]} f$ et donc la convergence uniforme sur tout compact de $f_n$ vers $f$.
			\end{ex}

	\section{Suites de fonctions et opérations d'intégration et de dérivation}
		\subsection{Passage à la limite dans une intégrale de Riemann}
			Soit $X$ un pavé de $\R^d$ (donc $X = \prod_{i=1}^d[a_i, b_i]$ avec $a_i < b_i \forall i \in \{1, \ldots, d\}$).

			\begin{thm} Soit $f_n : X \to \R$ intégrables au sens de Riemann sur $X$. Supposons $f_n \CVU Xn\pinfty f$. Alors~:
			\begin{itemize}
				\item $f$ est intégrable au sens de Riemann~;
				\item la $\left(\int_X f_n(x)\dif x\right)_n$ converge vers $\int_Xf(x)\dif x$.\footnote{Cela veut dire que~:
					\[\lim_{n \to \pinfty}\int_X f_n(x)\dif x = \int_X \lim_{n \to \pinfty} f_n(x)\dif x.\]}
			\end{itemize}
			\end{thm}

			\begin{proof} On note $\mathcal E(X, \R) \coloneqq \{f : X \to \R \tq f \text{ est élémentaire}\}$.

			Soit $\varepsilon > 0$. Par la convergence uniforme, on sait qu'il existe $N \in \N$ tel que~:
			\[\forall n \geq N : \norm {f_n - f}_\infty \leq \frac \varepsilon{4\abs X},\]
			où $\abs X = \prod_{i=1}^d(b_i - a_i)$.

			Par intégrabilité de $f_N$, on sait qu'il existe $\varphi, \psi \in \mathcal E(X, \R)$ telles que~:
			\[\psi \leq f_N \leq \varphi \qquad\qquad \text{ et } \qquad\qquad \int_X(\varphi - \psi) < \frac \varepsilon2.\]

			On a alors~:
			\[\psi - f_N \leq f \leq \varphi + f_N,\]
			ou encore~:
			\[\psi - \frac \varepsilon{4\abs X} \leq f \leq \varphi + \frac \varepsilon{4\abs X}.\]

			En posant $\overline \psi \coloneqq \psi - \frac \varepsilon{4\abs X}$ et $\overline \varphi \coloneqq \varphi + \frac \varepsilon{4\abs X}$, on a
			$\overline \psi, \overline \varphi \in \mathcal E(X, \R)$. De plus~:
			\[\int_X(\psi-\varphi) = \int_X\left(\psi + \frac \varepsilon{4\abs X} - \left(\varphi - \frac \varepsilon{4\abs X}\right)\right)
				= \frac \varepsilon{2\abs X}\abs X + \int_X \psi - \varphi < 2\frac \varepsilon2 = \varepsilon.\]

			Dès lors, on en déduit $f$ intégrable au sens de Riemann.

			Fixons $\varepsilon > 0$. Par convergence uniforme de $f_n$ vers $f$ sur $X$, on sait que~:
			\[\exists N \in \N \tq \forall n \geq N : \norm {f_n - f}_\infty < \frac \varepsilon{\abs X}\]

			Et donc~:
			\[\abs {\int_X f_n(x)\dif x - \int_X f(x)\dif x} = \abs {\int_X (f_n-f)(x)\dif x} \leq \abs {\int_X\norm {f_n-f}_\infty\dif x}
				= \abs X\norm{f_n-f}_\infty \leq \abs X\frac \varepsilon{\abs X} = \varepsilon.\]

			Finalement, la suite $\left(\int_X f_n(x)\dif x\right)_n$ converge dans $\R$ vers $\int_X f(x)\dif x$.
			\end{proof}

			\begin{rmq}~
			\begin{enumerate}
				\item Il est possible d'avoir les résultats sans vérifier les hypothèses. Par exemple, $X = [0, 1] \subset \R = Y$, avec $f_n(x) = x^n$.
				      On sait que $f_n \CVS Xn\pinfty 1_{\{x=1\}}$ et que la convergence n'est pas uniforme sur $[0, 1]$. On remarque alors~:
				      \[\lim_{n\to\pinfty}\int_0^1f_n(x)\dif x = \lim_{n\to\pinfty}\frac 1{n+1} = 0 = \int_0^11_{\{x=1\}}(x)\dif x
				      	= \int_0^1\lim_{n\to\pinfty}f_n(x)\dif x~;\]
				\item si les hypothèses ne sont pas vérifiées, la conclusion peut être fausse. Par exemple, $X = [0, 1] \subset \R = Y$. On définit ($n \geq 1$)~:
				      \[f_n(x) = \begin{cases}
				      	2n\alpha_nx &\text{ si } 0 \leq x < \frac 1{2n} \\
				        2\alpha_n - 2n\alpha_nx &\text{ si }\frac 1{2n} \leq x < \frac 1n \\
				        0 &\text{ sinon}\end{cases},\]
				      où $\alpha_n \in \R^+_0 \tq \forall n \in \N^* : \int_0^1f_n(x)\dif x = 1$, donc $\alpha_n = 2n$.

				      On a alors $f_n \CVS Xn\pinfty 0 = f$. La fonction nulle $0(x)$ est intégrable au sens de Riemann sur $[0, 1]$.

				      Finalement, on a~:
				      \[\int_0^1\lim_{n \to \pinfty}f_n(x)\dif x = \int_0^1f(x)\dif x = 0 \qquad \text{ et } \qquad \lim_{n \to \pinfty}\int_0^1f_n(x)\dif x = 1.\]

				      Dans ce cas précis, on ne peut pas passer à la limite.
			\end{enumerate}
			\end{rmq}

		\subsection{Passage à la limite dans une dérivation ordinaire ou partielle}
			\begin{thm} Soit $\emptyset \neq \Omega \subset \R^d$, un ouvert. Soient $f_n : \Omega \to \R$, toutes de classe $C^1$ sur $\Omega$. Supposons~:
			\begin{itemize}
				\item $f_n \CVSc \Omega n\pinfty f$~;
				\item $\forall i \in \intint 1d : \pd {f_n}{x_i} \CVU \Omega n\pinfty g_i$.
			\end{itemize}

			Alors~:
			\begin{enumerate}
				\item $f \in C^1(\Omega, \R)$~;
				\item $\forall i \in \intint 1d : \pd f{x_i} = \lim_{n \to \pinfty} \pd {f_n}{x_i}$ dans $\Omega$~;
				\item $f_n \CVUc \Omega n\pinfty f$.
			\end{enumerate}
			\end{thm}
			
			\begin{proof} Soit $x \in \Omega$. Par ouverture de $\Omega$, on sait qu'il existe $\delta \gneqq 0$ tel que $B(x, \delta[ \subset \Omega$.
			On en déduit que $B(x, \frac \delta2]$ est incluse dans $B(x, \delta[$. Or $B(x, \frac \delta2]$ est fermé et borné par définition.
			$B(x \frac \delta2]$ est donc un compact de $\Omega$.

			Soient $i \in \intint 1d$ et $h \in [\pm \frac \delta2]$. On a alors~:
			\[\forall n \in \N : f_n(x + he_i) = f_n(x) + \int_0^h\pd f{x_i}(x+s e_i)\dif s.\]

			Or comme $f_n \CVSc \Omega n\pinfty f$ et pour tout $i$, $\pd {f_n}{x_i}$ converge uniformément vers $g_i$ sur $B(x, \frac \delta2]$, il vient~:
			\[f_n(x + he_i) = f_n(x) + \int_0^hg_i(x+se_i)\dif s,\]
			où $\{e_1, \ldots e_d\}$ est la base canonique de $\R^d$.

			On en déduit alors que $f$ admet une dérivée partielle par rapport à $x_i$ en $x$~:
			\[\pd f{x_i}(x) = g_i(x).\]

			De plus, les $f_n$ sont $C^1(\Omega, \R)$, et donc les dérivées partielles $\pd {f_n}{x_i}$ sont $C^0(\Omega, \R)$ pour tout $i$ et par convergence
			uniforme sur les compacts, $g_i \in C^0(\Omega, \R)$ (Proposition~\ref{prp:cvuccontinues}).

			On en déduit alors $f \in C^1(\Omega, \R)$ avec $\pd f{x_i} = g_i$ pour tout $i$ dans $\Omega$ (points 1 et 2 à montrer).

			Il reste donc à montrer le point 3.

			Soit $K \subset \Omega$, un compact. Par ouverture de $\Omega$, on sait que pour tout $a \in K$, on a~:
			\[\exists r_a \gneqq 0 \tq B(a, r_a[ \subset \Omega.\]

			Dès lors, on sait que~:
			\[K \subset \bigcup_{a \in K}B\left(a, \frac {r_a}2\right[.\]

			Par complétude, on sait qu'il existe un sous-recouvrement fini de $K$, c'est-à-dire $p \in \N^*$ et $(a_i)_{i \in \intint 1p} \in K^p$ tel que~:
			\[K \subset \bigcup_{i=1}^pB\left(a_i, \frac {r_{a_i}}2\right[.\]

			Par convergence simple de $f_n$ vers $f$, et puisque les $a_i$ sont en nombre fini, on peut alors exprimer~:
			\[\forall \varepsilon > 0 : \exists N \in \N \tq \forall n \geq N : \forall k \in \intint 1p\abs {f_n(a_i) - f(a_i)} < \varepsilon.\]

			Fixons donc $\varepsilon > 0$, soit $N$ correspondant et soit $x \in K$. Il existe $k \in \intint 1p$ tel que $x \in B(a_k, \frac {r_{a_k}}2[$ car
			les boules ouvertes forment un recouvrement de $K$. On a alors~:
			\[f_n(x) = f_n(a_k) + \int_0^1\scpr {\nabla f_n(a_k + t(x - a_k))}{(x - a_k)}\dif t,\]
			et~:
			\[f(x) = f(a_k) + \int_0^1\scpr {\nabla f(a_k + t(x - a_k))}{(x - a_k)}\dif t.\]

			Par différence, on a~:
			\[\abs {f_n(x) - f(x)} \leq \abs {f_n(a_k) - f(a_k)} + \int_0^1\norm {\nabla f_n(a_k + t(x - a_k)) - \nabla f(a_k + t(x - a_k))}\dif t \cdot \norm {x - a_k}.\]

			Par convergence uniforme sur $\left(\bigcup_{i=1}^pB(a_i, \frac {r_{a_i}}2]\right)$ de $\nabla f_n$ vers $\nabla f$, on sait que~:
			\[\exists N \in \N \tq \forall n \geq N : \norm {\nabla f_n - \nabla f}_{\infty, \bigcup_{i=1}^pB\left(a_i, \frac {r_{a_i}}2\right]} <
				\frac {2\varepsilon}{\displaystyle \max_{i \in \intint 1p} r_{a_i}}.\]

			Finalement, on a~:
			\[\forall x \in K : \forall n \geq N : \norm {f_n(x) - f(x)} \leq
				\varepsilon + \frac {r_{a_k}}{2} \cdot \frac {2\varepsilon}{\displaystyle \max_{i \in \intint 1d}r_{a_i}} \leq 2\varepsilon.\]
				
			Ainsi, pour $n \geq N$, on a~:
			\[\norm {f_n - f}_{\infty, K} \leq 2\varepsilon.\]
			\end{proof}

			\begin{rmq} Ce théorème est vrai en particulier pour $d = 1$, et $\Omega$ un segment de $\R$. \end{rmq}

			\begin{ex}[Contre-exemples ne vérifiant pas les hypothèses donc ne pouvant faire passer la limite dans la dérivation]
			\begin{enumerate}
				\item $f_n(x) = \frac {\sin(n^2x)}n, n \geq 1, x \in X = \R$. On a donc $f_n \CVU \R n\pinfty 0 = f$ car $\abs {f_n(x) - f(x)} \leq \frac 1n \to 0$
				      avec $\frac 1n$ ne dépendant pas de $x$. Les $f^n$ sont $C^\infty(\R)$ et sont donc dérivables~:
				      \[\od {f_n}x = n\cos(n^2x),\]
				      et donc~:
				      \[\od {f_n}x\sVert[3]_{x=0} = n \to \pinfty.\]

			          On en déduit~:
				      \[\lnot\left(\od {f_n}x \xrightarrow[n \to \pinfty]{} \od fx\right).\]

				\item $f_n(x) = \frac {x^n}n, n \geq 1, x \in X = [0, 1]$. On a $f_n \CVU Xn\pinfty 0$. Puisque les $f_n$ sont $C^\infty(X, \R)$, on a~:
				      \[\od {f_n}x = x^{n-1} \xrightarrow[n \to \pinfty]{} 1_{\{x=1\}},\]
				      qui n'est pas une dérivée. À nouveau, la suite des dérivées des $f_n$ ne tend pas vers la dérivée de $f$.
			\end{enumerate}
			\end{ex}

			\begin{cor} Soient $p \in \N^*, \Omega \subset \R^d$, un ouvert non-vide. Soit $f_n : \Omega \to \R$ de classe $C^p(\Omega, \R)$. Suposons~:
			\begin{itemize}
				\item $\displaystyle \forall q \in \intint 0{p-1} : \forall (i_1, \ldots, i_q) \in \intint 1d^q :
					\frac {\partial^qf_n}{\partial x_{i_1}\ldots\partial x_{i_q}} \CVU \Omega n\pinfty g_{i_1, \ldots, i_q}$~;
				\item $\displaystyle \forall (i_1, \ldots i_p) \in \intint 1d^p :
					\frac {\partial^pf_n}{\partial x_{i_1}\ldots\partial x_{i_p}} \CVU \Omega n\pinfty g_{i_1, \ldots i_q}$.
			\end{itemize}
			Alors~:
			\begin{enumerate}
				\item $f = g_\emptyset \in C^p(\Omega, \R)$~;
				\item $\displaystyle \forall q \in \intint 1p : \forall (i_1, \ldots, i_q) \in \intint 1d^q :
					\frac {\partial^qf}{\partial x_{i_1}\ldots\partial x_{i_q}} = g_{i_1, \ldots, i_q}$~;
				\item $\forall q \in \intint 0{p-1} : \forall (i_1, \ldots, i_q) \in \intint 1d^q :
					\frac {\partial^qf_n}{\partial x_{i_1}\ldots\partial x_{i_q}} \CVUc \Omega n\pinfty g_{i_1, \ldots, i_q}$.
			\end{enumerate}
			\end{cor}

			\begin{proof} EXERCICE. (Récurrence sur $p$ par le résultat précédent)
			\end{proof}

	\section{Séries de fonctions}
		\subsection{Retranscription des résultats sur les suites}
			\begin{déf} Soit $u_n : X \to Y$ où $X \neq \emptyset$ et $Y$ est un \evn. On appelle \textit{somme partielle d'ordre $n$ de la série de terme
			général $u_n$} la fonction suivante~:
			\[S_n : X \to Y : x \mapsto \sum_{k=0}^nu_k(x).\]

			On dit que la série de terme général $u_n$ \textit{converge simplement} sur $X$ lorsque $S_n$ converge simplement sur $X$. De même pour la
			\textit{convergence uniforme} sur $X$ et la \textit{convergence uniforme sur tout compact} de $X$.
			\end{déf}

			\begin{thm} Soient $(X, d)$ un espace métrique et $Y$ un \evn. Soit $u_n : X \to Y$. Si $\forall n \in \N : u_n$ est continue en $a \in X$ et si
			la série de terme général $u_n$ converge uniformément sur $X$, alors~:
			\[S \coloneqq \lim_{n \to \pinfty}S_n \text{ est continue en } a.\]
			\end{thm}

			\begin{proof} EXERCICE.
			\end{proof}

			\begin{thm} Soit $X \neq \emptyset$, un pavé de $\R^d$ et soit $u_n : X \to Y \tq \sum_{n \geq 0}u_n \CVU Xn\pinfty S$ avec $u_n$ intégrable au
			sens de Riemann pour tout $n$. Alors~:
			\begin{enumerate}
				\item $S$ est intégrable au sens de Riemann sur $X$~;
				\item la suite $\int_XS_n(x)\dif x$ converge vers $\int_XS(x)\dif x$.
			\end{enumerate}
			\end{thm}

			\begin{proof} EXERCICE.
			\end{proof}

			\begin{thm} Soit $\Omega \subset \R^d$, un ouvert non-nul et soit $u_n : \Omega \to \R$ de classe $C^p(\Omega, \R)$ avec $p \in \N^*$.
			Supposons~:
			\begin{itemize}
				\item $\sum_{n \geq 0}u_n \CVS \Omega n\pinfty S$~;
				\item $\displaystyle \forall \alpha \in \N^d \tq \abs \alpha \coloneqq \sum_{i=1}^d\alpha_i \leq p :
					\sum_{n \geq 0}\frac {\partial^{\abs \alpha}}{\partial x_1^{\alpha_1}\ldots\partial x_d^{\alpha_d}}u_n \CVS \Omega n\pinfty s_{\alpha}$~;
				\item lorsque $\abs \alpha = p$, la convergence ci-dessus est uniforme sur les compacts de $\Omega$.
			\end{itemize}

			Alors~:
			\begin{enumerate}
				\item $S \in C^p(\Omega, \R)$~;
				\item $\displaystyle \forall \alpha \in \N^d :
					\frac {\partial^{\abs \alpha}}{\partial x_1^{\alpha_1}\ldots\partial x_d^{\alpha_d}}S =
						\sum_{n \geq 0}\frac {\partial^{\abs \alpha}}{\partial x_1^{\alpha_1}\ldots\partial x_d^{\alpha_d}}u_n$~;
				\item Il y a convergence uniforme sur les compacts de $\Omega$ des séries de dérivées partielles d'ordre $0$ à $p-1$.
			\end{enumerate}
			\end{thm}

		\subsection{Convergence normale}
			\begin{déf} Soient $X \neq \emptyset$ et $Y$ un \evn. On dit que la série de terme général $u_n : X \to Y$ converge normalement sur $X$ lorsque~:
			\[\sum_{n \geq 0}\norm {u_n}_{\infty, X} < \pinfty.\]
			\end{déf}

			\begin{déf} On dit que la série de terme général $u_n : X \to Y$ vérifie le critère de Weierstrass lorsqu'il existe $\seq Mn\N$ telle que~:
			\begin{itemize}
				\item $\displaystyle \forall n \in \N : \forall x \in X : \norm {u_n(x)}_E \leq M_n$~;
				\item $\displaystyle \sum_{n \geq 0}M_n < \pinfty$.
			\end{itemize}
			\end{déf}

			\begin{rmq} $\sum_{n \geq 0}u_n$ converge normalement sur $X$ si et seulement si elle vérifie le critère de Weierstrass.
			\end{rmq}

			\begin{prp} Si $(E, \norm \cdot_E)$ est un \evnc, et si $\sum_{n \geq 0}u_n$ converge normalement sur $X$ alors $\sum_{n \geq 0}$
			converge uniformément sur $X$.
			\end{prp}

			\begin{proof} Écrivons $S_n = \sum_{k=0}^nu_k \in B(X, E)$. Par convergence normale, la suite $\sigma_n = \sum_{k \geq 0}\norm {u_k}_{\infty, X}$
			converge. De plus, $(\sigma_n)$ est de Cauchy dans $\Rp$. Donc~:
			\[\forall \varepsilon > 0 : \exists N \in \N \tq \forall n, m \geq N : \abs {\sigma_n - \sigma_m} < \varepsilon.\]

			Ainsi~:
			\begin{align*}
				\norm {S_n - S_m}_{\infty, X} &= \norm {\sum_{k=\min(m, n)+1}^{\max(m, n)}u_k}_{\infty, X} \leq \sum_{k=\min(m, n)+1}^{\max(m, n)}\norm {u_k}_{\infty, X} \\
				&\leq \abs {\sigma_n - \sigma_m} < \varepsilon.
			\end{align*}

			Donc $(S_n)_n$ est de Cauchy dans $\left(B(X, E), \norm \cdot_{\infty, X}\right)$. Cet espace est complet car $(E, \norm \cdot_E)$ l'est
			(Théorème~\ref{thm:BXEcpltssiEcplt}). Et donc, $(S_n)_n$ converge uniformément sur $X$.
			\end{proof}

			\begin{rmq} On peut écrire~:
			\[CVN \underset {\text{complet}}\Rightarrow CVU \Rightarrow CVS,\]
			mais les réciproques sont habituellement fausses.
			\end{rmq}
\end{document}
